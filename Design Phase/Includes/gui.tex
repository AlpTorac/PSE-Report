%Components
\begin{package}{Components}
    \struct{+}{i}{AutoResizing}[
        \func{+}[void]{setResizer}[Component=Component to set the resizer on. | ComponentResizer=Resizer to use for the component.]{Associates a resizer with a component, making it resize it.}
    ]{
        Interface for components that are to be resized automatically.
    }

    \struct{+}{e}{ResizeMode}?
        ABSOLUTE\_TOP\_LEFT=component stays fixed in relation to the top/left of the parent component;
        ABSOLUTE\_BOTTOM\_RIGHT=component stays fixed in relation to the bottom/right of the parent component;
        RELATIVE=component keeps a relative distance to the parent componennt's edges.
    ?{
        Enum values represent resizing behaviour.
    }

    \struct{+}{c}{Resizer}<
        \func{+}{Resizer}[ResizeMode:t=Top edge alignment. | ResizeMode:r=Right edge alignment. | ResizeMode:b=Bottom edge alignment. | ResizeMode:l=Left edge alignment.]{Creates new resizer with specified attributes.}
    >[
        \func{+}[void]{resize}[Component=Component to resize. | int:width=From width. | int:height=From height. | int:newWidth=To width. | int:newHeight=To height.]{Resizes specified component to fit specified dimensions of parent component.}
    ]{
        Classes that automatically resize use this class to define behaviour when resizing.
    }

    \struct{+}{c}{Window:JFrame}?
        Map(Component, ComponentResizer)
    ?{
        Works like JFrame, but with automatic resizing.
    }

    \struct{+}{c}{Panel:JPanel}?
        Map(Component, ComponentResizer)
    ?{
        Works like JPanel, but with automatic resizing.
    }

    \struct{+}{c}{InputBox:JTextField}{Adapted version of the parent Swing component.}
    \struct{+}{c}{TextPane:JTextPane}{Adapted version of the parent Swing component.}
    \struct{+}{c}{DropdownSelector:JComboBox}{Adapted version of the parent Swing component.}
    \struct{+}{c}{Tree:JTree}{Adapted version of the parent Swing component.}
    \struct{+}{c}{MenuBar:JMenuBar}{Adapted version of the parent Swing component.}
    \struct{+}{c}{Checkbox:JCheckBox}{Adapted version of the parent Swing component.}
    \struct{+}{c}{Button:JButton}{Adapted version of the parent Swing component.}
    \struct{+}{c}{Label:JLabel}{Adapted version of the parent Swing component.}
    \struct{+}{c}{ScrollPane:JScrollPane}{Adapted version of the parent Swing component.}
\end{package}

\newpage
%Composites
\begin{package}{Composites}
    \struct{+}{c}{MainWindow:Window}?
        MenuBar:mainMenu=Menu bar with options that affect the entire project/
    ?<
        \func{+}{MainWindow}{Creates the main window and instantiates child components.}
    >[
        \func{+}<static>[void]{main}[String[]:args]{The program entry point.}
    ]{
        Class representing the main window. It manages all of the GUI elements a user can interact with.
    }

    \struct{+}{e}{InfoBarID}?
            InfoBarID:VERSION=Identifies version text label;
            InfoBarID:SELECTED=Identifies selection text label;
            InfoBarID:LASTACTION=Identifies last action text label;
            InfoBarID:ERROR=Identifies error text label
        ?{
            Identifies a data section of an InfoBar.
        }

    \struct{+}{c}{InfoBar:Panel}?
        Map(InfoBarID, TextBox)=Map to identify parts of the info bar.
        ?{
            Panel with a collection of Info about the application status to display to the user.
        }

    \struct{+}{c}{Outliner:Panel}?
            MenuBar:menu;
            Tree;
            Model
        ?<
            \func{+}{Outliner}[Model:data]{Creates outliner and associates it with a Model.}
        >{
            Interactive panel that shows a tree view of elements to the user.
        }

    \struct{+}{c}{SubWindowArea:Panel}?
        Map(SubWindow\&, SubWindow):subWindows=Aggregation of all SubWindows currently present in the panel.
    ?[
        \func{+}[void]{addSubWindow}[SubWindow:sub=SubWindow to add]{Adds a SubWindow to the panel.};
        \func{+}[void]{removeSubWindow}[SubWindow:sub=SubWindow to add]{Removes a SubWindow from the panel.}
    ]{
        Panel where SubWindows are opened/displayed.
    }

    \struct{+}{c}{SubWindow:Panel}?
        MenuBar:menu
        ElementManipulator[]:manipulators
    ?<
        \func{+}{SubWindow}[Element]{Creates new SubWindow and associates Element with it.}
    >[
        \func{+}{setElement}[Element]{Associates Element with the SubWindow.}
    ]{
        Panel that displays an element for the user to interact with depending on the set ElementManipulator.
    }

    \struct{+}{i}{ModelUser}[
        \func{+}[void]{update}{Called whenever the Model is updated.}
    ]{
        Components that use data of the Model package implement this to be able to get notified when the Model changes.
    }

    \struct{+}{a}{ElementManipulator:Panel}[
        \func{+}{setElement}[Element]{Associates Element with the manipulator.}
    ]{
        Implementing classes display elements and let the User interact with them depending on their implementation.
    }

    \struct{+}{c}{TextEditor:ElementManipulator}?
        TextPane
    ?<
        \func{+}{TextEditor}[Element]{Creates new TextEditor and asscoiates Element with it.}
    >{
        Text editor with which the User can directly edit the associated Element.
    }

    \struct{+}{c}{Visualizer:ElementManipulator}?
        TextArea:info ;
        MenuBar:diagramOptions ;
        MenuBar:statisticsoptions
    ?<
        \func{+}{Visualizer}[Element]{Associates Element with the Visualizer.}
    >{
        Visualizer to visualize the associated Element.
    }

    \struct{+}{c}{Comparer:ElementManipulator}<
        \func{+}{Comparer}[Element:e1 | Element:e2]{Creates new Comparer to compare specified elements.}
    >{
        Panel to show a comparison between two elements in.
    }

    \struct{+}{c}{MergeDialog:Window}{
        Dialog popup that is shown whenever merge conflicts arise. It lets the user resolve said conflicts in multiple ways.
    }
   
    \struct{+}{c}{SettingsDialog:Window}{DIalog popup that is used to provide the user a way to change settings that affect the entire application.}
\end{package}