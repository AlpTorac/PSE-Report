\documentclass[10pt,a4paper]{report}

\usepackage[T1]{fontenc}
\usepackage{titlesec, blindtext, color}
\newcommand{\hsp}{\hspace{10pt}}

\usepackage[utf8]{inputenc}
\usepackage{amsmath}
\usepackage{amsfonts}
\usepackage{amssymb}
\usepackage{enumitem}

\usepackage{caption}
\usepackage{float}
\usepackage{graphicx}
\usepackage{hyperref}
\usepackage{xcolor}

\usepackage{tocloft}
\usepackage{etoc}
\usepackage[at]{easylist}

\usepackage{etoolbox}
\usepackage{url}

\title{\includegraphics[scale=6.0]{App Icon}\\Design for Graphical Editor for Liberty Files}

\author{Alp Toraç Genç, Kerem Kara, Xhulio Pernoca, Manuel Schenk, Ege Uzhan \leavevmode \\---------------------------------------------------------------------------------------\\
Submitted to: Om Prakash, Georgios Zervakis, Faeze Faghih}
\date{\today}

% Format for the chapters/sections/subsections
\titleformat{\chapter}[hang]{\Huge\bfseries}{\thechapter\hsp{}}{0pt}{\Huge\bfseries}
\titlespacing{\chapter}{0cm}{0cm}{0.5cm} % distance from {right}{left}{next line} for chapter
\titlespacing{\subsection}{0cm}{0cm}{0.5cm} % distance from {right}{left}{next line} for subsection

%Macros
% Macro for referencing another section in the document
% #1 Name of referenced section
% #2 Text of link
\definecolor{col:reference}{HTML}{2f5399}
\newcommand{\refer}[2]{\hyperref[#1]{\textcolor{col:reference}{#2}}}

%Macro for Highlighting
% #1 Text to highlight
\definecolor{col:highlight}{HTML}{63b53e}
\newcommand{\h}[1]{\textcolor{col:highlight}{#1}}

% Aligns captions to the left of the images
\captionsetup{
  font=footnotesize,
  justification=raggedright,
  singlelinecheck=false
}

% Macro for including images
\newcommand{\includeimage}[5]{
    \begin{figure}[H]
        #1
        \includegraphics[scale=#2]{#3.png}
        \caption{#4}
        \label{fig:#5}
    \end{figure}
}

% Macro for adding graph images
\newcommand{\includegraph}[7]{
    \graphicspath{{#1}}
    \includeimage{#2}{#3}{#4}{#5}{#6}
    \graphicspath{{#7}}
}

\newcommand{\imagepath}{Images/}
\newcommand{\graphpath}{Images/Graphs/}

%Macros for keywords
\definecolor{col:public}{HTML}{499632}
\definecolor{col:private}{HTML}{bf5050}
\definecolor{col:protected}{HTML}{808000}
\definecolor{col:class}{HTML}{9a50bf}
\definecolor{col:interface}{HTML}{9a50bf}
\definecolor{col:sub}{HTML}{5052bf}
\definecolor{col:impl}{HTML}{5052bf}
\definecolor{col:throws}{HTML}{5052bf}
\definecolor{col:static}{HTML}{bfb650}
\definecolor{col:default}{HTML}{bfb650}
\definecolor{col:abstract}{HTML}{bfb650}
\definecolor{col:final}{HTML}{bfb650}
\definecolor{col:generic}{HTML}{9a50bf}

\newcommand{\public}{\textcolor{col:public}{public }}
\newcommand{\private}{\textcolor{col:private}{private }}
\newcommand{\proted}{\textcolor{col:protected}{protected }}

\newcommand{\class}{\textcolor{col:class}{class }}
\newcommand{\interface}{\textcolor{col:interface}{interface }}
\newcommand{\type}[1]{\textcolor{col:class}{#1}}
\newcommand{\generic}[1]{\textcolor{col:generic}{#1}}
\newcommand{\typewgeneric}[2]{\textcolor{col:class}{#1}<\generic{#2}>}

\newcommand{\extends}{\textcolor{col:sub}{extends }}
\newcommand{\implements}{\textcolor{col:impl}{implements }}
\newcommand{\throws}{\textcolor{col:throws}{throws }}

\newcommand{\static}{\textcolor{col:static}{static }}
\newcommand{\deflt}{\textcolor{col:default}{default }}
\newcommand{\abstr}{\textcolor{col:abstract}{abstract }}
\newcommand{\final}{\textcolor{col:final}{final }}

\newcommand{\packagebeginning}{edu.kit.informatik.pse.gelf} %root package name

%Macro for label name
\newcommand{\lblroot}{lbl} % root cascading label name (\casclabel)
\newcommand{\lblpackage}{} % package name (without \packagebeginning)
\newcommand{\lblpackageelement}{} % element inside the package (class/interface)
\newcommand{\lblpackageelementmember}{} % member of the element inside the package (method/constructor/attribute)
\newcommand{\lblpackageelementmemberparameter}{} % parameter of the member of the element inside the package (parameter)
\newcommand{\casclabelname}{\lblroot\lblpackage\lblpackageelement\lblpackageelementmember\lblpackageelementmemberparameter}
\newcommand{\casclabel}{\label{\casclabelname}}

%Macro for placing labels
%Places a label and resets the appropriate place in the hierarchy in the end
%#1 one of these as representative of the hierarchy level: \lblpackage \lblpackageelement \lblpackageelementmember \lblpackageelementmemberparameter
%#2 name to be added to \casclabel
\newcommand{\putcasclabel}[3]{
    \renewcommand{#1}{:#2}
    \casclabel
    #3
    \renewcommand{#1}{}
}

%Macro for describing functions that have a boolean value
%Tip: if #2 is self explanatory after #1, just write "not"
%#1 true description (dot will be added in the end automatically)
%#2 false description (dot will be added in the end automatically)
\newcommand{\booldesc}[2]{
    \begin{itemize}
        \item True, if #1.
        \item False, if #2.
    \end{itemize}
}

%Macro for adding notes to descriptions
%#1 note to be added to the description
\newcommand{\descnote}[1]{
\leavevmode \\
\leavevmode \\
    Note: #1
}

%Macro for temporarily overriding \casclabelname
%used for making labels of duplicate method names
%or for same method and attribute names
%#1 temporal name
%#2 area, in which the overriding takes place
\newcommand{\overridecasclabelname}[2]{
    \renewcommand{\casclabelname}{#1}
    #2
    \renewcommand{\casclabelname}{\lblroot\lblpackage\lblpackageelement\lblpackageelementmember\lblpackageelementmemberparameter}
}

%Macro for class/interface declarations
%#1 Declaration of the class/interface
%#2 Description
\newcommand{\decl}[2]{
#1
\leavevmode \\
\leavevmode \\
#2
}

%Macro for Method description
%#1 visibility modifier
%#2 optional keyword
%#3 output type
%#4 method name
%#5 parameters
%#6 description
%#7 list of parameter descriptions
%#8 output description
\newcommand{\methoddesc}[8]{
\putcasclabel{\lblpackageelementmember}{#4}{
    #4
    \leavevmode \\
    \leavevmode \\
    #1#2\type{#3} #4(#5)
    \ifblank{#6#7#8}{}{
        \begin{itemize}[label=-]
            \ifblank{#6}{}{
                \item Description: \leavevmode \\ #6 \leavevmode \\
            }
            \ifblank{#7}{}{
                \item Parameters: \leavevmode \\ #7 \leavevmode \\
            }
            \ifblank{#8}{}{
                \item Returns: \leavevmode \\ #8
            }
        \end{itemize}
    }
}
}

%Macro for overridden Method description
%#1 visibility modifier
%#2 optional keyword
%#3 output type
%#4 method name
%#5 parameters
%#6 origin class of the overridden method
%#7 description
%#8 list of parameter descriptions
%#9 output description
\newcommand{\ovrdnmethoddesc}[9]{
\putcasclabel{\lblpackageelementmember}{#4}{
    #4
    \leavevmode \\
    \leavevmode \\
    #1#2\type{#3} #4(#5) overrides the version of the method from #6
    \ifblank{#7#8#9}{}{
        \begin{itemize}[label=-]
            \ifblank{#7}{}{
                \item Description: \leavevmode \\ #7 \leavevmode \\
            }
            \ifblank{#8}{}{
                \item Parameters: \leavevmode \\ #8 \leavevmode \\
            }
            \ifblank{#9}{}{
                \item Returns: \leavevmode \\ #9
            }
        \end{itemize}
    }
}
}

%Macro for implemented Method description
%#1 visibility modifier
%#2 optional keyword
%#3 output type
%#4 method name
%#5 parameters
%#6 origin interface of the implemented method
%#7 description
%#8 list of parameter descriptions
%#9 output description
\newcommand{\implmethoddesc}[9]{
\putcasclabel{\lblpackageelementmember}{#4}{
    #4
    \leavevmode \\
    \leavevmode \\
    #1#2\type{#3} #4(#5) implements the method from #6
    \ifblank{#7#8#9}{}{
        \begin{itemize}[label=-]
            \ifblank{#7}{}{
                \item Description: \leavevmode \\ #7 \leavevmode \\
            }
            \ifblank{#8}{}{
                \item Parameters: \leavevmode \\ #8 \leavevmode \\
            }
            \ifblank{#9}{}{
                \item Returns: \leavevmode \\ #9
            }
        \end{itemize}
    }
}
}

%Macro for parameter description
%#1 parameter name
%#2 parameter description
\newcommand{\paramdesc}[2]{
\putcasclabel{\lblpackageelementmemberparameter}{#1}{
    \textbf{#1}: #2
}
}

%Macro for subclass declaration
%#1 subclass name
\newcommand{\subclsdec}[1]{
    #1
}

%Macro for implementing class declaration
%#1 implementing class name
\newcommand{\implclsdec}[1]{
    #1
}

%Macro for subinterface declaration
%#1 subinterface name
\newcommand{\subintdec}[1]{
    #1
}

%Macro for attribute descriptions
%#1 visibility
%#2 optional keywords
%#3 attribute type
%#4 attribute name
%#5 attribute description
\newcommand{\atrdesc}[5]{
\putcasclabel{\lblpackageelementmember}{#4}{
    #1#2\type{#3} #4
    \ifblank{#5}{}{
        : #5
    }
}
}

%Macro for Enum fields
%#1 Enum name
%#2 Field name (all UPPER case and words separated with -)
%#3 Field description
\newcommand{\fielddesc}[3]{
\putcasclabel{\lblpackageelementmember}{#2}{
    \public \static \final \type{#1} #2  \leavevmode \\
    \begin{itemize}[label=-]
        \item #3
    \end{itemize}
}
}

%Macro for constructor description
%#1 visibility modifier
%#2 constructor name
%#3 parameters
%#4 description
%#5 parameter description
\newcommand{\constrdesc}[5]{
\putcasclabel{\lblpackageelementmember}{#2}{
    #2
    \leavevmode \\
    \leavevmode \\
    #1#2(#3) \leavevmode \\
    \ifblank{#4#5}{}{
        \begin{itemize}[label=-]
            \ifblank{#4}{}{
                \item \textbf{Description:} \leavevmode \\ #4
            }
            \ifblank{#5}{}{
                \item \textbf{Parameters:} \leavevmode \\ #5
            }
        \end{itemize}
    }
}
}

%Macros for headers of package/class/interface/abstract class descriptions
%#1 place in hierarchy
%#2 type
%#3 name
\newcommand{\desc}[3]{
    #1{#2 #3}
}

%Macro for class descriptions
%#1 name
%#2 Declaration
%#3 Attributes
%#4 Constructors
%#5 Methods
\newcommand{\classdesc}[5]{
\putcasclabel{\lblpackageelement}{#1}{
    \desc{\subsection}{Class}{#1} \leavevmode \\
    \ifblank{#2}{}{
            \textbf{Declaration:} \leavevmode \\
            #2 \leavevmode \\
    }
    \ifblank{#3}{}{
            \textbf{Attributes:} \leavevmode \\
            #3 \leavevmode \\
    }
    \ifblank{#4}{}{
            \textbf{Constructors:} \leavevmode \\
            #4 \leavevmode \\
    }
    \ifblank{#5}{}{
            \textbf{Methods:} \leavevmode \\
            #5
    }
}
}

%Macro for abstract class descriptions
%#1 Name
%#2 Declaration
%#3 Attributes
%#4 All known subclasses
%#5 Constructors
%#6 Methods
\newcommand{\absclassdesc}[6]{
\putcasclabel{\lblpackageelement}{#1}{
    \desc{\subsection}{Abstract Class}{#1} \leavevmode \\
    \ifblank{#2}{}{
            \textbf{Declaration:} \leavevmode \\
            #2 \leavevmode \\
    }
    \ifblank{#3}{}{
            \textbf{Attributes:} \leavevmode \\
            #3 \leavevmode \\
    }
    \ifblank{#4}{}{
            \textbf{All known subclasses:} \leavevmode \\
            #4 \leavevmode \\
    }
    \ifblank{#5}{}{
            \textbf{Constructors:} \leavevmode \\
            #5 \leavevmode \\
    }
    \ifblank{#6}{}{
            \textbf{Methods:} \leavevmode \\
            #6
    }
}
}

%Macro for interface descriptions
%#1 name
%#2 Declaration
%#3 All known subinterfaces
%#4 All known implementing classes
%#5 Methods
\newcommand{\interfacedesc}[5]{
\putcasclabel{\lblpackageelement}{#1}{
    \desc{\subsection}{Interface}{#1} \leavevmode \\
    \ifblank{#2}{}{
            \textbf{Declaration:} \leavevmode \\
            #2 \leavevmode \\
    }
    \ifblank{#3}{}{
            \textbf{All known subinterfaces:} \leavevmode \\
            #3 \leavevmode \\
    }
    \ifblank{#4}{}{
            \textbf{All known implementing classes:} \leavevmode \\
            #4 \leavevmode \\
    }
    \ifblank{#5}{}{
            \textbf{Methods:} \leavevmode \\
            #5
    }
}
}

%Macro for enum descriptions
%#1 name
%#2 Declaration
%#3 Fields (Enum values)
%#4 Methods
\newcommand{\enumdesc}[4]{
\putcasclabel{\lblpackageelement}{#1}{
    \desc{\subsection}{Enum}{#1} \leavevmode \\
    \ifblank{#2}{}{
            \textbf{Declaration:} \leavevmode \\
            #2 \leavevmode \\
    }
    \ifblank{#3}{}{
            \textbf{Fields:} \leavevmode \\
            #3 \leavevmode \\
    }
    \ifblank{#4}{}{
            \textbf{Methods:} \leavevmode \\
            #4 \leavevmode \\
    }
}
}

%Macro for package descriptions
%#1 name
%#2 package content
\newcommand{\packagedesc}[2]{
\putcasclabel{\lblpackage}{#1}{
    \desc{\section}{Package}{\packagebeginning.#1}
    \renewcommand{\contentsname}{\small\textit{Package contents}
    \hfill
    \small\textit{Page}}
    \setlength{\cftbeforetoctitleskip}{0em} % removes space before (local) table of contents
    \setlength{\cftaftertoctitleskip}{0em} % removes space between the heading and the list parts of (local) table of contents
    \setlength{\cftsubsecindent}{1em} % shortens the spacing from left
    \localtableofcontents % makes a local table of contents
    #2
}
}

%Macros for design patterns
%#1 place in hierarchy
%#2 name
\newcommand{\patternentry}[2]{
    #1{#2}
}
\newcommand{\pattern}[1]{
    \patternentry{\section}{#1}
}

%Macro for defining classes
%#1 declaration
%#2 attributes
%#3 constructors
%#4 methods
\newcommand{\defineclass}[4]{
\textbf{Declaration:}
\\\indent #1
\\\textbf{Attributes:}
\\\indent#2
\\\textbf{Constructors:}
\\\indent#3
\\\textbf{Methods:}
\\\indent#4
}

%Macro for defining methods
%#1 signature
%#2 return type
%#3 definition
\newcommand{\method}{

}

\graphicspath{\imagepath}

\begin{document}
\maketitle
\label{sec:title}
\tableofcontents

\chapter{Packages}
    \section{Model}
    \section{View}
    \section{Controller}

\chapter{Class Descriptions}
%GUI
\packagedesc{view.gui}{
\classdesc{MainWindow}{a}{a}
    {
        \constrdesc{\public}{MainWindow}{a}{a}{a}
    }
    {a}
\classdesc{SubWindowArea}{a}{a}
    {
        \constrdesc{\public}{SubWindowArea}{a}{a}{a}
    }
    {a}
\classdesc{Outliner}{a}{a}
    {
        \constrdesc{\public}{Outliner}{a}{a}{a}
    }
    {a}
\classdesc{InfoBar}{a}{a}
    {
        \constrdesc{\public}{InfoBar}{a}{a}{a}
    }
    {a}
\enumdesc{InfoBarID}{a}{
    \begin{itemize}
        \item \fielddesc{InfoBarID}{VERSION}{a}
        \item \fielddesc{InfoBarID}{SELECTED}{a}
        \item \fielddesc{InfoBarID}{LASTACTION}{a}
    \end{itemize}
}{a}
\classdesc{SubWindow}{a}{a}
    {
        \constrdesc{\public}{SubWindow}{a}{a}{a}
    }
    {a}
\absclassdesc{ElementManipulator}{a}{a}
    {
        \begin{itemize}
            \item \subclsdec{TextEditor}
            \item \subclsdec{Visualizer}
        \end{itemize}
    }
    {a}
    {a}
\classdesc{MergeDialog}{a}{a}
    {a}
    {a}
}
%Components
\packagedesc{view.components}{
\interfacedesc{AutoResizing}{a}
    {a}
    {a}
    {a}
\classdesc{Resizer}{a}{a}
    {a}
    {a}
\enumdesc{ResizeMode}{a}{
    \begin{itemize}
        \item \fielddesc{ResizeMode}{ABSOLUTE-TOP-LEFT}{a}
        \item \fielddesc{ResizeMode}{ABSOLUTE-BOTTOM-RIGHT}{a}
        \item \fielddesc{ResizeMode}{RELATIVE}{a}
    \end{itemize}
}{a}
\classdesc{Window}{a}{a}
    {a}
    {a}
\classdesc{Panel}{a}{a}
    {a}
    {a}
}
%Diagrams
\packagedesc{view.diagrams}{
\interfacedesc{IDiagram}{
    \decl{\public \interface IDiagram}{An interface that is implemented by all diagrams.}
}{}{
    \begin{itemize}
        \item \implclsdec{Diagram}
    \end{itemize}
}{
    \begin{itemize}
        \item \methoddesc{\public}{\typewgeneric{Collection}{?}}{}{cloneData}{}
        {
            Makes a deep copy of the \refer{\lblroot:view.diagrams.type:Diagram:data}{data}.
        }{}{
            A deep copy of the \refer{\lblroot:view.diagrams.type:Diagram:data}{data} of the diagram.
        }
        \item \methoddesc{\public}{}{void}{refresh}{}
        {
            Re-draws the diagram.
        }{}{}
        \item \methoddesc{\public}{}{void}{update}{\type{DiagramData} data}
        {
            Replaces the \refer{\lblroot:view.diagrams.type:Diagram:data}{data} by the given.
        }{
            \begin{itemize}
                \item \paramdesc{data}{The data to replace the current \refer{\lblroot:view.diagrams.type:Diagram:data}{data}.}
            \end{itemize}
        }{}
        \item \methoddesc{\public}{}{boolean}{addDiagramViewHelper}{\type{DiagramViewHelper} dvh}
        {
            Adds the given \refer{\lblroot:view.diagrams.indicator:DiagramViewHelper}{DiagramViewHelper}.
        }{
            \begin{itemize}
                \item \paramdesc{dvh}{The \refer{\lblroot:view.diagrams.indicator:DiagramViewHelper}{DiagramViewHelper} instance to be added.}
            \end{itemize}
        }{}
        \item \methoddesc{\public}{}{boolean}{removeDiagramViewHelper}{\type{IndicatorIdentifier} id}
        {}{
            \begin{itemize}
                \item \paramdesc{id}{The unique identifier of the \refer{\lblroot:view.diagrams.indicator:DiagramViewHelper}{DiagramViewHelper} to be removed.}
            \end{itemize}
        }{}
        \item \methoddesc{\public}{}{boolean}{showDiagramViewHelper}{\type{IndicatorIdentifier} id}
        {}{
            \begin{itemize}
                \item \paramdesc{id}{The unique identifier of the \refer{\lblroot:view.diagrams.indicator:DiagramViewHelper}{DiagramViewHelper} to be shown.}
            \end{itemize}
        }{}
        \item \methoddesc{\public}{}{boolean}{hideDiagramViewHelper}{\type{IndicatorIdentifier} id}
        {}{
            \begin{itemize}
                \item \paramdesc{id}{The unique identifier of the \refer{\lblroot:view.diagrams.indicator:DiagramViewHelper}{DiagramViewHelper} to be hidden.}
            \end{itemize}
        }{}
        \item \methoddesc{\public}{}{DiagramComponent[]}{getNonValueDisplayDiagramComponentPrototypes}{}
        {}{}
        {Deep copies of \refer{\lblroot:view.diagrams.components:DiagramComponent}{DiagramComponents} that do not represent any value.}
        \item \methoddesc{\public}{}{DiagramValueDisplayComponent[]}{getDiagramValueDisplayComponentPrototypes}{}
        {}{}
        {Deep copies of \refer{\lblroot:view.diagrams.components:DiagramValueDisplayComponent}{DiagramValueDisplayComponents}}
    \end{itemize}
}

\interfacedesc{IDiagramOverlayer}{
    \decl{\public \interface IDiagramOverlayer}{An interface implemented by \refer{\lblroot:view.diagrams.overlayer:DiagramOverlayer}{DiagramOverlayer} \descnote{The amount of \refer{\lblroot:view.diagrams}{IDiagrams} that can be overlaid at once can vary between different implementing classes of \refer{\lblroot:view.diagrams:IDiagramOverlayer}{IDiagramOverlayer}.}}
}{}{
    \begin{itemize}
        \item \implclsdec{DiagramOverlayer}
    \end{itemize}
}{
    \begin{itemize}
        \item \methoddesc{\public}{}{IDiagram}{getDiagram}{\type{int} index}
        {}{
            \begin{itemize}
                \item \paramdesc{index}{The index of the wanted \refer{\lblroot:view.diagrams}{IDiagram} in the \refer{\lblroot:view.diagrams.overlayer:DiagramOverlayer:diagrams}{diagrams}}
            \end{itemize}
        }{
            The wanted \refer{\lblroot:view.diagrams}{\type{IDiagram}} from \refer{\lblroot:view.diagrams.overlayer:DiagramOverlayer:diagrams}{diagrams}.
        }
        \item \methoddesc{\public}{}{void}{setDiagram}{\type{int} index, IDiagram diagram}
        {}{
            \begin{itemize}
                \item \paramdesc{index}{The index of the \refer{\lblroot:view.diagrams}{IDiagram} in the \refer{\lblroot:view.diagrams.overlayer:DiagramOverlayer:diagrams}{diagrams} to be set}
                \item \paramdesc{diagram}{The new \refer{\lblroot:view.diagrams}{IDiagram}}
            \end{itemize}
        }{}
        \item \methoddesc{\public}{}{boolean}{addDiagram}{\type{IDiagram} diagram}
        {}{
            \begin{itemize}
                \item \paramdesc{diagram}{The \refer{\lblroot:view.diagrams}{IDiagram} to be added.}
            \end{itemize}
        }{
            \booldesc
            {the \refer{\lblroot:view.diagrams}{IDiagram} is added successfully}
            {not}
        }
        \item \methoddesc{\public}{}{boolean}{removeDiagram}{\type{int} index}
        {}{
            \begin{itemize}
                \item \paramdesc{index}{The index of the \refer{\lblroot:view.diagrams}{IDiagram} to be removed.}
            \end{itemize}
        }{
            \booldesc
            {the \refer{\lblroot:view.diagrams}{IDiagram} is removed successfully}
            {not}
        }
        \item \overridecasclabelname{\lblroot\lblpackage\lblpackageelement:(inds)overlay}{\methoddesc{\public}{}{\type{IDiagram}}{overlay}{\type{int[]} indices}
        {
            Overlays the \refer{\lblroot:view.diagrams}{IDiagrams} specified by the indices in \refer{\lblroot:view.diagrams.overlayer:DiagramOverlayer:diagrams}{diagrams}
            \descnote{The amount of \refer{\lblroot:view.diagrams}{IDiagrams} that can be overlaid at once can vary between different implementing classes of \refer{\lblroot:view.diagrams:IDiagramOverlayer}{IDiagramOverlayer}.}
        }{
            \begin{itemize}
                \item \overridecasclabelname{\lblroot\lblpackage\lblpackageelement:(inds)overlay:indices}{\paramdesc{indices}{The indices of the \refer{\lblroot:view.diagrams}{IDiagrams}} in \refer{\lblroot:view.diagrams.overlayer:DiagramOverlayer:diagrams}{diagrams} to be overlaid.}
            \end{itemize}
        }{
            The result of overlaying the \refer{\lblroot:view.diagrams}{IDiagrams} on top of each other.
        }}
        \item \overridecasclabelname{\lblroot\lblpackage\lblpackageelement:(dgrms)overlay}{\methoddesc{\public}{}{IDiagram}{overlay}{\type{IDiagram[]} diagrams}
        {
            Overlays the given \refer{\lblroot:view.diagrams}{IDiagrams} on top of each other and replaces \refer{\lblroot:view.diagrams.overlayer:DiagramOverlayer:diagrams}{diagrams} \descnote{The amount of \refer{\lblroot:view.diagrams}{IDiagrams} that can be overlaid at once can vary between different implementing classes of \refer{\lblroot:view.diagrams:IDiagramOverlayer}{IDiagramOverlayer}.}
        }{
            \begin{itemize}
                \item \overridecasclabelname{\lblroot\lblpackage\lblpackageelement:(dgrms)overlay:diagrams}{\paramdesc{diagrams}{The \refer{\lblroot:view.diagrams}{IDiagrams} to be overlaid}}
            \end{itemize}
        }{
            The result of overlaying the \refer{\lblroot:view.diagrams}{IDiagrams} on top of each other.
        }}
    \end{itemize}
}

\classdesc{DiagramDirector}{
    \decl{\public \class DiagramDirector}{The class, which is responsible for initiating the building of a \refer{\lblroot:view.diagrams}{IDiagrams} and returning the result}
}{
    \begin{itemize}
        \item \atrdesc{\private}{}{DiagramBuilder}{builder}{The \refer{\lblroot:view.diagrams.builder:DiagramBuilder}{DiagramBuilder} of the \refer{\lblroot:view.diagrams}{IDiagram} to build.}
        \item \atrdesc{\private}{}{DiagramData}{data}{The \refer{\lblroot:view.diagrams.data:DiagramData}{DiagramData} of the \refer{\lblroot:view.diagrams}{IDiagram} to build.}
        \item \atrdesc{\private}{\static}{DiagramDirector}{instance}{The only instance of the class.}
    \end{itemize}
}{
    \begin{itemize}
        \item \constrdesc{\private}{DiagramDirector}{}{}{}
    \end{itemize}
}{
    \begin{itemize}
        \item \methoddesc{\public}{}{DiagramDirector}{getDiagramDirector}{}
        {}{}{
            The only instance of the class.
        }
        \item \methoddesc{\public}{}{void}{changeBuilder}{\type{DiagramBuilder} builder}
        {
            Changes the active \refer{\lblroot:view.diagrams:DiagramDirector:builder}{builder} with the given one.
        }{
            \begin{itemize}
                \item \paramdesc{builder}{The new \refer{\lblroot:view.diagrams.builder:DiagramBuilder}{DiagramBuilder}}
            \end{itemize}
        }{}
        \item \methoddesc{\public}{}{void}{setDiagramData}{\type{DiagramData} data}
        {
            Sets the active \refer{\lblroot:view.diagrams:DiagramDirector:data}{data}.
        }{
            \begin{itemize}
                \item \paramdesc{data}{The given \refer{\lblroot:view.diagrams.data:DiagramData}{data}, which will be used to build the \refer{\lblroot:view.diagrams:IDiagram}{IDiagram}.}
            \end{itemize}
        }{}
        \item \methoddesc{\public}{}{IDiagram}{build}{}
        {
            Starts the building process of the \refer{\lblroot:view.diagrams:IDiagram}{IDiagram}.
        }{}{
            The built \refer{\lblroot:view.diagrams:IDiagram}{IDiagram} defined by the attributes of this class.
        }
    \end{itemize}
}

\classdesc{SettingsProvider}{
    The class, which is responsible for containing and distributing the latest \refer{\lblroot:model.project:Settings}{Settings}.
}{
    \begin{itemize}
        \item \atrdesc{\private}{\static}{SettingsProvider}{instance}{The only instance of the class.}
        \item \atrdesc{\private}{}{Settings}{s}{The latest \refer{\lblroot:model.project:Settings}{Settings}.}
    \end{itemize}
}{
    \begin{itemize}
        \item \constrdesc{\private}{SettingsProvider}{}{}{}
    \end{itemize}  
}{
    \begin{itemize}
        \item \methoddesc{\public}{}{SettingsProvider}{getInstance}{}
        {}{}{
            The only instance of the class.
        }
        \item \methoddesc{\public}{}{void}{changeSettings}{Settings s}
        {
            Sets \refer{\lblroot:view.diagrams:SettingsProvider:s}{s} with the given \refer{\lblroot:model.project:Settings}{Settings}.
        }{
            \begin{itemize}
                \item \paramdesc{s}{The given (latest) \refer{\lblroot:model.project:Settings}{Settings}.}
            \end{itemize}
        }{}
        \item \methoddesc{\public}{}{Settings}{getSettings}{}
        {}{}{
            \refer{\lblroot:view.diagrams:SettingsProvider:s}{s}
        }        
    \end{itemize}
}
}
\packagedesc{view.diagrams.overlayer}{

\interfacedesc{IDiagramOverlayStrategy}{
    \decl{\public \interface IDiagramOverlayStrategy}{An interface implemented by overlay strategies used by the \refer{\lblroot:view.diagrams.overlayer:DiagramOverlayer}{DiagramOverlayer}}
}{}{
    \begin{itemize}
        \item \implclsdec{FunctionGraphOverlayStrategy}
        \item \implclsdec{HistogramOverlayStrategy}
        \item \implclsdec{BarChartOverlayStrategy}
    \end{itemize}
}{
    \begin{itemize}
        \item \methoddesc{\public}{}{IDiagram}{overlay}{}
        {
            Overlays the \refer{\lblroot:view.diagrams:IDiagram}{IDiagrams} stored.
        }{}{
            The result of overlaying the specified \refer{\lblroot:view.diagrams:IDiagram}{IDiagrams}
        }
    \end{itemize}
}

\classdesc{DiagramOverlayer}{
    \decl{\public \class DiagramOverlayer}{Stores \refer{\lblroot:view.diagrams:IDiagram}{IDiagrams}} and overlays the specified ones. The stored \refer{\lblroot:view.diagrams:IDiagram}{IDiagrams} must be of the same type.
}{
    \begin{itemize}
        \item \atrdesc{\private}{\type{Collection}<\generic{?} \extends \type{IDiagram}>}{}{diagrams}{The stored \refer{\lblroot:view.diagrams:IDiagram}{IDiagrams}}
        \item \atrdesc{\private}{}{IDiagramOverlayStrategy}{overlayStrategy}{The active \refer{\lblroot:view.diagrams.overlayer:IDiagramOverlayStrategy}{IDiagramOverlayStrategy}}
    \end{itemize}
}{
    \begin{itemize}
        \item \constrdesc{\public}{DiagramOverlayer}{\type{IDiagram[]} diagrams}{
            Initializes an instance with the given \refer{\lblroot:view.diagrams:IDiagram}{IDiagrams}.
        }{
            \begin{itemize}
                \item \paramdesc{diagrams}{The given \refer{\lblroot:view.diagrams:IDiagram}{IDiagrams}}
            \end{itemize}
        }
    \end{itemize}
}{
    \begin{itemize}
        \item \methoddesc{\private}{}{void}{setOverlayStrategy}{}
        {
            Sets \refer{\lblroot:view.diagrams.overlayer:DiagramOverlayer:overlayStrategy}{overlayStrategy} based on the type of \refer{\lblroot:view.diagrams:IDiagram}{IDiagrams} stored in \refer{\lblroot:view.diagrams.overlayer:DiagramOverlayer:diagrams}{diagrams}.
        }{}{}
    \end{itemize}
}

\classdesc{FunctionGraphOverlayStrategy}{
    \decl{\public \class FunctionGraphOverlayStrategy}{The implementation of \refer{\lblroot:view.diagrams.overlayer:IDiagramOverlayStrategy}{IDiagramOverlayStrategy} for \refer{\lblroot:view.diagrams.type:FunctionGraph}{FunctionGraphs}.
    \descnote{Currently arbitrary amount of \refer{\lblroot:view.diagrams.type:FunctionGraph}{FunctionGraphs} can be overlaid.}}
}{
    \begin{itemize}
        \item \atrdesc{\private}{}{FunctionGraph[]}{functionGraphs}{The \refer{\lblroot:view.diagrams.type:FunctionGraph}{FunctionGraphs} to be overlaid.}
    \end{itemize}
}{
    \begin{itemize}
        \item \constrdesc{\public}{FunctionGraphOverlayStrategy}{\type{FunctionGraph[]} functionGraphs}{}{
            \begin{itemize}
                \item \paramdesc{functionGraphs}{The \refer{\lblroot:view.diagrams.type:FunctionGraph}{FunctionGraphs} to be overlaid.}
            \end{itemize}
        }
    \end{itemize}
}{}

\classdesc{HistogramOverlayStrategy}{
    \decl{\public \class HistogramOverlayStrategy}{The implementation of \refer{\lblroot:view.diagrams.overlayer:IDiagramOverlayStrategy}{IDiagramOverlayStrategy} for \refer{\lblroot:view.diagrams.type:Histogram}{Histograms}.
    \descnote{Currently only 2 given \refer{\lblroot:view.diagrams.type:Histogram}{Histograms} can be overlaid.}}
}{
    \begin{itemize}
        \item \atrdesc{\private}{}{Histogram}{histogram1}{A given \refer{\lblroot:view.diagrams.type:Histogram}{Histogram}}
        \item \atrdesc{\private}{}{Histogram}{histogram2}{Another given \refer{\lblroot:view.diagrams.type:Histogram}{Histogram}}
    \end{itemize}
}{
    \begin{itemize}
        \item \constrdesc{\public}{HistogramOverlayStrategy}{\type{Histogram} histogram1, \type{Histogram} histogram2}{}{
            \begin{itemize}
                \item \paramdesc{histogram1}{A given \refer{\lblroot:view.diagrams.type:Histogram}{Histogram}}
                \item \paramdesc{histogram2}{Another given \refer{\lblroot:view.diagrams.type:Histogram}{Histogram}}
            \end{itemize}
        }
    \end{itemize}
}{}

\classdesc{BarChartOverlayStrategy}{
    \decl{\public \class BarChartOverlayStrategy}{The implementation of \refer{\lblroot:view.diagrams.overlayer:IDiagramOverlayStrategy}{IDiagramOverlayStrategy} for \refer{\lblroot:view.diagrams.type:BarChart}{BarCharts}.
    \descnote{Currently only 2 given \refer{\lblroot:view.diagrams.type:BarChart}{BarCharts} can be overlaid.}}
}{
    \begin{itemize}
        \item \atrdesc{\private}{}{BarChart}{barChart1}{A given \refer{\lblroot:view.diagrams.type:BarChart}{BarChart}}
        \item \atrdesc{\private}{}{BarChart}{barChart2}{Another given \refer{\lblroot:view.diagrams.type:BarChart}{BarChart}}
    \end{itemize}
}{
    \begin{itemize}
        \item \constrdesc{\public}{BarChartOverlayStrategy}{\type{BarChart} barChart1, \type{BarChart} barChart2}{}{
            \begin{itemize}
                \item \paramdesc{barChart1}{A given \refer{\lblroot:view.diagrams.type:BarChart}{BarChart}}
                \item \paramdesc{barChart2}{Another given \refer{\lblroot:view.diagrams.type:BarChart}{BarChart}}
            \end{itemize}
        }
    \end{itemize}
}{}
}
\packagedesc{view.diagrams.components}{

\interfacedesc{Hoverable}{
    \decl{\public \interface Hoverable}{An interface implemented by \refer{\lblroot:view.diagrams.components:DiagramValueDisplayComponent}{DiagramValueDisplayComponents}.
    \leavevmode \\
    \leavevmode \\
    This interface is responsible for displaying the \refer{\lblroot:view.diagrams.components:HoverLabel}{HoverLabel}. The methods responsible for displaying the \refer{\lblroot:view.diagrams.components:HoverLabel}{HoverLabel} are implemented inside this interface by default.}
}{}{
    \begin{itemize}
        \item \implclsdec{DiagramValueDisplayComponent}
    \end{itemize}
}{
    \begin{itemize}
        \item \methoddesc{\public}{\deflt}{boolean}{isBeingHovered}{}
        {}{}{
            \booldesc{the implementing class is being hovered with the mouse pointer}{not}
        }
        \item \methoddesc{\public}{\deflt}{void}{hoverAction}{}
        {
            Performs the action that will occur once the implementing class is being hovered with the mouse pointer.
        }{}{}
        \item \methoddesc{\public}{\deflt}{void}{refreshHoverLabelPosition}{}
        {
            Refreshes the position of the \refer{\lblroot:view.diagrams.components:HoverLabel}{HoverLabel} in relation to the mouse pointer.
        }{}{}
        \item \methoddesc{\public}{\deflt}{void}{showHoverLabel}{}
        {
            Displays the \refer{\lblroot:view.diagrams.components:HoverLabel}{HoverLabel}.
        }{}{}
        \item \methoddesc{\public}{\deflt}{void}{hideHoverLabel}{}
        {
            Hides the \refer{\lblroot:view.diagrams.components:HoverLabel}{HoverLabel}.
        }{}{}
    \end{itemize}
}

\absclassdesc{PositionInDiagram}{
    \decl{\public \abstr \class PositionInDiagram}{The abstract class inherited by the classes that represent a position inside an \refer{\lblroot:view.diagrams:IDiagram}{IDiagram}.}
}{
    \begin{itemize}
        \item \atrdesc{\private}{}{DiagramAxis[]}{axes}{The axes, according to which the position inside the \refer{\lblroot:view.diagrams:IDiagram}{IDiagram} will be calculated.}
        \item \atrdesc{\private}{}{Number[]}{positionsInAxes}{The coordinates on the \refer{\lblroot:view.diagrams.components:DiagramAxis}{DiagramAxes} given in axes attribute.}
    \end{itemize}
}{
    \subclsdec{PositionIn2DDiagram}
}{    
    \begin{itemize}
        \item \constrdesc{\public}{PositionInDiagram}{\type{DiagramAxis[]} axes, \type{Number[]} coordinatesInAxes}{}{
            \begin{itemize}
                \item \paramdesc{axes}{The axes, according to which the position inside the \refer{\lblroot:view.diagrams:IDiagram}{IDiagram} will be calculated.}
                \item \paramdesc{coordinatesInAxes}{The coordinates on the \refer{\lblroot:view.diagrams.components:DiagramAxis}{DiagramAxes} given in axes attribute.}
            \end{itemize}
        }
    \end{itemize}}
{
    \begin{itemize}
        \item \methoddesc{\public}{}{Number}{axisCoordinateToFrameCoordinate}{\type{int} index}
        {
            Transforms the specified coordinate in \refer{\lblroot:view.diagrams.components:PositionInDiagram:positionsInAxes}{positionsInAxes} to the corresponding coordinate inside the frame the \refer{\lblroot:view.diagrams:IDiagram}{IDiagram} is on top of.
        }{
            \begin{itemize}
                \item \paramdesc{index}{The index of the coordinate to be transformed to a frame coordinate.}
            \end{itemize}
        }{
            The frame coordinate that corresponds to the specified coordinate on \refer{\lblroot:view.diagrams.components:PositionInDiagram:positionsInAxes}{positionsInAxes}.
        }
        \item \methoddesc{\public}{}{PositionInFrame}{toPositionInFrame}{}
        {}{}{
            The frame coordinates that correspond to the coordinates on \refer{\lblroot:view.diagrams.components:PositionInDiagram:positionsInAxes}{positionsInAxes}.
        }
        \item \methoddesc{\proted}{}{void}{setAxisCoordinate}{\type{int} index, \type{Number} position}
        {
            Sets the specified coordinate in \refer{\lblroot:view.diagrams.components:PositionInDiagram:positionsInAxes}{positionsInAxes} with the given one.
        }{
            \begin{itemize}
                \item \paramdesc{index}{The index of the wanted coordinate in \refer{\lblroot:view.diagrams.components:PositionInDiagram:positionsInAxes}{positionsInAxes}.}
                \item \paramdesc{position}{The new value of the coordinate at the specified index.}
            \end{itemize}
        }{}
        \item \methoddesc{\proted}{}{void}{setAxisCoordinates}{\type{Number[]} coordinates}
        {
            Sets \refer{\lblroot:view.diagrams.components:PositionInDiagram:positionsInAxes}{positionsInAxes} with the given new coordinates.
        }{
            \begin{itemize}
                \item \paramdesc{coordinates}{The coordinates to replace \refer{\lblroot:view.diagrams.components:PositionInDiagram:positionsInAxes}{positionsInAxes}}
            \end{itemize}
        }{}
        \item \methoddesc{\proted}{}{Number}{getAxisPos}{\type{int} index}
        {}{
            \begin{itemize}
                \item \paramdesc{index}{}
            \end{itemize}
        }{
            The specified coordinate in \refer{\lblroot:view.diagrams.components:PositionInDiagram:positionsInAxes}{positionsInAxes}.
        }
    \end{itemize}
}

\absclassdesc{DiagramComponent}{
    \decl{\public \abstr \class DiagramComponent}{The abstract class that is implemented by classes that represents parts of \refer{\lblroot:view.diagrams.type:Diagram}{Diagrams}.}
}{
    \begin{itemize}
        \item \atrdesc{\private}{}{Color}{color}{The color of the instance.}
    \end{itemize}
}{
    \begin{itemize}
        \item \constrdesc{\proted}{DiagramComponent}{\type{Color} color}{}{
            \begin{itemize}
                \item \paramdesc{color}{The color of the instance.}
            \end{itemize}
        }
    \end{itemize}
}{
    \begin{itemize}
        \item \subclsdec{DiagramValueDisplayComponent}
        \item \subclsdec{DiagramAxis}
        \item \subclsdec{DiagramLabel}
        \item \subclsdec{DiagramLine}
        \item \subclsdec{DiagramColorScale}
    \end{itemize}
}{
    \begin{itemize}
        \item \methoddesc{\public}{\abstr}{DiagramComponent}{clone}{}
        {}{}{
            A deep copy of the instance.
        }
        \item \methoddesc{\public}{}{void}{setColor}{\type{Color} color}
        {
            Sets the \refer{\lblroot:view.diagrams.components:DiagramComponent:color}{color} of the instance with the given one.
        }{
            \begin{itemize}
                \item \paramdesc{color}{The new color of the instance}
            \end{itemize}
        }{}
        \item \methoddesc{\public}{}{Color}{getColor}{}
        {}{}{
            The \refer{\lblroot:view.diagrams.components:DiagramComponent:color}{color} attribute.
        }
        \item \methoddesc{\public}{\abstr}{void}{show}{}
        {
            Shows the instance.
        }{}{}
        \item \methoddesc{\public}{\abstr}{void}{hide}{}
        {
            Hides the instance.
        }{}{}
    \end{itemize}
}

\absclassdesc{DiagramValueDisplayComponent}{
    \decl{\public \abstr \class DiagramValueDisplayComponent}{The abstract class that is implemented by classes that represents parts of \refer{\lblroot:view.diagrams.type:Diagram}{Diagrams}, which are responsible for displaying values inside \refer{\lblroot:view.diagrams.type:Diagram}{Diagrams}.}
}{
    \begin{itemize}
        \item \atrdesc{\private}{}{Number}{value}{The value, which will be represented by the instance on the \refer{\lblroot:view.diagrams.type:Diagram}{Diagrams}.}
    \end{itemize}
}{
    \begin{itemize}
        \item \subclsdec{DiagramBar}
        \item \subclsdec{DiagramValueLabel}
        \item \subclsdec{DiagramPoint}
    \end{itemize}
}{
    \begin{itemize}
        \item \constrdesc{\proted}{DiagramValueDisplayComponent}{\type{Color} color, \type{Number} value}{}{
            \begin{itemize}
                \item \paramdesc{color}{The color of the instance.}
                \item \paramdesc{value}{The value, which will be represented by the instance on the \refer{\lblroot:view.diagrams.type:Diagram}{Diagrams}.}
            \end{itemize}
        }
    \end{itemize}
}{
    \begin{itemize}
        \item \methoddesc{\public}{}{void}{setValue}{}
        {
            Sets the \refer{\lblroot:view.diagrams.components:DiagramValueDisplayComponent:value}{value} attribute.
        }{}{}
        \item \methoddesc{\public}{}{Number}{getValue}{}
        {}{}{
            The \refer{\lblroot:view.diagrams.components:DiagramValueDisplayComponent:value}{value} attribute.
        }
        \item \methoddesc{\public}{\abstr}{void}{refreshValueRelevantAttributes}{}
        {
            Performs the appropriate actions when the \refer{\lblroot:view.diagrams.components:DiagramValueDisplayComponent:value}{value} attribute is modified.
        }{}{}
    \end{itemize}
}

\absclassdesc{DiagramBar}{
    \decl{\public \abstr \class DiagramBar}{The abstract class inherited by the classes that represent bars of \refer{\lblroot:view.diagrams.type:Diagram}{Diagrams}.}
}{
    \begin{itemize}
        \item \atrdesc{\private}{}{PositionIn2DDiagram}{bottomLeft}{The bottom left coordinates of the bar instance in the \refer{\lblroot:view.diagrams.type:Diagram}{Diagram} (on axes).}
        \item \atrdesc{\private}{}{PositionIn2DDiagram}{topRight}{The top right coordinates of the bar instance in the \refer{\lblroot:view.diagrams.type:Diagram}{Diagram} (on axes).}
        \item \atrdesc{\private}{}{Number}{borderThickness}{The thickness of the borders of the bar instance.}
    \end{itemize}
}{
    \begin{itemize}
        \item \subclsdec{HistogramBar}
        \item \subclsdec{BarChartBar}
    \end{itemize}
}{
    \begin{itemize}
        \item \constrdesc{\proted}{DiagramBar}{\type{Color} color, \type{Number} value, \type{PositionIn2DDiagram} bottomLeft, \type{PositionIn2DDiagram} topRight, \type{Number} borderThickness}{}{
            \begin{itemize}
                \item \paramdesc{color}{The color of the bar instance.}
                \item \paramdesc{value}{The value to be represented by the bar instance.}
                \item \paramdesc{bottomLeft}{The bottom left coordinates of the bar instance in the \refer{\lblroot:view.diagrams.type:Diagram}{Diagram} (on axes).}
                \item \paramdesc{topRight}{The top right coordinates of the bar instance in the \refer{\lblroot:view.diagrams.type:Diagram}{Diagram} (on axes).}
                \item \paramdesc{borderThickness}{The thickness of the borders of the bar instance.}
            \end{itemize}
        }
    \end{itemize}
}{
    \begin{itemize}
        \item \methoddesc{\public}{}{Number}{getHeight}{}
        {}{}{
            The height of the bar instance.
        }
        \item \methoddesc{\public}{}{Number}{getWidth}{}
        {}{}{
            The width of the bar instance
        }
        \item \methoddesc{\public}{}{void}{setBottomLeftInDiagram}{\type{Number} x1, \type{Number} y1}
        {}{
            \begin{itemize}
                \item \paramdesc{x1}{The new x-Coordinate of the bottom left corner.}
                \item \paramdesc{y1}{The new y-Coordinate of the bottom left corner.}
            \end{itemize}
        }{
            Sets \refer{\lblroot:view.diagrams.components:DiagramBar:bottomLeft}{bottomLeft}.
        }
        \item \methoddesc{\public}{}{void}{setTopRightInDiagram}{\type{Number} x2, \type{Number} y2}
        {}{
            \begin{itemize}
                \item \paramdesc{x2}{The new x-Coordinate of the top right corner.}
                \item \paramdesc{y2}{The new y-Coordinate of the top right corner.}
            \end{itemize}
        }{
            Sets \refer{\lblroot:view.diagrams.components:DiagramBar:topRight}{topRight}.
        }
        \item \methoddesc{\public}{}{PositionIn2DDiagram}{getBottomLeftInDiagram}{}
        {}{}{
            \refer{\lblroot:view.diagrams.components:DiagramBar:bottomLeft}{bottomLeft}
        }
        \item \methoddesc{\public}{}{PositionIn2DDiagram}{getTopRightInDiagram}{}
        {}{}{
            \refer{\lblroot:view.diagrams.components:DiagramBar:topRight}{topRight}.
        }
        \item \ovrdnmethoddesc{\public}{\abstr}{void}{refreshValueRelevantAttributes}{}{DiagramValueDisplayComponent}
        {
            Sets the y-Coordinate of \refer{\lblroot:view.diagrams.components:DiagramBar:topRight}{topRight} and the value attribute to correlate each other.
        }{}{}
    \end{itemize}
}

\absclassdesc{DiagramAxis}{
    \decl{\public \abstr \class DiagramAxis}{The abstract class inherited by the classes that represent axes in a \refer{\lblroot:view.diagrams.type:Diagram}{Diagram}.}
}{
    \begin{itemize}
        \item \atrdesc{\private}{}{Number}{min}{The minimum value on the axis.}
        \item \atrdesc{\private}{}{Number}{max}{The maximum value on the axis.}
        \item \atrdesc{\private}{}{int}{steps}{The amount of partitions the axis will have.}
        \item \overridecasclabelname{\lblroot\lblpackage\lblpackageelement:(atr)showValues}{\atrdesc{\private}{}{boolean}{showValues}{Indicates whether the values for each partition is shown.}}
        \item \atrdesc{\private}{}{DiagramLine}{axisLine}{The \refer{\lblroot:view.diagrams.components:DiagramLine}{DiagramLine} part of the axis.}
    \end{itemize}
}{
    \begin{itemize}
        \item \subclsdec{SolidAxis}
    \end{itemize}
}{
    \begin{itemize}
        \item \constrdesc{\proted}{DiagramAxis}{\type{DiagramLine} axisLine, \type{Number} min, \type{Number} max, \type{int} steps}{}{
            \begin{itemize}
                \item \paramdesc{axisLine}{The \refer{\lblroot:view.diagrams.components:DiagramLine}{DiagramLine} part of the axis.}
                \item \paramdesc{min}{The minimum value on the axis.}
                \item \paramdesc{max}{The maximum value on the axis.}
                \item \paramdesc{steps}{The amount of partitions the axis will have.}
            \end{itemize}
        }
    \end{itemize}
}{
    \begin{itemize}
        \item \methoddesc{\public}{}{void}{setMin}{\type{Number} min}
        {
            Sets \refer{\lblroot:view.diagrams.components:DiagramAxis:min}{min} attribute.
        }{
            \begin{itemize}
                \item \paramdesc{min}{The new \refer{\lblroot:view.diagrams.components:DiagramAxis:min}{min}.}
            \end{itemize}
        }{}
        \item \methoddesc{\public}{}{Number}{getMin}{}
        {}{}{
            \refer{\lblroot:view.diagrams.components:DiagramAxis:min}{min}.
        }
        \item \methoddesc{\public}{}{void}{setMax}{\type{Number} max}
        {
            Sets \refer{\lblroot:view.diagrams.components:DiagramAxis:max}{max} attribute.
        }{
            \begin{itemize}
                \item \paramdesc{max}{The new \refer{\lblroot:view.diagrams.components:DiagramAxis:max}{max}.}
            \end{itemize}
        }{}
        \item \methoddesc{\public}{}{Number}{getMax}{}
        {}{}{
            \refer{\lblroot:view.diagrams.components:DiagramAxis:min}{min}.
        }
        \item \methoddesc{\public}{}{void}{setSteps}{\type{int} steps}
        {
            Sets \refer{\lblroot:view.diagrams.components:DiagramAxis:steps}{steps} attribute.
        }{
            \begin{itemize}
                \item \paramdesc{steps}{The new \refer{\lblroot:view.diagrams.components:DiagramAxis:steps}{steps}.}
            \end{itemize}
        }{}
        \item \methoddesc{\public}{}{int}{getSteps}{}
        {}{}{
            \refer{\lblroot:view.diagrams.components:DiagramAxis:steps}{steps}.
        }
        \item \overridecasclabelname{\lblroot\lblpackage\lblpackageelement:(mtd)showValues}{\methoddesc{\public}{}{void}{showValues}{}
        {
            Shows the values painted on the partitions of the axis.
        }{}{}}
        \item \methoddesc{\public}{}{void}{hideValues}{}
        {
            Hides the values painted on the partitions of the axis.
        }{}{}
        \item \methoddesc{\public}{}{PositionInFrame}{valueToCoordinate}{\type{Number} value}
        {
            Transforms the given coordinate on the axis (\refer{\lblroot:view.diagrams.components:DiagramValueDisplayComponent:value}{value}) to the corresponding frame coordinates.
        }{
            \begin{itemize}
                \item \paramdesc{value}{The given coordinate on the axis.}
            \end{itemize}
        }{
            The corresponding frame coordinates.
        }
        \item \methoddesc{\public}{}{Number}{CoordinateToValue}{\type{PositionInFrame} coordinate}
        {
            Transforms the given coordinates on the frame to the corresponding axis coordinate (\refer{\lblroot:view.diagrams.components:DiagramValueDisplayComponent:value}{value}).
        }{
            \begin{itemize}
                \item \paramdesc{coordinate}{The given coordinates on the frame.}
            \end{itemize}
        }{
            The corresponding coordinate on the axis (\refer{\lblroot:view.diagrams.components:DiagramValueDisplayComponent:value}{value}).
        }
        \item \methoddesc{\public}{}{void}{setLineByPos}{\type{Number} minValXPos, \type{Number} minValYPos, \type{Number} maxValXPos, \type{Number} maxValYPos}
        {
            Sets the position of \refer{\lblroot:view.diagrams.components:DiagramAxis:axisLine}{axisLine} to the given ones.
        }{
            \begin{itemize}
                \item \paramdesc{minValXPos}{The x-Coordinate of where the minimum value on the axis will be.}
                \item \paramdesc{minValYPos}{The y-Coordinate of where the minimum value on the axis will be.}
                \item \paramdesc{maxValXPos}{The x-Coordinate of where the maximum value on the axis will be.}
                \item \paramdesc{maxValYPos}{The y-Coordinate of where the maximum value on the axis will be.}
            \end{itemize}
        }{}
        \item \methoddesc{\public}{}{void}{setLineColor}{\type{Color} color}
        {
            Sets the color of the \refer{\lblroot:view.diagrams.components:DiagramAxis:axisLine}{axisLine}.
        }{
            \begin{itemize}
                \item \paramdesc{color}{The new color of \refer{\lblroot:view.diagrams.components:DiagramAxis:axisLine}{axisLine}.}
            \end{itemize}
        }{}
        \item \methoddesc{\public}{}{void}{setLineThickness}{\type{Number} thickness}
        {
            Sets the thickness of the \refer{\lblroot:view.diagrams.components:DiagramAxis:axisLine}{axisLine}.
        }{
            \begin{itemize}
                \item \paramdesc{thickness}{The new thickness of the \refer{\lblroot:view.diagrams.components:DiagramAxis:axisLine}{axisLine}.}
            \end{itemize}
        }{}
        \item \methoddesc{\public}{}{Number}{getLineLength}{}
        {}{}{
            The length of \refer{\lblroot:view.diagrams.components:DiagramAxis:axisLine}{axisLine}
        }
        \item \ovrdnmethoddesc{\public}{}{void}{show}{}{DiagramComponent}
        {
            Shows the axis.
        }{}{}
        \item \ovrdnmethoddesc{\public}{}{void}{hide}{}{DiagramComponent}
        {
            Hides the axis.
        }{}{}
    \end{itemize}
}

\absclassdesc{DiagramLabel}{
    \decl{\public \abstr \class DiagramLabel}{The abstract class inherited by classes that represent labels that are not responsible for displaying values on \refer{\lblroot:view.diagrams.type:Diagram}{Diagrams}.}
}{
    \begin{itemize}
        \item \atrdesc{\private}{}{String}{caption}{The text that will be displayed by the label.}
        \item \atrdesc{\private}{}{PositionInFrame}{bottomLeft}{The position of the bottom left corner of the label in the frame.}
        \item \atrdesc{\private}{}{PositionInFrame}{topRight}{The position of the top right corner of the label in the frame.}
        \item \atrdesc{\private}{}{Number}{borderThickness}{The thickness of the border of the label.}
    \end{itemize}
}{
    \begin{itemize}
        \item \subclsdec{DescriptionLabel}
    \end{itemize}
}{
    \begin{itemize}
        \item \constrdesc{\proted}{DiagramLabel}{\type{PositionInFrame} bottomLeft, \type{PositionInFrame} topRight, \type{Color} color, \type{String} caption, \type{Number} borderThickness}{}{
            \begin{itemize}
                \item \paramdesc{bottomLeft}{The position of the bottom left corner of the label in the frame.}
                \item \paramdesc{topRight}{The position of the top right corner of the label in the frame.}
                \item \paramdesc{color}{The color of the label.}
                \item \paramdesc{caption}{The text that will be displayed by the label.}
                \item \paramdesc{borderThickness}{The thickness of the border of the label.}
            \end{itemize}
        }
    \end{itemize}
}{
    \begin{itemize}
        \item \methoddesc{\public}{}{void}{setCaption}{\type{String} caption}
        {
            Changes \refer{\lblroot:view.diagrams.components:DiagramLabel:caption}{caption} to the given one.
        }{
            \begin{itemize}
                \item \paramdesc{caption}{The text that will be displayed by the label.}
            \end{itemize}
        }{}
        \item \methoddesc{\public}{}{String}{getCaption}{}
        {}{}{
            \refer{\lblroot:view.diagrams.components:DiagramLabel:caption}{caption}.
        }
        \item \methoddesc{\public}{}{void}{setBottomLeftInFrame}{\type{Number} x1, \type{Number} y1}
        {}{
            \begin{itemize}
                \item \paramdesc{x1}{The new x-Coordinate of the bottom left corner.}
                \item \paramdesc{y1}{The new y-Coordinate of the bottom left corner.}
            \end{itemize}
        }{}
        \item \methoddesc{\public}{}{void}{setTopRightInFrame}{\type{Number} x2, \type{Number} y2}
        {}{
            \begin{itemize}
                \item \paramdesc{x2}{The new x-Coordinate of the top right corner.}
                \item \paramdesc{y2}{The new y-Coordinate of the top right corner.}
            \end{itemize}
        }{}
        \item \methoddesc{\public}{}{PositionInFrame}{getBottomLeftInFrame}{}
        {}{}{
            \refer{\lblroot:view.diagrams.components:DiagramLabel:bottomLeft}{bottomLeft}
        }
        \item \methoddesc{\public}{}{PositionInFrame}{getTopRightInFrame}{}
        {}{}{
            \refer{\lblroot:view.diagrams.components:DiagramLabel:topRight}{topRight}
        }
    \end{itemize}
}

\absclassdesc{DiagramLine}{
    \decl{\public \abstr \class DiagramLine}{The abstract class inherited by the classes that represent a line in a \refer{\lblroot:view.diagrams.type:Diagram}{Diagram}.}
}{
    \begin{itemize}
        \item \atrdesc{\private}{}{PositionInFrame}{start}{The position (in frame) of the start of the line (x1, y1).}
        \item \atrdesc{\private}{}{PositionInFrame}{end}{The position (in frame) of the end of the line (x2, y2).}
        \item \atrdesc{\private}{}{Number}{thickness}{The thickness of the line.}
    \end{itemize}
}{
    \begin{itemize}
        \item \subclsdec{SolidLine}
        \item \subclsdec{CoordinateIndicatorLine}
        \item \subclsdec{ValueLine}
    \end{itemize}
}{
    \begin{itemize}
        \item \constrdesc{\proted}{DiagramLine}{\type{PositionInFrame} start, \type{PositionInFrame} end, \type{Color} color, \type{Number} thickness}{}{
            \begin{itemize}
                \item \paramdesc{start}{The position (in frame) of the start of the line (x1, y1).}
                \item \paramdesc{end}{The position (in frame) of the end of the line (x2, y2).}
                \item \paramdesc{color}{The color of the line.}
                \item \paramdesc{thickness}{The thickness of the line.}
            \end{itemize}
        }
    \end{itemize}
}{
    \begin{itemize}
        \item \methoddesc{\proted}{}{Number}{calculateLength}{}
        {}{}{
            The length of the line.
        }
        \item \methoddesc{\public}{}{void}{setThickness}{\type{Number} thickness}
        {
            Set \refer{\lblroot:view.diagrams.components:DiagramLine:thickness}{thickness} to the given thickness.
        }{
            \begin{itemize}
                \item \paramdesc{thickness}{The given thickness.}
            \end{itemize}
        }{}
        \item \methoddesc{\public}{}{void}{setStartInFrame}{\type{Number} x1, \type{Number} y1}
        {}{
            \begin{itemize}
                \item \paramdesc{x1}{The new x-Coordinate of the start of the line.}
                \item \paramdesc{y1}{The new y-Coordinate of the start of the line.}
            \end{itemize}
        }{}
        \item \methoddesc{\public}{}{void}{setEndInFrame}{\type{Number} x2, \type{Number} y2}
        {}{
            \begin{itemize}
                \item \paramdesc{x2}{The new x-Coordinate of the end of the line.}
                \item \paramdesc{y2}{The new y-Coordinate of the end of the line.}
            \end{itemize}
        }{}
        \item \methoddesc{\public}{}{PositionInFrame}{getStartInFrame}{}
        {}{}{
            \refer{\lblroot:view.diagrams.components:DiagramLine:start}{start}.
        }
        \item \methoddesc{\public}{}{PositionInFrame}{getEndInFrame}{}
        {}{}{
            \refer{\lblroot:view.diagrams.components:DiagramLine:end}{end}.
        }
        \item \methoddesc{\public}{}{Number}{getLength}{}
        {}{}{
            The length of the line, based on \refer{\lblroot:view.diagrams.components:DiagramLine:start}{start} and \refer{\lblroot:view.diagrams.components:DiagramLine:end}{end}.
        }
        \item \methoddesc{\public}{}{Number}{getThickness}{}
        {}{}{
            \refer{\lblroot:view.diagrams.components:DiagramLine:thickness}{thickness}.
        }
    \end{itemize}
}

\absclassdesc{DiagramValueLabel}{
    \decl{\public \abstr \class DiagramValueLabel}{The abstract class implemented by the classes that represent labels that are responsible for displaying values.}
}{
    \begin{itemize}
        \item \atrdesc{\private}{}{PositionIn2DDiagram}{topRight}{The position of the top right corner of the label.}
        \item \atrdesc{\private}{}{PositionIn2DDiagram}{bottomLeft}{The position of the bottom left corner of the label.}
        \item \atrdesc{\private}{}{String}{caption}{The text that the label will display.}
        \item \atrdesc{\private}{}{Number}{borderThickness}{The thickness of the border of the label.}
    \end{itemize}
}{
    \begin{itemize}
        \item \subclsdec{HeatMapLabel}
    \end{itemize}
}{
    \begin{itemize}
        \item \constrdesc{\proted}{DiagramValueLabel}{\type{PositionIn2DDiagram} bottomLeft, \type{PositionIn2DDiagram} topRight, \type{Color} color, \type{Number} value, \type{Number} borderThickness}{}{
            \begin{itemize}
                \item \paramdesc{bottomLeft}{The position of the bottom left corner of the label.}
                \item \paramdesc{topRight}{The position of the top right corner of the label.}
                \item \paramdesc{color}{The color of the label.}
                \item \paramdesc{value}{The value that will be displayed by the label.}
                \item \paramdesc{borderThickness}{The thickness of the border of the label.}
            \end{itemize}
        }
    \end{itemize}
}{
    \begin{itemize}
        \item \methoddesc{\proted}{}{void}{refreshCaption}{}
        {
            Refreshes the \refer{\lblroot:view.diagrams.components:DiagramValueLabel:caption}{caption}, when events that affect the caption occur.
        }{}{}
        \item \methoddesc{\proted}{}{void}{setCaption}{\type{String} caption}
        {
            Sets the \refer{\lblroot:view.diagrams.components:DiagramValueLabel:caption}{caption}.
        }{
            \begin{itemize}
                \item \paramdesc{caption}{The new \refer{\lblroot:view.diagrams.components:DiagramValueLabel:caption}{caption}.}
            \end{itemize}
        }{}
        \item \methoddesc{\public}{}{String}{getCaption}{}
        {}{}{
            \refer{\lblroot:view.diagrams.components:DiagramValueLabel:caption}{caption}.
        }
        \item \methoddesc{\public}{}{void}{setBottomLeftInDiagram}{\type{Number} x1, \type{Number} y1}
        {}{
            \begin{itemize}
                \item \paramdesc{x1}{The new x-Coordinate of the bottom left corner.}
                \item \paramdesc{y1}{The new y-Coordinate of the bottom left corner.}
            \end{itemize}
        }{}
        \item \methoddesc{\public}{}{void}{setTopRightInDiagram}{\type{Number} x2, \type{Number} y2}
        {}{
            \begin{itemize}
                \item \paramdesc{x2}{The new x-Coordinate of the top right corner.}
                \item \paramdesc{y2}{The new y-Coordinate of the top right corner.}
            \end{itemize}
        }{}
        \item \methoddesc{\public}{}{PositionIn2DDiagram}{getBottomLeftInDiagram}{}
        {}{}{
            \refer{\lblroot:view.diagrams.components:DiagramValueLabel:bottomLeft}{bottomLeft}
        }
        \item \methoddesc{\public}{}{PositionIn2DDiagram}{getTopRightInDiagram}{}
        {}{}{
            \refer{\lblroot:view.diagrams.components:DiagramValueLabel:topRight}{topRight}
        }
    \end{itemize}
}

\absclassdesc{DiagramPoint}{
    \decl{\public \abstr \class DiagramPoint}{The abstract class inherited by the classes that represent a point on a \refer{\lblroot:view.diagrams.type:Diagram}{Diagram}.}
}{
    \begin{itemize}
        \item \atrdesc{\private}{}{PositionIn2DDiagram}{position}{The position of the point in the \refer{\lblroot:view.diagrams.type:Diagram}{Diagram}.}
        \item \atrdesc{\private}{}{Number}{size}{The size of the point.}
    \end{itemize}
}{
    \begin{itemize}
        \item \subclsdec{ValueDisplayPoint}
    \end{itemize}
}{
    \begin{itemize}
        \item \constrdesc{\proted}{DiagramPoint}{\type{PositionIn2DDiagram} position, \type{Color} color, \type{Number} value, \type{Number} size}{}{
            \begin{itemize}
                \item \paramdesc{position}{The position of the point in the \refer{\lblroot:view.diagrams.type:Diagram}{Diagram}.}
                \item \paramdesc{color}{The color of the point.}
                \item \paramdesc{value}{The value that the point represents.}
                \item \paramdesc{size}{The size of the point.}
            \end{itemize}
        }
    \end{itemize}
}{
    \begin{itemize}
        \item \methoddesc{\public}{}{void}{setSize}{\type{Number} size}
        {
            Sets the \refer{\lblroot:view.diagrams.components:DiagramPoint:size}{size}.
        }{
            \begin{itemize}
                \item \paramdesc{size}{The new \refer{\lblroot:view.diagrams.components:DiagramPoint:size}{size}.}
            \end{itemize}
        }{}
        \item \methoddesc{\public}{}{Number}{getSize}{}
        {}{}{
            \refer{\lblroot:view.diagrams.components:DiagramPoint:size}{size}.
        }
        \item \methoddesc{\public}{}{void}{setPositionInDiagram}{\type{Number} x, \type{Number} y}
        {}{
            \begin{itemize}
                \item \paramdesc{x}{The new x-Coordinate of the point in the \refer{\lblroot:view.diagrams.type:Diagram}{Diagram}.}
                \item \paramdesc{y}{The new y-Coordinate of the point in the \refer{\lblroot:view.diagrams.type:Diagram}{Diagram}.}
            \end{itemize}
        }{}
        \item \methoddesc{\public}{}{PositionIn2DDiagram}{getPositionInDiagram}{}
        {}{}{
            \refer{\lblroot:view.diagrams.components:DiagramPoint:position}{position}.
        }
    \end{itemize}
}

\absclassdesc{DiagramColorScale}{
    \decl{\public \abstr \class}{The abstract class inherited by the classes that represent color scales in \refer{\lblroot:view.diagrams.type:Diagram}{Diagrams}. \descnote{The color scale will have an array of values and another array of the same size with colors. Each value that is not in the array, which lies in the range of the minimum and the maximum value, will have a mixture of the colors of the values, to which it is the nearest to.}}
}{
    \begin{itemize}
        \item \atrdesc{\private}{}{PositionIn2DDiagram}{bottomLeft}{The position of the bottom left corner of the color scale.}
        \item \atrdesc{\private}{}{PositionIn2DDiagram}{topRight}{The position of the top right corner of the color scale.}
        \item \atrdesc{\private}{}{Number}{borderThickness}{The thickness of the border of the color scale.}
        \item \atrdesc{\private}{}{Number[]}{values}{The values that have a certain color.}
        \item \atrdesc{\private}{}{Color[]}{valueColors}{The colors that represent the values.}
    \end{itemize}
}{
    \begin{itemize}
        \item \subclsdec{2ColorScale}
    \end{itemize}
}{
    \begin{itemize}
        \item \constrdesc{\proted}{}{\type{PositionIn2DDiagram} bottomLeft, \type{PositionIn2DDiagram} topRight, \type{Color} borderColor, \type{Number[]} values, \type{Color[]} valueColors, \type{Number} borderThickness}
        {
            Constructs and initializes the instance. \refer{\lblroot:view.diagrams.components:DiagramColorScale:values}{Values} and \refer{\lblroot:view.diagrams.components:DiagramColorScale:valueColors}{ValueColors} must have the same sizes. A value with the index i v[i] will be associated with the valueColor at the same index c[i].
        }{
            \begin{itemize}
                \item \paramdesc{bottomLeft}{The position of the bottom left corner of the color scale.}
                \item \paramdesc{topRight}{The position of the top right corner of the color scale.}
                \item \paramdesc{borderColor}{The color of the border of the color scale.}
                \item \paramdesc{values}{The values that have a certain color.}
                \item \paramdesc{valueColors}{The colors that represent the values.}
                \item \paramdesc{borderThickness}{The thickness of the border of the color scale.}
            \end{itemize}
        }
    \end{itemize}
}{
    \begin{itemize}
        \item \methoddesc{\public}{}{Color}{valueToColor}{\type{Number} value}
        {}{
            \begin{itemize}
                \item \paramdesc{value}{The given value.}
            \end{itemize}
        }{
            The color, which corresponds to the given value.
        }
        \item \methoddesc{\public}{}{Number[]}{getValues}{}
        {}{}{
            \refer{\lblroot:view.diagrams.components:DiagramColorScale:values}{values}.
        }
        \item \methoddesc{\public}{}{Color[]}{getColors}{}
        {}{}{
            \refer{\lblroot:view.diagrams.components:DiagramColorScale:valueColors}{valueColors}.
        }
        \item \methoddesc{\public}{}{void}{setBottomLeftInFrame}{\type{Number} x1, \type{Number} y1}
        {}{
            \begin{itemize}
                \item \paramdesc{x1}{The new x-Coordinate of the bottom left corner of the color scale.}
                \item \paramdesc{y1}{The new y-Coordinate of the bottom left corner of the color scale.}
            \end{itemize}
        }{}
        \item \methoddesc{\public}{}{void}{setTopRightInFrame}{\type{Number} x2, \type{Number} y2}
        {}{
            \begin{itemize}
                \item \paramdesc{x2}{The new x-Coordinate of the top right corner of the color scale.}
                \item \paramdesc{y2}{The new y-Coordinate of the top right corner of the color scale.}
            \end{itemize}
        }{}
        \item \methoddesc{\public}{}{PositionInFrame}{getBottomLeftInFrame}{}
        {}{}{
            \refer{\lblroot:view.diagrams.components:DiagramColorScale:bottomLeft}{bottomLeft}.
        }
        \item \methoddesc{\public}{}{PositionInFrame}{getTopRightInFrame}{}
        {}{}{
            \refer{\lblroot:view.diagrams.components:DiagramColorScale:topRight}{topRight}.
        }
    \end{itemize}
}

\classdesc{PositionInFrame}{
    \decl{\public \class PositionInFrame}{The class, which represents a position in the frame by storing the x- and y-Coordinates.}
}{
    \begin{itemize}
        \item \atrdesc{\private}{}{Number}{xPos}{The x-Coordinate in the frame.}
        \item \atrdesc{\private}{}{Number}{yPos}{The y-Coordinate in the frame.}
    \end{itemize}
}{
    \begin{itemize}
        \item \constrdesc{\public}{PositionInFrame}{\type{Number} xPos, \type{Number} yPos}{}{
            \begin{itemize}
                \item \paramdesc{xPos}{The x-Coordinate in the frame.}
                \item \paramdesc{yPos}{The y-Coordinate in the frame.}
            \end{itemize}
        }
    \end{itemize}
}{
    \begin{itemize}
        \item \methoddesc{\public}{}{Number}{getXPos}{}
        {}{}{
            \refer{\lblroot:view.diagrams.components:PositionInFrame:xPos}{xPos}.
        }
        \item \methoddesc{\public}{}{void}{setXPos}{\type{Number} xPos}
        {
            Sets \refer{\lblroot:view.diagrams.components:PositionInFrame:xPos}{xPos} to the given one.
        }{
            \begin{itemize}
                \item \paramdesc{xPos}{The new x-Coordinate in the frame.}
            \end{itemize}
        }{}
        \item \methoddesc{\public}{}{Number}{getYPos}{}
        {}{}{
            \refer{\lblroot:view.diagrams.components:PositionInFrame:yPos}{yPos}.
        }
        \item \methoddesc{\public}{}{void}{setYPos}{\type{Number} yPos}
        {
            Sets \refer{\lblroot:view.diagrams.components:PositionInFrame:yPos}{yPos} to the given one.
        }{
            \begin{itemize}
                \item \paramdesc{yPos}{The new y-Coordinate in the frame.}
            \end{itemize}
        }{}
    \end{itemize}
}

\classdesc{PositionIn2DDiagram}{
    \decl{\public \class PositionIn2DDiagram}{The class that represents a position in a 2 dimensional \refer{\lblroot:view.diagrams:IDiagram}{IDiagram} according to its x- and y-axis.\descnote{The x- and y-coordinates and axes will be stored in \refer{\lblroot:view.diagrams.components:PositionInDiagram:positionsInAxes}{positionsInAxes} and \refer{\lblroot:view.diagrams.components:PositionInDiagram:axes}{axes} attributes of the \refer{\lblroot:view.diagrams.components:PositionInDiagram}{PositionInDiagram} instance.}}
}{

}{
    \begin{itemize}
        \item \constrdesc{\public}{PositionIn2DDiagram}{\type{DiagramAxis} xAxis, \type{Number} xCoordinate, \type{DiagramAxis} yAxis, \type{Number} yCoordinate}{}{
            \begin{itemize}
                \item \paramdesc{xAxis}{The x-axis of the \refer{\lblroot:view.diagrams:IDiagram}{IDiagram}.}
                \item \paramdesc{xPos}{The x-Coordinate on the x-axis of the \refer{\lblroot:view.diagrams:IDiagram}{IDiagram}.}
                \item \paramdesc{yAxis}{The y-axis of the \refer{\lblroot:view.diagrams:IDiagram}{IDiagram}.}
                \item \paramdesc{yPos}{The y-Coordinate on the x-axis of the \refer{\lblroot:view.diagrams:IDiagram}{IDiagram}.}
            \end{itemize}
        }
    \end{itemize}
}{
    \begin{itemize}
        \item \methoddesc{\public}{}{void}{setXCoordinate}{\type{Number} xCoordinate}
        {
            Sets the x-coordinate to the given one.
        }{
            \begin{itemize}
                \item \paramdesc{xCoordinate}{The given x-coordinate.}
            \end{itemize}
        }{}
        \item \methoddesc{\public}{}{Number}{getXCoordinate}{}
        {}{}{
            The x-coordinate.
        }
        \item \methoddesc{\public}{}{void}{setYCoordinate}{\type{Number} yCoordinate}
        {
            Sets the y-coordinate to the given one.
        }{
            \begin{itemize}
                \item \paramdesc{yCoordinate}{The given y-coordinate.}
            \end{itemize}
        }{}
        \item \methoddesc{\public}{}{Number}{getYCoordinate}{}
        {}{}{
            The y-coordinate.
        }
    \end{itemize}
}

\classdesc{SolidLine}{}{
    \begin{itemize}
        \item \atrdesc{\private}{}{Line}{line}{}
    \end{itemize}
}{
    \begin{itemize}
        \item \constrdesc{\proted}{SolidLine}{\type{PositionInFrame} start, \type{PositionInFrame} end, \type{Color} color, \type{Number} thickness}{}{
            \begin{itemize}
                \item \paramdesc{start}{}
                \item \paramdesc{end}{}
                \item \paramdesc{color}{}
                \item \paramdesc{thickness}{}
            \end{itemize}
        }
    \end{itemize}
}{
    \begin{itemize}
        \item \ovrdnmethoddesc{\public}{}{void}{show}{}{DiagramComponent}
        {}{}{}
        \item \ovrdnmethoddesc{\public}{}{void}{hide}{}{DiagramComponent}
        {}{}{}
        \item \ovrdnmethoddesc{\public}{}{DiagramComponent}{clone}{}{DiagramComponent}
        {}{}{}
    \end{itemize}
}

\classdesc{2ColorScale}{}{
    \begin{itemize}
        \item \atrdesc{\private}{}{WritableImage}{colorScale}{}
        \item \atrdesc{\private}{}{Color}{minValueColor}{}
        \item \atrdesc{\private}{}{Color}{maxValueColor}{}
        \item \atrdesc{\private}{}{Number}{minValue}{}
        \item \atrdesc{\private}{}{Number}{maxValue}{}
    \end{itemize}
}{
    \begin{itemize}
        \item \constrdesc{\proted}{2ColorScale}{\type{PositionIn2DDiagram} bottomLeft, \type{PositionIn2DDiagram} topRight, \type{Color} borderColor, \type{Number} minVal, \type{Number} maxVal, \type{Color} minValColor, \type{Color} maxValColor, \type{Number} borderThickness}{}{
            \begin{itemize}
                \item \paramdesc{bottomLeft}{}
                \item \paramdesc{topRight}{}
                \item \paramdesc{borderColor}{}
                \item \paramdesc{minVal}{}
                \item \paramdesc{maxVal}{}
                \item \paramdesc{minValColor}{}
                \item \paramdesc{maxValColor}{}
                \item \paramdesc{borderThickness}{}
            \end{itemize}
        }
    \end{itemize}
}{
    \begin{itemize}
        \item \methoddesc{\public}{}{void}{setMinValueColor}{\type{Color} minValueColor}
        {}{
            \begin{itemize}
                \item \paramdesc{minValueColor}{}
            \end{itemize}
        }{}
        \item \methoddesc{\public}{}{Color}{getMinValueColor}{}
        {}{}{}
        \item \methoddesc{\public}{}{void}{setMaxValueColor}{\type{Color} maxValueColor}
        {}{
            \begin{itemize}
                \item \paramdesc{maxValueColor}{}
            \end{itemize}
        }{}
        \item \methoddesc{\public}{}{Color}{getMaxValueColor}{}
        {}{}{}
        \item \methoddesc{\public}{}{void}{setMinValue}{\type{Number} minValue}
        {}{
            \begin{itemize}
                \item \paramdesc{minValue}{}
            \end{itemize}
        }{}
        \item \methoddesc{\public}{}{Number}{getMinValue}{}
        {}{}{}
        \item \methoddesc{\public}{}{void}{setMaxValue}{\type{Number} maxValue}
        {}{
            \begin{itemize}
                \item \paramdesc{maxValue}{}
            \end{itemize}
        }{}
        \item \methoddesc{\public}{}{Number}{getMaxValue}{}
        {}{}{}
        \item \ovrdnmethoddesc{\public}{}{void}{show}{}{DiagramComponent}
        {}{}{}
        \item \ovrdnmethoddesc{\public}{}{void}{hide}{}{DiagramComponent}
        {}{}{}
        \item \ovrdnmethoddesc{\public}{}{DiagramComponent}{clone}{}{DiagramComponent}
        {}{}{}
    \end{itemize}
}

\classdesc{ValueDisplayPoint}{}{
    \begin{itemize}
        \item \atrdesc{\private}{}{Point}{point}{}
    \end{itemize}
}{
    \begin{itemize}
        \item \constrdesc{\proted}{ValueDisplayPoint}{\type{Color} color, \type{Number} value, \type{Number} size, \type{PositionIn2DDiagram} position}{}{
            \begin{itemize}
                \item \paramdesc{color}{}
                \item \paramdesc{value}{}
                \item \paramdesc{size}{}
                \item \paramdesc{position}{}
            \end{itemize}
        }
    \end{itemize}
}{
    \begin{itemize}
        \item \ovrdnmethoddesc{\public}{}{void}{show}{}{DiagramComponent}
        {}{}{}
        \item \ovrdnmethoddesc{\public}{}{void}{hide}{}{DiagramComponent}
        {}{}{}
        \item \ovrdnmethoddesc{\public}{}{DiagramComponent}{clone}{}{DiagramComponent}
        {}{}{}
    \end{itemize}
}

\classdesc{HistogramBar}{}{
    \begin{itemize}
        \item \atrdesc{\private}{}{Label}{label}{}
    \end{itemize}
}{
    \begin{itemize}
        \item \constrdesc{\proted}{HistogramBar}{\type{Color} color, \type{Number} value, \type{PositionIn2DDiagram} bottomLeft, \type{PositionIn2DDiagram} topRight}{}{
            \begin{itemize}
                \item \paramdesc{color}{}
                \item \paramdesc{value}{}
                \item \paramdesc{bottomLeft}{}
                \item \paramdesc{topRight}{}
            \end{itemize}
        }
    \end{itemize}
}{
    \begin{itemize}
        \item \ovrdnmethoddesc{\public}{}{void}{show}{}{DiagramComponent}
        {}{}{}
        \item \ovrdnmethoddesc{\public}{}{void}{hide}{}{DiagramComponent}
        {}{}{}
        \item \ovrdnmethoddesc{\public}{}{DiagramComponent}{clone}{}{DiagramComponent}
        {}{}{}
    \end{itemize}
}

\classdesc{BarChartBar}{}{
    \begin{itemize}
        \item \atrdesc{\private}{}{Label}{label}{}
    \end{itemize}
}{
    \begin{itemize}
        \item \constrdesc{\proted}{BarChartBar}{\type{Color} color, \type{Number} value, \type{Number} width, \type{PositionIn2DDiagram} bottomLeft, \type{PositionIn2DDiagram} topRight}{}{
            \begin{itemize}
                \item \paramdesc{color}{}
                \item \paramdesc{value}{}
                \item \paramdesc{width}{}
                \item \paramdesc{bottomLeft}{}
                \item \paramdesc{topRight}{}
            \end{itemize}
        }
    \end{itemize}
}{
    \begin{itemize}
        \item \ovrdnmethoddesc{\public}{}{void}{show}{}{DiagramComponent}
        {}{}{}
        \item \ovrdnmethoddesc{\public}{}{void}{hide}{}{DiagramComponent}
        {}{}{}
        \item \ovrdnmethoddesc{\public}{}{DiagramComponent}{clone}{}{DiagramComponent}
        {}{}{}
    \end{itemize}
}

\classdesc{SolidAxis}{}{}{
\begin{itemize}
    \item \constrdesc{\proted}{SolidAxis}{\type{SolidLine} axisLine, \type{Number} min, \type{Number} max, \type{int} steps}{}{
        \begin{itemize}
            \item \paramdesc{axisLine}{}
            \item \paramdesc{min}{}
            \item \paramdesc{max}{}
            \item \paramdesc{steps}{}
        \end{itemize}
    }
\end{itemize}
}{}

\classdesc{HeatMapLabel}{}{
    \begin{itemize}
        \item \atrdesc{\private}{}{Label}{label}{}
    \end{itemize}
}{
    \begin{itemize}
        \item \constrdesc{\proted}{HeatMapLabel}{\type{DiagramColorScale} cs, \type{Number} value, \type{PositionIn2DDiagram} bottomLeft, \type{PositionIn2DDiagram} topRight}{}{
            \begin{itemize}
                \item \paramdesc{cs}{}
                \item \paramdesc{value}{}
                \item \paramdesc{bottomLeft}{}
                \item \paramdesc{topRight}{}
            \end{itemize}
        }
    \end{itemize}
}{
    \begin{itemize}
        \item \ovrdnmethoddesc{\public}{}{void}{show}{}{DiagramComponent}
        {}{}{}
        \item \ovrdnmethoddesc{\public}{}{void}{hide}{}{DiagramComponent}
        {}{}{}
        \item \ovrdnmethoddesc{\public}{}{DiagramComponent}{clone}{}{DiagramComponent}
        {}{}{}
    \end{itemize}
}

\classdesc{DescriptionLabel}{}{
    \begin{itemize}
        \item \atrdesc{\private}{}{Label}{label}{}
    \end{itemize}
}{
    \begin{itemize}
        \item \constrdesc{\proted}{DescriptionLabel}{\type{Color} color, \type{String} caption, \type{PositionInFrame} bottomLeft, \type{PositionInFrame} topRight}{}{
            \begin{itemize}
                \item \paramdesc{color}{}
                \item \paramdesc{caption}{}
                \item \paramdesc{bottomLeft}{}
                \item \paramdesc{topRight}{}
            \end{itemize}
        }
    \end{itemize}
}{
    \begin{itemize}
        \item \ovrdnmethoddesc{\public}{}{void}{show}{}{DiagramComponent}
        {}{}{}
        \item \ovrdnmethoddesc{\public}{}{void}{hide}{}{DiagramComponent}
        {}{}{}
        \item \ovrdnmethoddesc{\public}{}{DiagramComponent}{clone}{}{DiagramComponent}
        {}{}{}
    \end{itemize}
}

\classdesc{HoverLabel}{}{
    \begin{itemize}
        \item \atrdesc{\private}{}{String}{caption}{}
        \item \atrdesc{\private}{}{Theme}{theme}{}
        \item \atrdesc{\private}{}{Number}{xPos}{}
        \item \atrdesc{\private}{}{Number}{yPos}{}
        \item \atrdesc{\private}{}{Number}{width}{}
        \item \atrdesc{\private}{}{Number}{height}{}
        \item \atrdesc{\private}{\static}{HoverLabel}{hoverLabel}{}
    \end{itemize}
}{
    \begin{itemize}
        \item \constrdesc{\private}{HoverLabel}{}{}{}
    \end{itemize}
}{
    \begin{itemize}
        \item \methoddesc{\public}{\static}{HoverLabel}{getHoverLabel}{}
        {}{}{}
        \item \methoddesc{\public}{}{void}{show}{}
        {}{}{}
        \item \methoddesc{\public}{}{void}{hide}{}
        {}{}{}
        \item \methoddesc{\public}{}{void}{setWidth}{\type{Number} width}
        {}{
            \begin{itemize}
                \item \paramdesc{width}{}
            \end{itemize}
        }{}
        \item \methoddesc{\public}{}{Number}{getWidth}{}
        {}{}{}
        \item \methoddesc{\public}{}{void}{setHeight}{\type{Number} height}
        {}{
            \begin{itemize}
                \item \paramdesc{height}{}
            \end{itemize}
        }{}
        \item \methoddesc{\public}{}{Number}{getHeight}{}
        {}{}{}
    \end{itemize}
}

\classdesc{DiagramComponentFactory}{}{
    \begin{itemize}
        \item \atrdesc{\private}{\static}{DiagramComponentFactory}{instance}{}
    \end{itemize}
}{
    \begin{itemize}
        \item \constrdesc{\private}{DiagramComponentFactory}{}{}{}
    \end{itemize}
}{
    \begin{itemize}
        \item \methoddesc{\public}{}{DiagramComponentFactory}{getDiagramComponentFactory}{}
        {}{}{}
        \item \methoddesc{\public}{}{DiagramPoint}{createPoint}{\type{Number} value, \type{PositionIn2DDiagram} position, \type{Number} size}
        {}{
            \begin{itemize}
                \item \paramdesc{value}{}
                \item \paramdesc{position}{}
                \item \paramdesc{size}{}
            \end{itemize}
        }{}
        \item \methoddesc{\public}{}{DiagramValueLabel}{createValueLabel}{\type{Number} value, \type{PositionIn2DDiagram} bottomLeft, \type{PositionIn2DDiagram} topRight, \type{Number} borderThickness}
        {}{
            \begin{itemize}
                \item \paramdesc{value}{}
                \item \paramdesc{bottomLeft}{}
                \item \paramdesc{topRight}{}
                \item \paramdesc{borderThickness}{}
            \end{itemize}
        }{}
        \item \methoddesc{\public}{}{DiagramBar}{createBar}{\type{Number} value, \type{PositionIn2DDiagram} bottomLeft, \type{PositionIn2DDiagram} topRight, \type{Number} borderThickness}
        {}{
            \begin{itemize}
                \item \paramdesc{value}{}
                \item \paramdesc{bottomLeft}{}
                \item \paramdesc{topRight}{}
                \item \paramdesc{borderThickness}{}
            \end{itemize}
        }{}
        \item \methoddesc{\public}{}{DiagramLabel}{createLabel}{\type{PositionInFrame} bottomLeft, \type{PositionInFrame} topRight, \type{Color} color, \type{String} caption, \type{Number} borderThickness}
        {}{
            \begin{itemize}
                \item \paramdesc{bottomLeft}{}
                \item \paramdesc{topRight}{}
                \item \paramdesc{color}{}
                \item \paramdesc{caption}{}
                \item \paramdesc{borderThickness}{}
            \end{itemize}
        }{}
        \item \methoddesc{\public}{}{DiagramAxis}{createAxis}{\type{DiagramLine} axisLine, \type{Number} min, \type{Number} max, \type{int} steps}
        {}{
            \begin{itemize}
                \item \paramdesc{axisLine}{}
                \item \paramdesc{min}{}
                \item \paramdesc{max}{}
                \item \paramdesc{steps}{}
            \end{itemize}
        }{}
        \item \methoddesc{\public}{}{DiagramColorScale}{createColorScale}{\type{PositionInFrame} bottomLeft, \type{PositionInFrame} topRight, \type{Color} borderColor, \type{Number[]} values, \type{Color[]} valueColors, \type{Number} borderThickness}
        {}{
            \begin{itemize}
                \item \paramdesc{bottomLeft}{}
                \item \paramdesc{topRight}{}
                \item \paramdesc{borderColor}{}
                \item \paramdesc{values}{}
                \item \paramdesc{valueColors}{}
                \item \paramdesc{borderThickness}{}
            \end{itemize}
        }{}
        \item \methoddesc{\public}{}{DiagramLine}{createLine}{\type{PositionInFrame} start, \type{PositionInFrame} end, \type{Color} color, \type{Number} thickness}
        {}{
            \begin{itemize}
                \item \paramdesc{start}{}
                \item \paramdesc{end}{}
                \item \paramdesc{color}{}
                \item \paramdesc{thickness}{}
            \end{itemize}
        }{}
    \end{itemize}
}
}
\packagedesc{view.diagrams.data}{

\absclassdesc{DiagramDataFormatter}{}{}{
    \begin{itemize}
        \item \constrdesc{\public}{DiagramDataFormatter}{}{}{}
    \end{itemize}
}{
    \begin{itemize}
        \item \subclsdec{ArrayListDataFormatter}
        \item \subclsdec{ArrayDataFormatter}
    \end{itemize}
}{
    \begin{itemize}
        \item \methoddesc{\public}{<\generic{T} \extends \typewgeneric{Collection}{?} >}{Object}{format}{\generic{T} data}
        {}{
            \begin{itemize}
                \item \paramdesc{data}{}
            \end{itemize}
        }{}
    \end{itemize}
}

\classdesc{DiagramData}{}{
    \begin{itemize}
        \item \atrdesc{\private}{\typewgeneric{Collection}{?}}{}{data}{}
        \item \atrdesc{\private}{}{DiagramDataFormatter}{ddf}{}
    \end{itemize}
}{
    \begin{itemize}
        \item \constrdesc{\public}{DiagramData}{\typewgeneric{Collection}{?} data}{}{
            \begin{itemize}
                \item \paramdesc{data}{}
            \end{itemize}
        }
    \end{itemize}
}{
    \begin{itemize}
        \item \methoddesc{\public}{<\generic{T} \extends \typewgeneric{Collection}{?} >}{\generic{T}}{getData}{}
        {}{}{}
        \item \methoddesc{\public}{}{void}{update}{}
        {}{}{}
        \item \methoddesc{\public}{}{void}{setFormat}{\type{DiagramDataFormatter} ddf}
        {}{
            \begin{itemize}
                \item \paramdesc{ddf}{}
            \end{itemize}
        }{}
    \end{itemize}
}

\classdesc{ArrayListDataFormatter}{}{}{
    \begin{itemize}
        \item \constrdesc{\public}{ArrayListDataFormatter}{}{}{}
    \end{itemize}
}{
    \begin{itemize}
        \item \ovrdnmethoddesc{\public}{<\generic{T} \extends \typewgeneric{Collection}{?} > }{Object}{format}{\generic{T} data}
        {DiagramDataFormatter}{}{
            \begin{itemize}
                \item \paramdesc{data}{}
            \end{itemize}
        }{}
    \end{itemize}
}

\classdesc{ArrayDataFormatter}{}{}{
    \begin{itemize}
        \item \constrdesc{\public}{ArrayDataFormatter}{}{}{}
    \end{itemize}
}{
    \begin{itemize}
        \item \ovrdnmethoddesc{\public}{<\generic{E}, \generic{T} \extends \typewgeneric{Collection}{E} > }{Object}{format}{\generic{T} data}
        {DiagramDataFormatter}{}{
            \begin{itemize}
                \item \paramdesc{data}{}
            \end{itemize}
        }{}
    \end{itemize}
}
}
\packagedesc{view.diagrams.type}{

\absclassdesc{Diagram}{}{
    \begin{itemize}
        \item \atrdesc{\private}{}{DiagramData}{data}{}
        \item \atrdesc{\private}{}{DiagramAxis[]}{axes}{}
        \item \atrdesc{\private}{}{DiagramValueDisplayComponent[]}{valueDisplayComponents}{}
        \item \atrdesc{\private}{}{DiagramComponent[]}{nonValueDisplayComponents}{}
        \item \atrdesc{\private}{\typewgeneric{EnumMap}{IndicatorIdentifier, DiagramViewHelper}}{}{viewHelpers}{}
    \end{itemize}
}{
    \begin{itemize}
        \item \constrdesc{\public}{Diagram}{\type{DiagramData} data}{}{
            \begin{itemize}
                \item \paramdesc{data}{}
            \end{itemize}
        }
    \end{itemize}
}{
    \begin{itemize}
        \item \subclsdec{Histogram}
        \item \subclsdec{BarChart}
        \item \subclsdec{HeatMap}
        \item \subclsdec{FunctionGraph}
    \end{itemize}
}{
    \begin{itemize}
        \item \methoddesc{\proted}{}{void}{assemble}{\type{DiagramAxis[]} axes, \type{DiagramValueDisplayComponent[]} valueDisplayComponents, \type{DiagramComponent[]} nonValueDisplayComponents}
        {}{
            \begin{itemize}
                \item \paramdesc{axes}{}
                \item \paramdesc{valueDisplayComponents}{}
                \item \paramdesc{nonValueDisplayComponents}{}
            \end{itemize}
        }{}
        \item \methoddesc{\public}{}{boolean}{addDiagramViewHelper}{\type{DiagramViewHelper} dvh}
        {}{
            \begin{itemize}
                \item \paramdesc{dvh}{}
            \end{itemize}
        }{}
        \item \methoddesc{\public}{}{boolean}{removeDiagramViewHelper}{\type{IndicatorIdentifier} id}
        {}{
            \begin{itemize}
                \item \paramdesc{id}{}
            \end{itemize}
        }{}
        \item \methoddesc{\public}{}{boolean}{showDiagramViewHelper}{\type{IndicatorIdentifier} id}
        {}{
            \begin{itemize}
                \item \paramdesc{id}{}
            \end{itemize}
        }{}
        \item \methoddesc{\public}{}{boolean}{hideDiagramViewHelper}{\type{IndicatorIdentifier} id}
        {}{
            \begin{itemize}
                \item \paramdesc{id}{}
            \end{itemize}
        }{}
    \end{itemize}
}

\classdesc{Histogram}{}{}{
    \begin{itemize}
        \item \constrdesc{\public}{Histogram}{\type{DiagramData} data}{}{
            \begin{itemize}
                \item \paramdesc{data}{}
            \end{itemize}
        }
    \end{itemize}
}{}

\classdesc{BarChart}{}{}{
    \begin{itemize}
        \item \constrdesc{\public}{BarChart}{\type{DiagramData} data}{}{
            \begin{itemize}
                \item \paramdesc{data}{}
            \end{itemize}
        }
    \end{itemize}
}{}

\classdesc{HeatMap}{}{}{
    \begin{itemize}
        \item \constrdesc{\public}{HeatMap}{\type{DiagramData} data}{}{
            \begin{itemize}
                \item \paramdesc{data}{}
            \end{itemize}
        }
    \end{itemize}
}{}

\classdesc{FunctionGraph}{}{}{
    \begin{itemize}
        \item \constrdesc{\public}{FunctionGraph}{\type{DiagramData} data}{}{
            \begin{itemize}
                \item \paramdesc{data}{}
            \end{itemize}
        }
    \end{itemize}
}{}
}
\packagedesc{view.diagrams.indicator}{

\absclassdesc{DiagramViewHelper}{}{
    \begin{itemize}
        \item \atrdesc{\private}{}{int}{layer}{}
        \item \atrdesc{\private}{\typewgeneric{List}{ViewHelperComponent}}{}{helperComponents}{}
        \item \atrdesc{\private}{}{IndicatorIdentifier}{id}{}
        \item \atrdesc{\private}{}{IDiagram}{diagram}{}
    \end{itemize}
}{
    \begin{itemize}
        \item \constrdesc{\public}{DiagramViewHelper}{\type{IDiagram} diagram, \type{int} layer, \type{IndicatorIdentifier} id}{}{
            \begin{itemize}
                \item \paramdesc{diagram}{}
                \item \paramdesc{layer}{}
                \item \paramdesc{id}{}
            \end{itemize}
        }
    \end{itemize}
}{
    \begin{itemize}
        \item \subclsdec{HelperLineDisplayer}
        \item \subclsdec{HelperComponentDisplayer}
    \end{itemize}
}{
    \begin{itemize}
        \item \methoddesc{\public}{}{int}{getLayerNumber}{}
        {}{}{}
        \item \methoddesc{\public}{}{void}{remove}{}
        {}{}{}
        \item \methoddesc{\public}{}{void}{show}{}
        {}{}{}
        \item \methoddesc{\public}{}{void}{hide}{}
        {}{}{}
        \item \methoddesc{\public}{}{void}{update}{}
        {}{}{}
        \item \methoddesc{\public}{}{boolean}{addViewHelperComponent}{\type{ViewHelperComponent} vhc}
        {}{
            \begin{itemize}
                \item \paramdesc{vhc}{}
            \end{itemize}
        }{}
        \item \methoddesc{\public}{}{boolean}{removeViewHelperComponent}{\type{ViewHelperComponent} vhc}
        {}{
            \begin{itemize}
                \item \paramdesc{vhc}{}
            \end{itemize}
        }{}
        \item \methoddesc{\public}{}{boolean}{clearViewHelperComponents}{}
        {}{}{}
        \item \methoddesc{\public}{}{IndicatorIdentifier}{getID}{}
        {}{}{}
    \end{itemize}
}

\absclassdesc{HelperLineDisplayer}{}{}{
    \begin{itemize}
        \item \constrdesc{\public}{HelperLineDisplayer}{\type{IDiagram} diagram, \type{IndicatorIdentifier} id}{}{
            \begin{itemize}
                \item \paramdesc{diagram}{}
                \item \paramdesc{id}{}
            \end{itemize}
        }
    \end{itemize}
}{
    \begin{itemize}
        \item \subclsdec{ValueLineDisplayer}
        \item \subclsdec{CoordinateIndicatorLineDisplayer}
    \end{itemize}
}{
    \begin{itemize}
        \item \methoddesc{\proted}{\abstr}{void}{generateHelperComponents}{}
        {}{}{}
    \end{itemize}
}

\absclassdesc{HelperComponentDisplayer}{}{}{
    \begin{itemize}
        \item \constrdesc{\public}{HelperComponentDisplayer}{\type{IDiagram} diagram, \type{IndicatorIdentifier} id}{}{
            \begin{itemize}
                \item \paramdesc{diagram}{}
                \item \paramdesc{id}{}
            \end{itemize}
        }
    \end{itemize}
}{
    \begin{itemize}
        \item \subclsdec{ValueFixColorDisplayer}
        \item \subclsdec{ValueScaleColorDisplayer}
    \end{itemize}
}{}

\enumdesc{IndicatorIdentifier}{}{
    \begin{itemize}
        \item \fielddesc{IndicatorIdentifier}{MIN}{}
        \item \fielddesc{IndicatorIdentifier}{MAX}{}
        \item \fielddesc{IndicatorIdentifier}{AVG}{}
        \item \fielddesc{IndicatorIdentifier}{MED}{}
        \item \fielddesc{IndicatorIdentifier}{X-COORDINATE-INDICATOR}{}
        \item \fielddesc{IndicatorIdentifier}{Y-COORDINATE-INDICATOR}{}
    \end{itemize}
}{}

\classdesc{DiagramViewHelperFactory}{}{
    \begin{itemize}
        \item \atrdesc{\private}{\static}{DiagramViewHelperFactory}{instance}{}
    \end{itemize}
}{
    \begin{itemize}
        \item \constrdesc{\private}{DiagramViewHelperFactory}{}{}{}
    \end{itemize}
}{
    \begin{itemize}
        \item \methoddesc{\public}{}{HelperComponentDisplayer}{createValueColorDisplayer}{\type{IDiagram} diagram, \type{IndicatorIdentifier} id}
        {}{
            \begin{itemize}
                \item \paramdesc{diagram}{}
                \item \paramdesc{id}{}
            \end{itemize}
        }{}
        \item \methoddesc{\public}{}{HelperLineDisplayer}{createCoordinateGridDisplayer}{\type{IDiagram} diagram, \type{DiagramAxis} axis, \type{IndicatorIdentifier} id}
        {}{
            \begin{itemize}
                \item \paramdesc{diagram}{}
                \item \paramdesc{axis}{}
                \item \paramdesc{id}{}
            \end{itemize}
        }{}
        \item \methoddesc{\public}{}{HelperLineDisplayer}{createValueLineDisplayer}{\type{IDiagram} diagram, \type{Number} value, \type{IndicatorIdentifier} id}
        {}{
            \begin{itemize}
                \item \paramdesc{diagram}{}
                \item \paramdesc{value}{}
                \item \paramdesc{id}{}
            \end{itemize}
        }{}
    \end{itemize}
}

\classdesc{ValueLineDisplayer}{}{
    \begin{itemize}
        \item \atrdesc{\private}{}{DiagramAxis}{parallelAxis}{}
        \item \atrdesc{\private}{}{Color}{color}{}
        \item \atrdesc{\private}{}{Number}{thickness}{}
        \item \atrdesc{\private}{}{Number}{value}{}
    \end{itemize}
}{
    \begin{itemize}
        \item \constrdesc{\proted}{ValueLineDisplayer}{\type{IDiagram} diagram, \type{DiagramAxis} parallelAxis, \type{Color} color, \type{Number} thickness, \type{Number} value, \type{IndicatorIdentifier} id}{}{
            \begin{itemize}
                \item \paramdesc{diagram}{}
                \item \paramdesc{parallelAxis}{}
                \item \paramdesc{color}{}
                \item \paramdesc{thickness}{}
                \item \paramdesc{value}{}
                \item \paramdesc{id}{}
            \end{itemize}
        }
    \end{itemize}
}{
    \begin{itemize}
        \item \methoddesc{\private}{}{void}{createValueLine}{}
        {}{}{}
        \item \ovrdnmethoddesc{\proted}{}{void}{generateHelperComponents}{}{HelperLineDisplayer}
        {}{}{}
    \end{itemize}
}

\classdesc{ValueFixColorDisplayer}{}{
    \begin{itemize}
        \item \atrdesc{\private}{\typewgeneric{TreeMap}{Number, Color}}{}{mapping}{}
    \end{itemize}
}{
    \begin{itemize}
        \item \constrdesc{\proted}{ValueFixColorDisplayer}{\type{IDiagram} diagram, \typewgeneric{TreeMap}{Number, Color} mapping, \type{IndicatorIdentifier} id}{}{
            \begin{itemize}
                \item \paramdesc{diagram}{}
                \item \paramdesc{mapping}{}
                \item \paramdesc{id}{}
            \end{itemize}
        }
    \end{itemize}
}{}

\classdesc{ValueScaleColorDisplayer}{}{
    \begin{itemize}
        \item \atrdesc{\private}{}{DiagramColorScale}{colorScale}{}
    \end{itemize}
}{
    \begin{itemize}
        \item \constrdesc{\proted}{ValueScaleColorDisplayer}{\type{IDiagram} diagram, \type{DiagramColorScale} colorScale, \type{IndicatorIdentifier} id}{}{
            \begin{itemize}
                \item \paramdesc{diagram}{}
                \item \paramdesc{colorScale}{}
                \item \paramdesc{id}{}
            \end{itemize}
        }
    \end{itemize}
}{}

\classdesc{CoordinateIndicatorLineDisplayer}{}{
    \begin{itemize}
        \item \atrdesc{\private}{}{DiagramAxis}{axis}{}
        \item \atrdesc{\private}{}{Color}{color}{}
        \item \atrdesc{\private}{}{Number}{thickness}{}
    \end{itemize}
}{
    \begin{itemize}
        \item \constrdesc{\proted}{CoordinateIndicatorLineDisplayer}{\type{IDiagram} diagram, \type{DiagramAxis} axis, \type{Color} color, \type{Number} thickness, \type{IndicatorIdentifier} id}{}{
            \begin{itemize}
                \item \paramdesc{diagram}{}
                \item \paramdesc{axis}{}
                \item \paramdesc{color}{}
                \item \paramdesc{thickness}{}
                \item \paramdesc{id}{}
            \end{itemize}
        }
    \end{itemize}
}{
    \begin{itemize}
        \item \methoddesc{\private}{}{void}{createCoordinateIndicatorLines}{}
        {}{}{}
        \item \ovrdnmethoddesc{\proted}{}{void}{generateHelperComponents}{}{HelperLineDisplayer}
        {}{}{}
    \end{itemize}
}

\classdesc{ViewHelperComponent}{}{}{
    \begin{itemize}
        \item \constrdesc{\proted}{ViewHelperComponent}{\type{DiagramComponent} dc}{}{
            \begin{itemize}
                \item \paramdesc{dc}{}
            \end{itemize}
        }
    \end{itemize}
}{
    \begin{itemize}
        \item \methoddesc{\public}{}{void}{show}{}
        {}{}{}
        \item \methoddesc{\public}{}{void}{hide}{}
        {}{}{}
    \end{itemize}
}

\classdesc{CoordinateIndicatorLine}{}{}{
    \begin{itemize}
        \item \constrdesc{\proted}{CoordinateIndicatorLine}{\type{DiagramAxis} parallelAxis, \type{Number} value, \type{Color} color, \type{Number} thickness}{}{
            \begin{itemize}
                \item \paramdesc{parallelAxis}{}
                \item \paramdesc{value}{}
                \item \paramdesc{color}{}
                \item \paramdesc{thickness}{}
            \end{itemize}
        }
    \end{itemize}
}{}

\classdesc{ValueLine}{}{}{
    \begin{itemize}
        \item \constrdesc{\proted}{ValueLine}{\type{DiagramAxis} parallelAxis, \type{Number} value, \type{Color} color, \type{Number} thickness}{}{
            \begin{itemize}
                \item \paramdesc{parallelAxis}{}
                \item \paramdesc{value}{}
                \item \paramdesc{color}{}
                \item \paramdesc{thickness}{}
            \end{itemize}
        }
    \end{itemize}
}{}
}
\packagedesc{view.diagrams.builder}{

\absclassdesc{DiagramBuilder}{}{
    \begin{itemize}
        \item \atrdesc{\private}{}{DiagramData}{data}{}
    \end{itemize}
}{
    \begin{itemize}
        \item \constrdesc{\public}{DiagramBuilder}{\type{DiagramData} data}{}{
            \begin{itemize}
                \item \paramdesc{data}{}
            \end{itemize}
        }
    \end{itemize}
}{
    \begin{itemize}
        \item \subclsdec{BarChartBuilder}
        \item \subclsdec{HistogramBuilder}
        \item \subclsdec{FunctionGraphBuilder}
        \item \subclsdec{HeatMapBuilder}
    \end{itemize}
}{
    \begin{itemize}
        \item \methoddesc{\proted}{}{DiagramAxis}{buildAxes}{}
        {}{}{}
        \item \methoddesc{\proted}{}{DiagramValueDisplayComponent[]}{buildValueDisplayComponents}{}
        {}{}{}
        \item \methoddesc{\proted}{}{DiagramComponent[]}{buildDiagramSpecificComponent}{}
        {}{}{}
        \item \methoddesc{\public}{}{IDiagram}{buildDiagram}{}
        {}{}{}
    \end{itemize}
}

\classdesc{BarChartBuilder}{}{}{
    \begin{itemize}
        \item \constrdesc{\public}{BarChartBuilder}{\type{DiagramData} data}{}{
            \begin{itemize}
                \item \paramdesc{data}{}
            \end{itemize}
        }
    \end{itemize}
}{}

\classdesc{HistogramBuilder}{}{}{
    \begin{itemize}
        \item \constrdesc{\public}{HistogramBuilder}{\type{DiagramData} data}{}{
            \begin{itemize}
                \item \paramdesc{data}{}
            \end{itemize}
        }
    \end{itemize}
}{}

\classdesc{FunctionGraphBuilder}{}{}{
    \begin{itemize}
        \item \constrdesc{\public}{FunctionGraphBuilder}{\type{DiagramData} data}{}{
            \begin{itemize}
                \item \paramdesc{data}{}
            \end{itemize}
        }
    \end{itemize}
}{}
\classdesc{HeatMapBuilder}{}{}{
    \begin{itemize}
        \item \constrdesc{\public}{HeatMapBuilder}{\type{DiagramData} data}{}{
            \begin{itemize}
                \item \paramdesc{data}{}
            \end{itemize}
        }
    \end{itemize}
}{}
}

\packagedesc{view.representation}{
\interfacedesc{ICellImageGenerator}{\interface ICellImageGenerator}{
}{CellImageGenerator}{
    \begin{itemize}
        \item \methoddesc{\public}{}{void}{buildCell}{\type{int} inputPins, \type{int} outputPins}
        {Builds a cell representation image with the given amount of input and output pins.
}{
        \begin{itemize}
            \item \paramdesc{inputPins}{Number of input pins of the cell}
            \item \paramdesc{outputPins}{Number of output pins of the cell.}
        \end{itemize}}{}
    \end{itemize}}

    
\classdesc{CellImageGenerator}{\proted \class CellImageGenerator implements ICellImageGenerator, import java.awt.image.BufferedImage
}{
    \begin{itemize}
        \item \atrdesc{\private}{}{BufferedImage}{pinIcon}{Image for pins of a cell}
        \item \atrdesc{\private}{}{BufferedImage}{cellIcon}{Image for a cell representation}
    \end{itemize}
}{
    \begin{itemize}
        \item \constrdesc{\public}{CellImageGenerator}{}{}{}
    \end{itemize}
}{  \begin{itemize}
        \item \methoddesc{\public}{}{void}{buildCell}{\type{int} inputPins, \type{int} outputPins}
        {Builds a cell representation image with the given amount of input and output pins.
}{
        \begin{itemize}
            \item \paramdesc{inputPins}{Number of input pins of the cell}
            \item \paramdesc{outputPins}{Number of output pins of the cell.}
        \end{itemize}}{}
    \end{itemize}}{}
\classdesc{DataPanel}{}{
    \begin{itemize}
        \item \atrdesc{\private}{}{Label}{label}{Label for the data text.}
        \item \atrdesc{\private}{}{String}{text}{Certain information about the opened liberty file.}
    \end{itemize}
}{
    \begin{itemize}
        \item \constrdesc{\public}{DataPanel}{}{}{}
    \end{itemize}}{
    \begin{itemize}
        \item \methoddesc{\public}{}{void}{setText}{\type{String} text}
        {Changes the text in the data panel.
}{
        \begin{itemize}
            \item \paramdesc{text}{New data to be shown on the data panel.}
        \end{itemize}
        }{}
    \end{itemize}}{}
\classdesc{CellPanel}{}{
    \begin{itemize}
        \item \atrdesc{\private}{}{Label}{label}{Label for the text on the cell panel}
        \item \atrdesc{\private}{}{Button[]}{buttons}{Array of all buttons for the pins}
        \item \atrdesc{\private}{}{Checkbox[]}{checkboxes}{Checkboxes for the used pins}
        \item \atrdesc{\private}{}{Element}{cell}{Cell opened in the visualizer.}
        \item \atrdesc{\private}{}{Element[]}{pins}{Pins of the opened cell.}
        \item \atrdesc{\private}{}{BufferedImage}{cellImage}{Cell image which is generated by the cellGenerator.}
        \item \atrdesc{\private}{}{CellImageGenerator}{cellGenerator}{Builds the cell representation image}
    \end{itemize}
}{
    \begin{itemize}
        \item \constrdesc{\public}{CellPanel}{\type{Element} element}{}{
        \begin{itemize}
            \item \paramdesc{element}{Pin/Cell to be opened in the cell panel.}
        \end{itemize}
        }
    \end{itemize}
}{
    \begin{itemize}
        \item \methoddesc{\public}{}{void}{switchToLibrary}{}
        {Switches to the parent library panel.}{}{}
        \item \methoddesc{\public}{}{void}{switchToPin}{\type{Element} element}
        {Switches the panel for a selected pin of the cell.}{
        \begin{itemize}
            \item \paramdesc{element}{Target pin}
        \end{itemize}}{}
        \item \methoddesc{\public}{}{void}{switchToCell}{\type{Element} element}
        {Switches the panel for the parent cell.}{
        \begin{itemize}
            \item \paramdesc{element}{Target cell}
        \end{itemize}
        }{}
    \end{itemize}
}{}
\classdesc{LibraryPanel}{}{
    \begin{itemize}
        \item \atrdesc{\private}{}{List<Button>}{buttons}{List of the buttons for each cell.}
        \item \atrdesc{\private}{}{List<Cell>}{cells}{List of every cell in the library}
        \item \atrdesc{\private}{}{Library}{selectedLibrary}{Library which has been opened in the visualizer.}
    \end{itemize}
}{
    \begin{itemize}
        \item \constrdesc{\public}{LibraryPanel}{\type{Library} Library}{}{}
    \end{itemize}
}{
    \begin{itemize}
        \item \methoddesc{\public}{}{void}{switchToCell}{\type{Element} element}
        {Switches from the library panel to the cell panel of the selected cell.}{
        \begin{itemize}
            \item \paramdesc{element}{Target cell.}
        \end{itemize}}{}
    \end{itemize}
}{}
}


\packagedesc{model.elements.attributes}{
\absclassdesc{Attribute}{Attribute class}
{\begin{itemize}
    \item \atrdesc{\proted}{}{\type{Stat}}{stats}{}
\end{itemize}
}{
\begin{itemize}
    \item \implclsdec{Leakage}
    \item \implclsdec{InAttribute}
    \item \implclsdec{OutAttribute}
\end{itemize}
}
{a}
{
\begin{itemize}
        \item \methoddesc{\public}{}{void}{calculate}{}{}
        {}{}
        \item \methoddesc{\private}{}{void}{scale}{\type{float} value}
        {{\begin{itemize}
        \item \paramdesc{value} {The scaling value.}
        \end{itemize}}}{}{}
        \item \methoddesc{\public}{}{\type{Stats}}{getStats}{}{}{}{
        
        Stats will be returned.}
\end{itemize}
}

\absclassdesc{InAttribute}{}
{\begin{itemize}
    \item \atrdesc{\proted}{}{\type{float[]}}{index1}{}
    \item \atrdesc{\proted}{}{\type{float[]}}{values}{}
\end{itemize}}
{\begin{itemize}
    \item \implclsdec{InputPower}
\end{itemize}}
{a}
{\begin{itemize}
        \item \methoddesc{\public}{}{void}{calculate}{}{}
        {}{}
        \item \methoddesc{\public}{}{void}{scale}{\type{float} value}
        {{\begin{itemize}
        \item \paramdesc{value} {The scaling value.}
        \end{itemize}}}{}{}
        \item \methoddesc{\public}{}{\type{float[]}}{getIndex1}{}{}{}{
        
        1-dimensional index will be returned.}
        \item \methoddesc{\public}{}{\type{float[]}}{getValues}{}{}{}{
        
        The array of values will be returned.}
        \item \methoddesc{\public}{}{void}{setIndex1}{\type{float[]} index}
        {{\begin{itemize}
        \item \paramdesc{index} {The 1-dimensional index of the input pin attributes.}
        \end{itemize}}}{}{}
        \item \methoddesc{\public}{}{void}{setValues}{\type{float[]} values}
        {{\begin{itemize}
        \item \paramdesc{values} {The array of input pin attribute values.}
        \end{itemize}}}{}{}
\end{itemize}}


\absclassdesc{OutAttribute}{}
{\begin{itemize}
    \item \atrdesc{\proted}{}{\type{float[]}}{index1}{}
    \item \atrdesc{\proted}{}{\type{float[]}}{index2}{}
    \item \atrdesc{\proted}{}{\type{float[][]}}{values}{}
    \item \atrdesc{\proted}{}{\type{InputPin}}{relatedPin}{}
\end{itemize}}
{\begin{itemize}
    \item \implclsdec{OutputPower}
    \item \implclsdec{Timing}
\end{itemize}}
{a}
{\begin{itemize}
        \item \methoddesc{\public}{}{void}{calculate}{}{}
        {}{}
        \item \methoddesc{\public}{}{void}{scale}{\type{float} value}
        {{\begin{itemize}
        \item \paramdesc{value} {The scaling value.}
        \end{itemize}}}{}{}
        \item \methoddesc{\public}{}{\type{float[]}}{getIndex1}{}{}{}{
        
        1-dimensional first index array will be returned.}
        \item \methoddesc{\public}{}{\type{float[]}}{getIndex2}{}{}{}{
        
        1-dimensional second index array will be returned.}
        \item \methoddesc{\public}{}{\type{float[][]}}{getValues}{}{}{}{
        
        The array of values will be returned.}
        \item \methoddesc{\public}{}{void}{setIndex1}{\type{float[]} index}
        {{\begin{itemize}
        \item \paramdesc{index} {The 1-dimensional first index array of the output pin attributes.}
        \end{itemize}}}{}{}
        \item \methoddesc{\public}{}{void}{setIndex2}{\type{float[]} index}
        {{\begin{itemize}
        \item \paramdesc{index} {The 1-dimensional second index array of the output pin attributes.}
        \end{itemize}}}{}{}
        \item \methoddesc{\public}{}{void}{setValues}{\type{float[][]} values}
        {{\begin{itemize}
        \item \paramdesc{values} {The array of input pin attribute values.}
        \end{itemize}}}{}{}
        \item
        \methoddesc{\public}{}{void}{setRelatedPin}{\type{InputPin} inpin}
        {{\begin{itemize}
        \item \paramdesc{inpin} {The input pin that the output pin is going to be related to.}
        \end{itemize}}}{}{}
\end{itemize}}


\classdesc{Leakage}{}
{\begin{itemize}
    \item \atrdesc{\private}{}{\type{float[]}}{values}{}
\end{itemize}}
{\begin{itemize}
        \item \constrdesc{\public}{Leakage}{\type{float[]} values}{{\begin{itemize}
        \item \paramdesc{values} {The array of leakages of a cell.}
        \end{itemize}}}{}
    \end{itemize}}
{}


\classdesc{InputPower}{}
{\begin{itemize}
    \item \atrdesc{\private}{}{\type{PowerGroup}}{powgroup}{}
\end{itemize}}
{{\begin{itemize}
        \item \constrdesc{\public}{InputPower}{{\type{PowerGroup} powgroup, \type{float[]} values}}
        {{\begin{itemize}
        \item \paramdesc{powgroup} {The power group of the values.}
        \item \paramdesc{values} {The array of power values of an input pin.}
        \end{itemize}}}{}
    \end{itemize}}}{}
    
    
\classdesc{OutputPower}{}
{\begin{itemize}
    \item \atrdesc{\private}{}{\type{PowerGroup}}{powgroup}{}
\end{itemize}}
{\begin{itemize}
        \item \constrdesc{\public}{OutputPower}{{\type{PowerGroup} powgroup, \type{float[][]} values}}
        {{\begin{itemize}
        \item \paramdesc{powgroup} {The power group of the values.}
        \item \paramdesc{values} {The 2 dimensional array of power values of an output pin.}
        \end{itemize}}}{}
    \end{itemize}}
{}


\classdesc{Timing}{}
{\begin{itemize}
    \item \atrdesc{\private}{}{\type{TimingSense}}{timsense}{}
    \item \atrdesc{\private}{}{\type{TimingType}}{timtype}{}
    \item \atrdesc{\private}{}{\type{TimingGroup}}{timgroup}{}
\end{itemize}}
{\begin{itemize}
        \item \constrdesc{\public}{Timing}{{\type{TimingSense} timsense, \type{TimingType} timtype, \type{TimingGroup} timgroup, \type{InputPin} relatedPin, \type{float[][]} values}}
        {{\begin{itemize}
        \item \paramdesc{timsense} {The timing sense of the values.}
        \item \paramdesc{timtype} {The timing type of the values.}
        \item \paramdesc{timgroup} {The timing group of the values.}
        \item \paramdesc{relatedPin} {The input pin that is related to the output pin for the values.}
        \item \paramdesc{values} {The 2 dimensional array of the values.}
        \end{itemize}}}{}
    \end{itemize}}
{}

\enumdesc{PowerGroup}
{
Keeps track of the power group of the input or output pin.
}{
    \begin{itemize}
        \item \fielddesc{PowerGroup}{RISEPOWER}{RisePower: RisePower of the input or output pin.}
        \item \fielddesc{PowerGroup}{FALLPOWER}{FallPower: FallPower of the input or output pin.}
        \item \fielddesc{PowerGroup}{POWER}{Power: Default power of the input or output pin.}
    \end{itemize}
}{}



\enumdesc{TimingGroup}
{
Keeps track of the timing group of the output pin.
}{
    \begin{itemize}
        \item \fielddesc{TimingGroup}{CELLRISE}{}
        \item
        \fielddesc{TimingGroup}{CELLFALL}{}
        \item \fielddesc{TimingGroup}{FALLTRANSITION}{}
        \item \fielddesc{TimingGroup}{RISETRANSITION}{}
    \end{itemize}
}{}



\enumdesc{TimingType}
{
Keeps track of the timing type of the output pin.
}{
    \begin{itemize}
        \item \fielddesc{TimingType}{COMBINATIONAL}{}
        \item \fielddesc{TimingType}{COMBRISE}{}
        \item \fielddesc{TimingType}{COMBFALL}{}
        \item \fielddesc{TimingType}{TSDISABLE}{}
        \item \fielddesc{TimingType}{TSENABLE}{}
        \item \fielddesc{TimingType}{TSDISABLERISE}{}
        \item \fielddesc{TimingType}{TSDISABLEFALL}{}
        \item \fielddesc{TimingType}{TSENABLERISE}{}
        \item \fielddesc{TimingType}{TSENABLEFALL}{}
    \end{itemize}
}{}
\enumdesc{TimingSense}
{
Keeps track of the timing sense of the output pin.
}{
    \begin{itemize}
        \item \fielddesc{TimingSense}{POSITIVE}{}
        \item
        \fielddesc{TimingSense}{NEGATIVE}{}
        \item \fielddesc{TimingSense}{NON}{}
    \end{itemize}
}{}
}


\packagedesc{model.elements}{

\absclassdesc{Element}{}
{\begin{itemize}
    \item \atrdesc{\proted}{}{\type{boolean}}{filtered}{}
    \item \atrdesc{\proted}{}{\type{boolean}}{searched}{}
    \item \atrdesc{\proted}{}{\type{String}}{name}{}
\end{itemize}}
{\begin{itemize}
    \item \implclsdec{HigherElement}
    \item \implclsdec{Pin}
\end{itemize}}
{a}
{\begin{itemize}
        \item \methoddesc{\public}{}{void}{calculate}{}{}
        {}{}
        \item \methoddesc{\public}{}{\type{int}}{compare}{\type{Element} firstEl, \type{Element} secondEl}{}{}{
        
        An integer will be returned.}
\end{itemize}
}
{}



\absclassdesc{HigherElement}{}
{\begin{itemize}
    \item \atrdesc{\proted}{}{\type{ArrayList<\type{TimingSense}>}}{availableTimSen}{}
    \item \atrdesc{\proted}{}{\type{boolean}}{searched}{}
    \item \atrdesc{\proted}{}{\type{String}}{name}{}
\end{itemize}}
{\begin{itemize}
    \item \implclsdec{HigherElement}
    \item \implclsdec{Pin}
\end{itemize}}
{a}
{}{}


\classdesc{Library}{}{}{}{}
\classdesc{Cell}{}{}{}{}
\absclassdesc{Pin}{}{}{}{}{}
\classdesc{InputPin}{}{}{}{}
\classdesc{OutputPin}{}{}{}{}
\classdesc{Stat}{}{}{}{}
}



\packagedesc{model.commands}{
\interfacedesc{Command}
    {
    An interface that is implemented by all commands.
    }
    {}
    {
    \begin{itemize}
        \item \implclsdec{OpenFileAction}
        \item \implclsdec{RemoveAction}
        \item \implclsdec{AddFilterAction}
        \item \implclsdec{RemoveFilterAction}
        \item \implclsdec{DeleteAction}
        \item \implclsdec{MergeAction}
        \item \implclsdec{MoveAction}
        \item \implclsdec{RenameAction}
        \item \implclsdec{SelectAction}
        \item \implclsdec{CreateLibraryAction}
        \item \implclsdec{ScaleAction}
        \item \implclsdec{UndoAction}
        \item \implclsdec{TextEditAction}
    \end{itemize}
    }
    {
    \begin{itemize}
        \item \methoddesc{\public}{}{void}{execute}{}
        {}{}{}{}
        \item \methoddesc{\public}{}{void}{undo}{}
        {}{}{}
    \end{itemize}
    }
\classdesc{OpenFileAction}{Opens a file.} {   \begin{itemize}
        \item \atrdesc{\private}{}{\type{Library}}{openedLibrary}{}
    \end{itemize}
}
{
    {\begin{itemize}
        \item \constrdesc{\public}{OpenFileAction}{}{}{}
    \end{itemize}
    }}
    
    {\begin{itemize}
        \item \methoddesc{\public}{}{void}{execute}{}
        {}{}{}
        \item \methoddesc{\public}{}{void}{undo}{}
        {}{}{}
    \end{itemize}
    }

\classdesc{RemoveAction}{Removes a library from the view.}
    {\begin{itemize}
        \item \atrdesc{\private}{}{\type{Library}}{removedLibrary}{}
    \end{itemize}} 
{   \begin{itemize}
        \item \constrdesc{\public}{RemoveAction}{\type{Library} library}{}
        {
            \begin{itemize}
            \item \paramdesc{library}{The library that is going to be removed from the view.}
            \end{itemize}
        }
    \end{itemize}
}
    {\begin{itemize}
        \item \methoddesc{\public}{}{void}{execute}{}
        {}{}{}
        \item \methoddesc{\public}{}{void}{undo}{}
        {}{}{}
    \end{itemize}
    }
    
    
\classdesc{AddFilterAction}{Adds a filter.}
    {\begin{itemize}
        \item \atrdesc{\private}{}{\type{Filter}}{addedFilter}{}
    \end{itemize}} 
{   \begin{itemize}
        \item \constrdesc{\public}{AddFilterAction}{\type{Filter} filter}{}
        {
            \begin{itemize}
            \item \paramdesc{filter}{The filter that is going to be added.}
            \end{itemize}
        }
    \end{itemize}
}
    {\begin{itemize}
        \item \methoddesc{\public}{}{void}{execute}{}
        {}{}{}
        \item \methoddesc{\public}{}{void}{undo}{}
        {}{}{}
    \end{itemize}
    }



\classdesc{RemoveFilterAction}{Removes a filter.}
    {\begin{itemize}
        \item \atrdesc{\private}{}{\type{Filter}}{removedFilter}{}
    \end{itemize}} 
{   \begin{itemize}
        \item \constrdesc{\public}{RemoveFilterAction}{\type{Filter} filter}{}
        {
            \begin{itemize}
            \item \paramdesc{filter}{The filter that is going to be removed.}
            \end{itemize}
        }
    \end{itemize}
}
    {\begin{itemize}
        \item \methoddesc{\public}{}{void}{execute}{}
        {}{}{}
        \item \methoddesc{\public}{}{void}{undo}{}
        {}{}{}
    \end{itemize}
    }


\classdesc{DeleteAction}{Deletes a cell from a library.}
    {\begin{itemize}
        \item \atrdesc{\private}{}{\type{Cell}}{deletedCell}{}
    \end{itemize}} 
{   \begin{itemize}
        \item \constrdesc{\public}{DeleteAction}{\type{Cell} cell}{}
        {
            \begin{itemize}
            \item \paramdesc{cell}{The cell that is going to be deleted.}
            \end{itemize}
        }
    \end{itemize}
}
    {\begin{itemize}
        \item \methoddesc{\public}{}{void}{execute}{}
        {}{}{}
        \item \methoddesc{\public}{}{void}{undo}{}
        {}{}{}
    \end{itemize}
    }


\classdesc{MergeAction}{Merges multiple libraries to a single library.}
    {\begin{itemize}
        \item \atrdesc{\private}{}{\type{Library[]}}{mergedLibraries}{}
        \item
        \atrdesc{\private}{}{\type{Library}}{productLibrary}{}
    \end{itemize}} 
{   \begin{itemize}
        \item \constrdesc{\public}{MergeAction}{\type{Library[]} libraries}{}
        {
            \begin{itemize}
            \item \paramdesc{libraries}{Array of libraries that are going to be merged.}
            \end{itemize}
        }
    \end{itemize}
}
    {\begin{itemize}
        \item \methoddesc{\public}{}{void}{execute}{}
        {}{}{}
        \item \methoddesc{\public}{}{void}{undo}{}
        {}{}{}
    \end{itemize}
    }


\classdesc{MoveAction}{Moves selected cells to a desired library.}
    {\begin{itemize}
        \item \atrdesc{\private}{}{\type{Map<\type{Cell}, \type{Library}>}}{initialPositions}{}
        \item
        \atrdesc{\private}{}{Library}{destinationLibrary}{}
    \end{itemize}} 
{   \begin{itemize}
        \item \constrdesc{\public}{MoveAction}{\type{Cell[]} cells, \type{Library} library}{}
        {
            \begin{itemize}
            \item \paramdesc{cells}{Array of the selected cells that are going to be moved.}
            \item \paramdesc{library}{The library that the cells are going to be moved to.}
            \end{itemize}
        }
    \end{itemize}
}
    {\begin{itemize}
        \item \methoddesc{\public}{}{void}{execute}{}
        {}{}{}
        \item \methoddesc{\public}{}{void}{undo}{}
        {}{}{}
    \end{itemize}
    }

\classdesc{RenameAction}{Changes the name of an element.}
    {\begin{itemize}
        \item \atrdesc{\private}{}{\type{Element}} {renamedElement}{}
        \item \atrdesc{\private}{}{\type{String}} {oldName}{}
        \item
        \atrdesc{\private}{}{\type{String}}{newName}{}
    \end{itemize}} 
{   \begin{itemize}
        \item \constrdesc{\public}{RenameAction}{\type{Element} element, \type{String} name}{}
        {
            \begin{itemize}
            \item \paramdesc{element}{Element that is going to be renamed.}
            \item \paramdesc{name}{The new name of the element.}
            \end{itemize}
        }
    \end{itemize}
}
    {\begin{itemize}
        \item \methoddesc{\public}{}{void}{execute}{}
        {}{}{}
        \item \methoddesc{\public}{}{void}{undo}{}
        {}{}{}
    \end{itemize}
    }


\classdesc{SelectAction}{Selects an element.}
    {\begin{itemize}
        \item \atrdesc{\private}{}{\type{HashSet<Element>}} {selectedElements}{}
        \item \atrdesc{\private}{}{\type{HashSet<Element>}} {deselectedElements}{}
    \end{itemize}} 
{   \begin{itemize}
        \item \constrdesc{\public}{SelectAction}{\type{Element} element}{}
        {
            \begin{itemize}
            \item \paramdesc{element}{Element that is going to be selected.}
            \end{itemize}
        }
    \end{itemize}
}
    {\begin{itemize}
        \item \methoddesc{\public}{}{void}{execute}{}
        {}{}{}
        \item \methoddesc{\public}{}{void}{undo}{}
        {}{}{}
    \end{itemize}
    }

\classdesc{CreateLibraryAction}{Creates a library.}
    {\begin{itemize}
        \item \atrdesc{\private}{}{\type{Library}} {createdLibrary}{}
    \end{itemize}} 
{   \begin{itemize}
        \item \constrdesc{\public}{CreateLibraryAction}{\type{String} name}{}
        {
            \begin{itemize}
            \item \paramdesc{name}{Name of the created library.}
            \end{itemize}
        }
    \end{itemize}
}
    {\begin{itemize}
        \item \methoddesc{\public}{}{void}{execute}{}
        {}{}{}
        \item \methoddesc{\public}{}{void}{undo}{}
        {}{}{}
    \end{itemize}
    }{}
\classdesc{ScaleAction}{Scales the values of an attribute by a given float.}
    {\begin{itemize}
        \item \atrdesc{\private}{}{\type{Attribute}} {attribute}{}
        \item \atrdesc{\private}{}{\type{float}} {scale}{}
    \end{itemize}} 
{   \begin{itemize}
        \item \constrdesc{\public}{ScaleAction}{\type{Attribute} attribute, \type{float} scale}{}
        {
            \begin{itemize}
            \item \paramdesc{attribute}{Attribute that is going to be scaled.}
            \item \paramdesc{scale}{The value of scale.}
            \end{itemize}
        }
    \end{itemize}
}
    {\begin{itemize}
        \item \methoddesc{\public}{}{void}{execute}{}
        {}{}{}
        \item \methoddesc{\public}{}{void}{undo}{}
        {}{}{}
    \end{itemize}
    }{}
\classdesc{UndoAction}{Undoes an action.}{}
{\begin{itemize}
        \item \constrdesc{\public}{UndoAction}{}{}{}
    \end{itemize}
}
    {\begin{itemize}
        \item \methoddesc{\public}{}{void}{execute}{}
        {}{}{}
        \item \methoddesc{\public}{}{void}{undo}{}
        {}{}{}
    \end{itemize}
    }{}
\classdesc{TextEditAction}{Makes changes on the text.}
    {\begin{itemize}
        \item \atrdesc{\private}{}{\type{String}} {oldContent}{}
        \item \atrdesc{\private}{}{\type{String}} {newContent}{}
        \item \atrdesc{\private}{}{\type{Element}} {element}{}
    \end{itemize}} 
{   \begin{itemize}
        \item \constrdesc{\public}{TextEditAction}{\type{String} oldContent, \type{String} newContent, \type{Element} element}{}
        {
            \begin{itemize}
            \item \paramdesc{oldContent}{Old content of the changed text.}
            \item \paramdesc{newContent}{New content of the changed text.}
            \item \paramdesc{element}{The text of this element is going to be changed.}
            \end{itemize}
        }
    \end{itemize}
}
    {\begin{itemize}
        \item \methoddesc{\public}{}{void}{execute}{}
        {}{}{}
        \item \methoddesc{\public}{}{void}{undo}{}
        {}{}{}
    \end{itemize}
    }{}
\classdesc{ActionHistory}{Keeps the history of the executed Commands and manages it.}
    {\begin{itemize}
        \item \atrdesc{\private}{}{\type{Command[]}} {actions}{}
        \item \atrdesc{\private}{}{\type{Command[]}} {undoneActions}{}
        \item \atrdesc{\private}{}{\type{int}} {undoNumber}{}
    \end{itemize}} 
{   \begin{itemize}
        \item \constrdesc{\public}{ActionHistory}{}{}{}
    \end{itemize}
}{
    \begin{itemize}
        \item \methoddesc{\public}{}{void}{setUndoNumber}{\type{int} undoNumber}{}
        {{\begin{itemize}
        \item \paramdesc{undoNumber} {The number that decides how many undo operations can be done.}
        \end{itemize}}
        }{}
        \item \methoddesc{\private}{}{void}{resetUndoneActions}{}
        {}{}{}
        \item \methoddesc{\public}{}{void}{AddAction}{\type{Command} action}
        {}{{\begin{itemize}
        \item \paramdesc{action} {An action that is added to the undo history.}
        \end{itemize}}
        }{}
        \item \methoddesc{\public}{}{void}{resetActions}{}
        {}{}{}
        \item \methoddesc{\public}{}{\type{Command}}{getLatestActions}{}
        {}{}{{\begin{itemize}
        \item \paramdesc{Command} {Latest action will be returned.}
        \end{itemize}}}
    \end{itemize}
    }{}
}
\packagedesc{model.parsers}{
\classdesc{LibertyParser}{
Provides functionality to parse pieces of Liberty File text format into their corresponding data objects.
}{}{
    \begin{itemize}
        \item \constrdesc{\private}{LibertyParser}{}{}{}
    \end{itemize}
}{
    \begin{itemize}
        \item \methoddesc{\public}{\static}{Library}{parseLibrary}{String content}
        {Parses the content in Liberty format into a library object}{
        \begin{itemize}
            \item \paramdesc{content}{String Content to be parsed}
        \end{itemize}
        }{}
        \item \methoddesc{\public}{\static}{Cell}{parseCell}{String content}
        {Parses the content in Liberty format into a Cell object}{
        \begin{itemize}
            \item \paramdesc{content}{String Content to be parsed}
        \end{itemize}
        }{}
        \item \methoddesc{\public}{\static}{Pin}{parsePin}{String content}
        {Parses the content in Liberty format into a Pin object}{
        \begin{itemize}
            \item \paramdesc{content}{String Content to be parsed}
        \end{itemize}
        }{}
    \end{itemize}
}{}
}
\packagedesc{model.compiler}{
\interfacedesc{Compiler}{
    An interface that is implemented by all compilers.
}{}{
    \begin{itemize}
        \item \implclsdec{CSVCompiler}
        \item \implclsdec{LibertyCompiler}
    \end{itemize}
}{
    \begin{itemize}
        \item \methoddesc{\public}{\static}{String}{compile}{Library library}
        {Compiles a library into a certain Format}{
        \begin{itemize}
            \item \paramdesc{library}{Library object to be compiled}
        \end{itemize}
        }{}
        \item \methoddesc{\public}{\static}{String}{compile}{Cell cell}
        {Compiles a cell into a certain Format}{
        \begin{itemize}
            \item \paramdesc{cell}{Cell object to be compiled}
        \end{itemize}
        }{}
        \item \methoddesc{\public}{\static}{String}{compile}{Pin pin}
        {Compiles a pin into a certain Format}{
        \begin{itemize}
            \item \paramdesc{pin}{Pin object to be compiled}
        \end{itemize}
        }{}
    \end{itemize}
}

\classdesc{CSVCompiler}{
Provides functionality to compile Element Data Objects into their corresponding CSV text format.
}{}{
    \begin{itemize}
        \item \constrdesc{\private}{CSVCompiler}{}{}{}
    \end{itemize}
}{
    \begin{itemize}
        \item \methoddesc{\public}{\static}{String}{compile}{Library library}
        {Compiles a library into CSV Format}{
        \begin{itemize}
            \item \paramdesc{library}{Library object to be compiled}
        \end{itemize}
        }{}
        \item \methoddesc{\public}{\static}{String}{compile}{Cell cell}
        {Compiles a cell into CSV Format}{
        \begin{itemize}
            \item \paramdesc{cell}{Cell object to be compiled}
        \end{itemize}
        }{}
        \item \methoddesc{\public}{\static}{String}{compile}{Pin pin}
        {Compiles a pin into CSV Format}{
        \begin{itemize}
            \item \paramdesc{pin}{Pin object to be compiled}
        \end{itemize}
        }{}
    \end{itemize}
}{}
\classdesc{LibertyCompiler}{
Provides functionality to compile Element Data Objects into their corresponding Liberty File text format.
}{}{
    \begin{itemize}
        \item \constrdesc{\private}{LibertyCompiler}{}{}{}
    \end{itemize}
}{
    \begin{itemize}
        \item \methoddesc{\public}{\static}{String}{compile}{Library library}
        {Compiles a library into Liberty Format}{
        \begin{itemize}
            \item \paramdesc{library}{Library object to be compiled}
        \end{itemize}
        }{}
        \item \methoddesc{\public}{\static}{String}{compile}{Cell cell}
        {Compiles a cell into Liberty Format}{
        \begin{itemize}
            \item \paramdesc{cell}{Cell object to be compiled}
        \end{itemize}
        }{}
        \item \methoddesc{\public}{\static}{String}{compile}{Pin pin}
        {Compiles a pin into Liberty Format}{
        \begin{itemize}
            \item \paramdesc{pin}{Pin object to be compiled}
        \end{itemize}
        }{}
    \end{itemize}
}{}
}

\packagedesc{model.project}{
\classdesc{Model}{
Keeps track of the objects of the entire model package by ensuring the singularity of Project, Settings and Shortcuts classes as well as managing their Files.
}{
    \begin{itemize}
        \item \atrdesc{\private}{\static}{Model}{instance}{The single instance of the Model class}
        \item \atrdesc{\private}{}{Project}{currentProject}{The active Project class}
        \atrdesc{\private}{}{Settings}{currentSettings}{The active Settings class}
        \atrdesc{\private}{}{Project}{currentShortcuts}{The active Shortcuts class}
    \end{itemize}
}{
    \begin{itemize}
        \item \constrdesc{\private}{Model}{}{}{}
    \end{itemize}
}{  
    \begin{itemize}
        \item \methoddesc{\public}{\static}{Model}{getInstance}{}
        {Returns the sole instance of the Model}{}{}
        \item \methoddesc{\public}{}{void}{saveProject}{}
        {Opens the OS File Manager in order to select where to save the current Project Object in JSON format.}{}{}
        \item \methoddesc{\public}{}{void}{loadProject}{}
        {Opens the OS File Manager in order to select a File in JSON Format that replaces the current Project Object if the format fits.}{}{}
        \item \methoddesc{\public}{}{void}{saveSettings}{}
        {Saves the Settings Object in the Program files (so that it reloads on rerun).}{}{}
        \item \methoddesc{\public}{}{void}{resetSettings}{}
        {Loads the default Settings Object from the Program Files and calls saveSettings.}{}{}
        \item \methoddesc{\public}{}{void}{saveShortcuts}{}
        {Saves the Shortcuts Object in the Program files (so that it reloads on rerun).}{}{}
        \item \methoddesc{\public}{}{void}{resetShortcuts}{}
        {Loads the default Shortcuts Object from the Program Files and calls saveShortcut.}{}{}
        \item \methoddesc{\public}{}{void}{notify}{}
        {Notifies the view of made changes}{}{}
    \end{itemize}
}
\classdesc{Project}{
Keeps track of the Elements loaded into the program and manages their Files
}{
    \begin{itemize}
        \item \atrdesc{\private}{}{ArrayList<Library>}{libraries}{The array that keeps track of the libraries loaded into the program}
        \item \atrdesc{\private}{}{HashSet<Element>}{selectedElements}{The Set that keeps track of the selected elements}
        \atrdesc{\private}{}{HashSet<Element>}{openedInTextElements}{The Set that keeps track of the elements opened in a text editor}
    \end{itemize}
}{
    \begin{itemize}
        \item \constrdesc{\public}{Project}{}{}{}
    \end{itemize}
}{
    \begin{itemize}
        \item \methoddesc{\public}{\static}{void}{saveLibrary}{\type{Library} library}
        {Saves library as a Liberty File in the path specified in the Library object. If the library doesn't have a physical copy yet, it instead calls the saveLibraryAs method of the same class}{
        \begin{itemize}
            \item \paramdesc{library}{Library to be saved}
        \end{itemize}
        }{}
        \item \methoddesc{\public}{\static}{void}{saveLibraryAs}{\type{Library} library}
        {Opens the OS File Manager in order to select where to save the Library as a Liberty File. Updates Path on Library object.}{
        \begin{itemize}
            \item \paramdesc{library}{Library to be saved}
        \end{itemize}
        }{}
        \item \methoddesc{\public}{\static}{void}{saveAsCSV}{\type{Element} Element}
        {Opens the OS File Manager in order to select where to save the Element as a CSV File. Updates Path on Library object.}{
        \begin{itemize}
            \item \paramdesc{library}{Element to be saved}
        \end{itemize}
        }{}
    \end{itemize}
}{}
\classdesc{Interpolator}{
Provides interpolation functionality.
}{}{
    \begin{itemize}
        \item \constrdesc{\public}{Intepolator}{}{}{}
    \end{itemize}
}{
    \begin{itemize}
        \item \methoddesc{\public}{\static}{\type{float[]}}{interpolator}{\type{float[]} indexes, \type{float[]} values, \type{float[]} newIndexes}
        {Interpolates a set of 2D coordinates (index1 and value) and gives a set of values for a given set of indexes.\newline
        Imports org.apache.commons.math.analysis.interpolation}{
        \begin{itemize}
            \item \paramdesc{indexes}{Original indexes}
            \item \paramdesc{values}{Original values}
            \item \paramdesc{newIndexes}{New required indexes}
        \end{itemize}
        }{}
        \item \methoddesc{\public}{\static}{\type{float[]}}{bicubicInterpolate}{\type{float[]} indexes1, {float[]} indexes2 \type{float[]} values, \type{float[]} newIndexes1, \type{float[]} newIndexes1}
        {Interpolates a set of 3D coordinates (index1, index2 and value) and gives a set of values for a given set of indexes. \newline
        Imports org.apache.commons.math3.analysis.interpolation}{
        \begin{itemize}
            \item \paramdesc{indexes1}{Original indexes1}
            \item \paramdesc{indexes2}{Original indexes2}
            \item \paramdesc{values}{Original}
            \item \paramdesc{newIndexes1}{New required indexes1}
            \item \paramdesc{newIndexes2}{New requires indexes2}
        \end{itemize}
        }{}
    \end{itemize}
}{}

\classdesc{Shortcuts}{
Maps the characters from pressed Key events with their corresponding Actions with the Event enum.}{
    \begin{itemize}
        \item \atrdesc{\private}{}{HashMap<char, Event>}{commands}{The map of characters and their corresponding actions}
    \end{itemize}
}{
    \begin{itemize}
        \item \constrdesc{\public}{Shortcuts}{}{}{}
    \end{itemize}
}{
    \begin{itemize}
        \item \methoddesc{\public}{}{void}{setKey}{\type{char} key, \type{Event} action}
        {Binds a char key to the event in the controller}{
        \begin{itemize}
                \item \paramdesc{key}{The pressed key.}
                \item \paramdesc{action}{The action that corresponds to the key.}
        \end{itemize}
        }{}
        \item \methoddesc{\public}{}{void}{removeKey}{\type{char} key}
        {Removes a key char from the binding}{
        \begin{itemize}
                \item \paramdesc{key}{The pressed key.}
        \end{itemize}
        }{}
        \item \methoddesc{\public}{}{ArrayList<char>}{getKeys}{}
        {Returns all the set char keys}{}{}
        \item \methoddesc{\public}{}{Event}{getAction}{\type{char} key}
        {Returns the action corresponding to a keystroke char}{
        \begin{itemize}
                \item \paramdesc{key}{The pressed key.}
        \end{itemize}
        }{}
    \end{itemize}
}{}
\classdesc{FileManager}{
Provides functionality to save and load Files.
}{}{
    \begin{itemize}
        \item \constrdesc{\public}{FileManager}{}{}{}
    \end{itemize}
}{
    \begin{itemize}
        \item \methoddesc{\public}{\static}{\type{File}}{openFile}{}
        {Opens the OS File Manager in order to select which file will be opened}{}{}
        \item \methoddesc{\public}{\static}{\type{File}}{openFile}{\type{String} path}
        {Opens the file in the specified path}{}{}
        \item \methoddesc{\public}{\static}{void}{saveFile}{\type{String} content, \type{String[]} extensions}
        {Opens the OS File Manager in order to select where the file will be saved}{
        \begin{itemize}
                \item \paramdesc{content}{The content of the file.}
                \item \paramdesc{extensions}{The extensions the file can be saved as.}
        \end{itemize}
        }{}
        \item \methoddesc{\public}{\static}{void}{saveFile}{\type{String} content, \type{String} extension, \type{String} path}
        {Saves the file in the specified path}{
        \begin{itemize}
                \item \paramdesc{content}{The content of the file.}
                \item \paramdesc{extension}{The extension the file will be saved as.}
                \item \paramdesc{path}{The path where the file will be saved.}
        \end{itemize}
        }{}
    \end{itemize}
}{}
\classdesc{Filter}{
Keeps track of the filters, a filters specific data and provides filtering functionality.
}{
    \begin{itemize}
        \item \atrdesc{\private}{\static}{ArrayList<Filter>}{filters}{The array list of all currently active filters}
        \item \atrdesc{\private}{}{Attribute}{attribute}{The attribute that is being filtered.}
        \atrdesc{\private}{}{Mode}{mode}{The way the filtered attribute is calculated.}
        \atrdesc{\private}{}{float}{value}{The value that the attribute is being compared to.}
        \atrdesc{\private}{}{Operation}{operation}{The operation that is being executed in the filter.}
    \end{itemize}
}{
    \begin{itemize}
        \item \constrdesc{\public}{Filter}{\type{Attribute} attribute, \type{Mode} mode, \type{float} value, \type{Operation} operation}{}{
        \begin{itemize}
            \item \paramdesc{attribute}{}
            \item \paramdesc{mode}{}
            \item \paramdesc{value}{}
            \item \paramdesc{operation}{}
        \end{itemize}
        }
    \end{itemize}
}{
    \begin{itemize}
        \item \methoddesc{\public}{\static}{void}{filter}{}
        {Filters all Elements by changing their Filtered attribute.}{}{}
        \item \methoddesc{\public}{\static}{void}{saveDefaultFilters}{}
        {Saves the current array of Filters in the program files.}{}{}
        \item \methoddesc{\public}{\static}{void}{loadDefaultFilters}{}
        {Loads the saved array of Filters from program files into the current Filter class.}{}{}
    \end{itemize}
}{}
\classdesc{Settings}{
Keeps track off the settings
}{
    \begin{itemize}
        \item \atrdesc{\private}{}{Language}{currentLanguage}{The currently active Language}
        \item \atrdesc{\private}{}{ColorTheme}{colors}{The currently active ColorTheme}
        \atrdesc{\private}{}{String}{fontType}{The set Font type}
        \atrdesc{\private}{}{int}{fontSize}{Ther set Font Size}
        \atrdesc{\private}{}{int}{barCount}{The number of bars set to be displayed in a bar chart}
        \atrdesc{\private}{}{int}{undoCount}{The set number of undoable Actions saved in the ActionHistory}
        \atrdesc{\private}{}{boolean}{editorOpenedFirst}{If true, the Text Editor opens first upon opening an Element in the works space.}
    \end{itemize}
}{
    \begin{itemize}
        \item \constrdesc{\public}{Settings}{}{}{}
    \end{itemize}
}{
    \begin{itemize}
        \item \methoddesc{\public}{}{void}{setLanguage}{\type{LanguageEnum} language}
        {}{
        \begin{itemize}
            \item \paramdesc{language}{}
        \end{itemize}
        }{}
        \item \methoddesc{\public}{}{void}{setColorTheme}{\type{ColorThemeEnum} colorTheme}
        {}{
        \begin{itemize}
            \item \paramdesc{colorTheme}{}
        \end{itemize}
        }{}
        \item \methoddesc{\public}{}{void}{setFontType}{\type{String} fontType}
        {}{
        \begin{itemize}
            \item \paramdesc{fontType}{}
        \end{itemize}
        }{}
        \item \methoddesc{\public}{}{void}{setFontSize}{\type{int} fontSize}
        {}{
        \begin{itemize}
            \item \paramdesc{fontSize}{}
        \end{itemize}
        }{}
        \item \methoddesc{\public}{}{void}{setBarCount}{\type{int} barCount}
        {}{
        \begin{itemize}
            \item \paramdesc{barCount}{}
        \end{itemize}
        }{}
        \item \methoddesc{\public}{}{void}{setUndoCount}{\type{int} undoCount}
        {}{
        \begin{itemize}
            \item \paramdesc{undoCount}{}
        \end{itemize}
        }{}
        \item \methoddesc{\public}{}{void}{setEditorOpenedFirst}{\type{boolean} editorOpenedFirst}
        {}{
        \begin{itemize}
            \item \paramdesc{editorOpenedFirst}{}
        \end{itemize}
        }{}
        \item \methoddesc{\public}{}{LanguageEnum}{getLanguage}{}
        {}{}{}
        \item \methoddesc{\public}{}{colorTheme}{getColorTheme}{}
        {}{}{}
        \item \methoddesc{\public}{}{String}{getFontType}{}
        {}{}{}
        \item \methoddesc{\public}{}{int}{getFontSize}{}
        {}{}{}
        \item \methoddesc{\public}{}{int}{getBarCount}{}
        {}{}{}
        \item \methoddesc{\public}{}{int}{getUndoCount}{}
        {}{}{}
        \item \methoddesc{\public}{}{boolean}{getEditorOpenedFirst}{}
        {}{}{}
        \item \methoddesc{\private}{}{void}{notify}{}
        {}{}{}
    \end{itemize}
}{}

\classdesc{ColorTheme}{
Stores the colors included in the color theme.
}{
    \begin{itemize}
        \item \atrdesc{\private}{}{int[]}{colors}{The colors corresponding to the theme}
    \end{itemize}
}{
    \begin{itemize}
        \item \constrdesc{\public}{ColorTheme}{}{}{}
    \end{itemize}
}{
    \begin{itemize}
        \item \methoddesc{\public}{}{int[]}{getColors}{}
        {}{}{}
    \end{itemize}
}
\classdesc{Language}{
Stores the snippets of text that correspond to a specific language
}{
    \begin{itemize}
        \item \atrdesc{\private}{}{String[]}{librety}{The element, attributes, modes names in the specified language}
        \atrdesc{\private}{}{String[]}{menus}{The interface menu names in the specified language}
        \atrdesc{\private}{}{String[]}{methods}{The method names in the specified language}
        \atrdesc{\private}{}{String[]}{attributes}{The attribute names in the specified language}
        \atrdesc{\private}{}{String[]}{errors}{The error texts in the specified language}
        \atrdesc{\private}{}{String[]}{misc}{any other text element in the specified language}
    \end{itemize}
}{
    \begin{itemize}
        \item \constrdesc{\public}{Language}{}{}{}
    \end{itemize}
}{}
\enumdesc{Mode}{
Keeps track of the mode used to calculate the attribute used on the filter
}{
    \begin{itemize}
        \item \fielddesc{Mode}{MAX}{Maximum: Takes the maximum value for the attribute}
        \item \fielddesc{Mode}{MIN}{Minimum: Takes the minimal value for the attribute}
        \item \fielddesc{Mode}{MIN}{Average: Takes the average value for the attribute}
        \item \fielddesc{Mode}{Med}{Median: Takes the median value for the attribute}
    \end{itemize}
}{}
\enumdesc{Operation}{Keeps track of the type of operation done by the Filter}{
    \begin{itemize}
        \item \fielddesc{Operation}{LESS}{Checks if attribute is less than value}
        \item \fielddesc{Operation}{EQUAL}{Checks if attribute is equal than value}
        \item \fielddesc{Operation}{BIGGER}{Checks if attribute is bigger than value}
    \end{itemize}
}{}
\enumdesc{ColorThemeEnum}{Keeps track of the Color themes}{
    \begin{itemize}
        \item \fielddesc{ColorThemeEnum}{THEME1}{Theme name 1}
        \item \fielddesc{ColorThemeEnum}{THEME2}{Theme name 2}
        \item \fielddesc{ColorThemeEnum}{AND-SO-ON}{And further theme names}
    \end{itemize}
}{}
\enumdesc{LanguageEnum}{Keeps track of the languages available for the program}{
    \begin{itemize}
        \item \fielddesc{LanguageEnum}{ENGLISH}{The default English language}
        \item \fielddesc{LanguageEnum}{DEUTSCH}{German}
        \item \fielddesc{LanguageEnum}{TURK}{Turkish}
        \item \fielddesc{LanguageEnum}{SHQIP}{Albanian}
        \item \fielddesc{LanguageEnum}{FRANCAIS}{French}
    \end{itemize}
}{}
}

\packagedesc{model.exceptions}{
\classdesc{TooManyPanelsOpenedException}{}{}{}{}{}
\classdesc{SearchedStringNotFoundException}{}{}{}{}{}
\classdesc{InvalidNameException}{}{}{}{}{}
\classdesc{InvalidComparisonException}{}{}{}{}{}
\classdesc{InvalidFileFormatException}{}{}{}{}{}
\classdesc{ExceedingFileSizeException}{}{}{}{}{}
\classdesc{TooManySelectedException}{}{}{}{}{}
}


\packagedesc{controller.listeners}{
\classdesc{EventManager}{public class EventManager}{
    \begin{itemize}
        \item \atrdesc{\private}{}{MainWindow}{view}{Application GUI}
        \item \atrdesc{\private}{}{Model}{model}{Main model data}
        \item \atrdesc{\private}{}{Map<Events,ActionListener>}{listeners}{Map of all view listeners for subscription and initialization.}
        \atrdesc{\private}{}{List<IObserver>}{handlers}{List of all model listeners}
    \end{itemize}
    }{
    \begin{itemize}
        \item \constrdesc{\public}{EventManager}{\type{MainWindow} {view}, \type{Model} model}{}{}
    \end{itemize}
}{ 
    \begin{itemize}
        \item \methoddesc{\public}{}{Map<Events, EventListener>}{getListeners}{}{Initializes the view listeners}{}{}
        \item \methoddesc{\public}{}{void}{subscribeHandlers}{}{Registers the update handlers}{}{}
        \item \methoddesc{\public}{}{void}{removeHandler}{\type{IObserver} handlers}{Unsubscribes a handler}{}{}
        \item \methoddesc{\public}{}{void}{notifyHandlers}{\type{Model} model}{Notifies the handlers of the changes in the model.}{
        \begin{itemize}
            \item \paramdesc{model}{Current model class}
            \end{itemize}}{}
    \end{itemize}}{}

\enumdesc{Event}
{
Identifies each view listener for subscription.
}{
    \begin{itemize}
        \item \fielddesc{Event}{LOAD}{}
        \item\fielddesc{Event}{OPEN}{}
        \item \fielddesc{Event}{DELETE}{}
        \item \fielddesc{Event}{REMOVE}{}
        \item \fielddesc{Event}{SAVE}{}
        \item \fielddesc{Event}{SAVEAS}{}
        \item \fielddesc{Event}{SELECT}{}
        \item \fielddesc{Event}{EDIT}{}
        \item \fielddesc{Event}{RENAME}{}
        \item \fielddesc{Event}{MERGE}{}
        \item \fielddesc{Event}{COPY}{}
        \item \fielddesc{Event}{MOVE}{}
        \item \fielddesc{Event}{PASTE}{}
        \item \fielddesc{Event}{UNDO}{}
        \item \fielddesc{Event}{REDO}{}
        \item \fielddesc{Event}{SCALE}{}
        \item \fielddesc{Event}{COMPARE}{}
        \item \fielddesc{Event}{INTERPOLATE}{}
        \item \fielddesc{Event}{ADDFILTER}{}
        \item \fielddesc{Event}{REMOVEFILTER}{}
        \item \fielddesc{Event}{LOADPROJECT}{}
        \item \fielddesc{Event}{SCSETTINGS}{}
        \item \fielddesc{Event}{PRSETTINGS}{}
        \item \fielddesc{Event}{LASETTINGS}{}
        \item \fielddesc{Event}{ATTRDROPDOWN}{}
        \item \fielddesc{Event}{GRAPHDROPDOWN}{}
        \item \fielddesc{Event}{SUBATTRDROPDOWN}{}
        \item \fielddesc{Event}{PIN}{}
        \item \fielddesc{Event}{STATISTICS}{}
    \end{itemize}
}{}



\classdesc{LoadLibraryListener}{Listener for loading a liberty file into the app}{
    \begin{itemize}
        \item \atrdesc{\private}{}{Command}{command}{Command for loading a liberty file}
        \item \atrdesc{\private}{}{File}{file}{To be loaded file}
    \end{itemize}
}{
    \begin{itemize}
        \item \constrdesc{\public}{LoadLibraryListener}{}{}{}
    \end{itemize}
}{
    \begin{itemize}
        \item \methoddesc{\public}{}{void}{actionPerformed}{\type{ActionEvent} e}
        {}{\begin{itemize}
            \item \paramdesc{e}{Performed action event on the component.}
            \end{itemize}}{}
    \end{itemize}
}{}
\classdesc{OpenElementListener}{Listener for opening a file in the working area.
}{
    \begin{itemize}
        \item \atrdesc{\private}{}{Command}{command}{Command for opening a liberty file in working area}
        \item \atrdesc{\private}{}{Element}{element}{Opened element}
    \end{itemize}
}{
    \begin{itemize}
        \item \constrdesc{\public}{OpenElementListener}{}{}{}
    \end{itemize}
}{
    \begin{itemize}
        \item \methoddesc{\public}{}{void}{actionPerformed}{\type{ActionEvent} e}
        {}{}{}
    \end{itemize}
    }{}
\classdesc{DeleteCellListener}{Listener for deleting a cell.}{
    \begin{itemize}
        \item \atrdesc{\private}{}{Command}{command}{Command for deleting a cell.}
        \item \atrdesc{\private}{}{Cell[]}{cells}{Cells to be deleted.}
    \end{itemize}
}{
    \begin{itemize}
        \item \constrdesc{\public}{DeleteListener}{}{}{}
    \end{itemize}
}{  
    \begin{itemize}
        \item \methoddesc{\public}{}{void}{actionPerformed}{\type{ActionEvent} e}
        {}{}{}
    \end{itemize}
}{}
\classdesc{RemoveListener}{Listener for removing a file from the project.}{
    \begin{itemize}
        \item \atrdesc{\private}{}{Command}{command}{Command for removing a library}
        \item \atrdesc{\private}{}{Element[]}{elements}{Libraries to be removed.}
    \end{itemize}
}{
    \begin{itemize}
        \item \constrdesc{\public}{RemoveListener}{}{}{}
    \end{itemize}
}{
    \begin{itemize}
        \item \methoddesc{\public}{}{void}{actionPerformed}{\type{ActionEvent} e}
        {}{}{}
    \end{itemize}
}{}
\classdesc{SelectListener}{Listener for selections in the outliner.}{
    \begin{itemize}
        \item \atrdesc{\private}{}{Command}{command}{Command for selecting elements in the outliner.}
        \item \atrdesc{\private}{}{Element[]}{element}{Selected elements}
    \end{itemize}
}{
    \begin{itemize}
        \item \constrdesc{\public}{SelectListener}{\type{Outliner} outliner}{}{}
    \end{itemize}
}{
    \begin{itemize}
        \item \methoddesc{\public}{}{void}{actionPerformed}{\type{ActionEvent} e}
        {}{}{}
    \end{itemize}}{}
\classdesc{EditListener}{Listener for changed values in the text editor.}{
    \begin{itemize}
        \item \atrdesc{\private}{}{Command}{command}{Command for editing a liberty file in the text editor}
        \item \atrdesc{\private}{}{Element}{element}{Edited liberty file}
        \item \atrdesc{\private}{}{String}{newText}{Edited text in liberty file.}
    \end{itemize}
}{
    \begin{itemize}
        \item \constrdesc{\public}{EditListener}{}{}{}
    \end{itemize}
}{
    \begin{itemize}
        \item \methoddesc{\public}{}{void}{actionPerformed}{\type{ActionEvent} e}
        {}{}{}
    \end{itemize}}{}
\classdesc{RenameListener}{Listener for renaming in the outliner.}{
    \begin{itemize}
        \item \atrdesc{\private}{}{Command}{command}{Command for renaming an element.}
        \item \atrdesc{\private}{}{Element}{element}{To be renamed element.}
    \end{itemize}
}{
    \begin{itemize}
        \item \constrdesc{\public}{RenameListener}{}{}{}
    \end{itemize}
}{
    \begin{itemize}
        \item \methoddesc{\public}{}{void}{actionPerformed}{\type{ActionEvent} e}
        {}{}{}
    \end{itemize}}{}
\classdesc{SaveListener}{Listener for saving changes made in the liberty file}{
    \begin{itemize}
        \item \atrdesc{\private}{}{Model}{model}{Model holding the project data.}
    \end{itemize}
}{
    \begin{itemize}
        \item \constrdesc{\public}{SaveListener}{}{}{}
    \end{itemize}
}{
    \begin{itemize}
        \item \methoddesc{\public}{}{void}{actionPerformed}{\type{ActionEvent} e}
        {}{}{}
    \end{itemize}}{}
\classdesc{SaveAsListener}{Listener for saving changes made in the liberty file as a new file.}{
    \begin{itemize}
        \item \atrdesc{\private}{}{Model}{model}{Model holding tghe project data.}
    \end{itemize}
}{
    \begin{itemize}
        \item \constrdesc{\public}{SaveAsListener}{}{}{}
    \end{itemize}
}{
    \begin{itemize}
        \item \methoddesc{\public}{}{void}{actionPerformed}{\type{ActionEvent} e}
        {}{}{}
    \end{itemize}}{}
\classdesc{MergeListener}{Listener for merging multiple libraries}{
    \begin{itemize}
        \item \atrdesc{\private}{}{Command}{command}{Command for merging action}
        \item \atrdesc{\private}{}{Library[]}{libraries}{Libraries selected for merge action.}
    \end{itemize}
}{
    \begin{itemize}
        \item \constrdesc{\public}{MergeListener}{}{}{}
    \end{itemize}
}{
    \begin{itemize}
        \item \methoddesc{\public}{}{void}{actionPerformed}{\type{ActionEvent} e}
        {}{}{}
    \end{itemize}}{}
\classdesc{ScaleListener}{Listener for scaling.}{
    \begin{itemize}
        \item \atrdesc{\private}{}{Command}{command}{Command for scaling action}
    \end{itemize}
}{
    \begin{itemize}
        \item \constrdesc{\public}{ScaleListener}{}{}{}
    \end{itemize}
}{
    \begin{itemize}
        \item \methoddesc{\public}{}{void}{actionPerformed}{\type{ActionEvent} e}
        {}{}{}
    \end{itemize}}{}
\classdesc{InterpolationListener}{Listener for interpolation action.}{
    \begin{itemize}
        \item \atrdesc{\private}{}{Command}{command}{Command for interpolation action.}
    \end{itemize}
}{
    \begin{itemize}
        \item \constrdesc{\public}{InterpolationListener}{}{}{}
    \end{itemize}
}{
    \begin{itemize}
        \item \methoddesc{\public}{}{void}{actionPerformed}{\type{ActionEvent} e}
        {}{}{}
    \end{itemize}}{}
\classdesc{UndoListener}{Listener for the undo button.}{
    \begin{itemize}
        \item \atrdesc{\private}{}{Command}{command}{Command for the undo action.}
    \end{itemize}
}{
    \begin{itemize}
        \item \constrdesc{\public}{UndoListener}{}{}{}
    \end{itemize}
}{
    \begin{itemize}
        \item \methoddesc{\public}{}{void}{actionPerformed}{\type{ActionEvent} e}
        {}{}{}
    \end{itemize}}{}
\classdesc{RedoListener}{Listener for the redo button.}{
    \begin{itemize}
        \item \atrdesc{\private}{}{Command}{command}{Command for the redo action.}
    \end{itemize}
}{
    \begin{itemize}
        \item \constrdesc{\public}{RedoListener}{}{}{}
    \end{itemize}
}{
    \begin{itemize}
        \item \methoddesc{\public}{}{void}{actionPerformed}{\type{ActionEvent} e}
        {}{}{}
    \end{itemize}}{}
\classdesc{StatisticsListener}{Listener for the statistics checkboxes.}{
    \begin{itemize}
        \item \atrdesc{\private}{}{Command}{command}{}
    \end{itemize}
}{
    \begin{itemize}
        \item \constrdesc{\public}{StatisticsListener}{}{}{}
    \end{itemize}
}{
    \begin{itemize}
        \item \methoddesc{\public}{}{void}{actionPerformed}{\type{ActionEvent} e}
        {}{}{}
    \end{itemize}}{}
\classdesc{CompareListener}{Listener for the compare action.}{
    \begin{itemize}
        \item \atrdesc{\private}{}{Command}{command}{Command for comparing.}
    \end{itemize}
}{
    \begin{itemize}
        \item \constrdesc{\public}{CompareListener}{}{}{}
    \end{itemize}
}{
    \begin{itemize}
        \item \methoddesc{\public}{}{void}{actionPerformed}{\type{ActionEvent} e}
        {}{}{}
    \end{itemize}}{}
\classdesc{MoveListener}{Listener for moving selected cells to another library.}{
    \begin{itemize}
        \item \atrdesc{\private}{}{Command}{command}{Move action.}
        \item \atrdesc{\private}{}{Element[]}{element}{Selected cells.}
        \item \atrdesc{\private}{}{Library}{library}{Target library.}
    \end{itemize}
}{
    \begin{itemize}
        \item \constrdesc{\public}{MoveListener}{}{}{}
    \end{itemize}
}{
    \begin{itemize}
        \item \methoddesc{\public}{}{void}{actionPerformed}{\type{ActionEvent} e}
        {}{}{}
    \end{itemize}}{}
\classdesc{CopyListener}{Listener for copying selected cells to another library.}{
    \begin{itemize}
        \item \atrdesc{\private}{}{Command}{command}{Copy action}
        \item \atrdesc{\private}{}{Element[]}{element}{Elements selected for copying.}
        \item \atrdesc{\private}{}{Library}{library}{Target library}
    \end{itemize}
}{
    \begin{itemize}
        \item \constrdesc{\public}{CopyListener}{}{}{}
    \end{itemize}
}{
    \begin{itemize}
        \item \methoddesc{\public}{}{void}{actionPerformed}{\type{ActionEvent} e}
        {}{}{}
    \end{itemize}}{}
\classdesc{AddFilterListener}{Listener for add filter button.}{
    \begin{itemize}
        \item \atrdesc{\private}{}{Command}{command}{Filter adding action.}
        \item \atrdesc{\private}{}{Filter}{filter}{Filter to be added.}
    \end{itemize}
}{
    \begin{itemize}
        \item \constrdesc{\public}{AddFilterListener}{}{}{}
    \end{itemize}
}{
    \begin{itemize}
        \item \methoddesc{\public}{}{void}{actionPerformed}{\type{ActionEvent} e}
        {}{}{}
    \end{itemize}}{}
\classdesc{RemoveFilterListener}{Listener for remove filter button.}{
    \begin{itemize}
        \item \atrdesc{\private}{}{Command}{command}{Filter removing action.}
        \item \atrdesc{\private}{}{Filter}{filter}{Filter to be removed.}
    \end{itemize}
}{
    \begin{itemize}
        \item \constrdesc{\public}{RemoveFilterListener}{}{}{}
    \end{itemize}
}{
    \begin{itemize}
        \item \methoddesc{\public}{}{void}{actionPerformed}{\type{ActionEvent} e}
        {}{}{}
    \end{itemize}}{}

\classdesc{LoadProjectListener}{Listener for importing a project}{
    \begin{itemize}
        \item \atrdesc{\private}{}{Command}{command}{Command for importing a project.}
        \item \atrdesc{\private}{}{Project}{project}{Loaded project file}
    \end{itemize}
}{
    \begin{itemize}
        \item \constrdesc{\public}{LoadProjectListener}{}{}{}
    \end{itemize}
}{
    \begin{itemize}
        \item \methoddesc{\public}{}{void}{actionPerformed}{\type{ActionEvent} e}
        {}{}{}
    \end{itemize}}{}

\classdesc{ShortcutSettingsListener}{Listener for changing shortcuts component.}{
    \begin{itemize}
        \item \atrdesc{\private}{}{Shortcuts}{shortcuts}{Shortcut data in the model.}
    \end{itemize}
}{
    \begin{itemize}
        \item \constrdesc{\public}{ShortcutSettingsListener}{}{}{}
    \end{itemize}
}{
    \begin{itemize}
        \item \methoddesc{\public}{}{void}{actionPerformed}{\type{ActionEvent} e}
        {}{}{}
    \end{itemize}}{}
\classdesc{PreferencesSettingsListener}{Listener for application settings}{
    \begin{itemize}
        \item \atrdesc{\private}{}{Model}{model}{Preferences data in the model.}
    \end{itemize}
}{
    \begin{itemize}
        \item \constrdesc{\public}{PreferencesSettingsListener}{}{}{}
    \end{itemize}
}{
    \begin{itemize}
        \item \methoddesc{\public}{}{void}{actionPerformed}{\type{ActionEvent} e}
        {}{}{}
    \end{itemize}}{}
\classdesc{LanguageSettingsListener}{Listener for language dropdown component.}{
    \begin{itemize}
        \item \atrdesc{\private}{}{Model}{Model}{Model holding the language data.}
    \end{itemize}
}{
    \begin{itemize}
        \item \constrdesc{\public}{LanguageSettingsListener}{}{}{}
    \end{itemize}
}{
    \begin{itemize}
        \item \methoddesc{\public}{}{void}{actionPerformed}{\type{ActionEvent} e}
        {}{}{}
    \end{itemize}}{}
    
\classdesc{AttributeDropdownListener}{Listener for the attribute dropdown menu in the visualizer}{
    \begin{itemize}
        \item \atrdesc{\private}{}{Element[]}{element}{}
    \end{itemize}
}{
    \begin{itemize}
        \item \constrdesc{\public}{AttributeDropdownListener}{}{}{}
    \end{itemize}
}{
    \begin{itemize}
        \item \methoddesc{\public}{}{void}{actionPerformed}{\type{ActionEvent} e}
        {}{}{}
    \end{itemize}}{}
\classdesc{SubAttributeDropdownListener}{}{
    \begin{itemize}
        \item \atrdesc{\private}{}{Command}{command}{}
        \item \atrdesc{\private}{}{Element[]}{element}{}
    \end{itemize}
}{
    \begin{itemize}
        \item \constrdesc{\public}{SubAttributeDropdownListener}{}{}{}
    \end{itemize}
}{
    \begin{itemize}
        \item \methoddesc{\public}{}{void}{actionPerformed}{\type{ActionEvent} e}
        {}{}{}
    \end{itemize}}{}
\classdesc{GraphDropdownListener}{}{
    \begin{itemize}
        \item \atrdesc{\private}{}{Command}{command}{}
        \item \atrdesc{\private}{}{Element[]}{element}{}
    \end{itemize}
}{
    \begin{itemize}
        \item \constrdesc{\public}{GraphDropdownListener}{}{}{}
    \end{itemize}
}{
    \begin{itemize}
        \item \methoddesc{\public}{}{void}{actionPerformed}{\type{ActionEvent} e}
        {}{}{}
    \end{itemize}}{}
\classdesc{PinSelectListener}{}{
    \begin{itemize}
        \item \atrdesc{\private}{}{Command}{command}{}
        \item \atrdesc{\private}{}{Element[]}{element}{}
    \end{itemize}
}{
    \begin{itemize}
        \item \constrdesc{\public}{PinSelectListener}{}{}{}
    \end{itemize}
}{
    \begin{itemize}
        \item \methoddesc{\public}{}{void}{actionPerformed}{\type{ActionEvent} e}
        {}{}{}
    \end{itemize}}{}

}
\packagedesc{controller.updatehandlers}{
\classdesc{DataUpdateHandler}{}{
    \begin{itemize}
        \item \atrdesc{\private}{}{Outliner}{outliner}{Application GUI}
        \item \atrdesc{\private}{}{List<Element>}{elements}{Libraries from the model.}
    \end{itemize}
}{
    \begin{itemize}
        \item \constrdesc{\public}{DataUpdateHandler}{\type{Outliner} outliner}{}{}
    \end{itemize}
}{
    \begin{itemize}
        \item \methoddesc{\public}{}{void}{update}{\type{List<Library>} libraries}
        {Updates the view with the changed data.}{}{}
    \end{itemize}
}{}
\classdesc{SettingsUpdateHandler}{}{
    \begin{itemize}
        \item \atrdesc{\private}{}{Panel}{panel}{Settings window of the GUI}
    \end{itemize}
}{
    \begin{itemize}
        \item \constrdesc{\public}{SettingsUpdateHandler}{\type{Panel} panel}{}{}
    \end{itemize}
}{
    \begin{itemize}
        \item \methoddesc{\public}{}{void}{update}{\type{Settings} settings}
        {Updates the view with the changed settings data.}{}{}
    \end{itemize}}{}
\classdesc{FilterUpdateHandler}{}{
    \begin{itemize}
        \item \atrdesc{\private}{}{Panel}{panel}{Filter window of the UI}
    \end{itemize}
}{
    \begin{itemize}
        \item \constrdesc{\public}{FilterUpdateHandler}{\type{Panel} panel}{}{}
    \end{itemize}}{\begin{itemize}
        \item \methoddesc{\public}{}{void}{update}{\type{List<Filter>} filters}
        {Updates the view with the changed filters.}{}{}
    \end{itemize}}{}
\classdesc{InterpolationUpdateHandler}{}{
    \begin{itemize}
        \item \atrdesc{\private}{}{Panel}{panel}{Panel for interpolation}
    \end{itemize}
}{
    \begin{itemize}
        \item \constrdesc{\public}{InterpolationUpdateHandler}{\type{Panel} panel}{}{}
    \end{itemize}}{\begin{itemize}
        \item \methoddesc{\public}{}{void}{update}{\type{Interpolation} interpolation}
        {Updates the view with the changed data.}{}{}
    \end{itemize}}{}
}
\chapter{Class Diagrams}

\chapter{Sequence Diagrams}

\chapter{Design Patterns}
\pattern{Strategy}
\pattern{Builder}
\pattern{Singleton}

\end{document}
