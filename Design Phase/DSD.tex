\documentclass[10pt,a4paper]{report}

\usepackage[a4paper,margin=1in,left=1in, right=1in]{geometry}

\usepackage[T1]{fontenc}
\usepackage{titlesec, blindtext, color}
\newcommand{\hsp}{\hspace{10pt}}

\usepackage[utf8]{inputenc}
\usepackage{amsmath}
\usepackage{amsfonts}
\usepackage{amssymb}
\usepackage{enumitem}

\usepackage{caption}
\usepackage{float}
\usepackage{graphicx}
\usepackage{hyperref}
\usepackage{xcolor}

\usepackage{tocloft}
\usepackage{etoc}
\usepackage[at]{easylist}

\usepackage{etoolbox}
\usepackage{url}

\usepackage{Includes/PSEMacros}

\title{\includegraphics[scale=6.0]{App Icon}\\Design for Graphical Editor for Liberty Files}

\author{Alp Toraç Genç, Kerem Kara, Xhulio Pernoca, Manuel Schenk, Ege Uzhan \leavevmode \\---------------------------------------------------------------------------------------\\
Submitted to: Om Prakash, Georgios Zervakis, Faeze Faghih}
\date{\today}

% Format for the chapters/sections/subsections
\titleformat{\chapter}[hang]{\Huge\bfseries}{\thechapter\hsp{}}{0pt}{\Huge\bfseries}
\titlespacing{\chapter}{0cm}{0cm}{0.5cm} % distance from {right}{left}{next line} for chapter
\titlespacing{\subsection}{0cm}{0cm}{0.5cm} % distance from {right}{left}{next line} for subsection

%Macros
% Macro for referencing another section in the document
% #1 Name of referenced section
% #2 Text of link
\definecolor{col:reference}{HTML}{2f5399}
\newcommand{\refer}[2]{\hyperref[#1]{\textcolor{col:reference}{#2}}}

%Macro for Highlighting
% #1 Text to highlight
\definecolor{col:highlight}{HTML}{63b53e}
\newcommand{\h}[1]{\textcolor{col:highlight}{#1}}

% Aligns captions to the left of the images
\captionsetup{
  font=footnotesize,
  justification=raggedright,
  singlelinecheck=false
}

% Macro for including images
\newcommand{\includeimage}[5]{
    \begin{figure}[H]
        #1
        \includegraphics[scale=#2]{#3.png}
        \caption{#4}
        \label{fig:#5}
    \end{figure}
}

% Macro for adding graph images
\newcommand{\includegraph}[7]{
    \graphicspath{{#1}}
    \includeimage{#2}{#3}{#4}{#5}{#6}
    \graphicspath{{#7}}
}

\newcommand{\imagepath}{Images/}
\newcommand{\graphpath}{Images/Graphs/}

%Macros for keywords
\definecolor{col:public}{HTML}{499632}
\definecolor{col:private}{HTML}{bf5050}
\definecolor{col:protected}{HTML}{808000}
\definecolor{col:class}{HTML}{9a50bf}
\definecolor{col:enum}{HTML}{9a50bf}
\definecolor{col:interface}{HTML}{9a50bf}
\definecolor{col:sub}{HTML}{5052bf}
\definecolor{col:impl}{HTML}{5052bf}
\definecolor{col:throws}{HTML}{5052bf}
\definecolor{col:static}{HTML}{bfb650}
\definecolor{col:default}{HTML}{bfb650}
\definecolor{col:abstract}{HTML}{bfb650}
\definecolor{col:final}{HTML}{bfb650}
\definecolor{col:generic}{HTML}{9a50bf}

\newcommand{\public}{\textcolor{col:public}{public }}
\newcommand{\private}{\textcolor{col:private}{private }}
\newcommand{\proted}{\textcolor{col:protected}{protected }}

\newcommand{\class}{\textcolor{col:class}{class }}
\newcommand{\interface}{\textcolor{col:interface}{interface }}
\newcommand{\enum}{\textcolor{col:enum}{enum }}
\newcommand{\type}[1]{\textcolor{col:class}{#1}}
\newcommand{\generic}[1]{\textcolor{col:generic}{#1}}
\newcommand{\typewgeneric}[2]{\textcolor{col:class}{#1}<\generic{#2}>}

\newcommand{\extends}{\textcolor{col:sub}{extends }}
\newcommand{\implements}{\textcolor{col:impl}{implements }}
\newcommand{\throws}{\textcolor{col:throws}{throws }}

\newcommand{\static}{\textcolor{col:static}{static }}
\newcommand{\deflt}{\textcolor{col:default}{default }}
\newcommand{\abstr}{\textcolor{col:abstract}{abstract }}
\newcommand{\final}{\textcolor{col:final}{final }}

\newcommand{\packagebeginning}{edu.kit.informatik.pse.gelf} %root package name

%Macro for label name
\newcommand{\lblroot}{lbl} % root cascading label name (\casclabel)
\newcommand{\lblpackage}{} % package name (without \packagebeginning)
\newcommand{\lblpackageelement}{} % element inside the package (class/interface)
\newcommand{\lblpackageelementmember}{} % member of the element inside the package (method/constructor/attribute)
\newcommand{\lblpackageelementmemberparameter}{} % parameter of the member of the element inside the package (parameter)
\newcommand{\casclabelname}{\lblroot\lblpackage\lblpackageelement\lblpackageelementmember\lblpackageelementmemberparameter}
\newcommand{\casclabel}{\label{\casclabelname}}

%Macro for placing labels
%Places a label and resets the appropriate place in the hierarchy in the end
%#1 one of these as representative of the hierarchy level: \lblpackage \lblpackageelement \lblpackageelementmember \lblpackageelementmemberparameter
%#2 name to be added to \casclabel
\newcommand{\putcasclabel}[3]{
    \renewcommand{#1}{:#2}
    \casclabel
    #3
    \renewcommand{#1}{}
}

%Macro for describing functions that have a boolean value
%Tip: if #2 is self explanatory after #1, just write "not"
%#1 true description (dot will be added in the end automatically)
%#2 false description (dot will be added in the end automatically)
\newcommand{\booldesc}[2]{
    \begin{itemize}
        \item True, if #1.
        \item False, if #2.
    \end{itemize}
}

%Macro for adding notes to descriptions
%#1 note to be added to the description
\newcommand{\descnote}[1]{
\leavevmode \\
\leavevmode \\
    Note: #1
}

%Macro for temporarily overriding \casclabelname
%used for making labels of duplicate method names
%or for same method and attribute names
%#1 temporal name
%#2 area, in which the overriding takes place
\newcommand{\overridecasclabelname}[2]{
    \renewcommand{\casclabelname}{#1}
    #2
    \renewcommand{\casclabelname}{\lblroot\lblpackage\lblpackageelement\lblpackageelementmember\lblpackageelementmemberparameter}
}

%Macro for class/interface declarations
%#1 Declaration of the class/interface
%#2 Description
\newcommand{\decl}[2]{
#1
\leavevmode \\
\leavevmode \\
#2
}

%Macro for Method description
%#1 visibility modifier
%#2 optional keyword
%#3 output type
%#4 method name
%#5 parameters
%#6 description
%#7 list of parameter descriptions
%#8 output description
\newcommand{\methoddesc}[8]{
\putcasclabel{\lblpackageelementmember}{#4}{
    #4
    \leavevmode \\
    \leavevmode \\
    #1#2\type{#3} #4(#5)
    \ifblank{#6#7#8}{}{
        \begin{itemize}[label=-]
            \ifblank{#6}{}{
                \item Description: \leavevmode \\ #6 \leavevmode \\
            }
            \ifblank{#7}{}{
                \item Parameters: \leavevmode \\ #7 \leavevmode \\
            }
            \ifblank{#8}{}{
                \item Returns: \leavevmode \\ #8
            }
        \end{itemize}
    }
}
}

%Macro for overridden Method description
%#1 visibility modifier
%#2 optional keyword
%#3 output type
%#4 method name
%#5 parameters
%#6 origin class of the overridden method
%#7 description
%#8 list of parameter descriptions
%#9 output description
\newcommand{\ovrdnmethoddesc}[9]{
\putcasclabel{\lblpackageelementmember}{#4}{
    #4
    \leavevmode \\
    \leavevmode \\
    #1#2\type{#3} #4(#5) overrides the version of the method from #6
    \ifblank{#7#8#9}{}{
        \begin{itemize}[label=-]
            \ifblank{#7}{}{
                \item Description: \leavevmode \\ #7 \leavevmode \\
            }
            \ifblank{#8}{}{
                \item Parameters: \leavevmode \\ #8 \leavevmode \\
            }
            \ifblank{#9}{}{
                \item Returns: \leavevmode \\ #9
            }
        \end{itemize}
    }
}
}

%Macro for implemented Method description
%#1 visibility modifier
%#2 optional keyword
%#3 output type
%#4 method name
%#5 parameters
%#6 origin interface of the implemented method
%#7 description
%#8 list of parameter descriptions
%#9 output description
\newcommand{\implmethoddesc}[9]{
\putcasclabel{\lblpackageelementmember}{#4}{
    #4
    \leavevmode \\
    \leavevmode \\
    #1#2\type{#3} #4(#5) implements the method from #6
    \ifblank{#7#8#9}{}{
        \begin{itemize}[label=-]
            \ifblank{#7}{}{
                \item Description: \leavevmode \\ #7 \leavevmode \\
            }
            \ifblank{#8}{}{
                \item Parameters: \leavevmode \\ #8 \leavevmode \\
            }
            \ifblank{#9}{}{
                \item Returns: \leavevmode \\ #9
            }
        \end{itemize}
    }
}
}

%Macro for parameter description
%#1 parameter name
%#2 parameter description
\newcommand{\paramdesc}[2]{
\putcasclabel{\lblpackageelementmemberparameter}{#1}{
    \textbf{#1}: #2
}
}

%Macro for subclass declaration
%#1 subclass name
\newcommand{\subclsdec}[1]{
    #1
}

%Macro for implementing class declaration
%#1 implementing class name
\newcommand{\implclsdec}[1]{
    #1
}

%Macro for subinterface declaration
%#1 subinterface name
\newcommand{\subintdec}[1]{
    #1
}

%Macro for attribute descriptions
%#1 visibility
%#2 optional keywords
%#3 attribute type
%#4 attribute name
%#5 attribute description
\newcommand{\atrdesc}[5]{
\putcasclabel{\lblpackageelementmember}{#4}{
    #1#2\type{#3} #4
    \ifblank{#5}{}{
        : #5
    }
}
}

%Macro for Enum fields
%#1 Enum name
%#2 Field name (all UPPER case and words separated with -)
%#3 Field description
\newcommand{\fielddesc}[3]{
\putcasclabel{\lblpackageelementmember}{#2}{
    \public \static \final \type{#1} #2  \leavevmode \\
    \begin{itemize}[label=-]
        \item #3
    \end{itemize}
}
}

%Macro for constructor description
%#1 visibility modifier
%#2 constructor name
%#3 parameters
%#4 description
%#5 parameter description
\newcommand{\constrdesc}[5]{
\putcasclabel{\lblpackageelementmember}{#2}{
    #2
    \leavevmode \\
    \leavevmode \\
    #1#2(#3) \leavevmode \\
    \ifblank{#4#5}{}{
        \begin{itemize}[label=-]
            \ifblank{#4}{}{
                \item \textbf{Description:} \leavevmode \\ #4
            }
            \ifblank{#5}{}{
                \item \textbf{Parameters:} \leavevmode \\ #5
            }
        \end{itemize}
    }
}
}

%Macros for headers of package/class/interface/abstract class descriptions
%#1 place in hierarchy
%#2 type
%#3 name
\newcommand{\desc}[3]{
    #1{#2 #3}
}

%Macro for class descriptions
%#1 name
%#2 Declaration
%#3 Attributes
%#4 Constructors
%#5 Methods
\newcommand{\classdesc}[5]{
\putcasclabel{\lblpackageelement}{#1}{
    \desc{\subsection}{Class}{#1} \leavevmode \\
    \ifblank{#2}{}{
            \textbf{Declaration:} \leavevmode \\
            #2 \leavevmode \\
    }
    \ifblank{#3}{}{
            \textbf{Attributes:} \leavevmode \\
            #3 \leavevmode \\
    }
    \ifblank{#4}{}{
            \textbf{Constructors:} \leavevmode \\
            #4 \leavevmode \\
    }
    \ifblank{#5}{}{
            \textbf{Methods:} \leavevmode \\
            #5
    }
}
}

%Macro for abstract class descriptions
%#1 Name
%#2 Declaration
%#3 Attributes
%#4 All known subclasses
%#5 Constructors
%#6 Methods
\newcommand{\absclassdesc}[6]{
\putcasclabel{\lblpackageelement}{#1}{
    \desc{\subsection}{Abstract Class}{#1} \leavevmode \\
    \ifblank{#2}{}{
            \textbf{Declaration:} \leavevmode \\
            #2 \leavevmode \\
    }
    \ifblank{#3}{}{
            \textbf{Attributes:} \leavevmode \\
            #3 \leavevmode \\
    }
    \ifblank{#4}{}{
            \textbf{All known subclasses:} \leavevmode \\
            #4 \leavevmode \\
    }
    \ifblank{#5}{}{
            \textbf{Constructors:} \leavevmode \\
            #5 \leavevmode \\
    }
    \ifblank{#6}{}{
            \textbf{Methods:} \leavevmode \\
            #6
    }
}
}

%Macro for interface descriptions
%#1 name
%#2 Declaration
%#3 All known subinterfaces
%#4 All known implementing classes
%#5 Methods
\newcommand{\interfacedesc}[5]{
\putcasclabel{\lblpackageelement}{#1}{
    \desc{\subsection}{Interface}{#1} \leavevmode \\
    \ifblank{#2}{}{
            \textbf{Declaration:} \leavevmode \\
            #2 \leavevmode \\
    }
    \ifblank{#3}{}{
            \textbf{All known subinterfaces:} \leavevmode \\
            #3 \leavevmode \\
    }
    \ifblank{#4}{}{
            \textbf{All known implementing classes:} \leavevmode \\
            #4 \leavevmode \\
    }
    \ifblank{#5}{}{
            \textbf{Methods:} \leavevmode \\
            #5
    }
}
}

%Macro for enum descriptions
%#1 name
%#2 Declaration
%#3 Fields (Enum values)
%#4 Methods
\newcommand{\enumdesc}[4]{
\putcasclabel{\lblpackageelement}{#1}{
    \desc{\subsection}{Enum}{#1} \leavevmode \\
    \ifblank{#2}{}{
            \textbf{Declaration:} \leavevmode \\
            #2 \leavevmode \\
    }
    \ifblank{#3}{}{
            \textbf{Fields:} \leavevmode \\
            #3 \leavevmode \\
    }
    \ifblank{#4}{}{
            \textbf{Methods:} \leavevmode \\
            #4 \leavevmode \\
    }
}
}

%Macro for package descriptions
%#1 name
%#2 package content
\newcommand{\packagedesc}[2]{
\putcasclabel{\lblpackage}{#1}{
    \desc{\section}{Package}{\packagebeginning.#1}
    \renewcommand{\contentsname}{\small\textit{Package contents}
    \hfill
    \small\textit{Page}}
    \setlength{\cftbeforetoctitleskip}{0em} % removes space before (local) table of contents
    \setlength{\cftaftertoctitleskip}{0em} % removes space between the heading and the list parts of (local) table of contents
    \setlength{\cftsubsecindent}{1em} % shortens the spacing from left
    \localtableofcontents % makes a local table of contents
    #2
}
}

%Macros for design patterns
%#1 place in hierarchy
%#2 name
\newcommand{\patternentry}[2]{
    #1{#2}
}
\newcommand{\pattern}[2]{
    \patternentry{\section}{#1}
    {#2}
}

%Macro for defining classes
%#1 declaration
%#2 attributes
%#3 constructors
%#4 methods
\newcommand{\defineclass}[4]{
\textbf{Declaration:}
\\\indent #1
\\\textbf{Attributes:}
\\\indent#2
\\\textbf{Constructors:}
\\\indent#3
\\\textbf{Methods:}
\\\indent#4
}

%Macro for defining methods
%#1 signature
%#2 return type
%#3 definition
\newcommand{\method}{

}

\graphicspath{\imagepath}

\begin{document}
\maketitle
\label{sec:title}
\tableofcontents

\chapter{Packages}
    \includeimage{}{0.3}{PackageDiagram}{PackageDiagram}{PackageDiagram}
    \section{Model}
    \includeimage{}{0.1}{Model}{Model}{Model}{}
    \subsection{model.commands}
    This package contains actions.
    \subsection{model.elements}
    This package contains elements of the parsed liberty file.
    \subsubsection{model.elements.attributes}
    This package contains different attributes of the elements.
    \subsection{model.parsers}
    This package contains the parsers, that are used to parse the file.
    \subsection{model.compilers}
    This package contains the compilers that compile the files.
    \subsection{model.exceptions}
    This package contains all of the exceptions.
    \subsection{model.project}
    This package provides classes for the management of the project.
    \section{View}
    \includeimage{}{0.035}{View}{View}{View}
    \subsection{view.diagrams}
    The package that encapsulates the interfaces and subpackages about the visual part of diagrams.
    \subsubsection{view.diagrams.overlayer}
    This package houses the classes that overlay same types of diagrams.
    \subsubsection{view.diagrams.components}
    The package, which contains all the components (and their creator) required to build the diagrams.
    \subsubsection{view.diagrams.data}
    This package holds the classes that are responsible for containing the information needed to build diagrams.
    \subsubsection{view.diagrams.type}
    The classes of this package represent the diagrams specified in the requirement phase report.
    \subsubsection{view.diagrams.indicator}
    The components that are not a part of a diagram but serve an indication purpose (such as displaying statistics or displaying coordinate grids) reside in this package.
    \subsubsection{view.diagrams.builder}
    The contents of this package are responsible for building the diagrams by utilizing view.diagrams.components package.
    \section{Controller}

\chapter{Class Descriptions}

%GUI
%Components
\begin{package}{Components}
    \struct{+}{i}{AutoResizing}[
        \func{+}[void]{setResizer}[Component=Component to set the resizer on. | ComponentResizer=Resizer to use for the component.]{Associates a resizer with a component, making it resize it.}
    ]{
        Interface for components that are to be resized automatically.
    }

    \struct{+}{e}{ResizeMode}?
        ABSOLUTE\_TOP\_LEFT=component stays fixed in relation to the top/left of the parent component;
        ABSOLUTE\_BOTTOM\_RIGHT=component stays fixed in relation to the bottom/right of the parent component;
        RELATIVE=component keeps a relative distance to the parent componennt's edges.
    ?{
        Enum values represent resizing behaviour.
    }

    \struct{+}{c}{Resizer}<
        \func{+}{Resizer}[ResizeMode:t=Top edge alignment. | ResizeMode:r=Right edge alignment. | ResizeMode:b=Bottom edge alignment. | ResizeMode:l=Left edge alignment.]{Creates new resizer with specified attributes.}
    >[
        \func{+}[void]{resize}[Component=Component to resize. | int:width=From width. | int:height=From height. | int:newWidth=To width. | int:newHeight=To height.]{Resizes specified component to fit specified dimensions of parent component.}
    ]{
        Classes that automatically resize use this class to define behaviour when resizing.
    }

    \struct{+}{c}{Window:JFrame}?
        Map(Component, ComponentResizer)
    ?{
        Works like JFrame, but with automatic resizing.
    }

    \struct{+}{c}{Panel:JPanel}?
        Map(Component, ComponentResizer)
    ?{
        Works like JPanel, but with automatic resizing.
    }

    \struct{+}{c}{InputBox:JTextField}{Adapted version of the parent Swing component.}
    \struct{+}{c}{TextPane:JTextPane}{Adapted version of the parent Swing component.}
    \struct{+}{c}{DropdownSelector:JComboBox}{Adapted version of the parent Swing component.}
    \struct{+}{c}{Tree:JTree}{Adapted version of the parent Swing component.}
    \struct{+}{c}{MenuBar:JMenuBar}{Adapted version of the parent Swing component.}
    \struct{+}{c}{Checkbox:JCheckBox}{Adapted version of the parent Swing component.}
    \struct{+}{c}{Button:JButton}{Adapted version of the parent Swing component.}
    \struct{+}{c}{Label:JLabel}{Adapted version of the parent Swing component.}
    \struct{+}{c}{ScrollPane:JScrollPane}{Adapted version of the parent Swing component.}
\end{package}

\newpage
%Composites
\begin{package}{Composites}
    \struct{+}{c}{MainWindow:Window}?
        MenuBar:mainMenu=Menu bar with options that affect the entire project/
    ?<
        \func{+}{MainWindow}{Creates the main window and instantiates child components.}
    >[
        \func{+}<static>[void]{main}[String[]:args]{The program entry point.}
    ]{
        Class representing the main window. It manages all of the GUI elements a user can interact with.
    }

    \struct{+}{e}{InfoBarID}?
            InfoBarID:VERSION=Identifies version text label;
            InfoBarID:SELECTED=Identifies selection text label;
            InfoBarID:LASTACTION=Identifies last action text label;
            InfoBarID:ERROR=Identifies error text label
        ?{
            Identifies a data section of an InfoBar.
        }

    \struct{+}{c}{InfoBar:Panel}?
        Map(InfoBarID, TextBox)=Map to identify parts of the info bar.
        ?{
            Panel with a collection of Info about the application status to display to the user.
        }

    \struct{+}{c}{Outliner:Panel}?
            MenuBar:menu;
            Tree;
            Model
        ?<
            \func{+}{Outliner}[Model:data]{Creates outliner and associates it with a Model.}
        >{
            Interactive panel that shows a tree view of elements to the user.
        }

    \struct{+}{c}{SubWindowArea:Panel}?
        Map(SubWindow\&, SubWindow):subWindows=Aggregation of all SubWindows currently present in the panel.
    ?[
        \func{+}[void]{addSubWindow}[SubWindow:sub=SubWindow to add]{Adds a SubWindow to the panel.};
        \func{+}[void]{removeSubWindow}[SubWindow:sub=SubWindow to add]{Removes a SubWindow from the panel.}
    ]{
        Panel where SubWindows are opened/displayed.
    }

    \struct{+}{c}{SubWindow:Panel}?
        MenuBar:menu
        ElementManipulator[]:manipulators
    ?<
        \func{+}{SubWindow}[Element]{Creates new SubWindow and associates Element with it.}
    >[
        \func{+}{setElement}[Element]{Associates Element with the SubWindow.}
    ]{
        Panel that displays an element for the user to interact with depending on the set ElementManipulator.
    }

    \struct{+}{i}{ModelUser}[
        \func{+}[void]{update}{Called whenever the Model is updated.}
    ]{
        Components that use data of the Model package implement this to be able to get notified when the Model changes.
    }

    \struct{+}{a}{ElementManipulator:Panel}[
        \func{+}{setElement}[Element]{Associates Element with the manipulator.}
    ]{
        Implementing classes display elements and let the User interact with them depending on their implementation.
    }

    \struct{+}{c}{TextEditor:ElementManipulator}?
        TextPane
    ?<
        \func{+}{TextEditor}[Element]{Creates new TextEditor and asscoiates Element with it.}
    >{
        Text editor with which the User can directly edit the associated Element.
    }

    \struct{+}{c}{Visualizer:ElementManipulator}?
        TextArea:info ;
        MenuBar:diagramOptions ;
        MenuBar:statisticsoptions
    ?<
        \func{+}{Visualizer}[Element]{Associates Element with the Visualizer.}
    >{
        Visualizer to visualize the associated Element.
    }

    \struct{+}{c}{Comparer:ElementManipulator}<
        \func{+}{Comparer}[Element:e1 | Element:e2]{Creates new Comparer to compare specified elements.}
    >{
        Panel to show a comparison between two elements in.
    }

    \struct{+}{c}{MergeDialog:Window}{
        Dialog popup that is shown whenever merge conflicts arise. It lets the user resolve said conflicts in multiple ways.
    }
   
    \struct{+}{c}{SettingsDialog:Window}{DIalog popup that is used to provide the user a way to change settings that affect the entire application.}
\end{package}
%\packagedesc{view.gui}{
%\classdesc{MainWindow}{a}{a}
%    {
%        \constrdesc{\public}{MainWindow}{a}{a}{a}
%    }
%    {a}
%\classdesc{SubWindowArea}{a}{a}
%    {
%        \constrdesc{\public}{SubWindowArea}{a}{a}{a}
%    }
%    {a}
%\classdesc{Outliner}{a}{a}
%    {
%        \constrdesc{\public}{Outliner}{a}{a}{a}
%    }
%    {a}
%\classdesc{InfoBar}{a}{a}
%    {
%        \constrdesc{\public}{InfoBar}{a}{a}{a}
%    }
%    {a}
%\enumdesc{InfoBarID}{a}{
%    \begin{itemize}
%        \item \fielddesc{InfoBarID}{VERSION}{a}
%        \item \fielddesc{InfoBarID}{SELECTED}{a}
%        \item \fielddesc{InfoBarID}{LASTACTION}{a}
%    \end{itemize}
%}{a}
%\classdesc{SubWindow}{a}{a}
%    {
%        \constrdesc{\public}{SubWindow}{a}{a}{a}
%    }
%    {a}
%\absclassdesc{ElementManipulator}{a}{a}
%    {
%        \begin{itemize}
%            \item \subclsdec{TextEditor}
%            \item \subclsdec{Visualizer}
%        \end{itemize}
%    }
%    {a}
%    {a}
%\classdesc{MergeDialog}{a}{a}
%    {a}
%    {a}
%}
%%Components
%\packagedesc{view.components}{
%\interfacedesc{AutoResizing}{a}
%    {a}
%    {a}
%    {a}
%\classdesc{Resizer}{a}{a}
%    {a}
%    {a}
%\enumdesc{ResizeMode}{a}{
%    \begin{itemize}
%        \item \fielddesc{ResizeMode}{ABSOLUTE-TOP-LEFT}{a}
%        \item \fielddesc{ResizeMode}{ABSOLUTE-BOTTOM-RIGHT}{a}
%        \item \fielddesc{ResizeMode}{RELATIVE}{a}
%    \end{itemize}
%}{a}
%\classdesc{Window}{a}{a}
%    {a}
%    {a}
%\classdesc{Panel}{a}{a}
%    {a}
%    {a}
%}
%Diagrams
\packagedesc{view.diagrams}{
\interfacedesc{IDiagram}{
    \decl{\public \interface IDiagram}{An interface that is implemented by all diagrams.}
}{}{
    \begin{itemize}
        \item \implclsdec{Diagram}
    \end{itemize}
}{
    \begin{itemize}
        \item \methoddesc{\public}{\typewgeneric{Collection}{?}}{}{cloneData}{}
        {
            Makes a deep copy of the \refer{\lblroot:view.diagrams.type:Diagram:data}{data}.
        }{}{
            A deep copy of the \refer{\lblroot:view.diagrams.type:Diagram:data}{data} of the diagram.
        }
        \item \methoddesc{\public}{}{void}{refresh}{}
        {
            Re-draws the diagram.
        }{}{}
        \item \methoddesc{\public}{}{void}{update}{\type{DiagramData} data}
        {
            Replaces the \refer{\lblroot:view.diagrams.type:Diagram:data}{data} by the given.
        }{
            \begin{itemize}
                \item \paramdesc{data}{The data to replace the current \refer{\lblroot:view.diagrams.type:Diagram:data}{data}.}
            \end{itemize}
        }{}
        \item \methoddesc{\public}{}{boolean}{addDiagramViewHelper}{\type{DiagramViewHelper} dvh}
        {
            Adds the given \refer{\lblroot:view.diagrams.indicator:DiagramViewHelper}{DiagramViewHelper}.
        }{
            \begin{itemize}
                \item \paramdesc{dvh}{The \refer{\lblroot:view.diagrams.indicator:DiagramViewHelper}{DiagramViewHelper} instance to be added.}
            \end{itemize}
        }{}
        \item \methoddesc{\public}{}{boolean}{removeDiagramViewHelper}{\type{IndicatorIdentifier} id}
        {}{
            \begin{itemize}
                \item \paramdesc{id}{The unique identifier of the \refer{\lblroot:view.diagrams.indicator:DiagramViewHelper}{DiagramViewHelper} to be removed.}
            \end{itemize}
        }{}
        \item \methoddesc{\public}{}{boolean}{showDiagramViewHelper}{\type{IndicatorIdentifier} id}
        {}{
            \begin{itemize}
                \item \paramdesc{id}{The unique identifier of the \refer{\lblroot:view.diagrams.indicator:DiagramViewHelper}{DiagramViewHelper} to be shown.}
            \end{itemize}
        }{}
        \item \methoddesc{\public}{}{boolean}{hideDiagramViewHelper}{\type{IndicatorIdentifier} id}
        {}{
            \begin{itemize}
                \item \paramdesc{id}{The unique identifier of the \refer{\lblroot:view.diagrams.indicator:DiagramViewHelper}{DiagramViewHelper} to be hidden.}
            \end{itemize}
        }{}
        \item \methoddesc{\public}{}{DiagramComponent[]}{getNonValueDisplayDiagramComponentPrototypes}{}
        {}{}
        {Deep copies of \refer{\lblroot:view.diagrams.components:DiagramComponent}{DiagramComponents} that do not represent any value.}
        \item \methoddesc{\public}{}{DiagramValueDisplayComponent[]}{getDiagramValueDisplayComponentPrototypes}{}
        {}{}
        {Deep copies of \refer{\lblroot:view.diagrams.components:DiagramValueDisplayComponent}{DiagramValueDisplayComponents}}
    \end{itemize}
}

\interfacedesc{IDiagramOverlayer}{
    \decl{\public \interface IDiagramOverlayer}{An interface implemented by \refer{\lblroot:view.diagrams.overlayer:DiagramOverlayer}{DiagramOverlayer} \descnote{The amount of \refer{\lblroot:view.diagrams}{IDiagrams} that can be overlaid at once can vary between different implementing classes of \refer{\lblroot:view.diagrams:IDiagramOverlayer}{IDiagramOverlayer}.}}
}{}{
    \begin{itemize}
        \item \implclsdec{DiagramOverlayer}
    \end{itemize}
}{
    \begin{itemize}
        \item \methoddesc{\public}{}{IDiagram}{getDiagram}{\type{int} index}
        {}{
            \begin{itemize}
                \item \paramdesc{index}{The index of the wanted \refer{\lblroot:view.diagrams}{IDiagram} in the \refer{\lblroot:view.diagrams.overlayer:DiagramOverlayer:diagrams}{diagrams}}
            \end{itemize}
        }{
            The wanted \refer{\lblroot:view.diagrams}{\type{IDiagram}} from \refer{\lblroot:view.diagrams.overlayer:DiagramOverlayer:diagrams}{diagrams}.
        }
        \item \methoddesc{\public}{}{void}{setDiagram}{\type{int} index, IDiagram diagram}
        {}{
            \begin{itemize}
                \item \paramdesc{index}{The index of the \refer{\lblroot:view.diagrams}{IDiagram} in the \refer{\lblroot:view.diagrams.overlayer:DiagramOverlayer:diagrams}{diagrams} to be set}
                \item \paramdesc{diagram}{The new \refer{\lblroot:view.diagrams}{IDiagram}}
            \end{itemize}
        }{}
        \item \methoddesc{\public}{}{boolean}{addDiagram}{\type{IDiagram} diagram}
        {}{
            \begin{itemize}
                \item \paramdesc{diagram}{The \refer{\lblroot:view.diagrams}{IDiagram} to be added.}
            \end{itemize}
        }{
            \booldesc
            {the \refer{\lblroot:view.diagrams}{IDiagram} is added successfully}
            {not}
        }
        \item \methoddesc{\public}{}{boolean}{removeDiagram}{\type{int} index}
        {}{
            \begin{itemize}
                \item \paramdesc{index}{The index of the \refer{\lblroot:view.diagrams}{IDiagram} to be removed.}
            \end{itemize}
        }{
            \booldesc
            {the \refer{\lblroot:view.diagrams}{IDiagram} is removed successfully}
            {not}
        }
        \item \overridecasclabelname{\lblroot\lblpackage\lblpackageelement:(inds)overlay}{\methoddesc{\public}{}{\type{IDiagram}}{overlay}{\type{int[]} indices}
        {
            Overlays the \refer{\lblroot:view.diagrams}{IDiagrams} specified by the indices in \refer{\lblroot:view.diagrams.overlayer:DiagramOverlayer:diagrams}{diagrams}
            \descnote{The amount of \refer{\lblroot:view.diagrams}{IDiagrams} that can be overlaid at once can vary between different implementing classes of \refer{\lblroot:view.diagrams:IDiagramOverlayer}{IDiagramOverlayer}.}
        }{
            \begin{itemize}
                \item \overridecasclabelname{\lblroot\lblpackage\lblpackageelement:(inds)overlay:indices}{\paramdesc{indices}{The indices of the \refer{\lblroot:view.diagrams}{IDiagrams}} in \refer{\lblroot:view.diagrams.overlayer:DiagramOverlayer:diagrams}{diagrams} to be overlaid.}
            \end{itemize}
        }{
            The result of overlaying the \refer{\lblroot:view.diagrams}{IDiagrams} on top of each other.
        }}
        \item \overridecasclabelname{\lblroot\lblpackage\lblpackageelement:(dgrms)overlay}{\methoddesc{\public}{}{IDiagram}{overlay}{\type{IDiagram[]} diagrams}
        {
            Overlays the given \refer{\lblroot:view.diagrams}{IDiagrams} on top of each other and replaces \refer{\lblroot:view.diagrams.overlayer:DiagramOverlayer:diagrams}{diagrams} \descnote{The amount of \refer{\lblroot:view.diagrams}{IDiagrams} that can be overlaid at once can vary between different implementing classes of \refer{\lblroot:view.diagrams:IDiagramOverlayer}{IDiagramOverlayer}.}
        }{
            \begin{itemize}
                \item \overridecasclabelname{\lblroot\lblpackage\lblpackageelement:(dgrms)overlay:diagrams}{\paramdesc{diagrams}{The \refer{\lblroot:view.diagrams}{IDiagrams} to be overlaid}}
            \end{itemize}
        }{
            The result of overlaying the \refer{\lblroot:view.diagrams}{IDiagrams} on top of each other.
        }}
    \end{itemize}
}

\classdesc{DiagramDirector}{
    \decl{\public \class DiagramDirector}{The class, which is responsible for initiating the building of a \refer{\lblroot:view.diagrams}{IDiagrams} and returning the result}
}{
    \begin{itemize}
        \item \atrdesc{\private}{}{DiagramBuilder}{builder}{The \refer{\lblroot:view.diagrams.builder:DiagramBuilder}{DiagramBuilder} of the \refer{\lblroot:view.diagrams}{IDiagram} to build.}
        \item \atrdesc{\private}{}{DiagramData}{data}{The \refer{\lblroot:view.diagrams.data:DiagramData}{DiagramData} of the \refer{\lblroot:view.diagrams}{IDiagram} to build.}
        \item \atrdesc{\private}{\static}{DiagramDirector}{instance}{The only instance of the class.}
    \end{itemize}
}{
    \begin{itemize}
        \item \constrdesc{\private}{DiagramDirector}{}{}{}
    \end{itemize}
}{
    \begin{itemize}
        \item \methoddesc{\public}{}{DiagramDirector}{getDiagramDirector}{}
        {}{}{
            The only instance of the class.
        }
        \item \methoddesc{\private}{}{void}{setBuilder}{}
        {
            Changes the active \refer{\lblroot:view.diagrams:DiagramDirector:builder}{builder} to the appropriate one, according to \refer{\lblroot:view.diagrams:DiagramDirector:data}{data}.
        }{}{}
        \item \methoddesc{\public}{}{void}{setDiagramData}{\type{DiagramData} data}
        {
            Sets the active \refer{\lblroot:view.diagrams:DiagramDirector:data}{data}.
        }{
            \begin{itemize}
                \item \paramdesc{data}{The given \refer{\lblroot:view.diagrams.data:DiagramData}{data}, which will be used to build the \refer{\lblroot:view.diagrams:IDiagram}{IDiagram}.}
            \end{itemize}
        }{}
        \item \methoddesc{\public}{}{IDiagram}{build}{}
        {
            Starts the building process of the \refer{\lblroot:view.diagrams:IDiagram}{IDiagram}.
        }{}{
            The built \refer{\lblroot:view.diagrams:IDiagram}{IDiagram} defined by the attributes of this class.
        }
    \end{itemize}
}

\classdesc{SettingsProvider}{
    The class, which is responsible for containing and distributing the latest \refer{\lblroot:model.project:Settings}{Settings}.
}{
    \begin{itemize}
        \item \atrdesc{\private}{\static}{SettingsProvider}{instance}{The only instance of the class.}
        \item \atrdesc{\private}{}{Settings}{s}{The latest \refer{\lblroot:model.project:Settings}{Settings}.}
    \end{itemize}
}{
    \begin{itemize}
        \item \constrdesc{\private}{SettingsProvider}{}{}{}
    \end{itemize}  
}{
    \begin{itemize}
        \item \methoddesc{\public}{}{SettingsProvider}{getInstance}{}
        {}{}{
            The only instance of the class.
        }
        \item \methoddesc{\public}{}{void}{changeSettings}{Settings s}
        {
            Sets \refer{\lblroot:view.diagrams:SettingsProvider:s}{s} with the given \refer{\lblroot:model.project:Settings}{Settings}.
        }{
            \begin{itemize}
                \item \paramdesc{s}{The given (latest) \refer{\lblroot:model.project:Settings}{Settings}.}
            \end{itemize}
        }{}
        \item \methoddesc{\public}{}{Settings}{getSettings}{}
        {}{}{
            \refer{\lblroot:view.diagrams:SettingsProvider:s}{s}
        }        
    \end{itemize}
}
}
\packagedesc{view.diagrams.overlayer}{

\interfacedesc{IDiagramOverlayStrategy}{
    \decl{\public \interface IDiagramOverlayStrategy}{An interface implemented by overlay strategies used by the \refer{\lblroot:view.diagrams.overlayer:DiagramOverlayer}{DiagramOverlayer}}
}{}{
    \begin{itemize}
        \item \implclsdec{FunctionGraphOverlayStrategy}
        \item \implclsdec{HistogramOverlayStrategy}
        \item \implclsdec{BarChartOverlayStrategy}
    \end{itemize}
}{
    \begin{itemize}
        \item \methoddesc{\public}{}{IDiagram}{overlay}{}
        {
            Overlays the \refer{\lblroot:view.diagrams:IDiagram}{IDiagrams} stored.
        }{}{
            The result of overlaying the specified \refer{\lblroot:view.diagrams:IDiagram}{IDiagrams}
        }
    \end{itemize}
}

\classdesc{DiagramOverlayer}{
    \decl{\public \class DiagramOverlayer}{Stores \refer{\lblroot:view.diagrams:IDiagram}{IDiagrams}} and overlays the specified ones. The stored \refer{\lblroot:view.diagrams:IDiagram}{IDiagrams} must be of the same type.
}{
    \begin{itemize}
        \item \atrdesc{\private}{\typewgeneric{Collection}{IDiagram}}{}{diagrams}{The stored \refer{\lblroot:view.diagrams:IDiagram}{IDiagrams}}
        \item \atrdesc{\private}{}{IDiagramOverlayStrategy}{overlayStrategy}{The active \refer{\lblroot:view.diagrams.overlayer:IDiagramOverlayStrategy}{IDiagramOverlayStrategy}}
    \end{itemize}
}{
    \begin{itemize}
        \item \constrdesc{\public}{DiagramOverlayer}{\type{IDiagram[]} diagrams}{
            Initializes an instance with the given \refer{\lblroot:view.diagrams:IDiagram}{IDiagrams}.
        }{
            \begin{itemize}
                \item \paramdesc{diagrams}{The given \refer{\lblroot:view.diagrams:IDiagram}{IDiagrams}}
            \end{itemize}
        }
    \end{itemize}
}{}

\classdesc{FunctionGraphOverlayStrategy}{
    \decl{\public \class FunctionGraphOverlayStrategy}{The implementation of \refer{\lblroot:view.diagrams.overlayer:IDiagramOverlayStrategy}{IDiagramOverlayStrategy} for \refer{\lblroot:view.diagrams.type:FunctionGraph}{FunctionGraphs}.
    \descnote{Currently arbitrary amount of \refer{\lblroot:view.diagrams.type:FunctionGraph}{FunctionGraphs} can be overlaid.}}
}{
    \begin{itemize}
        \item \atrdesc{\private}{}{FunctionGraph[]}{functionGraphs}{The \refer{\lblroot:view.diagrams.type:FunctionGraph}{FunctionGraphs} to be overlaid.}
    \end{itemize}
}{
    \begin{itemize}
        \item \constrdesc{\public}{FunctionGraphOverlayStrategy}{\type{FunctionGraph[]} functionGraphs}{}{
            \begin{itemize}
                \item \paramdesc{functionGraphs}{The \refer{\lblroot:view.diagrams.type:FunctionGraph}{FunctionGraphs} to be overlaid.}
            \end{itemize}
        }
    \end{itemize}
}{}

\classdesc{HistogramOverlayStrategy}{
    \decl{\public \class HistogramOverlayStrategy}{The implementation of \refer{\lblroot:view.diagrams.overlayer:IDiagramOverlayStrategy}{IDiagramOverlayStrategy} for \refer{\lblroot:view.diagrams.type:Histogram}{Histograms}.
    \descnote{Currently only 2 given \refer{\lblroot:view.diagrams.type:Histogram}{Histograms} can be overlaid.}}
}{
    \begin{itemize}
        \item \atrdesc{\private}{}{Histogram}{histogram1}{A given \refer{\lblroot:view.diagrams.type:Histogram}{Histogram}}
        \item \atrdesc{\private}{}{Histogram}{histogram2}{Another given \refer{\lblroot:view.diagrams.type:Histogram}{Histogram}}
    \end{itemize}
}{
    \begin{itemize}
        \item \constrdesc{\public}{HistogramOverlayStrategy}{\type{Histogram} histogram1, \type{Histogram} histogram2}{}{
            \begin{itemize}
                \item \paramdesc{histogram1}{A given \refer{\lblroot:view.diagrams.type:Histogram}{Histogram}}
                \item \paramdesc{histogram2}{Another given \refer{\lblroot:view.diagrams.type:Histogram}{Histogram}}
            \end{itemize}
        }
    \end{itemize}
}{}

\classdesc{BarChartOverlayStrategy}{
    \decl{\public \class BarChartOverlayStrategy}{The implementation of \refer{\lblroot:view.diagrams.overlayer:IDiagramOverlayStrategy}{IDiagramOverlayStrategy} for \refer{\lblroot:view.diagrams.type:BarChart}{BarCharts}.
    \descnote{Currently only 2 given \refer{\lblroot:view.diagrams.type:BarChart}{BarCharts} can be overlaid.}}
}{
    \begin{itemize}
        \item \atrdesc{\private}{}{BarChart}{barChart1}{A given \refer{\lblroot:view.diagrams.type:BarChart}{BarChart}}
        \item \atrdesc{\private}{}{BarChart}{barChart2}{Another given \refer{\lblroot:view.diagrams.type:BarChart}{BarChart}}
    \end{itemize}
}{
    \begin{itemize}
        \item \constrdesc{\public}{BarChartOverlayStrategy}{\type{BarChart} barChart1, \type{BarChart} barChart2}{}{
            \begin{itemize}
                \item \paramdesc{barChart1}{A given \refer{\lblroot:view.diagrams.type:BarChart}{BarChart}}
                \item \paramdesc{barChart2}{Another given \refer{\lblroot:view.diagrams.type:BarChart}{BarChart}}
            \end{itemize}
        }
    \end{itemize}
}{}
}
\packagedesc{view.diagrams.components}{

\interfacedesc{Hoverable}{
    \decl{\public \interface Hoverable}{An interface implemented by \refer{\lblroot:view.diagrams.components:DiagramValueDisplayComponent}{DiagramValueDisplayComponents}.
    \leavevmode \\
    \leavevmode \\
    This interface is responsible for displaying the \refer{\lblroot:view.diagrams.components:HoverLabel}{HoverLabel}. The methods responsible for displaying the \refer{\lblroot:view.diagrams.components:HoverLabel}{HoverLabel} are implemented inside this interface by default.}
}{}{
    \begin{itemize}
        \item \implclsdec{DiagramValueDisplayComponent}
    \end{itemize}
}{
    \begin{itemize}
        \item \methoddesc{\public}{\deflt}{boolean}{isBeingHovered}{}
        {}{}{
            \booldesc{the implementing class is being hovered with the mouse pointer}{not}
        }
        \item \methoddesc{\public}{\deflt}{void}{hoverAction}{}
        {
            Performs the action that will occur once the implementing class is being hovered with the mouse pointer.
        }{}{}
        \item \methoddesc{\public}{\deflt}{void}{refreshHoverLabelPosition}{}
        {
            Refreshes the position of the \refer{\lblroot:view.diagrams.components:HoverLabel}{HoverLabel} in relation to the mouse pointer.
        }{}{}
        \item \methoddesc{\public}{\deflt}{void}{showHoverLabel}{}
        {
            Displays the \refer{\lblroot:view.diagrams.components:HoverLabel}{HoverLabel}.
        }{}{}
        \item \methoddesc{\public}{\deflt}{void}{hideHoverLabel}{}
        {
            Hides the \refer{\lblroot:view.diagrams.components:HoverLabel}{HoverLabel}.
        }{}{}
    \end{itemize}
}

\absclassdesc{PositionInDiagram}{
    \decl{\public \abstr \class PositionInDiagram}{The abstract class inherited by the classes that represent a position inside an \refer{\lblroot:view.diagrams:IDiagram}{IDiagram}.}
}{
    \begin{itemize}
        \item \atrdesc{\private}{}{DiagramAxis[]}{axes}{The axes, according to which the position inside the \refer{\lblroot:view.diagrams:IDiagram}{IDiagram} will be calculated.}
        \item \atrdesc{\private}{}{Number[]}{positionsInAxes}{The coordinates on the \refer{\lblroot:view.diagrams.components:DiagramAxis}{DiagramAxes} given in axes attribute.}
    \end{itemize}
}{
    \subclsdec{PositionIn2DDiagram}
}{    
    \begin{itemize}
        \item \constrdesc{\public}{PositionInDiagram}{\type{DiagramAxis[]} axes, \type{Number[]} coordinatesInAxes}{}{
            \begin{itemize}
                \item \paramdesc{axes}{The axes, according to which the position inside the \refer{\lblroot:view.diagrams:IDiagram}{IDiagram} will be calculated.}
                \item \paramdesc{coordinatesInAxes}{The coordinates on the \refer{\lblroot:view.diagrams.components:DiagramAxis}{DiagramAxes} given in axes attribute.}
            \end{itemize}
        }
    \end{itemize}}
{
    \begin{itemize}
        \item \methoddesc{\public}{}{Number}{axisCoordinateToFrameCoordinate}{\type{int} index}
        {
            Transforms the specified coordinate in \refer{\lblroot:view.diagrams.components:PositionInDiagram:positionsInAxes}{positionsInAxes} to the corresponding coordinate inside the frame the \refer{\lblroot:view.diagrams:IDiagram}{IDiagram} is on top of.
        }{
            \begin{itemize}
                \item \paramdesc{index}{The index of the coordinate to be transformed to a frame coordinate.}
            \end{itemize}
        }{
            The frame coordinate that corresponds to the specified coordinate on \refer{\lblroot:view.diagrams.components:PositionInDiagram:positionsInAxes}{positionsInAxes}.
        }
        \item \methoddesc{\public}{}{PositionInFrame}{toPositionInFrame}{}
        {}{}{
            The frame coordinates that correspond to the coordinates on \refer{\lblroot:view.diagrams.components:PositionInDiagram:positionsInAxes}{positionsInAxes}.
        }
        \item \methoddesc{\proted}{}{void}{setAxisCoordinate}{\type{int} index, \type{Number} position}
        {
            Sets the specified coordinate in \refer{\lblroot:view.diagrams.components:PositionInDiagram:positionsInAxes}{positionsInAxes} with the given one.
        }{
            \begin{itemize}
                \item \paramdesc{index}{The index of the wanted coordinate in \refer{\lblroot:view.diagrams.components:PositionInDiagram:positionsInAxes}{positionsInAxes}.}
                \item \paramdesc{position}{The new value of the coordinate at the specified index.}
            \end{itemize}
        }{}
        \item \methoddesc{\proted}{}{void}{setAxisCoordinates}{\type{Number[]} coordinates}
        {
            Sets \refer{\lblroot:view.diagrams.components:PositionInDiagram:positionsInAxes}{positionsInAxes} with the given new coordinates.
        }{
            \begin{itemize}
                \item \paramdesc{coordinates}{The coordinates to replace \refer{\lblroot:view.diagrams.components:PositionInDiagram:positionsInAxes}{positionsInAxes}}
            \end{itemize}
        }{}
        \item \methoddesc{\proted}{}{Number}{getAxisPos}{\type{int} index}
        {}{
            \begin{itemize}
                \item \paramdesc{index}{}
            \end{itemize}
        }{
            The specified coordinate in \refer{\lblroot:view.diagrams.components:PositionInDiagram:positionsInAxes}{positionsInAxes}.
        }
    \end{itemize}
}

\absclassdesc{DiagramComponent}{
    \decl{\public \abstr \class DiagramComponent}{The abstract class that is implemented by classes that represents parts of \refer{\lblroot:view.diagrams.type:Diagram}{Diagrams}.}
}{
    \begin{itemize}
        \item \atrdesc{\private}{}{Color}{color}{The color of the instance.}
    \end{itemize}
}{
    \begin{itemize}
        \item \constrdesc{\proted}{DiagramComponent}{\type{Color} color}{}{
            \begin{itemize}
                \item \paramdesc{color}{The color of the instance.}
            \end{itemize}
        }
    \end{itemize}
}{
    \begin{itemize}
        \item \subclsdec{DiagramValueDisplayComponent}
        \item \subclsdec{DiagramAxis}
        \item \subclsdec{DiagramLabel}
        \item \subclsdec{DiagramLine}
        \item \subclsdec{DiagramColorScale}
    \end{itemize}
}{
    \begin{itemize}
        \item \methoddesc{\public}{\abstr}{DiagramComponent}{clone}{}
        {}{}{
            A deep copy of the instance.
        }
        \item \methoddesc{\public}{}{void}{setColor}{\type{Color} color}
        {
            Sets the \refer{\lblroot:view.diagrams.components:DiagramComponent:color}{color} of the instance with the given one.
        }{
            \begin{itemize}
                \item \paramdesc{color}{The new color of the instance}
            \end{itemize}
        }{}
        \item \methoddesc{\public}{}{Color}{getColor}{}
        {}{}{
            The \refer{\lblroot:view.diagrams.components:DiagramComponent:color}{color} attribute.
        }
        \item \methoddesc{\public}{\abstr}{void}{show}{}
        {
            Shows the instance.
        }{}{}
        \item \methoddesc{\public}{\abstr}{void}{hide}{}
        {
            Hides the instance.
        }{}{}
    \end{itemize}
}

\absclassdesc{DiagramValueDisplayComponent}{
    \decl{\public \abstr \class DiagramValueDisplayComponent}{The abstract class that is implemented by classes that represents parts of \refer{\lblroot:view.diagrams.type:Diagram}{Diagrams}, which are responsible for displaying values inside \refer{\lblroot:view.diagrams.type:Diagram}{Diagrams}.}
}{
    \begin{itemize}
        \item \atrdesc{\private}{}{Number}{value}{The value, which will be represented by the instance on the \refer{\lblroot:view.diagrams.type:Diagram}{Diagrams}.}
    \end{itemize}
}{
    \begin{itemize}
        \item \subclsdec{DiagramBar}
        \item \subclsdec{DiagramValueLabel}
        \item \subclsdec{DiagramPoint}
    \end{itemize}
}{
    \begin{itemize}
        \item \constrdesc{\proted}{DiagramValueDisplayComponent}{\type{Color} color, \type{Number} value}{}{
            \begin{itemize}
                \item \paramdesc{color}{The color of the instance.}
                \item \paramdesc{value}{The value, which will be represented by the instance on the \refer{\lblroot:view.diagrams.type:Diagram}{Diagrams}.}
            \end{itemize}
        }
    \end{itemize}
}{
    \begin{itemize}
        \item \methoddesc{\public}{}{void}{setValue}{}
        {
            Sets the \refer{\lblroot:view.diagrams.components:DiagramValueDisplayComponent:value}{value} attribute.
        }{}{}
        \item \methoddesc{\public}{}{Number}{getValue}{}
        {}{}{
            The \refer{\lblroot:view.diagrams.components:DiagramValueDisplayComponent:value}{value} attribute.
        }
        \item \methoddesc{\public}{\abstr}{void}{refreshValueRelevantAttributes}{}
        {
            Performs the appropriate actions when the \refer{\lblroot:view.diagrams.components:DiagramValueDisplayComponent:value}{value} attribute is modified.
        }{}{}
    \end{itemize}
}

\absclassdesc{DiagramBar}{
    \decl{\public \abstr \class DiagramBar}{The abstract class inherited by the classes that represent bars of \refer{\lblroot:view.diagrams.type:Diagram}{Diagrams}.}
}{
    \begin{itemize}
        \item \atrdesc{\private}{}{PositionIn2DDiagram}{bottomLeft}{The bottom left coordinates of the bar instance in the \refer{\lblroot:view.diagrams.type:Diagram}{Diagram} (on axes).}
        \item \atrdesc{\private}{}{PositionIn2DDiagram}{topRight}{The top right coordinates of the bar instance in the \refer{\lblroot:view.diagrams.type:Diagram}{Diagram} (on axes).}
        \item \atrdesc{\private}{}{Number}{borderThickness}{The thickness of the borders of the bar instance.}
    \end{itemize}
}{
    \begin{itemize}
        \item \subclsdec{HistogramBar}
        \item \subclsdec{BarChartBar}
    \end{itemize}
}{
    \begin{itemize}
        \item \constrdesc{\proted}{DiagramBar}{\type{Color} color, \type{Number} value, \type{PositionIn2DDiagram} bottomLeft, \type{PositionIn2DDiagram} topRight, \type{Number} borderThickness}{}{
            \begin{itemize}
                \item \paramdesc{color}{The color of the bar instance.}
                \item \paramdesc{value}{The value to be represented by the bar instance.}
                \item \paramdesc{bottomLeft}{The bottom left coordinates of the bar instance in the \refer{\lblroot:view.diagrams.type:Diagram}{Diagram} (on axes).}
                \item \paramdesc{topRight}{The top right coordinates of the bar instance in the \refer{\lblroot:view.diagrams.type:Diagram}{Diagram} (on axes).}
                \item \paramdesc{borderThickness}{The thickness of the borders of the bar instance.}
            \end{itemize}
        }
    \end{itemize}
}{
    \begin{itemize}
        \item \methoddesc{\public}{}{Number}{getHeight}{}
        {}{}{
            The height of the bar instance.
        }
        \item \methoddesc{\public}{}{Number}{getWidth}{}
        {}{}{
            The width of the bar instance
        }
        \item \methoddesc{\public}{}{void}{setBottomLeftInDiagram}{\type{Number} x1, \type{Number} y1}
        {}{
            \begin{itemize}
                \item \paramdesc{x1}{The new x-Coordinate of the bottom left corner.}
                \item \paramdesc{y1}{The new y-Coordinate of the bottom left corner.}
            \end{itemize}
        }{
            Sets \refer{\lblroot:view.diagrams.components:DiagramBar:bottomLeft}{bottomLeft}.
        }
        \item \methoddesc{\public}{}{void}{setTopRightInDiagram}{\type{Number} x2, \type{Number} y2}
        {}{
            \begin{itemize}
                \item \paramdesc{x2}{The new x-Coordinate of the top right corner.}
                \item \paramdesc{y2}{The new y-Coordinate of the top right corner.}
            \end{itemize}
        }{
            Sets \refer{\lblroot:view.diagrams.components:DiagramBar:topRight}{topRight}.
        }
        \item \methoddesc{\public}{}{PositionIn2DDiagram}{getBottomLeftInDiagram}{}
        {}{}{
            \refer{\lblroot:view.diagrams.components:DiagramBar:bottomLeft}{bottomLeft}
        }
        \item \methoddesc{\public}{}{PositionIn2DDiagram}{getTopRightInDiagram}{}
        {}{}{
            \refer{\lblroot:view.diagrams.components:DiagramBar:topRight}{topRight}.
        }
        \item \ovrdnmethoddesc{\public}{\abstr}{void}{refreshValueRelevantAttributes}{}{DiagramValueDisplayComponent}
        {
            Sets the y-Coordinate of \refer{\lblroot:view.diagrams.components:DiagramBar:topRight}{topRight} and the value attribute to correlate each other.
        }{}{}
    \end{itemize}
}

\absclassdesc{DiagramAxis}{
    \decl{\public \abstr \class DiagramAxis}{The abstract class inherited by the classes that represent axes in a \refer{\lblroot:view.diagrams.type:Diagram}{Diagram}.}
}{
    \begin{itemize}
        \item \atrdesc{\private}{}{Number}{min}{The minimum value on the axis.}
        \item \atrdesc{\private}{}{Number}{max}{The maximum value on the axis.}
        \item \atrdesc{\private}{}{int}{steps}{The amount of partitions the axis will have.}
        \item \overridecasclabelname{\lblroot\lblpackage\lblpackageelement:(atr)showValues}{\atrdesc{\private}{}{boolean}{showValues}{Indicates whether the values for each partition is shown.}}
        \item \atrdesc{\private}{}{DiagramLine}{axisLine}{The \refer{\lblroot:view.diagrams.components:DiagramLine}{DiagramLine} part of the axis.}
    \end{itemize}
}{
    \begin{itemize}
        \item \subclsdec{SolidAxis}
    \end{itemize}
}{
    \begin{itemize}
        \item \constrdesc{\proted}{DiagramAxis}{\type{DiagramLine} axisLine, \type{Number} min, \type{Number} max, \type{int} steps}{}{
            \begin{itemize}
                \item \paramdesc{axisLine}{The \refer{\lblroot:view.diagrams.components:DiagramLine}{DiagramLine} part of the axis.}
                \item \paramdesc{min}{The minimum value on the axis.}
                \item \paramdesc{max}{The maximum value on the axis.}
                \item \paramdesc{steps}{The amount of partitions the axis will have.}
            \end{itemize}
        }
    \end{itemize}
}{
    \begin{itemize}
        \item \methoddesc{\public}{}{void}{setMin}{\type{Number} min}
        {
            Sets \refer{\lblroot:view.diagrams.components:DiagramAxis:min}{min} attribute.
        }{
            \begin{itemize}
                \item \paramdesc{min}{The new \refer{\lblroot:view.diagrams.components:DiagramAxis:min}{min}.}
            \end{itemize}
        }{}
        \item \methoddesc{\public}{}{Number}{getMin}{}
        {}{}{
            \refer{\lblroot:view.diagrams.components:DiagramAxis:min}{min}.
        }
        \item \methoddesc{\public}{}{void}{setMax}{\type{Number} max}
        {
            Sets \refer{\lblroot:view.diagrams.components:DiagramAxis:max}{max} attribute.
        }{
            \begin{itemize}
                \item \paramdesc{max}{The new \refer{\lblroot:view.diagrams.components:DiagramAxis:max}{max}.}
            \end{itemize}
        }{}
        \item \methoddesc{\public}{}{Number}{getMax}{}
        {}{}{
            \refer{\lblroot:view.diagrams.components:DiagramAxis:min}{min}.
        }
        \item \methoddesc{\public}{}{void}{setSteps}{\type{int} steps}
        {
            Sets \refer{\lblroot:view.diagrams.components:DiagramAxis:steps}{steps} attribute.
        }{
            \begin{itemize}
                \item \paramdesc{steps}{The new \refer{\lblroot:view.diagrams.components:DiagramAxis:steps}{steps}.}
            \end{itemize}
        }{}
        \item \methoddesc{\public}{}{int}{getSteps}{}
        {}{}{
            \refer{\lblroot:view.diagrams.components:DiagramAxis:steps}{steps}.
        }
        \item \overridecasclabelname{\lblroot\lblpackage\lblpackageelement:(mtd)showValues}{\methoddesc{\public}{}{void}{showValues}{}
        {
            Shows the values painted on the partitions of the axis.
        }{}{}}
        \item \methoddesc{\public}{}{void}{hideValues}{}
        {
            Hides the values painted on the partitions of the axis.
        }{}{}
        \item \methoddesc{\public}{}{PositionInFrame}{valueToCoordinate}{\type{Number} value}
        {
            Transforms the given coordinate on the axis (\refer{\lblroot:view.diagrams.components:DiagramValueDisplayComponent:value}{value}) to the corresponding frame coordinates.
        }{
            \begin{itemize}
                \item \paramdesc{value}{The given coordinate on the axis.}
            \end{itemize}
        }{
            The corresponding frame coordinates.
        }
        \item \methoddesc{\public}{}{Number}{CoordinateToValue}{\type{PositionInFrame} coordinate}
        {
            Transforms the given coordinates on the frame to the corresponding axis coordinate (\refer{\lblroot:view.diagrams.components:DiagramValueDisplayComponent:value}{value}).
        }{
            \begin{itemize}
                \item \paramdesc{coordinate}{The given coordinates on the frame.}
            \end{itemize}
        }{
            The corresponding coordinate on the axis (\refer{\lblroot:view.diagrams.components:DiagramValueDisplayComponent:value}{value}).
        }
        \item \methoddesc{\public}{}{void}{setLineByPos}{\type{Number} minValXPos, \type{Number} minValYPos, \type{Number} maxValXPos, \type{Number} maxValYPos}
        {
            Sets the position of \refer{\lblroot:view.diagrams.components:DiagramAxis:axisLine}{axisLine} to the given ones.
        }{
            \begin{itemize}
                \item \paramdesc{minValXPos}{The x-Coordinate of where the minimum value on the axis will be.}
                \item \paramdesc{minValYPos}{The y-Coordinate of where the minimum value on the axis will be.}
                \item \paramdesc{maxValXPos}{The x-Coordinate of where the maximum value on the axis will be.}
                \item \paramdesc{maxValYPos}{The y-Coordinate of where the maximum value on the axis will be.}
            \end{itemize}
        }{}
        \item \methoddesc{\public}{}{void}{setLineColor}{\type{Color} color}
        {
            Sets the color of the \refer{\lblroot:view.diagrams.components:DiagramAxis:axisLine}{axisLine}.
        }{
            \begin{itemize}
                \item \paramdesc{color}{The new color of \refer{\lblroot:view.diagrams.components:DiagramAxis:axisLine}{axisLine}.}
            \end{itemize}
        }{}
        \item \methoddesc{\public}{}{void}{setLineThickness}{\type{Number} thickness}
        {
            Sets the thickness of the \refer{\lblroot:view.diagrams.components:DiagramAxis:axisLine}{axisLine}.
        }{
            \begin{itemize}
                \item \paramdesc{thickness}{The new thickness of the \refer{\lblroot:view.diagrams.components:DiagramAxis:axisLine}{axisLine}.}
            \end{itemize}
        }{}
        \item \methoddesc{\public}{}{Number}{getLineLength}{}
        {}{}{
            The length of \refer{\lblroot:view.diagrams.components:DiagramAxis:axisLine}{axisLine}
        }
        \item \ovrdnmethoddesc{\public}{}{void}{show}{}{DiagramComponent}
        {
            Shows the axis.
        }{}{}
        \item \ovrdnmethoddesc{\public}{}{void}{hide}{}{DiagramComponent}
        {
            Hides the axis.
        }{}{}
    \end{itemize}
}

\absclassdesc{DiagramLabel}{
    \decl{\public \abstr \class DiagramLabel}{The abstract class inherited by classes that represent labels that are not responsible for displaying values on \refer{\lblroot:view.diagrams.type:Diagram}{Diagrams}.}
}{
    \begin{itemize}
        \item \atrdesc{\private}{}{String}{caption}{The text that will be displayed by the label.}
        \item \atrdesc{\private}{}{PositionInFrame}{bottomLeft}{The position of the bottom left corner of the label in the frame.}
        \item \atrdesc{\private}{}{PositionInFrame}{topRight}{The position of the top right corner of the label in the frame.}
        \item \atrdesc{\private}{}{Number}{borderThickness}{The thickness of the border of the label.}
    \end{itemize}
}{
    \begin{itemize}
        \item \subclsdec{DescriptionLabel}
    \end{itemize}
}{
    \begin{itemize}
        \item \constrdesc{\proted}{DiagramLabel}{\type{PositionInFrame} bottomLeft, \type{PositionInFrame} topRight, \type{Color} color, \type{String} caption, \type{Number} borderThickness}{}{
            \begin{itemize}
                \item \paramdesc{bottomLeft}{The position of the bottom left corner of the label in the frame.}
                \item \paramdesc{topRight}{The position of the top right corner of the label in the frame.}
                \item \paramdesc{color}{The color of the label.}
                \item \paramdesc{caption}{The text that will be displayed by the label.}
                \item \paramdesc{borderThickness}{The thickness of the border of the label.}
            \end{itemize}
        }
    \end{itemize}
}{
    \begin{itemize}
        \item \methoddesc{\public}{}{void}{setCaption}{\type{String} caption}
        {
            Changes \refer{\lblroot:view.diagrams.components:DiagramLabel:caption}{caption} to the given one.
        }{
            \begin{itemize}
                \item \paramdesc{caption}{The text that will be displayed by the label.}
            \end{itemize}
        }{}
        \item \methoddesc{\public}{}{String}{getCaption}{}
        {}{}{
            \refer{\lblroot:view.diagrams.components:DiagramLabel:caption}{caption}.
        }
        \item \methoddesc{\public}{}{void}{setBottomLeftInFrame}{\type{Number} x1, \type{Number} y1}
        {}{
            \begin{itemize}
                \item \paramdesc{x1}{The new x-Coordinate of the bottom left corner.}
                \item \paramdesc{y1}{The new y-Coordinate of the bottom left corner.}
            \end{itemize}
        }{}
        \item \methoddesc{\public}{}{void}{setTopRightInFrame}{\type{Number} x2, \type{Number} y2}
        {}{
            \begin{itemize}
                \item \paramdesc{x2}{The new x-Coordinate of the top right corner.}
                \item \paramdesc{y2}{The new y-Coordinate of the top right corner.}
            \end{itemize}
        }{}
        \item \methoddesc{\public}{}{PositionInFrame}{getBottomLeftInFrame}{}
        {}{}{
            \refer{\lblroot:view.diagrams.components:DiagramLabel:bottomLeft}{bottomLeft}
        }
        \item \methoddesc{\public}{}{PositionInFrame}{getTopRightInFrame}{}
        {}{}{
            \refer{\lblroot:view.diagrams.components:DiagramLabel:topRight}{topRight}
        }
    \end{itemize}
}

\absclassdesc{DiagramLine}{
    \decl{\public \abstr \class DiagramLine}{The abstract class inherited by the classes that represent a line in a \refer{\lblroot:view.diagrams.type:Diagram}{Diagram}.}
}{
    \begin{itemize}
        \item \atrdesc{\private}{}{PositionInFrame}{start}{The position (in frame) of the start of the line (x1, y1).}
        \item \atrdesc{\private}{}{PositionInFrame}{end}{The position (in frame) of the end of the line (x2, y2).}
        \item \atrdesc{\private}{}{Number}{thickness}{The thickness of the line.}
    \end{itemize}
}{
    \begin{itemize}
        \item \subclsdec{SolidLine}
        \item \subclsdec{CoordinateIndicatorLine}
        \item \subclsdec{ValueLine}
    \end{itemize}
}{
    \begin{itemize}
        \item \constrdesc{\proted}{DiagramLine}{\type{PositionInFrame} start, \type{PositionInFrame} end, \type{Color} color, \type{Number} thickness}{}{
            \begin{itemize}
                \item \paramdesc{start}{The position (in frame) of the start of the line (x1, y1).}
                \item \paramdesc{end}{The position (in frame) of the end of the line (x2, y2).}
                \item \paramdesc{color}{The color of the line.}
                \item \paramdesc{thickness}{The thickness of the line.}
            \end{itemize}
        }
    \end{itemize}
}{
    \begin{itemize}
        \item \methoddesc{\proted}{}{Number}{calculateLength}{}
        {}{}{
            The length of the line.
        }
        \item \methoddesc{\public}{}{void}{setThickness}{\type{Number} thickness}
        {
            Set \refer{\lblroot:view.diagrams.components:DiagramLine:thickness}{thickness} to the given thickness.
        }{
            \begin{itemize}
                \item \paramdesc{thickness}{The given thickness.}
            \end{itemize}
        }{}
        \item \methoddesc{\public}{}{void}{setStartInFrame}{\type{Number} x1, \type{Number} y1}
        {}{
            \begin{itemize}
                \item \paramdesc{x1}{The new x-Coordinate of the start of the line.}
                \item \paramdesc{y1}{The new y-Coordinate of the start of the line.}
            \end{itemize}
        }{}
        \item \methoddesc{\public}{}{void}{setEndInFrame}{\type{Number} x2, \type{Number} y2}
        {}{
            \begin{itemize}
                \item \paramdesc{x2}{The new x-Coordinate of the end of the line.}
                \item \paramdesc{y2}{The new y-Coordinate of the end of the line.}
            \end{itemize}
        }{}
        \item \methoddesc{\public}{}{PositionInFrame}{getStartInFrame}{}
        {}{}{
            \refer{\lblroot:view.diagrams.components:DiagramLine:start}{start}.
        }
        \item \methoddesc{\public}{}{PositionInFrame}{getEndInFrame}{}
        {}{}{
            \refer{\lblroot:view.diagrams.components:DiagramLine:end}{end}.
        }
        \item \methoddesc{\public}{}{Number}{getLength}{}
        {}{}{
            The length of the line, based on \refer{\lblroot:view.diagrams.components:DiagramLine:start}{start} and \refer{\lblroot:view.diagrams.components:DiagramLine:end}{end}.
        }
        \item \methoddesc{\public}{}{Number}{getThickness}{}
        {}{}{
            \refer{\lblroot:view.diagrams.components:DiagramLine:thickness}{thickness}.
        }
    \end{itemize}
}

\absclassdesc{DiagramValueLabel}{
    \decl{\public \abstr \class DiagramValueLabel}{The abstract class implemented by the classes that represent labels that are responsible for displaying values.}
}{
    \begin{itemize}
        \item \atrdesc{\private}{}{PositionIn2DDiagram}{topRight}{The position of the top right corner of the label.}
        \item \atrdesc{\private}{}{PositionIn2DDiagram}{bottomLeft}{The position of the bottom left corner of the label.}
        \item \atrdesc{\private}{}{String}{caption}{The text that the label will display.}
        \item \atrdesc{\private}{}{Number}{borderThickness}{The thickness of the border of the label.}
    \end{itemize}
}{
    \begin{itemize}
        \item \subclsdec{HeatMapLabel}
    \end{itemize}
}{
    \begin{itemize}
        \item \constrdesc{\proted}{DiagramValueLabel}{\type{PositionIn2DDiagram} bottomLeft, \type{PositionIn2DDiagram} topRight, \type{Color} color, \type{Number} value, \type{Number} borderThickness}{}{
            \begin{itemize}
                \item \paramdesc{bottomLeft}{The position of the bottom left corner of the label.}
                \item \paramdesc{topRight}{The position of the top right corner of the label.}
                \item \paramdesc{color}{The color of the label.}
                \item \paramdesc{value}{The value that will be displayed by the label.}
                \item \paramdesc{borderThickness}{The thickness of the border of the label.}
            \end{itemize}
        }
    \end{itemize}
}{
    \begin{itemize}
        \item \methoddesc{\proted}{}{void}{refreshCaption}{}
        {
            Refreshes the \refer{\lblroot:view.diagrams.components:DiagramValueLabel:caption}{caption}, when events that affect the caption occur.
        }{}{}
        \item \methoddesc{\proted}{}{void}{setCaption}{\type{String} caption}
        {
            Sets the \refer{\lblroot:view.diagrams.components:DiagramValueLabel:caption}{caption}.
        }{
            \begin{itemize}
                \item \paramdesc{caption}{The new \refer{\lblroot:view.diagrams.components:DiagramValueLabel:caption}{caption}.}
            \end{itemize}
        }{}
        \item \methoddesc{\public}{}{String}{getCaption}{}
        {}{}{
            \refer{\lblroot:view.diagrams.components:DiagramValueLabel:caption}{caption}.
        }
        \item \methoddesc{\public}{}{void}{setBottomLeftInDiagram}{\type{Number} x1, \type{Number} y1}
        {}{
            \begin{itemize}
                \item \paramdesc{x1}{The new x-Coordinate of the bottom left corner.}
                \item \paramdesc{y1}{The new y-Coordinate of the bottom left corner.}
            \end{itemize}
        }{}
        \item \methoddesc{\public}{}{void}{setTopRightInDiagram}{\type{Number} x2, \type{Number} y2}
        {}{
            \begin{itemize}
                \item \paramdesc{x2}{The new x-Coordinate of the top right corner.}
                \item \paramdesc{y2}{The new y-Coordinate of the top right corner.}
            \end{itemize}
        }{}
        \item \methoddesc{\public}{}{PositionIn2DDiagram}{getBottomLeftInDiagram}{}
        {}{}{
            \refer{\lblroot:view.diagrams.components:DiagramValueLabel:bottomLeft}{bottomLeft}
        }
        \item \methoddesc{\public}{}{PositionIn2DDiagram}{getTopRightInDiagram}{}
        {}{}{
            \refer{\lblroot:view.diagrams.components:DiagramValueLabel:topRight}{topRight}
        }
    \end{itemize}
}

\absclassdesc{DiagramPoint}{
    \decl{\public \abstr \class DiagramPoint}{The abstract class inherited by the classes that represent a point on a \refer{\lblroot:view.diagrams.type:Diagram}{Diagram}.}
}{
    \begin{itemize}
        \item \atrdesc{\private}{}{PositionIn2DDiagram}{position}{The position of the point in the \refer{\lblroot:view.diagrams.type:Diagram}{Diagram}.}
        \item \atrdesc{\private}{}{Number}{size}{The size of the point.}
    \end{itemize}
}{
    \begin{itemize}
        \item \subclsdec{ValueDisplayPoint}
    \end{itemize}
}{
    \begin{itemize}
        \item \constrdesc{\proted}{DiagramPoint}{\type{PositionIn2DDiagram} position, \type{Color} color, \type{Number} value, \type{Number} size}{}{
            \begin{itemize}
                \item \paramdesc{position}{The position of the point in the \refer{\lblroot:view.diagrams.type:Diagram}{Diagram}.}
                \item \paramdesc{color}{The color of the point.}
                \item \paramdesc{value}{The value that the point represents.}
                \item \paramdesc{size}{The size of the point.}
            \end{itemize}
        }
    \end{itemize}
}{
    \begin{itemize}
        \item \methoddesc{\public}{}{void}{setSize}{\type{Number} size}
        {
            Sets the \refer{\lblroot:view.diagrams.components:DiagramPoint:size}{size}.
        }{
            \begin{itemize}
                \item \paramdesc{size}{The new \refer{\lblroot:view.diagrams.components:DiagramPoint:size}{size}.}
            \end{itemize}
        }{}
        \item \methoddesc{\public}{}{Number}{getSize}{}
        {}{}{
            \refer{\lblroot:view.diagrams.components:DiagramPoint:size}{size}.
        }
        \item \methoddesc{\public}{}{void}{setPositionInDiagram}{\type{Number} x, \type{Number} y}
        {}{
            \begin{itemize}
                \item \paramdesc{x}{The new x-Coordinate of the point in the \refer{\lblroot:view.diagrams.type:Diagram}{Diagram}.}
                \item \paramdesc{y}{The new y-Coordinate of the point in the \refer{\lblroot:view.diagrams.type:Diagram}{Diagram}.}
            \end{itemize}
        }{}
        \item \methoddesc{\public}{}{PositionIn2DDiagram}{getPositionInDiagram}{}
        {}{}{
            \refer{\lblroot:view.diagrams.components:DiagramPoint:position}{position}.
        }
    \end{itemize}
}

\absclassdesc{DiagramColorScale}{
    \decl{\public \abstr \class}{The abstract class inherited by the classes that represent color scales in \refer{\lblroot:view.diagrams.type:Diagram}{Diagrams}. \descnote{The color scale will have an array of values and another array of the same size with colors. Each value that is not in the array, which lies in the range of the minimum and the maximum value, will have a mixture of the colors of the values, to which it is the nearest to.}}
}{
    \begin{itemize}
        \item \atrdesc{\private}{}{PositionInFrame}{bottomLeft}{The position of the bottom left corner of the color scale.}
        \item \atrdesc{\private}{}{PositionInFrame}{topRight}{The position of the top right corner of the color scale.}
        \item \atrdesc{\private}{}{Number}{borderThickness}{The thickness of the border of the color scale.}
        \item \atrdesc{\private}{}{Number[]}{values}{The values that have a certain color.}
        \item \atrdesc{\private}{}{Color[]}{valueColors}{The colors that represent the values.}
    \end{itemize}
}{
    \begin{itemize}
        \item \subclsdec{2ColorScale}
    \end{itemize}
}{
    \begin{itemize}
        \item \constrdesc{\proted}{}{\type{PositionInFrame} bottomLeft, \type{PositionInFrame} topRight, \type{Color} borderColor, \type{Number[]} values, \type{Color[]} valueColors, \type{Number} borderThickness}
        {
            Constructs and initializes the instance. \refer{\lblroot:view.diagrams.components:DiagramColorScale:values}{Values} and \refer{\lblroot:view.diagrams.components:DiagramColorScale:valueColors}{ValueColors} must have the same sizes. A value with the index i v[i] will be associated with the valueColor at the same index c[i].
        }{
            \begin{itemize}
                \item \paramdesc{bottomLeft}{The position of the bottom left corner of the color scale.}
                \item \paramdesc{topRight}{The position of the top right corner of the color scale.}
                \item \paramdesc{borderColor}{The color of the border of the color scale.}
                \item \paramdesc{values}{The values that have a certain color.}
                \item \paramdesc{valueColors}{The colors that represent the values.}
                \item \paramdesc{borderThickness}{The thickness of the border of the color scale.}
            \end{itemize}
        }
    \end{itemize}
}{
    \begin{itemize}
        \item \methoddesc{\public}{}{Color}{valueToColor}{\type{Number} value}
        {}{
            \begin{itemize}
                \item \paramdesc{value}{The given value.}
            \end{itemize}
        }{
            The color, which corresponds to the given value.
        }
        \item \methoddesc{\public}{}{Number[]}{getValues}{}
        {}{}{
            \refer{\lblroot:view.diagrams.components:DiagramColorScale:values}{values}.
        }
        \item \methoddesc{\public}{}{Color[]}{getColors}{}
        {}{}{
            \refer{\lblroot:view.diagrams.components:DiagramColorScale:valueColors}{valueColors}.
        }
        \item \methoddesc{\public}{}{void}{setBottomLeftInFrame}{\type{Number} x1, \type{Number} y1}
        {}{
            \begin{itemize}
                \item \paramdesc{x1}{The new x-Coordinate of the bottom left corner of the color scale.}
                \item \paramdesc{y1}{The new y-Coordinate of the bottom left corner of the color scale.}
            \end{itemize}
        }{}
        \item \methoddesc{\public}{}{void}{setTopRightInFrame}{\type{Number} x2, \type{Number} y2}
        {}{
            \begin{itemize}
                \item \paramdesc{x2}{The new x-Coordinate of the top right corner of the color scale.}
                \item \paramdesc{y2}{The new y-Coordinate of the top right corner of the color scale.}
            \end{itemize}
        }{}
        \item \methoddesc{\public}{}{PositionInFrame}{getBottomLeftInFrame}{}
        {}{}{
            \refer{\lblroot:view.diagrams.components:DiagramColorScale:bottomLeft}{bottomLeft}.
        }
        \item \methoddesc{\public}{}{PositionInFrame}{getTopRightInFrame}{}
        {}{}{
            \refer{\lblroot:view.diagrams.components:DiagramColorScale:topRight}{topRight}.
        }
    \end{itemize}
}

\classdesc{PositionInFrame}{
    \decl{\public \class PositionInFrame}{The class, which represents a position in the frame by storing the x- and y-Coordinates.}
}{
    \begin{itemize}
        \item \atrdesc{\private}{}{Number}{xPos}{The x-Coordinate in the frame.}
        \item \atrdesc{\private}{}{Number}{yPos}{The y-Coordinate in the frame.}
    \end{itemize}
}{
    \begin{itemize}
        \item \constrdesc{\public}{PositionInFrame}{\type{Number} xPos, \type{Number} yPos}{}{
            \begin{itemize}
                \item \paramdesc{xPos}{The x-Coordinate in the frame.}
                \item \paramdesc{yPos}{The y-Coordinate in the frame.}
            \end{itemize}
        }
    \end{itemize}
}{
    \begin{itemize}
        \item \methoddesc{\public}{}{Number}{getXPos}{}
        {}{}{
            \refer{\lblroot:view.diagrams.components:PositionInFrame:xPos}{xPos}.
        }
        \item \methoddesc{\public}{}{void}{setXPos}{\type{Number} xPos}
        {
            Sets \refer{\lblroot:view.diagrams.components:PositionInFrame:xPos}{xPos} to the given one.
        }{
            \begin{itemize}
                \item \paramdesc{xPos}{The new x-Coordinate in the frame.}
            \end{itemize}
        }{}
        \item \methoddesc{\public}{}{Number}{getYPos}{}
        {}{}{
            \refer{\lblroot:view.diagrams.components:PositionInFrame:yPos}{yPos}.
        }
        \item \methoddesc{\public}{}{void}{setYPos}{\type{Number} yPos}
        {
            Sets \refer{\lblroot:view.diagrams.components:PositionInFrame:yPos}{yPos} to the given one.
        }{
            \begin{itemize}
                \item \paramdesc{yPos}{The new y-Coordinate in the frame.}
            \end{itemize}
        }{}
    \end{itemize}
}

\classdesc{PositionIn2DDiagram}{
    \decl{\public \class PositionIn2DDiagram}{The class that represents a position in a 2 dimensional \refer{\lblroot:view.diagrams:IDiagram}{IDiagram} according to its x- and y-axis.\descnote{The x- and y-coordinates and axes will be stored in \refer{\lblroot:view.diagrams.components:PositionInDiagram:positionsInAxes}{positionsInAxes} and \refer{\lblroot:view.diagrams.components:PositionInDiagram:axes}{axes} attributes of the \refer{\lblroot:view.diagrams.components:PositionInDiagram}{PositionInDiagram} instance.}}
}{

}{
    \begin{itemize}
        \item \constrdesc{\public}{PositionIn2DDiagram}{\type{DiagramAxis} xAxis, \type{Number} xCoordinate, \type{DiagramAxis} yAxis, \type{Number} yCoordinate}{}{
            \begin{itemize}
                \item \paramdesc{xAxis}{The x-axis of the \refer{\lblroot:view.diagrams:IDiagram}{IDiagram}.}
                \item \paramdesc{xPos}{The x-Coordinate on the x-axis of the \refer{\lblroot:view.diagrams:IDiagram}{IDiagram}.}
                \item \paramdesc{yAxis}{The y-axis of the \refer{\lblroot:view.diagrams:IDiagram}{IDiagram}.}
                \item \paramdesc{yPos}{The y-Coordinate on the x-axis of the \refer{\lblroot:view.diagrams:IDiagram}{IDiagram}.}
            \end{itemize}
        }
    \end{itemize}
}{
    \begin{itemize}
        \item \methoddesc{\public}{}{void}{setXCoordinate}{\type{Number} xCoordinate}
        {
            Sets the x-coordinate to the given one.
        }{
            \begin{itemize}
                \item \paramdesc{xCoordinate}{The given x-coordinate.}
            \end{itemize}
        }{}
        \item \methoddesc{\public}{}{Number}{getXCoordinate}{}
        {}{}{
            The x-coordinate.
        }
        \item \methoddesc{\public}{}{void}{setYCoordinate}{\type{Number} yCoordinate}
        {
            Sets the y-coordinate to the given one.
        }{
            \begin{itemize}
                \item \paramdesc{yCoordinate}{The given y-coordinate.}
            \end{itemize}
        }{}
        \item \methoddesc{\public}{}{Number}{getYCoordinate}{}
        {}{}{
            The y-coordinate.
        }
    \end{itemize}
}

\classdesc{SolidLine}{
    \decl{\public \class SolidLine}{Represents a solid line in an \refer{\lblroot:view.diagrams:IDiagram}{IDiagram}.}
}{
    \begin{itemize}
        \item \atrdesc{\private}{}{Line}{line}{The line object specified by the constructor parameters.}
    \end{itemize}
}{
    \begin{itemize}
        \item \constrdesc{\proted}{SolidLine}{\type{PositionInFrame} start, \type{PositionInFrame} end, \type{Color} color, \type{Number} thickness}{}{
            \begin{itemize}
                \item \paramdesc{start}{The start position of the line in the frame (x1, y1).}
                \item \paramdesc{end}{The end position of the line in the frame (x2, y2).}
                \item \paramdesc{color}{The color of the line.}
                \item \paramdesc{thickness}{The thickness of the line.}
            \end{itemize}
        }
    \end{itemize}
}{
    \begin{itemize}
        \item \ovrdnmethoddesc{\public}{}{void}{show}{}{DiagramComponent}
        {
            Shows the SolidLine.
        }{}{}
        \item \ovrdnmethoddesc{\public}{}{void}{hide}{}{DiagramComponent}
        {
            Hides the SolidLine.
        }{}{}
        \item \ovrdnmethoddesc{\public}{}{DiagramComponent}{clone}{}{DiagramComponent}
        {}{}{
            A deep copy of the SolidLine.
        }
    \end{itemize}
}

\classdesc{2ColorScale}{
    \decl{\public \class 2ColorScale}{Represents a \refer{\lblroot:view.diagrams.components:DiagramColorScale}{DiagramColorScale} that has only 2 value-color pairs (stored in \refer{\lblroot:view.diagrams.components:DiagramColorScale:values}{values} and \refer{\lblroot:view.diagrams.components:DiagramColorScale:valueColors}{valueColors}).}
}{
    \begin{itemize}
        \item \atrdesc{\private}{}{WritableImage}{colorScale}{The part of the color scale that shows the colors for the values between minValue and maxValue.}
        \item \atrdesc{\private}{}{Color}{minValueColor}{The color of the minimum value in the color scale.}
        \item \atrdesc{\private}{}{Color}{maxValueColor}{The color of the maximum value in the color scale.}
        \item \atrdesc{\private}{}{Number}{minValue}{The minimum value in the color scale.}
        \item \atrdesc{\private}{}{Number}{maxValue}{The maximum value in the color scale.}
    \end{itemize}
}{
    \begin{itemize}
        \item \constrdesc{\proted}{2ColorScale}{\type{PositionInFrame} bottomLeft, \type{PositionInFrame} topRight, \type{Color} borderColor, \type{Number} minVal, \type{Number} maxVal, \type{Color} minValColor, \type{Color} maxValColor, \type{Number} borderThickness}{}{
            \begin{itemize}
                \item \paramdesc{bottomLeft}{The bottom left corner of the color scale in frame.}
                \item \paramdesc{topRight}{The top right corner of the color scale in frame.}
                \item \paramdesc{borderColor}{The color of the border of the color scale.}
                \item \paramdesc{minVal}{The minimum value in the color scale.}
                \item \paramdesc{maxVal}{The maximum value in the color scale.}
                \item \paramdesc{minValColor}{The color of the minimum value in the color scale.}
                \item \paramdesc{maxValColor}{The color of the maximum value in the color scale.}
                \item \paramdesc{borderThickness}{The thickness of the border of the color scale.}
            \end{itemize}
        }
    \end{itemize}
}{
    \begin{itemize}
        \item \methoddesc{\public}{}{void}{setMinValueColor}{\type{Color} minValueColor}
        {
            Sets \refer{\lblroot:view.diagrams.components:2ColorScale:minValueColor}{minValueColor}.
        }{
            \begin{itemize}
                \item \paramdesc{minValueColor}{The new \refer{\lblroot:view.diagrams.components:2ColorScale:minValueColor}{minValueColor}.}
            \end{itemize}
        }{}
        \item \methoddesc{\public}{}{Color}{getMinValueColor}{}
        {}{}{
            \refer{\lblroot:view.diagrams.components:2ColorScale:minValueColor}{minValueColor}.
        }
        \item \methoddesc{\public}{}{void}{setMaxValueColor}{\type{Color} maxValueColor}
        {
            Sets \refer{\lblroot:view.diagrams.components:2ColorScale:maxValueColor}{maxValueColor}.
        }{
            \begin{itemize}
                \item \paramdesc{maxValueColor}{The new \refer{\lblroot:view.diagrams.components:2ColorScale:maxValueColor}{maxValueColor}.}
            \end{itemize}
        }{}
        \item \methoddesc{\public}{}{Color}{getMaxValueColor}{}
        {}{}{
            \refer{\lblroot:view.diagrams.components:2ColorScale:maxValueColor}{maxValueColor}.
        }
        \item \methoddesc{\public}{}{void}{setMinValue}{\type{Number} minValue}
        {
            Sets \refer{\lblroot:view.diagrams.components:2ColorScale:minValue}{minValue}.
        }{
            \begin{itemize}
                \item \paramdesc{minValue}{The new \refer{\lblroot:view.diagrams.components:2ColorScale:minValue}{minValue}.}
            \end{itemize}
        }{}
        \item \methoddesc{\public}{}{Number}{getMinValue}{}
        {}{}{
            \refer{\lblroot:view.diagrams.components:2ColorScale:minValue}{minValue}.
        }
        \item \methoddesc{\public}{}{void}{setMaxValue}{\type{Number} maxValue}
        {
            Sets \refer{\lblroot:view.diagrams.components:2ColorScale:maxValue}{maxValue}.
        }{
            \begin{itemize}
                \item \paramdesc{maxValue}{The new \refer{\lblroot:view.diagrams.components:2ColorScale:maxValue}{maxValue}.}
            \end{itemize}
        }{}
        \item \methoddesc{\public}{}{Number}{getMaxValue}{}
        {}{}{
            \refer{\lblroot:view.diagrams.components:2ColorScale:maxValue}{maxValue}.
        }
        \item \ovrdnmethoddesc{\public}{}{void}{show}{}{DiagramComponent}
        {
            Shows the 2ColorScale.
        }{}{}
        \item \ovrdnmethoddesc{\public}{}{void}{hide}{}{DiagramComponent}
        {
            Hides the 2ColorScale.
        }{}{}
        \item \ovrdnmethoddesc{\public}{}{DiagramComponent}{clone}{}{DiagramComponent}
        {}{}{
            A deep copy of the 2ColorScale.
        }
    \end{itemize}
}

\classdesc{ValueDisplayPoint}{
    \decl{\public \class ValueDisplayPoint}{Represents a point that is responsible for displaying a value.}
}{
    \begin{itemize}
        \item \atrdesc{\private}{}{Point}{point}{The point, via which the \refer{\lblroot:view.diagrams.components:ValueDisplayPoint:ValueDisplayPoint:value}{value} will be displayed.}
    \end{itemize}
}{
    \begin{itemize}
        \item \constrdesc{\proted}{ValueDisplayPoint}{\type{Color} color, \type{Number} value, \type{Number} size, \type{PositionIn2DDiagram} position}{}{
            \begin{itemize}
                \item \paramdesc{color}{The color of the \refer{\lblroot:view.diagrams.components:ValueDisplayPoint:point}{point}.}
                \item \paramdesc{value}{The value that will be displayed via the \refer{\lblroot:view.diagrams.components:ValueDisplayPoint:point}{point}.}
                \item \paramdesc{size}{The size of the \refer{\lblroot:view.diagrams.components:ValueDisplayPoint:point}{point}.}
                \item \paramdesc{position}{The position of the \refer{\lblroot:view.diagrams.components:ValueDisplayPoint:point}{point} in the \refer{\lblroot:view.diagrams:IDiagram}{IDiagram} (x, y).}
            \end{itemize}
        }
    \end{itemize}
}{
    \begin{itemize}
        \item \ovrdnmethoddesc{\public}{}{void}{show}{}{DiagramComponent}
        {
            Shows the ValueDisplayPoint.
        }{}{}
        \item \ovrdnmethoddesc{\public}{}{void}{hide}{}{DiagramComponent}
        {
            Hides the ValueDisplayPoint.
        }{}{}
        \item \ovrdnmethoddesc{\public}{}{DiagramComponent}{clone}{}{DiagramComponent}
        {
            A deep copy of the ValueDisplayPoint.
        }{}{}
    \end{itemize}
}

\classdesc{HistogramBar}{
    \decl{\public \class HistogramBar}{Represents a bar used by a \refer{\lblroot:view.diagrams.type:Histogram}{Histogram}.}
}{
    \begin{itemize}
        \item \atrdesc{\private}{}{Label}{label}{The representation of the bar in form of a label.}
    \end{itemize}
}{
    \begin{itemize}
        \item \constrdesc{\proted}{HistogramBar}{\type{Color} color, \type{Number} value, \type{PositionIn2DDiagram} bottomLeft, \type{PositionIn2DDiagram} topRight}{}{
            \begin{itemize}
                \item \paramdesc{color}{The color of the bar.}
                \item \paramdesc{value}{The value that will be displayed via the bar.}
                \item \paramdesc{bottomLeft}{The bottom left corner of the bar in the \refer{\lblroot:view.diagrams:IDiagram}{IDiagram}}
                \item \paramdesc{topRight}{The top right corner of the bar in the \refer{\lblroot:view.diagrams:IDiagram}{IDiagram}.}
            \end{itemize}
        }
    \end{itemize}
}{
    \begin{itemize}
        \item \ovrdnmethoddesc{\public}{}{void}{show}{}{DiagramComponent}
        {
            Shows the HistogramBar.
        }{}{}
        \item \ovrdnmethoddesc{\public}{}{void}{hide}{}{DiagramComponent}
        {
            Hides the HistogramBar.
        }{}{}
        \item \ovrdnmethoddesc{\public}{}{DiagramComponent}{clone}{}{DiagramComponent}
        {}{}{
            A deep copy of the HistogramBar.
        }
    \end{itemize}
}

\classdesc{BarChartBar}{
    \decl{\public \class BarChartBar}{Representation of a bar used by a \refer{\lblroot:view.diagrams.type:BarChart}{BarChart}.}
}{
    \begin{itemize}
        \item \atrdesc{\private}{}{Label}{label}{The representation of the bar in form of a label.}
    \end{itemize}
}{
    \begin{itemize}
        \item \constrdesc{\proted}{BarChartBar}{\type{Color} color, \type{Number} value, \type{Number} width, \type{PositionIn2DDiagram} bottomLeft, \type{PositionIn2DDiagram} topRight}{}{
            \begin{itemize}
                \item \paramdesc{color}{The color of the bar.}
                \item \paramdesc{value}{The value that will be displayed via the bar.}
                \item \paramdesc{width}{The width of the bar.}
                \item \paramdesc{bottomLeft}{The bottom left corner of the bar in the \refer{\lblroot:view.diagrams:IDiagram}{IDiagram}.}
                \item \paramdesc{topRight}{The rop right corner of the bar in the \refer{\lblroot:view.diagrams:IDiagram}{IDiagram}.}
            \end{itemize}
        }
    \end{itemize}
}{
    \begin{itemize}
        \item \ovrdnmethoddesc{\public}{}{void}{show}{}{DiagramComponent}
        {
            Shows the BarChartBar.
        }{}{}
        \item \ovrdnmethoddesc{\public}{}{void}{hide}{}{DiagramComponent}
        {
            Hides the BarChartBar.
        }{}{}
        \item \ovrdnmethoddesc{\public}{}{DiagramComponent}{clone}{}{DiagramComponent}
        {}{}{
            A deep copy of the BarChartBar.
        }
    \end{itemize}
}

\classdesc{SolidAxis}{
    \decl{\public \class SolidAxis}{Represents an axis, whose line is a \refer{\lblroot:view.diagrams.components:SolidLine}{SolidLine}.}
}{
    \begin{itemize}
        \item \atrdesc{\private}{}{Line}{line}{The line object specified by the constructor parameters.}
    \end{itemize}
}{
\begin{itemize}
    \item \constrdesc{\proted}{SolidAxis}{\type{SolidLine} axisLine, \type{Number} min, \type{Number} max, \type{int} steps}{}{
        \begin{itemize}
            \item \paramdesc{axisLine}{The line part of the axis.}
            \item \paramdesc{min}{The minimum value on the axis.}
            \item \paramdesc{max}{The maximum value on the axis.}
            \item \paramdesc{steps}{The number of partitions on the axis.}
        \end{itemize}
    }
\end{itemize}
}{
    \begin{itemize}
        \item \ovrdnmethoddesc{\public}{}{void}{show}{}{DiagramComponent}
        {
            Shows the SolidAxis.
        }{}{}
        \item \ovrdnmethoddesc{\public}{}{void}{hide}{}{DiagramComponent}
        {
            Hides the SolidAxis.
        }{}{}
        \item \ovrdnmethoddesc{\public}{}{DiagramComponent}{clone}{}{DiagramComponent}
        {}{}{
            A deep copy of the SolidAxis.
        }
    \end{itemize}
}

\classdesc{HeatMapLabel}{
    \decl{\public \class HeatMapLabel}{Represents a field in a \refer{\lblroot:view.diagrams.type:HeatMap}{HeatMap}.}
}{
    \begin{itemize}
        \item \atrdesc{\private}{}{Label}{label}{The label that represents a field in a \refer{\lblroot:view.diagrams.type:HeatMap}{HeatMap}.}
    \end{itemize}
}{
    \begin{itemize}
        \item \constrdesc{\proted}{HeatMapLabel}{\type{DiagramColorScale} cs, \type{Number} value, \type{PositionIn2DDiagram} bottomLeft, \type{PositionIn2DDiagram} topRight}{}{
            \begin{itemize}
                \item \paramdesc{cs}{The color scale, according to which the colors for the value of the \refer{\lblroot:view.diagrams.components:HeatMapLabel:label}{label}.}
                \item \paramdesc{value}{The value that will be displayed via the \refer{\lblroot:view.diagrams.components:HeatMapLabel:label}{label}.}
                \item \paramdesc{bottomLeft}{The bottom left corner of the \refer{\lblroot:view.diagrams.components:HeatMapLabel:label}{label} in the \refer{\lblroot:view.diagrams:IDiagram}{IDiagram}.}
                \item \paramdesc{topRight}{The top right corner of the \refer{\lblroot:view.diagrams.components:HeatMapLabel:label}{label} in the \refer{\lblroot:view.diagrams:IDiagram}{IDiagram}.}
            \end{itemize}
        }
    \end{itemize}
}{
    \begin{itemize}
        \item \ovrdnmethoddesc{\public}{}{void}{show}{}{DiagramComponent}
        {
            Shows the HeatMapLabel.
        }{}{}
        \item \ovrdnmethoddesc{\public}{}{void}{hide}{}{DiagramComponent}
        {
            Hides the HeatMapLabel.
        }{}{}
        \item \ovrdnmethoddesc{\public}{}{DiagramComponent}{clone}{}{DiagramComponent}
        {}{}{
            A deep copy of the HeatMapLabel.
        }
    \end{itemize}
}

\classdesc{DescriptionLabel}{
    \decl{\public \class DescriptionLabel}{Represents a label that holds a description, which does not serve a value displaying purpose.}
}{
    \begin{itemize}
        \item \atrdesc{\private}{}{Label}{label}{The label that displays the \refer{\lblroot:view.diagrams.components:DescriptionLabel:DescriptionLabel:caption}{caption}.}
    \end{itemize}
}{
    \begin{itemize}
        \item \constrdesc{\proted}{DescriptionLabel}{\type{Color} color, \type{String} caption, \type{PositionInFrame} bottomLeft, \type{PositionInFrame} topRight}{}{
            \begin{itemize}
                \item \paramdesc{color}{The color of the \refer{\lblroot:view.diagrams.components:DescriptionLabel:label}{label}.}
                \item \paramdesc{caption}{The text the \refer{\lblroot:view.diagrams.components:DescriptionLabel:label}{label} will display.}
                \item \paramdesc{bottomLeft}{The bottom left corner of the \refer{\lblroot:view.diagrams.components:DescriptionLabel:label}{label}.}
                \item \paramdesc{topRight}{The top right corner of the \refer{\lblroot:view.diagrams.components:DescriptionLabel:label}{label}.}
            \end{itemize}
        }
    \end{itemize}
}{
    \begin{itemize}
        \item \ovrdnmethoddesc{\public}{}{void}{show}{}{DiagramComponent}
        {
            Shows the label.
        }{}{}
        \item \ovrdnmethoddesc{\public}{}{void}{hide}{}{DiagramComponent}
        {
            Hides the label.
        }{}{}
        \item \ovrdnmethoddesc{\public}{}{DiagramComponent}{clone}{}{DiagramComponent}
        {}{}{
            A deep copy of the DescriptionLabel.
        }
    \end{itemize}
}

\classdesc{HoverLabel}{
    \decl{\public \class HoverLabel}{Represents the label that will follow the mouse cursor and display some information about the instance from a class implementing \refer{\lblroot:view.diagrams.components:Hoverable}{Hoverable}.}
}{
    \begin{itemize}
        \item \atrdesc{\private}{}{String}{caption}{The text the label will display.}
        \item \atrdesc{\private}{}{Theme}{theme}{The theme the label will have.}
        \item \atrdesc{\private}{}{PositionInFrame}{mousePointer}{The position of the mouse pointer.}
        \item \atrdesc{\private}{}{Number}{width}{The width of the label.}
        \item \atrdesc{\private}{}{Number}{height}{The height of the label.}
        \item \atrdesc{\private}{}{Label}{label}{The label specified by the attributes.}
        \item \atrdesc{\private}{\static}{HoverLabel}{hoverLabel}{The only instance of the class.}
    \end{itemize}
}{
    \begin{itemize}
        \item \constrdesc{\private}{HoverLabel}{}{}{}
    \end{itemize}
}{
    \begin{itemize}
        \item \methoddesc{\public}{\static}{HoverLabel}{getHoverLabel}{}
        {}{}{
            \refer{\lblroot:view.diagrams.components:HoverLabel:hoverLabel}{hoverLabel}.
        }
        \item \methoddesc{\public}{}{void}{show}{}
        {
            Shows the \refer{\lblroot:view.diagrams.components:HoverLabel:hoverLabel}{hoverLabel}.
        }{}{}
        \item \methoddesc{\public}{}{void}{hide}{}
        {
            Hides the \refer{\lblroot:view.diagrams.components:HoverLabel:hoverLabel}{hoverLabel}.
        }{}{}
        \item \methoddesc{\public}{}{void}{setWidth}{\type{Number} width}
        {
            Sets \refer{\lblroot:view.diagrams.components:HoverLabel:width}{width}.
        }{
            \begin{itemize}
                \item \paramdesc{width}{The new \refer{\lblroot:view.diagrams.components:HoverLabel:width}{width}.}
            \end{itemize}
        }{}
        \item \methoddesc{\public}{}{Number}{getWidth}{}
        {}{}{
            \refer{\lblroot:view.diagrams.components:HoverLabel:width}{width}.
        }
        \item \methoddesc{\public}{}{void}{setHeight}{\type{Number} height}
        {
            Sets \refer{\lblroot:view.diagrams.components:HoverLabel:height}{height}.
        }{
            \begin{itemize}
                \item \paramdesc{height}{The new \refer{\lblroot:view.diagrams.components:HoverLabel:height}{height}.}
            \end{itemize}
        }{}
        \item \methoddesc{\public}{}{Number}{getHeight}{}
        {}{}{
            \refer{\lblroot:view.diagrams.components:HoverLabel:height}{height}.
        }
    \end{itemize}
}

\classdesc{DiagramComponentFactory}{
    \decl{\public \class DiagramComponentFactory}{The class, which is responsible for the creation of concrete subclasses of \refer{\lblroot:view.diagrams.components:DiagramComponent}{DiagramComponent}.}
}{
    \begin{itemize}
        \item \atrdesc{\private}{\static}{DiagramComponentFactory}{instance}{The only instance of the class}
    \end{itemize}
}{
    \begin{itemize}
        \item \constrdesc{\private}{DiagramComponentFactory}{}{}{}
    \end{itemize}
}{
    \begin{itemize}
        \item \methoddesc{\public}{}{DiagramComponentFactory}{getDiagramComponentFactory}{}
        {}{}{
            \refer{\lblroot:view.diagrams.components:DiagramComponentFactory:instance}{instance}.
        }
        \item \methoddesc{\public}{}{DiagramPoint}{createPoint}{\type{Number} value, \type{PositionIn2DDiagram} position, \type{Number} size}
        {}{
            \begin{itemize}
                \item \paramdesc{value}{The value that will be displayed via the \refer{\lblroot:view.diagrams.components:DiagramPoint}{DiagramPoint}.}
                \item \paramdesc{position}{The position of the \refer{\lblroot:view.diagrams.components:DiagramPoint}{DiagramPoint} in the \refer{\lblroot:view.diagrams:IDiagram}{IDiagram}.}
                \item \paramdesc{size}{The size of the \refer{\lblroot:view.diagrams.components:DiagramPoint}{DiagramPoint}.}
            \end{itemize}
        }{
            A concrete instance of a concrete subclass of \refer{\lblroot:view.diagrams.components:DiagramPoint}{DiagramPoint} specified via the parameters.
        }
        \item \methoddesc{\public}{}{DiagramValueLabel}{createValueLabel}{\type{Number} value, \type{PositionIn2DDiagram} bottomLeft, \type{PositionIn2DDiagram} topRight, \type{Number} borderThickness}
        {}{
            \begin{itemize}
                \item \paramdesc{value}{The value the \refer{\lblroot:view.diagrams.components:DiagramValueLabel}{DiagramValueLabel} will display.}
                \item \paramdesc{bottomLeft}{The bottom left corner of the \refer{\lblroot:view.diagrams.components:DiagramValueLabel}{DiagramValueLabel}.}
                \item \paramdesc{topRight}{The top right corner of the \refer{\lblroot:view.diagrams.components:DiagramValueLabel}{DiagramValueLabel}.}
                \item \paramdesc{borderThickness}{The thickness of the borders of \refer{\lblroot:view.diagrams.components:DiagramValueLabel}{DiagramValueLabel}.}
            \end{itemize}
        }{
            A concrete instance of a concrete subclass of \refer{\lblroot:view.diagrams.components:DiagramValueLabel}{DiagramValueLabel} specified via the parameters.
        }
        \item \methoddesc{\public}{}{DiagramBar}{createBar}{\type{Number} value, \type{PositionIn2DDiagram} bottomLeft, \type{PositionIn2DDiagram} topRight, \type{Number} borderThickness}
        {}{
            \begin{itemize}
                \item \paramdesc{value}{The value the \refer{\lblroot:view.diagrams.components:DiagramBar}{DiagramBar} will display.}
                \item \paramdesc{bottomLeft}{The bottom left corner of the \refer{\lblroot:view.diagrams.components:DiagramBar}{DiagramBar}.}
                \item \paramdesc{topRight}{The top right corner of the \refer{\lblroot:view.diagrams.components:DiagramBar}{DiagramBar}.}
                \item \paramdesc{borderThickness}{The thickness of the borders of \refer{\lblroot:view.diagrams.components:DiagramBar}{DiagramBar}.}
            \end{itemize}
        }{
            A concrete instance of a concrete subclass of \refer{\lblroot:view.diagrams.components:DiagramBar}{DiagramBar} specified via the parameters.
        }
        \item \methoddesc{\public}{}{DiagramLabel}{createLabel}{\type{PositionInFrame} bottomLeft, \type{PositionInFrame} topRight, \type{Color} color, \type{String} caption, \type{Number} borderThickness}
        {}{
            \begin{itemize}
                \item \paramdesc{bottomLeft}{The bottom left corner of the \refer{\lblroot:view.diagrams.components:DiagramLabel}{DiagramLabel}.}
                \item \paramdesc{topRight}{The top right corner of the \refer{\lblroot:view.diagrams.components:DiagramLabel}{DiagramLabel}.}
                \item \paramdesc{color}{The color of the \refer{\lblroot:view.diagrams.components:DiagramLabel}{DiagramLabel}.}
                \item \paramdesc{caption}{The text to be displayed by \refer{\lblroot:view.diagrams.components:DiagramLabel}{DiagramLabel}.}
                \item \paramdesc{borderThickness}{The thickness of the borders of \refer{\lblroot:view.diagrams.components:DiagramLabel}{DiagramLabel}.}
            \end{itemize}
        }{
            A concrete instance of a concrete subclass of \refer{\lblroot:view.diagrams.components:DiagramLabel}{DiagramLabel} specified via the parameters.
        }
        \item \methoddesc{\public}{}{DiagramAxis}{createAxis}{\type{DiagramLine} axisLine, \type{Number} min, \type{Number} max, \type{int} steps}
        {}{
            \begin{itemize}
                \item \paramdesc{axisLine}{The line part of the \refer{\lblroot:view.diagrams.components:DiagramAxis}{DiagramAxis}.}
                \item \paramdesc{min}{The minimum value on the \refer{\lblroot:view.diagrams.components:DiagramAxis}{DiagramAxis}.}
                \item \paramdesc{max}{The maximum value on the \refer{\lblroot:view.diagrams.components:DiagramAxis}{DiagramAxis}.}
                \item \paramdesc{steps}{The amount of partitions in \refer{\lblroot:view.diagrams.components:DiagramAxis}{DiagramAxis}.}
            \end{itemize}
        }{
            A concrete instance of a concrete subclass of \refer{\lblroot:view.diagrams.components:DiagramAxis}{DiagramAxis} specified via the parameters.
        }
        \item \methoddesc{\public}{}{DiagramColorScale}{createColorScale}{\type{PositionInFrame} bottomLeft, \type{PositionInFrame} topRight, \type{Color} borderColor, \type{Number[]} values, \type{Color[]} valueColors, \type{Number} borderThickness}
        {}{
            \begin{itemize}
                \item \paramdesc{bottomLeft}{The bottom left corner of the \refer{\lblroot:view.diagrams.components:DiagramColorScale}{DiagramColorScale}.}
                \item \paramdesc{topRight}{The top right corner of the \refer{\lblroot:view.diagrams.components:DiagramColorScale}{DiagramColorScale}.}
                \item \paramdesc{borderColor}{The color of the border of the \refer{\lblroot:view.diagrams.components:DiagramColorScale}{DiagramColorScale}.}
                \item \paramdesc{values}{Array of values, for which \refer{\lblroot:view.diagrams.components:DiagramColorScale}{DiagramColorScale} has certain colors.}
                \item \paramdesc{valueColors}{Array of colors, for which \refer{\lblroot:view.diagrams.components:DiagramColorScale}{DiagramColorScale} has certain values.}
                \item \paramdesc{borderThickness}{The thickness of the border of the \refer{\lblroot:view.diagrams.components:DiagramColorScale}{DiagramColorScale}.}
            \end{itemize}
        }{
            A concrete instance of a concrete subclass of \refer{\lblroot:view.diagrams.components:DiagramColorScale}{DiagramColorScale} specified via the parameters.
        }
        \item \methoddesc{\public}{}{DiagramLine}{createLine}{\type{PositionInFrame} start, \type{PositionInFrame} end, \type{Color} color, \type{Number} thickness}
        {}{
            \begin{itemize}
                \item \paramdesc{start}{The start position of \refer{\lblroot:view.diagrams.components:DiagramLine}{DiagramLine} (x1, y1).}
                \item \paramdesc{end}{The end position of \refer{\lblroot:view.diagrams.components:DiagramLine}{DiagramLine} (x2, y2).}
                \item \paramdesc{color}{The color of the \refer{\lblroot:view.diagrams.components:DiagramLine}{DiagramLine}.}
                \item \paramdesc{thickness}{The thickness of the \refer{\lblroot:view.diagrams.components:DiagramLine}{DiagramLine}.}
            \end{itemize}
        }{
            A concrete instance of a concrete subclass of \refer{\lblroot:view.diagrams.components:DiagramLine}{DiagramLine} specified via the parameters.
        }
    \end{itemize}
}
}
\packagedesc{view.diagrams.data}{

\absclassdesc{DiagramDataFormatter}{
    \decl{\public \class DiagramDataFormatter}{The abstract class inherited by classes that are used format \refer{\lblroot:view.diagrams.data:DiagramData:data}{data}.\descnote{What the format of the data will be after the \refer{\lblroot:view.diagrams.data:DiagramDataFormatter:format}{format} function is defined by the classes' names.}}
}{}{
    \begin{itemize}
        \item \subclsdec{ArrayListDataFormatter}
        \item \subclsdec{ArrayDataFormatter}
    \end{itemize}
}{
    \begin{itemize}
        \item \constrdesc{\public}{DiagramDataFormatter}{}{}{}
    \end{itemize}
}{
    \begin{itemize}
        \item \methoddesc{\public}{<\generic{T} \extends \typewgeneric{Collection}{?} >}{Object}{format}{\generic{T} data}
        {}{
            \begin{itemize}
                \item \paramdesc{data}{The data to format.}
            \end{itemize}
        }{
            The formatted data.
        }
    \end{itemize}
}

\classdesc{DiagramData}{
    The class responsible for containing \refer{\lblroot:view.diagrams.data:DiagramData:data}{data} required to make an \refer{\lblroot:view.diagrams:IDiagram}{IDiagram}.
}{
    \begin{itemize}
        \item \atrdesc{\private}{\typewgeneric{Collection}{?}}{}{data}{The data required to make an \refer{\lblroot:view.diagrams:IDiagram}{IDiagram}.}
        \item \atrdesc{\private}{}{DiagramDataFormatter}{ddf}{The part that is responsible for changing the format of the data.}
    \end{itemize}
}{
    \begin{itemize}
        \item \constrdesc{\public}{DiagramData}{\typewgeneric{Collection}{?} data}{}{
            \begin{itemize}
                \item \paramdesc{data}{The data needed to make an \refer{\lblroot:view.diagrams:IDiagram}{IDiagram}.}
            \end{itemize}
        }
    \end{itemize}
}{
    \begin{itemize}
        \item \methoddesc{\public}{<\generic{T} \extends \typewgeneric{Collection}{?} >}{\generic{T}}{getData}{}
        {}{}{
            \refer{\lblroot:view.diagrams.data:DiagramData:data}{data}.
        }
        \item \methoddesc{\public}{}{void}{setData}{\typewgeneric{Collection}{?} data}
        {
            Sets \refer{\lblroot:view.diagrams.data:DiagramData:data}{data}.
        }{}{}
        \item \methoddesc{\public}{}{Object}{getFormattedData}{}
        {
            Uses \refer{\lblroot:view.diagrams.data:DiagramData:ddf}{ddf} to format \refer{\lblroot:view.diagrams.data:DiagramData:data}{data}.
        }{
        }{
            The formatted \refer{\lblroot:view.diagrams.data:DiagramData:data}{data}.
        }
        \item \methoddesc{\public}{}{void}{setFormat}{\type{DiagramDataFormatter} ddf}
        {
            Sets \refer{\lblroot:view.diagrams.data:DiagramData:ddf}{ddf}.
        }{
            \begin{itemize}
                \item \paramdesc{ddf}{The new \refer{\lblroot:view.diagrams.data:DiagramData:ddf}{ddf}.}
            \end{itemize}
        }{}
    \end{itemize}
}

\classdesc{ArrayListDataFormatter}{
    \decl{\public \class ArrayListDataFormatter}{The implementation of \refer{\lblroot:view.diagrams.data:DiagramDataFormatter}{DiagramDataFormatter}, which formats a given piece of data that extends \typewgeneric{Collection}{?} to \typewgeneric{ArrayList}{?}.}
}{}{
    \begin{itemize}
        \item \constrdesc{\public}{ArrayListDataFormatter}{}{}{}
    \end{itemize}
}{
    \begin{itemize}
        \item \ovrdnmethoddesc{\public}{<\generic{T} \extends \typewgeneric{Collection}{?} > }{Object}{format}{\generic{T} data}
        {DiagramDataFormatter}{}{
            \begin{itemize}
                \item \paramdesc{data}{The given piece of data.}
            \end{itemize}
        }{
            The given data as \typewgeneric{ArrayList}{?}.
        }
    \end{itemize}
}

\classdesc{ArrayDataFormatter}{
    \decl{\public \class ArrayDataFormatter}{The implementation of \refer{\lblroot:view.diagrams.data:DiagramDataFormatter}{DiagramDataFormatter}, which formats a given piece of data that extends \typewgeneric{Collection}{?} to an array.}
}{}{
    \begin{itemize}
        \item \constrdesc{\public}{ArrayDataFormatter}{}{}{}
    \end{itemize}
}{
    \begin{itemize}
        \item \ovrdnmethoddesc{\public}{<\generic{E}, \generic{T} \extends \typewgeneric{Collection}{E} > }{Object}{format}{\generic{T} data}
        {DiagramDataFormatter}{}{
            \begin{itemize}
                \item \paramdesc{data}{The given piece of data.}
            \end{itemize}
        }{
            The given data as array.
        }
    \end{itemize}
}
}
\packagedesc{view.diagrams.type}{

\absclassdesc{Diagram}{
    The abstract class inherited by the implementations of \refer{\lblroot:view.diagrams:IDiagram}{IDiagram} that are made of concrete subclasses of \refer{\lblroot:view.diagrams.components:DiagramComponent}{DiagramComponents}.
}{
    \begin{itemize}
        \item \atrdesc{\private}{}{DiagramData}{data}{The data required to make a Diagram.}
        \item \atrdesc{\private}{}{DiagramAxis[]}{axes}{An array of \refer{\lblroot:view.diagrams.components:DiagramAxis}{DiagramAxes} the Diagram has.}
        \item \atrdesc{\private}{}{DiagramValueDisplayComponent[]}{valueDisplayComponents}{An array of \refer{\lblroot:view.diagrams.components:DiagramValueDisplayComponent}{DiagramValueDisplayComponents} that are responsible for displaying values in Diagrams.}
        \item \atrdesc{\private}{}{DiagramComponent[]}{nonValueDisplayComponents}{An array of \refer{\lblroot:view.diagrams.components:DiagramComponent}{DiagramComponents} that are not responsible for displaying values in Diagrams.}
        \item \atrdesc{\private}{\typewgeneric{EnumMap}{IndicatorIdentifier, DiagramViewHelper}}{}{viewHelpers}{An \type{EnumMap} that stores the \refer{\lblroot:view.diagrams.indicator:DiagramViewHelper}{DiagramViewHelpers} and their unique \refer{\lblroot:view.diagrams.indicator:IndicatorIdentifier}{IndicatorIdentifier}.}
    \end{itemize}
}{
    \begin{itemize}
        \item \subclsdec{Histogram}
        \item \subclsdec{BarChart}
        \item \subclsdec{HeatMap}
        \item \subclsdec{FunctionGraph}
    \end{itemize}
}{
    \begin{itemize}
        \item \constrdesc{\public}{Diagram}{\type{DiagramData} data, \type{DiagramAxis[]} axes, \type{DiagramValueDisplayComponent[]} valueDisplayComponents, \type{DiagramComponent[]} nonValueDisplayComponents}{}{
            \begin{itemize}
                \item \paramdesc{data}{The data required to make a Diagram.}
                \item \paramdesc{axes}{The \refer{\lblroot:view.diagrams.components:DiagramAxis}{DiagramAxes} the Diagram has.}
                \item \paramdesc{valueDisplayComponents}{The array of \refer{\lblroot:view.diagrams.components:DiagramValueDisplayComponent}{DiagramValueDisplayComponents} the Diagram has.}
                \item \paramdesc{nonValueDisplayComponents}{The array of \refer{\lblroot:view.diagrams.components:DiagramComponent}{DiagramComponents} the Diagram that are not responsible for displaying values.}
            \end{itemize}
        }
    \end{itemize}
}{
    \begin{itemize}
        \item \methoddesc{\public}{}{boolean}{addDiagramViewHelper}{\type{DiagramViewHelper} dvh}
        {
            Adds the given \refer{\lblroot:view.diagrams.indicator:DiagramViewHelper}{DiagramViewHelper} to \refer{\lblroot:view.diagrams.type:Diagram:viewHelpers}{viewHelpers}.
        }{
            \begin{itemize}
                \item \paramdesc{dvh}{The given \refer{\lblroot:view.diagrams.indicator:DiagramViewHelper}{DiagramViewHelper}.}
            \end{itemize}
        }{
            \booldesc{the given \refer{\lblroot:view.diagrams.indicator:DiagramViewHelper}{DiagramViewHelper} was successfully added to \refer{\lblroot:view.diagrams.type:Diagram:viewHelpers}{viewHelpers}}{not}
        }
        \item \methoddesc{\public}{}{boolean}{removeDiagramViewHelper}{\type{IndicatorIdentifier} id}
        {
            Removes the specified \refer{\lblroot:view.diagrams.indicator:DiagramViewHelper}{DiagramViewHelper} from \refer{\lblroot:view.diagrams.type:Diagram:viewHelpers}{viewHelpers}.
        }{
            \begin{itemize}
                \item \paramdesc{id}{The given \refer{\lblroot:view.diagrams.indicator:IndicatorIdentifier}{IndicatorIdentifier} of a \refer{\lblroot:view.diagrams.indicator:DiagramViewHelper}{DiagramViewHelper}.}
            \end{itemize}
        }{
            \booldesc{the given \refer{\lblroot:view.diagrams.indicator:DiagramViewHelper}{DiagramViewHelper} was successfully removed from \refer{\lblroot:view.diagrams.type:Diagram:viewHelpers}{viewHelpers}}{not}
        }
        \item \methoddesc{\public}{}{boolean}{showDiagramViewHelper}{\type{IndicatorIdentifier} id}
        {
            Shows the specified \refer{\lblroot:view.diagrams.indicator:DiagramViewHelper}{DiagramViewHelper}.
        }{
            \begin{itemize}
                \item \paramdesc{id}{The given \refer{\lblroot:view.diagrams.indicator:IndicatorIdentifier}{IndicatorIdentifier} of a \refer{\lblroot:view.diagrams.indicator:DiagramViewHelper}{DiagramViewHelper}.}
            \end{itemize}
        }{
            \booldesc{the given \refer{\lblroot:view.diagrams.indicator:DiagramViewHelper}{DiagramViewHelper} was successfully shown}{not}
        }
        \item \methoddesc{\public}{}{boolean}{hideDiagramViewHelper}{\type{IndicatorIdentifier} id}
        {
            Hides the specified \refer{\lblroot:view.diagrams.indicator:DiagramViewHelper}{DiagramViewHelper}.
        }{
            \begin{itemize}
                \item \paramdesc{id}{The given \refer{\lblroot:view.diagrams.indicator:IndicatorIdentifier}{IndicatorIdentifier} of a \refer{\lblroot:view.diagrams.indicator:DiagramViewHelper}{DiagramViewHelper}.}
            \end{itemize}
        }{
            \booldesc{the given \refer{\lblroot:view.diagrams.indicator:DiagramViewHelper}{DiagramViewHelper} was successfully hidden}{not}
        }
    \end{itemize}
}

\classdesc{Histogram}{
    \decl{\public \class Histogram}{Represents a histogram, whose \refer{\lblroot:view.diagrams.components:HistogramBar}{HistogramBars}' height represent the value of a given index and whose widths' represent the differences between the adjacent indices.}
}{}{
    \begin{itemize}
        \item \constrdesc{\public}{Histogram}{\type{DiagramData} data, \type{DiagramAxis[]} axes, \type{DiagramValueDisplayComponent[]} valueDisplayComponents, \type{DiagramComponent[]} nonValueDisplayComponents}{}{
            \begin{itemize}
                \item \paramdesc{data}{The data required to make a Histogram.}
                \item \paramdesc{axes}{The \refer{\lblroot:view.diagrams.components:DiagramAxis}{DiagramAxes} the Histogram has.}
                \item \paramdesc{valueDisplayComponents}{The \refer{\lblroot:view.diagrams.components:HistogramBar}{HistogramBars} the Histogram has in form of an array of \refer{\lblroot:view.diagrams.components:DiagramValueDisplayComponent}{DiagramValueDisplayComponents}.}
                \item \paramdesc{nonValueDisplayComponents}{The array of \refer{\lblroot:view.diagrams.components:DiagramComponent}{DiagramComponents} the Histogram that are not responsible for displaying values.}
            \end{itemize}
        }
    \end{itemize}
}{}

\classdesc{BarChart}{
    \decl{\public \class BarChart}{Represents a bar chart, whose \refer{\lblroot:view.diagrams.components:BarChartBar}{BarChartBars}' height represents the value of the enum-like argument underneath each bar.}
}{}{
    \begin{itemize}
        \item \constrdesc{\public}{BarChart}{\type{DiagramData} data, \type{DiagramAxis[]} axes, \type{DiagramValueDisplayComponent[]} valueDisplayComponents, \type{DiagramComponent[]} nonValueDisplayComponents}{}{
            \begin{itemize}
                \item \paramdesc{data}{The data required to make a BarChart.}
                \item \paramdesc{axes}{The \refer{\lblroot:view.diagrams.components:DiagramAxis}{DiagramAxes} the BarChart has.}
                \item \paramdesc{valueDisplayComponents}{The \refer{\lblroot:view.diagrams.components:BarChartBar}{BarChartBars} the BarChart has in form of an array of \refer{\lblroot:view.diagrams.components:DiagramValueDisplayComponent}{DiagramValueDisplayComponents}.}
                \item \paramdesc{nonValueDisplayComponents}{The array of \refer{\lblroot:view.diagrams.components:DiagramComponent}{DiagramComponents} the BarChart that are not responsible for displaying values.}
            \end{itemize}
        }
    \end{itemize}
}{}

\classdesc{HeatMap}{
    \decl{\public \class HeatMap}{Represents a heat map, whose \refer{\lblroot:view.diagrams.components:HeatMapLabel}{HeatMapLabels}' color represent the value of a given index tuple of size 2.}
}{}{
    \begin{itemize}
        \item \constrdesc{\public}{HeatMap}{\type{DiagramData} data, \type{DiagramAxis[]} axes, \type{DiagramValueDisplayComponent[]} valueDisplayComponents, \type{DiagramComponent[]} nonValueDisplayComponents}{}{
            \begin{itemize}
                \item \paramdesc{data}{The data required to make a HeatMap.}
                \item \paramdesc{axes}{The \refer{\lblroot:view.diagrams.components:DiagramAxis}{DiagramAxes} the HeatMap has.}
                \item \paramdesc{valueDisplayComponents}{The \refer{\lblroot:view.diagrams.components:HeatMapLabel}{HeatMapLabels} the HeatMap has in form of an array of \refer{\lblroot:view.diagrams.components:DiagramValueDisplayComponent}{DiagramValueDisplayComponents}.}
                \item \paramdesc{nonValueDisplayComponents}{The array of \refer{\lblroot:view.diagrams.components:DiagramComponent}{DiagramComponents} the HeatMap that are not responsible for displaying values.}
            \end{itemize}
        }
    \end{itemize}
}{}

\classdesc{FunctionGraph}{
    \decl{\public \class FunctionGraph}{Represents a 2D function graph.}
}{}{
    \begin{itemize}
        \item \constrdesc{\public}{FunctionGraph}{\type{DiagramData} data, \type{DiagramAxis[]} axes, \type{DiagramValueDisplayComponent[]} valueDisplayComponents, \type{DiagramComponent[]} nonValueDisplayComponents}{}{
            \begin{itemize}
                \item \paramdesc{data}{The data required to make a FunctionGraph.}
                \item \paramdesc{axes}{The \refer{\lblroot:view.diagrams.components:DiagramAxis}{DiagramAxes} the FunctionGraph has.}
                \item \paramdesc{valueDisplayComponents}{The \refer{\lblroot:view.diagrams.components:ValueDisplayPoint}{ValueDisplayPoints} the FunctionGraph has in form of an array of \refer{\lblroot:view.diagrams.components:DiagramValueDisplayComponent}{DiagramValueDisplayComponents}.}
                \item \paramdesc{nonValueDisplayComponents}{The array of \refer{\lblroot:view.diagrams.components:DiagramComponent}{DiagramComponents} the FunctionGraph that are not responsible for displaying values.}
            \end{itemize}
        }
    \end{itemize}
}{}
}
\packagedesc{view.diagrams.indicator}{

\absclassdesc{DiagramViewHelper}{
    \decl{\public \abstr \class DiagramViewHelper}{The abstract class inherited by classes that are responsible for displaying \refer{\lblroot:view.diagrams.indicator:ViewHelperComponent}{ViewHelperComponents} on top of \refer{\lblroot:view.diagrams:IDiagram}{IDiagrams}.}
}{
    \begin{itemize}
        \item \atrdesc{\private}{}{int}{layer}{The layer, in which the DiagramViewHelper lies. DiagramViewHelpers with a larger layer value will be shown on top of the ones with a lower layer value.}
        \item \atrdesc{\private}{\typewgeneric{List}{ViewHelperComponent}}{}{helperComponents}{The list of the \refer{\lblroot:view.diagrams.indicator:ViewHelperComponent}{ViewHelperComponents} that the DiagramViewHelper is responsible of.}
        \item \atrdesc{\private}{}{IndicatorIdentifier}{id}{The unique identifier of the DiagramViewHelper.}
        \item \atrdesc{\private}{}{IDiagram}{diagram}{The \refer{\lblroot:view.diagrams:IDiagram}{IDiagram}, to which the DiagramViewHelper is attached to.}
    \end{itemize}
}{
    \begin{itemize}
        \item \subclsdec{HelperLineDisplayer}
        \item \subclsdec{HelperComponentDisplayer}
    \end{itemize}
}{
    \begin{itemize}
        \item \constrdesc{\public}{DiagramViewHelper}{\type{IDiagram} diagram, \type{int} layer, \type{IndicatorIdentifier} id}{}{
            \begin{itemize}
                \item \paramdesc{diagram}{The \refer{\lblroot:view.diagrams:IDiagram}{IDiagram}, to which the DiagramViewHelper is attached to.}
                \item \paramdesc{layer}{The layer, in which the DiagramViewHelper lies. DiagramViewHelpers with a larger layer value will be shown on top of the ones with a lower layer value.}
                \item \paramdesc{id}{The unique identifier of the DiagramViewHelper.}
            \end{itemize}
        }
    \end{itemize}
}{
    \begin{itemize}
        \item \methoddesc{\public}{}{int}{getLayerNumber}{}
        {}{}{
            \refer{\lblroot:view.diagrams.indicator:DiagramViewHelper:layer}{layer}.
        }
        \item \methoddesc{\public}{}{void}{remove}{}
        {
            Performs actions needed to be done (from the DiagramViewHelper's side), when the DiagramViewHelper is to be removed from the \refer{\lblroot:view.diagrams.indicator:DiagramViewHelper:diagram}{diagram} it is attached to.
        }{}{}
        \item \methoddesc{\public}{}{void}{show}{}
        {
            Shows the \refer{\lblroot:view.diagrams.indicator:ViewHelperComponent}{ViewHelperComponents} in \refer{\lblroot:view.diagrams.indicator:DiagramViewHelper:helperComponents}{helperComponents}.
        }{}{}
        \item \methoddesc{\public}{}{void}{hide}{}
        {
            Hides the \refer{\lblroot:view.diagrams.indicator:ViewHelperComponent}{ViewHelperComponents} in \refer{\lblroot:view.diagrams.indicator:DiagramViewHelper:helperComponents}{helperComponents}.
        }{}{}
        \item \methoddesc{\public}{}{void}{update}{}
        {
            Updates the DiagramViewHelper.
        }{}{}
        \item \methoddesc{\public}{}{boolean}{addViewHelperComponent}{\type{ViewHelperComponent} vhc}
        {
            Adds a given \refer{\lblroot:view.diagrams.indicator:ViewHelperComponent}{ViewHelperComponent} to \refer{\lblroot:view.diagrams.indicator:DiagramViewHelper:helperComponents}{helperComponents}.
        }{
            \begin{itemize}
                \item \paramdesc{vhc}{The given \refer{\lblroot:view.diagrams.indicator:ViewHelperComponent}{ViewHelperComponent}.}
            \end{itemize}
        }{
            \booldesc{the given \refer{\lblroot:view.diagrams.indicator:ViewHelperComponent}{ViewHelperComponent} has been added to \refer{\lblroot:view.diagrams.indicator:DiagramViewHelper:helperComponents}{helperComponents} successfully.}{not}
        }
        \item \methoddesc{\public}{}{boolean}{removeViewHelperComponent}{\type{ViewHelperComponent} vhc}
        {
            Removes a given \refer{\lblroot:view.diagrams.indicator:ViewHelperComponent}{ViewHelperComponent} from \refer{\lblroot:view.diagrams.indicator:DiagramViewHelper:helperComponents}{helperComponents}.
        }{
            \begin{itemize}
                \item \paramdesc{vhc}{The given \refer{\lblroot:view.diagrams.indicator:ViewHelperComponent}{ViewHelperComponent}.}
            \end{itemize}
        }{
            \booldesc{the given \refer{\lblroot:view.diagrams.indicator:ViewHelperComponent}{ViewHelperComponent} has been removed from \refer{\lblroot:view.diagrams.indicator:DiagramViewHelper:helperComponents}{helperComponents} successfully}{not}
        }
        \item \methoddesc{\public}{}{boolean}{clearViewHelperComponents}{}
        {
            Empties \refer{\lblroot:view.diagrams.indicator:DiagramViewHelper:helperComponents}{helperComponents}.
        }{}{
            \booldesc{\refer{\lblroot:view.diagrams.indicator:DiagramViewHelper:helperComponents}{helperComponents} has been emptied successfully}{not}
        }
        \item \methoddesc{\public}{}{IndicatorIdentifier}{getID}{}
        {}{}{
            \refer{\lblroot:view.diagrams.indicator:DiagramViewHelper:id}{id}.
        }
    \end{itemize}
}

\absclassdesc{HelperLineDisplayer}{
    \decl{\public \abstr \class HelperLineDisplayer}{The abstract class inherited by classes that are responsible for showing lines on \refer{\lblroot:view.diagrams:IDiagram}{IDiagrams}.}
}{}{
    \begin{itemize}
        \item \subclsdec{ValueLineDisplayer}
        \item \subclsdec{CoordinateIndicatorLineDisplayer}
    \end{itemize}
}{
    \begin{itemize}
        \item \constrdesc{\public}{HelperLineDisplayer}{\type{IDiagram} diagram, \type{int} layer, \type{IndicatorIdentifier} id}{}{
            \begin{itemize}
                \item \paramdesc{diagram}{The \refer{\lblroot:view.diagrams:IDiagram}{IDiagram}, to which the HelperLineDisplayer is attached to.}
                \item \paramdesc{layer}{The layer, in which the HelperLineDisplayer lies. HelperLineDisplayer with a larger layer value will be shown on top of the ones with a lower layer value.}
                \item \paramdesc{id}{The unique identifier of the HelperLineDisplayer.}
            \end{itemize}
        }
    \end{itemize}
}{
    \begin{itemize}
        \item \methoddesc{\proted}{\abstr}{void}{generateHelperComponents}{}
        {
            Generates \refer{\lblroot:view.diagrams.indicator:ValueLine}{ValueLines}, which will be shown on the \refer{\lblroot:view.diagrams:IDiagram}{IDiagrams}.
        }{}{}
    \end{itemize}
}

\absclassdesc{HelperComponentDisplayer}{
    \decl{\public \abstr \class HelperComponentDisplayer}{The abstract class inherited by classes that are responsible for showing cloned and manipulated \refer{\lblroot:view.diagrams.components:DiagramComponent}{DiagramComponents} on an \refer{\lblroot:view.diagrams:IDiagram}{IDiagram}.}
}{}{
    \begin{itemize}
        \item \subclsdec{ValueFixColorDisplayer}
        \item \subclsdec{ValueScaleColorDisplayer}
    \end{itemize}
}{
    \begin{itemize}
        \item \constrdesc{\public}{HelperComponentDisplayer}{\type{IDiagram} diagram, \type{int} layer, \type{IndicatorIdentifier} id}{}{
            \begin{itemize}
                \item \paramdesc{diagram}{The \refer{\lblroot:view.diagrams:IDiagram}{IDiagram}, to which the HelperComponentDisplayer is attached to.}
                \item \paramdesc{layer}{The layer, in which the HelperComponentDisplayer lies. HelperComponentDisplayer with a larger layer value will be shown on top of the ones with a lower layer value.}
                \item \paramdesc{id}{The unique identifier of the HelperComponentDisplayer.}
            \end{itemize}
        }
    \end{itemize}
}{}

\enumdesc{IndicatorIdentifier}{
    \decl{\public \enum IndicatorIdentifier}{The enum for the unique identifiers the \refer{\lblroot:view.diagrams.indicator:DiagramViewHelper}{DiagramViewHelpers} can use.}
}{
    \begin{itemize}
        \item \fielddesc{IndicatorIdentifier}{MIN}{The unique identifier for the \refer{\lblroot:view.diagrams.indicator:DiagramViewHelper}{DiagramViewHelpers} that make the minimum value more noticeable.}
        \item \fielddesc{IndicatorIdentifier}{MAX}{The unique identifier for the \refer{\lblroot:view.diagrams.indicator:DiagramViewHelper}{DiagramViewHelpers} that make the maximum value more noticeable.}
        \item \fielddesc{IndicatorIdentifier}{AVG}{The unique identifier for the \refer{\lblroot:view.diagrams.indicator:DiagramViewHelper}{DiagramViewHelpers} that make the average value more noticeable.}
        \item \fielddesc{IndicatorIdentifier}{MED}{The unique identifier for the \refer{\lblroot:view.diagrams.indicator:DiagramViewHelper}{DiagramViewHelpers} that make the median more noticeable.}
        \item \fielddesc{IndicatorIdentifier}{X-COORDINATE-INDICATOR}{The unique identifier for the \refer{\lblroot:view.diagrams.indicator:DiagramViewHelper}{DiagramViewHelpers} that are responsible for displaying \refer{\lblroot:view.diagrams.indicator:ValueLine}{ValueLines} for the x-coordinates.}
        \item \fielddesc{IndicatorIdentifier}{Y-COORDINATE-INDICATOR}{The unique identifier for the \refer{\lblroot:view.diagrams.indicator:DiagramViewHelper}{DiagramViewHelpers} that are responsible for displaying \refer{\lblroot:view.diagrams.indicator:ValueLine}{ValueLines} for the y-coordinates.}
    \end{itemize}
}{}

\classdesc{DiagramViewHelperFactory}{
    \decl{\public \class DiagramViewHelperFactory}{The class responsible for creating \refer{\lblroot:view.diagrams.indicator:DiagramViewHelper}{DiagramViewHelpers}.}
}{
    \begin{itemize}
        \item \atrdesc{\private}{\static}{DiagramViewHelperFactory}{instance}{The only instance of the class.}
    \end{itemize}
}{
    \begin{itemize}
        \item \constrdesc{\private}{DiagramViewHelperFactory}{}{}{}
    \end{itemize}
}{
    \begin{itemize}
        \item \methoddesc{\public}{}{HelperComponentDisplayer}{createValueColorDisplayer}{\type{IDiagram} diagram, \type{IndicatorIdentifier} id}
        {}{
            \begin{itemize}
                \item \paramdesc{diagram}{The \refer{\lblroot:view.diagrams:IDiagram}{IDiagram}, to which the \refer{\lblroot:view.diagrams.indicator:HelperComponentDisplayer}{HelperComponentDisplayer} will be attached to.}
                \item \paramdesc{id}{The unique identifier of the \refer{\lblroot:view.diagrams.indicator:HelperComponentDisplayer}{HelperComponentDisplayer}.}
            \end{itemize}
        }{
            The \refer{\lblroot:view.diagrams.indicator:HelperComponentDisplayer}{HelperComponentDisplayer} specified by the parameters.
        }
        \item \methoddesc{\public}{}{HelperLineDisplayer}{createCoordinateGridDisplayer}{\type{IDiagram} diagram, \type{DiagramAxis} axis, \type{IndicatorIdentifier} id}
        {}{
            \begin{itemize}
                \item \paramdesc{diagram}{The \refer{\lblroot:view.diagrams:IDiagram}{IDiagram}, to which the \refer{\lblroot:view.diagrams.indicator:HelperLineDisplayer}{HelperLineDisplayer} will be attached to.}
                \item \paramdesc{axis}{The \refer{\lblroot:view.diagrams.components:DiagramAxis}{DiagramAxis}, whose coordinates the \refer{\lblroot:view.diagrams.indicator:ValueLine}{ValueLines} of \refer{\lblroot:view.diagrams.indicator:HelperLineDisplayer}{HelperLineDisplayer} will indicate.}
                \item \paramdesc{id}{The unique identifier of the \refer{\lblroot:view.diagrams.indicator:HelperLineDisplayer}{HelperLineDisplayer}.}
            \end{itemize}
        }{
            The \refer{\lblroot:view.diagrams.indicator:HelperLineDisplayer}{HelperLineDisplayer} specified by the parameters.
        }
        \item \methoddesc{\public}{}{HelperLineDisplayer}{createValueLineDisplayer}{\type{IDiagram} diagram, \type{DiagramAxis} axis, \type{Number} value, \type{IndicatorIdentifier} id}
        {}{
            \begin{itemize}
                \item \paramdesc{diagram}{The \refer{\lblroot:view.diagrams:IDiagram}{IDiagram}, to which the \refer{\lblroot:view.diagrams.indicator:HelperLineDisplayer}{HelperLineDisplayer} will be attached to.}
                \item \paramdesc{axis}{The \refer{\lblroot:view.diagrams.components:DiagramAxis}{DiagramAxis}, on whom the value lies.}
                \item \paramdesc{value}{The coordinate on the axis, which will be displayed via \refer{\lblroot:view.diagrams.indicator:HelperLineDisplayer}{HelperLineDisplayer}.}
                \item \paramdesc{id}{The unique identifier of the \refer{\lblroot:view.diagrams.indicator:HelperLineDisplayer}{HelperLineDisplayer}.}
            \end{itemize}
        }{
            The \refer{\lblroot:view.diagrams.indicator:HelperLineDisplayer}{HelperLineDisplayer} specified by the parameters.
        }
    \end{itemize}
}

\classdesc{ValueLineDisplayer}{
    \decl{\public \class ValueLineDisplayer}{The class, which is responsible for displaying \refer{\lblroot:view.diagrams.indicator:ValueLine}{ValueLine(s)} for a given value on a given \refer{\lblroot:view.diagrams.components:DiagramAxis}{DiagramAxis}.}
}{
    \begin{itemize}
        \item \atrdesc{\private}{}{DiagramAxis}{axis}{The \refer{\lblroot:view.diagrams.components:DiagramAxis}{DiagramAxis}, on which the given coordinate lies.}
        \item \atrdesc{\private}{}{Color}{color}{The color of the \refer{\lblroot:view.diagrams.indicator:ValueLine}{ValueLines}.}
        \item \atrdesc{\private}{}{Number}{thickness}{The thickness of the \refer{\lblroot:view.diagrams.indicator:ValueLine}{ValueLines}.}
        \item \atrdesc{\private}{}{Number}{value}{A given coordinate on the given \refer{\lblroot:view.diagrams.components:DiagramAxis}{DiagramAxis}.}
    \end{itemize}
}{
    \begin{itemize}
        \item \constrdesc{\proted}{ValueLineDisplayer}{\type{IDiagram} diagram, \type{DiagramAxis} axis, \type{Color} color, \type{Number} thickness, \type{Number} value, \type{IndicatorIdentifier} id}{}{
            \begin{itemize}
                \item \paramdesc{diagram}{The \refer{\lblroot:view.diagrams:IDiagram}{IDiagram}, to which the ValueLineDisplayer will be attached to.}
                \item \paramdesc{axis}{The \refer{\lblroot:view.diagrams.components:DiagramAxis}{DiagramAxis}, on which the given coordinate lies.}
                \item \paramdesc{color}{The color of the \refer{\lblroot:view.diagrams.indicator:ValueLine}{ValueLines}.}
                \item \paramdesc{thickness}{The thickness of the \refer{\lblroot:view.diagrams.indicator:ValueLine}{ValueLines}.}
                \item \paramdesc{value}{A given coordinate on the given \refer{\lblroot:view.diagrams.components:DiagramAxis}{DiagramAxis}.}
                \item \paramdesc{id}{The unique identifier of ValueLineDisplayer.}
            \end{itemize}
        }
    \end{itemize}
}{
    \begin{itemize}
        \item \methoddesc{\private}{}{void}{createValueLine}{}
        {
            Generates the \refer{\lblroot:view.diagrams.indicator:ValueLine}{ValueLine} that will indicate the \refer{\lblroot:view.diagrams.indicator:ValueLineDisplayer:value}{value}.
        }{}{}
        \item \ovrdnmethoddesc{\proted}{}{void}{generateHelperComponents}{}{HelperLineDisplayer}
        {
            Generates the \refer{\lblroot:view.diagrams.indicator:ValueLine}{ValueLine(s)}, which will be shown on the \refer{\lblroot:view.diagrams:IDiagram}{IDiagrams}.
        }{}{}
    \end{itemize}
}

\classdesc{ValueFixColorDisplayer}{
    \decl{\public \class ValueFixColorDisplayer}{The class that is responsible for displaying cloned and manipulated \refer{\lblroot:view.diagrams.components:DiagramComponent}{DiagramComponents} on top of their \refer{\lblroot:view.diagrams:IDiagram}{IDiagrams} with the given colors.}
}{
    \begin{itemize}
        \item \atrdesc{\private}{\typewgeneric{TreeMap}{Number, Color}}{}{mapping}{A map, which maps given values to given colors.}
    \end{itemize}
}{
    \begin{itemize}
        \item \constrdesc{\proted}{ValueFixColorDisplayer}{\type{IDiagram} diagram, \typewgeneric{TreeMap}{Number, Color} mapping, \type{IndicatorIdentifier} id}{}{
            \begin{itemize}
                \item \paramdesc{diagram}{The \refer{\lblroot:view.diagrams:IDiagram}{IDiagram}, to which the ValueFixColorDisplayer is attached to.}
                \item \paramdesc{mapping}{A given mapping of values to colors.}
                \item \paramdesc{id}{The unique identifier of the ValueFixColorDisplayer.}
            \end{itemize}
        }
    \end{itemize}
}{}

\classdesc{ValueScaleColorDisplayer}{
    \decl{\public \class ValueScaleColorDisplayer}{The class that is responsible for displaying cloned and manipulated \refer{\lblroot:view.diagrams.components:DiagramComponent}{DiagramComponents} on top of their \refer{\lblroot:view.diagrams:IDiagram}{IDiagrams} using the given \refer{\lblroot:view.diagrams.components:DiagramColorScale}{DiagramColorScale}.}
}{
    \begin{itemize}
        \item \atrdesc{\private}{}{DiagramColorScale}{colorScale}{The given \refer{\lblroot:view.diagrams.components:DiagramColorScale}{DiagramColorScale}.}
    \end{itemize}
}{
    \begin{itemize}
        \item \constrdesc{\proted}{ValueScaleColorDisplayer}{\type{IDiagram} diagram, \type{DiagramColorScale} colorScale, \type{IndicatorIdentifier} id}{}{
            \begin{itemize}
                \item \paramdesc{diagram}{The \refer{\lblroot:view.diagrams:IDiagram}{IDiagram}, to which the ValueFixColorDisplayer is attached to.}
                \item \paramdesc{colorScale}{A given \refer{\lblroot:view.diagrams.components:DiagramColorScale}{DiagramColorScale}.}
                \item \paramdesc{id}{The unique identifier of the ValueScaleColorDisplayer.}
            \end{itemize}
        }
    \end{itemize}
}{}

\classdesc{CoordinateIndicatorLineDisplayer}{
    \decl{\public \class CoordinateIndicatorLineDisplayer}{The class that is responsible for indicating the coordinates on a given \refer{\lblroot:view.diagrams.components:DiagramAxis}{DiagramAxis}.}
}{
    \begin{itemize}
        \item \atrdesc{\private}{}{DiagramAxis}{axis}{A given \refer{\lblroot:view.diagrams.components:DiagramAxis}{DiagramAxis}.}
        \item \atrdesc{\private}{}{Color}{color}{The color of the \refer{\lblroot:view.diagrams.indicator:ValueLine}{ValueLines}.}
        \item \atrdesc{\private}{}{Number}{thickness}{The thickness of the \refer{\lblroot:view.diagrams.indicator:ValueLine}{ValueLines}.}
    \end{itemize}
}{
    \begin{itemize}
        \item \constrdesc{\proted}{CoordinateIndicatorLineDisplayer}{\type{IDiagram} diagram, \type{DiagramAxis} axis, \type{Color} color, \type{Number} thickness, \type{IndicatorIdentifier} id}{}{
            \begin{itemize}
                \item \paramdesc{diagram}{The \refer{\lblroot:view.diagrams:IDiagram}{IDiagram}, to which the ValueFixColorDisplayer is attached to.}
                \item \paramdesc{axis}{A given \refer{\lblroot:view.diagrams.components:DiagramAxis}{DiagramAxis}.}
                \item \paramdesc{color}{The color of the \refer{\lblroot:view.diagrams.indicator:ValueLine}{ValueLines}.}
                \item \paramdesc{thickness}{The thickness of the \refer{\lblroot:view.diagrams.indicator:ValueLine}{ValueLines}.}
                \item \paramdesc{id}{The unique identifier of the ValueScaleColorDisplayer.}
            \end{itemize}
        }
    \end{itemize}
}{
    \begin{itemize}
        \item \methoddesc{\private}{}{void}{createCoordinateIndicatorLines}{}
        {}{}{}
        \item \ovrdnmethoddesc{\proted}{}{void}{generateHelperComponents}{}{HelperLineDisplayer}
        {}{}{}
    \end{itemize}
}

\classdesc{ViewHelperComponent}{
    \decl{\public \class ViewHelperComponent}{The class, which wraps the \refer{\lblroot:view.diagrams.components:DiagramComponent}{DiagramComponents} cloned and manipulated to point out various aspects of an \refer{\lblroot:view.diagrams:IDiagram}{IDiagram}.}
}{}{
    \begin{itemize}
        \item \constrdesc{\proted}{ViewHelperComponent}{\type{DiagramComponent} dc}{}{
            \begin{itemize}
                \item \paramdesc{dc}{The cloned \refer{\lblroot:view.diagrams.components:DiagramComponent}{DiagramComponent}.}
            \end{itemize}
        }
    \end{itemize}
}{
    \begin{itemize}
        \item \methoddesc{\public}{}{void}{show}{}
        {
            Shows the ViewHelperComponent.
        }{}{}
        \item \methoddesc{\public}{}{void}{hide}{}
        {
            Hides the ViewHelperComponent.
        }{}{}
    \end{itemize}
}

\classdesc{CoordinateIndicatorLine}{
    \decl{\public \class CoordinateIndicatorLine}{The class that represents the lines that are used to indicate coordinates of a given \refer{\lblroot:view.diagrams.components:DiagramAxis}{DiagramAxis}.}
}{}{
    \begin{itemize}
        \item \constrdesc{\proted}{CoordinateIndicatorLine}{\type{DiagramAxis} axis, \type{Color} color, \type{Number} thickness}{}{
            \begin{itemize}
                \item \paramdesc{axis}{A given \refer{\lblroot:view.diagrams.components:DiagramAxis}{DiagramAxis}, on which the given value is.}
                \item \paramdesc{color}{The color of the CoordinateIndicatorLine.}
                \item \paramdesc{thickness}{The thickness of the CoordinateIndicatorLine.}
            \end{itemize}
        }
    \end{itemize}
}{}

\classdesc{ValueLine}{
    \decl{\public \class ValueLine}{The class that represents a line that is responsible for displaying a given value on a \refer{\lblroot:view.diagrams:IDiagram}{IDiagram}.}
}{}{
    \begin{itemize}
        \item \constrdesc{\proted}{ValueLine}{\type{DiagramAxis} axis, \type{Number} value, \type{Color} color, \type{Number} thickness}{}{
            \begin{itemize}
                \item \paramdesc{axis}{A given \refer{\lblroot:view.diagrams.components:DiagramAxis}{DiagramAxis}, on which the given value is.}
                \item \paramdesc{value}{A given value on the given \refer{\lblroot:view.diagrams.components:DiagramAxis}{DiagramAxis}.}
                \item \paramdesc{color}{The color of the ValueLine.}
                \item \paramdesc{thickness}{The thickness of the ValueLine.}
            \end{itemize}
        }
    \end{itemize}
}{}
}
\packagedesc{view.diagrams.builder}{

\absclassdesc{DiagramBuilder}{
    \decl{\public \abstr \class DiagramBuilder}{The abstract class inherited by classes that are responsible for creating \refer{\lblroot:view.diagrams.components:DiagramComponent}{DiagramComponents} for certain types of \refer{\lblroot:view.diagrams:IDiagram}{IDiagram}.}
}{
    \begin{itemize}
        \item \atrdesc{\private}{}{DiagramData}{data}{The data needed to build a \refer{\lblroot:view.diagrams:IDiagram}{IDiagram} (see: \refer{\lblroot:view.diagrams.type:Diagram:data}{data})}
    \end{itemize}
}{
    \begin{itemize}
        \item \subclsdec{BarChartBuilder}
        \item \subclsdec{HistogramBuilder}
        \item \subclsdec{FunctionGraphBuilder}
        \item \subclsdec{HeatMapBuilder}
    \end{itemize}
}{
    \begin{itemize}
        \item \constrdesc{\public}{DiagramBuilder}{\type{DiagramData} data}{}{
            \begin{itemize}
                \item \paramdesc{data}{The data needed to build a \refer{\lblroot:view.diagrams:IDiagram}{IDiagram} (see: \refer{\lblroot:view.diagrams.type:Diagram:data}{data})}
            \end{itemize}
        }
    \end{itemize}
}{
    \begin{itemize}
        \item \methoddesc{\proted}{}{DiagramAxis}{buildAxes}{}
        {}{}{
            The \refer{\lblroot:view.diagrams.components:DiagramAxis}{DiagramAxes} that will be used by the \refer{\lblroot:view.diagrams:IDiagram}{IDiagram}.
        }
        \item \methoddesc{\proted}{}{DiagramValueDisplayComponent[]}{buildValueDisplayComponents}{}
        {}{}{
            The \refer{\lblroot:view.diagrams.components:DiagramValueDisplayComponent}{DiagramValueDisplayComponent} that will be used by the \refer{\lblroot:view.diagrams:IDiagram}{IDiagram}.
        }
        \item \methoddesc{\proted}{}{DiagramComponent[]}{buildDiagramSpecificComponent}{}
        {}{}{
            The \refer{\lblroot:view.diagrams.components:DiagramComponent}{DiagramComponent} that will be used by the \refer{\lblroot:view.diagrams:IDiagram}{IDiagram}.
        }
        \item \methoddesc{\public}{}{IDiagram}{buildDiagram}{}
        {}{}{
            The result \refer{\lblroot:view.diagrams:IDiagram}{IDiagram} of the build methods.
        }
    \end{itemize}
}

\classdesc{BarChartBuilder}{
    \decl{\public \class BarChartBuilder}{The class that is responsible of building \refer{\lblroot:view.diagrams.type:BarChart}{BarCharts}.}
}{}{
    \begin{itemize}
        \item \constrdesc{\public}{BarChartBuilder}{\type{DiagramData} data}{}{
            \begin{itemize}
                \item \paramdesc{data}{The data needed to build a \refer{\lblroot:view.diagrams.type:BarChart}{BarChart}.}
            \end{itemize}
        }
    \end{itemize}
}{}

\classdesc{HistogramBuilder}{
    \decl{\public \class HistogramBuilder}{The class that is responsible of building \refer{\lblroot:view.diagrams.type:Histogram}{Histograms}.}
}{}{
    \begin{itemize}
        \item \constrdesc{\public}{HistogramBuilder}{\type{DiagramData} data}{}{
            \begin{itemize}
                \item \paramdesc{data}{The data needed to build a \refer{\lblroot:view.diagrams.type:Histogram}{Histogram}.}
            \end{itemize}
        }
    \end{itemize}
}{}

\classdesc{FunctionGraphBuilder}{
    \decl{\public \class FunctionGraphBuilder}{The class that is responsible of building \refer{\lblroot:view.diagrams.type:FunctionGraph}{FunctionGraph}.}
}{}{
    \begin{itemize}
        \item \constrdesc{\public}{FunctionGraphBuilder}{\type{DiagramData} data}{}{
            \begin{itemize}
                \item \paramdesc{data}{The data needed to build a \refer{\lblroot:view.diagrams.type:FunctionGraph}{FunctionGraph}.}
            \end{itemize}
        }
    \end{itemize}
}{}
\classdesc{HeatMapBuilder}{
    \decl{\public \class HeatMapBuilder}{The class that is responsible of building \refer{\lblroot:view.diagrams.type:HeatMap}{HeatMap}.}
}{}{
    \begin{itemize}
        \item \constrdesc{\public}{HeatMapBuilder}{\type{DiagramData} data}{}{
            \begin{itemize}
                \item \paramdesc{data}{The data needed to build a \refer{\lblroot:view.diagrams.type:HeatMap}{HeatMap}.}
            \end{itemize}
        }
    \end{itemize}
}{}
}

\packagedesc{view.representation}{
\interfacedesc{ICellImageGenerator}{\interface ICellImageGenerator \decl{}{Interface for the image generator}}{
}{CellImageGenerator}{
    \begin{itemize}
        \item \methoddesc{\public}{}{void}{buildCell}{\type{int} inputPins, \type{int} outputPins}
        {Builds a cell representation image with the given amount of input and output pins.
}{
        \begin{itemize}
            \item \paramdesc{inputPins}{Number of input pins of the cell}
            \item \paramdesc{outputPins}{Number of output pins of the cell.}
        \end{itemize}}{}
    \end{itemize}}

    
\classdesc{CellImageGenerator}{\proted \class CellImageGenerator implements ICellImageGenerator \decl{}{Represents a representation image builder for a cell.}
}{
    \begin{itemize}
        \item \atrdesc{\private}{}{BufferedImage}{pinIcon}{Image for pins of a cell}
        \item \atrdesc{\private}{}{BufferedImage}{cellIcon}{Image for a cell representation}
    \end{itemize}
}{
    \begin{itemize}
        \item \constrdesc{\public}{CellImageGenerator}{}{}{}
    \end{itemize}
}{  \begin{itemize}
        \item \methoddesc{\public}{}{void}{buildCell}{\type{int} inputPins, \type{int} outputPins}
        {Builds a cell representation image with the given amount of input and output pins.
}{
        \begin{itemize}
            \item \paramdesc{inputPins}{Number of input pins of the cell}
            \item \paramdesc{outputPins}{Number of output pins of the cell.}
        \end{itemize}}{}
    \end{itemize}}{}
\classdesc{DataPanel}{\public \class DataPanel \decl{}{Displays data about the opened element.}}{
    \begin{itemize}
        \item \atrdesc{\private}{}{Label}{label}{Label for the data text.}
        \item \atrdesc{\private}{}{String}{text}{Certain information about the opened liberty file.}
    \end{itemize}
}{
    \begin{itemize}
        \item \constrdesc{\public}{DataPanel}{}{}{}
    \end{itemize}}{
    \begin{itemize}
        \item \methoddesc{\public}{}{void}{setText}{\type{String} text}
        {Changes the text in the data panel.
}{
        \begin{itemize}
            \item \paramdesc{text}{New data to be shown on the data panel.}
        \end{itemize}
        }{}
    \end{itemize}}{}
\classdesc{CellPanel}{\public \class CellPanel \decl{}{Displays a representation of the opened cell.}}{
    \begin{itemize}
        \item \atrdesc{\private}{}{Label}{label}{Label for the text on the cell panel}
        \item \atrdesc{\private}{}{Button[]}{buttons}{Array of all buttons for the pins}
        \item \atrdesc{\private}{}{Checkbox[]}{checkboxes}{Checkboxes for the used pins}
        \item \atrdesc{\private}{}{Element}{cell}{Cell opened in the visualizer.}
        \item \atrdesc{\private}{}{Element[]}{pins}{Pins of the opened cell.}
        \item \atrdesc{\private}{}{BufferedImage}{cellImage}{Cell image which is generated by the cellGenerator.}
        \item \atrdesc{\private}{}{CellImageGenerator}{cellGenerator}{Builds the cell representation image}
    \end{itemize}
}{
    \begin{itemize}
        \item \constrdesc{\public}{CellPanel}{\type{Element} element}{}{
        \begin{itemize}
            \item \paramdesc{element}{Pin/Cell to be opened in the cell panel.}
        \end{itemize}
        }
    \end{itemize}
}{
    \begin{itemize}
        \item \methoddesc{\public}{}{void}{switchToLibrary}{}
        {Switches to the parent library panel.}{}{}
        \item \methoddesc{\public}{}{void}{switchToPin}{\type{Element} element}
        {Switches the panel for a selected pin of the cell.}{
        \begin{itemize}
            \item \paramdesc{element}{Target pin}
        \end{itemize}}{}
        \item \methoddesc{\public}{}{void}{switchToCell}{\type{Element} element}
        {Switches the panel for the parent cell.}{
        \begin{itemize}
            \item \paramdesc{element}{Target cell}
        \end{itemize}
        }{}
    \end{itemize}
}{}
\classdesc{LibraryPanel}{\public \class LibraryPanel \decl{}{Displays the cells inside the selected library.}}{
    \begin{itemize}
        \item \atrdesc{\private}{}{List<Button>}{buttons}{List of the buttons for each cell.}
        \item \atrdesc{\private}{}{List<Cell>}{cells}{List of every cell in the library}
        \item \atrdesc{\private}{}{Library}{selectedLibrary}{Library which has been opened in the visualizer.}
    \end{itemize}
}{
    \begin{itemize}
        \item \constrdesc{\public}{LibraryPanel}{\type{Library} Library}{}{}
    \end{itemize}
}{
    \begin{itemize}
        \item \methoddesc{\public}{}{void}{switchToCell}{\type{Element} element}
        {Switches from the library panel to the cell panel of the selected cell.}{
        \begin{itemize}
            \item \paramdesc{element}{Target cell.}
        \end{itemize}}{}
    \end{itemize}
}{}
}


\packagedesc{model.elements.attributes}{
\absclassdesc{Attribute}{Attribute class}
{\begin{itemize}
    \item \atrdesc{\proted}{}{\type{Stat}}{stats}{}
\end{itemize}
}{
\begin{itemize}
    \item \implclsdec{Leakage}
    \item \implclsdec{InAttribute}
    \item \implclsdec{OutAttribute}
\end{itemize}
}
{a}
{
\begin{itemize}
        \item \methoddesc{\public}{}{void}{calculate}{}{}
        {}{}
        \item \methoddesc{\private}{}{void}{scale}{\type{float} value}
        {{\begin{itemize}
        \item \paramdesc{value} {The scaling value.}
        \end{itemize}}}{}{}
        \item \methoddesc{\public}{}{\type{Stats}}{getStats}{}{}{}{
        
        Stats will be returned.}
\end{itemize}
}

\absclassdesc{InAttribute}{}
{\begin{itemize}
    \item \atrdesc{\proted}{}{\type{float[]}}{index1}{}
    \item \atrdesc{\proted}{}{\type{float[]}}{values}{}
\end{itemize}}
{\begin{itemize}
    \item \implclsdec{InputPower}
\end{itemize}}
{a}
{\begin{itemize}
        \item \methoddesc{\public}{}{void}{calculate}{}{}
        {}{}
        \item \methoddesc{\public}{}{void}{scale}{\type{float} value}
        {{\begin{itemize}
        \item \paramdesc{value} {The scaling value.}
        \end{itemize}}}{}{}
        \item \methoddesc{\public}{}{\type{float[]}}{getIndex1}{}{}{}{
        
        1-dimensional index will be returned.}
        \item \methoddesc{\public}{}{\type{float[]}}{getValues}{}{}{}{
        
        The array of values will be returned.}
        \item \methoddesc{\public}{}{void}{setIndex1}{\type{float[]} index}
        {{\begin{itemize}
        \item \paramdesc{index} {The 1-dimensional index of the input pin attributes.}
        \end{itemize}}}{}{}
        \item \methoddesc{\public}{}{void}{setValues}{\type{float[]} values}
        {{\begin{itemize}
        \item \paramdesc{values} {The array of input pin attribute values.}
        \end{itemize}}}{}{}
\end{itemize}}


\absclassdesc{OutAttribute}{}
{\begin{itemize}
    \item \atrdesc{\proted}{}{\type{float[]}}{index1}{}
    \item \atrdesc{\proted}{}{\type{float[]}}{index2}{}
    \item \atrdesc{\proted}{}{\type{float[][]}}{values}{}
    \item \atrdesc{\proted}{}{\type{InputPin}}{relatedPin}{}
\end{itemize}}
{\begin{itemize}
    \item \implclsdec{OutputPower}
    \item \implclsdec{Timing}
\end{itemize}}
{a}
{\begin{itemize}
        \item \methoddesc{\public}{}{void}{calculate}{}{}
        {}{}
        \item \methoddesc{\public}{}{void}{scale}{\type{float} value}
        {{\begin{itemize}
        \item \paramdesc{value} {The scaling value.}
        \end{itemize}}}{}{}
        \item \methoddesc{\public}{}{\type{float[]}}{getIndex1}{}{}{}{
        
        1-dimensional first index array will be returned.}
        \item \methoddesc{\public}{}{\type{float[]}}{getIndex2}{}{}{}{
        
        1-dimensional second index array will be returned.}
        \item \methoddesc{\public}{}{\type{float[][]}}{getValues}{}{}{}{
        
        The array of values will be returned.}
        \item \methoddesc{\public}{}{void}{setIndex1}{\type{float[]} index}
        {{\begin{itemize}
        \item \paramdesc{index} {The 1-dimensional first index array of the output pin attributes.}
        \end{itemize}}}{}{}
        \item \methoddesc{\public}{}{void}{setIndex2}{\type{float[]} index}
        {{\begin{itemize}
        \item \paramdesc{index} {The 1-dimensional second index array of the output pin attributes.}
        \end{itemize}}}{}{}
        \item \methoddesc{\public}{}{void}{setValues}{\type{float[][]} values}
        {{\begin{itemize}
        \item \paramdesc{values} {The array of input pin attribute values.}
        \end{itemize}}}{}{}
        \item
        \methoddesc{\public}{}{void}{setRelatedPin}{\type{InputPin} inpin}
        {{\begin{itemize}
        \item \paramdesc{inpin} {The input pin that the output pin is going to be related to.}
        \end{itemize}}}{}{}
\end{itemize}}


\classdesc{Leakage}{}
{\begin{itemize}
    \item \atrdesc{\private}{}{\type{float[]}}{values}{}
\end{itemize}}
{\begin{itemize}
        \item \constrdesc{\public}{Leakage}{\type{float[]} values}{{\begin{itemize}
        \item \paramdesc{values} {The array of leakages of a cell.}
        \end{itemize}}}{}
    \end{itemize}}
{}


\classdesc{InputPower}{}
{\begin{itemize}
    \item \atrdesc{\private}{}{\type{PowerGroup}}{powgroup}{}
\end{itemize}}
{{\begin{itemize}
        \item \constrdesc{\public}{InputPower}{{\type{PowerGroup} powgroup, \type{float[]} values}}
        {{\begin{itemize}
        \item \paramdesc{powgroup} {The power group of the values.}
        \item \paramdesc{values} {The array of power values of an input pin.}
        \end{itemize}}}{}
    \end{itemize}}}{}
    
    
\classdesc{OutputPower}{}
{\begin{itemize}
    \item \atrdesc{\private}{}{\type{PowerGroup}}{powgroup}{}
\end{itemize}}
{\begin{itemize}
        \item \constrdesc{\public}{OutputPower}{{\type{PowerGroup} powgroup, \type{float[][]} values}}
        {{\begin{itemize}
        \item \paramdesc{powgroup} {The power group of the values.}
        \item \paramdesc{values} {The 2 dimensional array of power values of an output pin.}
        \end{itemize}}}{}
    \end{itemize}}
{}


\classdesc{Timing}{}
{\begin{itemize}
    \item \atrdesc{\private}{}{\type{TimingSense}}{timsense}{}
    \item \atrdesc{\private}{}{\type{TimingType}}{timtype}{}
    \item \atrdesc{\private}{}{\type{TimingGroup}}{timgroup}{}
\end{itemize}}
{\begin{itemize}
        \item \constrdesc{\public}{Timing}{{\type{TimingSense} timsense, \type{TimingType} timtype, \type{TimingGroup} timgroup, \type{InputPin} relatedPin, \type{float[][]} values}}
        {{\begin{itemize}
        \item \paramdesc{timsense} {The timing sense of the values.}
        \item \paramdesc{timtype} {The timing type of the values.}
        \item \paramdesc{timgroup} {The timing group of the values.}
        \item \paramdesc{relatedPin} {The input pin that is related to the output pin for the values.}
        \item \paramdesc{values} {The 2 dimensional array of the values.}
        \end{itemize}}}{}
    \end{itemize}}
{}

\enumdesc{PowerGroup}
{
Keeps track of the power group of the input or output pin.
}{
    \begin{itemize}
        \item \fielddesc{PowerGroup}{RISEPOWER}{RisePower: RisePower of the input or output pin.}
        \item \fielddesc{PowerGroup}{FALLPOWER}{FallPower: FallPower of the input or output pin.}
        \item \fielddesc{PowerGroup}{POWER}{Power: Default power of the input or output pin.}
    \end{itemize}
}{}



\enumdesc{TimingGroup}
{
Keeps track of the timing group of the output pin.
}{
    \begin{itemize}
        \item \fielddesc{TimingGroup}{CELLRISE}{}
        \item
        \fielddesc{TimingGroup}{CELLFALL}{}
        \item \fielddesc{TimingGroup}{FALLTRANSITION}{}
        \item \fielddesc{TimingGroup}{RISETRANSITION}{}
    \end{itemize}
}{}



\enumdesc{TimingType}
{
Keeps track of the timing type of the output pin.
}{
    \begin{itemize}
        \item \fielddesc{TimingType}{COMBINATIONAL}{}
        \item \fielddesc{TimingType}{COMBRISE}{}
        \item \fielddesc{TimingType}{COMBFALL}{}
        \item \fielddesc{TimingType}{TSDISABLE}{}
        \item \fielddesc{TimingType}{TSENABLE}{}
        \item \fielddesc{TimingType}{TSDISABLERISE}{}
        \item \fielddesc{TimingType}{TSDISABLEFALL}{}
        \item \fielddesc{TimingType}{TSENABLERISE}{}
        \item \fielddesc{TimingType}{TSENABLEFALL}{}
    \end{itemize}
}{}
\enumdesc{TimingSense}
{
Keeps track of the timing sense of the output pin.
}{
    \begin{itemize}
        \item \fielddesc{TimingSense}{POSITIVE}{}
        \item
        \fielddesc{TimingSense}{NEGATIVE}{}
        \item \fielddesc{TimingSense}{NON}{}
    \end{itemize}
}{}
}


\packagedesc{model.elements}{

\absclassdesc{Element}{}
{\begin{itemize}
    \item \atrdesc{\proted}{}{\type{boolean}}{filtered}{}
    \item \atrdesc{\proted}{}{\type{boolean}}{searched}{}
    \item \atrdesc{\proted}{}{\type{String}}{name}{}
\end{itemize}}
{\begin{itemize}
    \item \implclsdec{HigherElement}
    \item \implclsdec{Pin}
\end{itemize}}
{   \begin{itemize}
        \item \constrdesc{\public}{Element}{}{}{}
    \end{itemize}}
{\begin{itemize}
        \item \methoddesc{\public}{\abstr}{void}{calculate}{}{}
        {}{}
        \item \methoddesc{\public}{}{\type{int}}{compareTo}{\type{Element} element}{}{}{}
        \item \methoddesc{\public}{}{\type{boolean}}{getFiltered}{}{}
        {}{}
        \item \methoddesc{\public}{}{void}{setFiltered}{\type{boolean} filtered}{}{}{}
        \item \methoddesc{\public}{}{\type{boolean}}{getSearched}{}{}
        {}{}
        \item \methoddesc{\public}{}{void}{setSearched}{\type{boolean} searched}{}{}{}
        \item \methoddesc{\public}{}{\type{String}}{getName}{}{}
        {}{}
        \item \methoddesc{\public}{}{void}{setName}{\type{String} name}{}{}{}
\end{itemize}
}
{}



\absclassdesc{HigherElement}{}
{\begin{itemize}
    \item \atrdesc{\private}{}{\type{ArrayList<\type{TimingSense}>}}{availableTimSen}{}
    \item \atrdesc{\private}{}{\type{ArrayList<\type{TimingGroup}>}}{availableTimGr}{}
    \item \atrdesc{\private}{}{\type{ArrayList<\type{TimingType}>}}{availableTimType}{}
    \item \atrdesc{\private}{}{\type{ArrayList<\type{PowerGroup}>}}{availableOutputPower}{}
    \item \atrdesc{\private}{}{\type{ArrayList<\type{PowerGroup}>}}{availableInputPower}{}
    \item \atrdesc{\private}{}{\type{Map<\type{Sense}, \type{Map<\type{Group}, Map<\type{Type}, \type{Stat}> > >}}}{timingStat}{}
    \item \atrdesc{\private}{}{\type{Map<\type{Group}, \type{Stat}>}}{inPowerStat}{}
    \item \atrdesc{\private}{}{\type{Map<\type{Group}, \type{Stat}>}}{outPowerStat}{}
    \item \atrdesc{\proted}{}{\type{boolean}}{hasShownElements}{}
    \item \atrdesc{\proted}{}{\type{Stat}}{leakage}{}
\end{itemize}}
{\begin{itemize}
    \item \implclsdec{Library}
    \item \implclsdec{Cell}
\end{itemize}}
{\begin{itemize}
        \item \constrdesc{\public}{HigherElement}{}{}{}
    \end{itemize}}
{\begin{itemize}
        \item \methoddesc{\public}{}{\type{ArrayList<PowerGroup>}}{getAvailableOutputPower}{}{}
        {}{}
        \item \methoddesc{\public}{}{void}{setAvailableOutputPower}{\type{ArrayList<PowerGroup>} availableOutPow}{}{}{}
        
        \item \methoddesc{\public}{}{\type{ArrayList<PowerGroup>}}{getAvailableInputPower}{}{}
        {}{}
        \item \methoddesc{\public}{}{void}{setAvailableInputPower}{\type{ArrayList<PowerGroup>} availableInPow}{}{}{}
        
        \item \methoddesc{\public}{}{\type{ArrayList<TimingSense>}}{getAvailableTimSen}{}{}
        {}{}
        \item \methoddesc{\public}{}{void}{setAvailableTimSen}{\type{ArrayList<TimingSense>} availableTimSen}{}{}{}
        
        \item \methoddesc{\public}{}{\type{ArrayList<TimingGroup>}}{getAvailableTimGr}{}{}
        {}{}
        \item \methoddesc{\public}{}{void}{setAvailableTimGr}{\type{ArrayList<TimingGroup>} availableTimGr}{}{}{}
        
        \item \methoddesc{\public}{}{\type{ArrayList<TimingType>}}{getAvailableTimType}{}{}
        {}{}
        \item \methoddesc{\public}{}{void}{setAvailableTimType}{\type{ArrayList<TimingType>} availableTimType}{}{}{}
        
        \item \methoddesc{\public}{}{\type{Map<Sense, Map<Group, Map<Type, Stat> > >}}{getTimingStat}{}{}
        {}{}
        
        \item \methoddesc{\public}{}{\type{Map<Group, Stat>}}{getInPowerStat}{}{}
        {}{}
        
        \item \methoddesc{\public}{}{\type{Map<Group, Stat>}}{getOutPowerStat}{}{}
        {}{}
        
        \item \methoddesc{\public}{}{\type{Stat}}{getLeakageStat}{}{}
        {}{}
        
        \item \methoddesc{\public}{}{\type{boolean}}{getHasShownElements}{}{}
        {}{}
        
        \item \methoddesc{\public}{}{void}{setHasShownElements}{\type{boolean} hasShownElements}{}
        {}{}
        
        \item \methoddesc{\public}{}{void}{calculateHasShownElements}{}{}
        {}{}
        
\end{itemize}}{}


\classdesc{Library}{}
{\begin{itemize}
    \item \atrdesc{\private}{}{\type{float[]}}{index1}{}
    \item \atrdesc{\private}{}{\type{float[]}}{index2}{}
    \item \atrdesc{\private}{}{\type{String}}{path}{}
    \item \atrdesc{\private}{}{\type{String[]}}{fileData}{}
    \item \atrdesc{\private}{}{\type{ArrayList<\type{Cell}>}}{cells}{}
    \item \atrdesc{\private}{}{\type{Stat}}{defaultLeakage}{}
\end{itemize}}
{\begin{itemize}
        \item \constrdesc{\public}{Library}{\type{float[]} index1, \type{float[]} index2}{}{}
    \end{itemize}}
{\begin{itemize}
        \item \methoddesc{\public}{\abstr}{void}{calculate}{}{}
        {}{}
        \item \methoddesc{\public}{}{\type{ArrayList<Cell>}}{getCells}{}{}{}{}
        \item \methoddesc{\public}{}{void}{setCells}{\type{ArrayList<Cell>} cells}{}{}{}
        \item \methoddesc{\public}{}{\type{boolean}}{getSearched}{}{}
        {}{}
        \item \methoddesc{\public}{}{void}{calculateLeakage}{}{}{}{}
        \item \methoddesc{\public}{}{void}{calculateInPow}{}{}
        {}{}
        \item \methoddesc{\public}{}{void}{calculateOutPow}{}{}
        {}{}
        \item \methoddesc{\public}{}{void}{calculateTiming}{}{}
        {}{}
        \item \methoddesc{\public}{}{void}{calculateDefaultLeakage}{}{}
        {}{}
        \item \methoddesc{\public}{}{\type{String}}{getPath}{}{}{}{}
        \item \methoddesc{\public}{}{void}{setPath}{\type{String} path}{}{}{}
        \item \methoddesc{\public}{}{\type{float[]}}{getIndex1}{}{}{}{}
        \item \methoddesc{\public}{}{\type{float[]}}{getIndex2}{}{}{}{}
        \item \methoddesc{\public}{}{\type{Stat}}{getDefaultLeakage}{}{}{}{}
        \item \methoddesc{\public}{\static}{void}{saveLibrary}{}
        {Saves library as a Liberty File in the path specified in the Library object. If the library doesn't have a physical copy yet, it instead calls the saveLibraryAs method of the same class}{}{}
        \item \methoddesc{\public}{\static}{void}{saveLibraryAs}{}
        {Opens the OS File Manager in order to select where to save the Library as a Liberty File. Updates Path on Library object.}{}{}
        \item \methoddesc{\public}{\static}{void}{saveAsCSV}{}
        {Opens the OS File Manager in order to select where to save the Element as a CSV File. Updates Path on Library object.}{}{}
\end{itemize}}



\classdesc{Cell}{}
{\begin{itemize}
    \item \atrdesc{\private}{}{\type{Library}}{parentLibrary}{}
    \item \atrdesc{\private}{}{\type{InputPin[]}}{inputPins}{}
    \item \atrdesc{\private}{}{\type{OutputPin[]}}{outputPins}{}
    \item \atrdesc{\private}{}{\type{Leakage[]}}{leakages}{}
    \item \atrdesc{\private}{}{\type{float}}{defaultLeakage}{}
\end{itemize}}
{\begin{itemize}
        \item \constrdesc{\public}{Cell}{}{}{}
    \end{itemize}}
{\begin{itemize}
        
        \item \methoddesc{\public}{}{\type{InputPin[]}}{getInPins}{}{}{}{}
        \item \methoddesc{\public}{}{void}{setInPins}{\type{InputPin[]} inPins}{}{}{}
        \item \methoddesc{\public}{}{\type{OutputPin[]}}{getOutPins}{}{}{}{}
        \item \methoddesc{\public}{}{void}{setOutPins}{\type{OutputPin[]} outPins}{}{}{}
        
        \item \methoddesc{\public}{}{\type{Library}}{getParent}{}{}{}{}
        \item \methoddesc{\public}{}{void}{setParent}{\type{Library} parentLibrary}{}{}{}
        
        \item \methoddesc{\public}{}{\type{Leakage[]}}{getLeakage}{}{}{}{}
        \item \methoddesc{\public}{}{void}{setLeakage}{\type{Leakage[]} leakages}{}{}{}
        
        \item \methoddesc{\public}{}{\type{float}}{getDefaultLeakage}{}{}{}{}
        \item \methoddesc{\public}{}{void}{setDefaultLeakage}{\type{float} defaultLeakage}{}{}{}

        \item \methoddesc{\public}{}{void}{calculateLeakage}{}{}{}{}
        \item \methoddesc{\public}{}{void}{calculateInPow}{}{}
        {}{}
        \item \methoddesc{\public}{}{void}{calculateOutPow}{}{}
        {}{}
        \item \methoddesc{\public}{}{void}{calculateTiming}{}{}
        {}{}
        \item \methoddesc{\public}{}{void}{interpolate}{\type{float[]} index1, \type{float[]} index2 }{}{}{}
        \item \methoddesc{\public}{\static}{void}{saveAsCSV}{}
        {Opens the OS File Manager in order to select where to save the Element as a CSV File. Updates Path on Library object.}{}{}

\end{itemize}}

\absclassdesc{Pin}{}
{\begin{itemize}
    \item \atrdesc{\proted}{}{\type{Cell}}{parentCell}{}
    \item \atrdesc{\proted}{}{\type{float}}{capacitance}{}
    \item \atrdesc{\proted}{}{\type{ArrayList<PowerGroup>}}{availablePower}{}
\end{itemize}}
{\begin{itemize}
    \item \implclsdec{InputPin}
    \item \implclsdec{OutputPin}
\end{itemize}}
{\begin{itemize}
        \item \constrdesc{\public}{Pin}{}{}{}
    \end{itemize}}
{\begin{itemize}
        
        \item \methoddesc{\public}{}{\type{Cell}}{getParent}{}{}{}{}
        \item \methoddesc{\public}{}{void}{setParent}{\type{Cell} parentCell}{}{}{}
        
        \item \methoddesc{\public}{}{\type{ArrayList<PowerGroup>}}{getAvailablePower}{}{}{}{}
        \item \methoddesc{\public}{}{void}{setAvailablePower}{\type{ArrayList<PowerGroup>} availablePowers}{}{}{}
\end{itemize}}



\classdesc{InputPin}{}
{\begin{itemize}
    \item \atrdesc{\private}{}{\type{ArrayList<InputPower>}}{inputPowers}{}
\end{itemize}}
{\begin{itemize}
        \item \constrdesc{\public}{InputPin}{}{}{}
    \end{itemize}}
{\begin{itemize}
        \item \methoddesc{\public}{}{void}{calculatePower}{}{}{}{}
        
        \item \methoddesc{\public}{}{\type{ArrayList<InputPower>}}{getInputPowers}{}{}{}{}
        \item \methoddesc{\public}{}{void}{setInputPowers}{\type{ArrayList<InputPower>} inputPowers}{}{}{}
        \item \methoddesc{\public}{}{void}{interpolate}{\type{float[]} index1}{}{}{}
        \item \methoddesc{\public}{\static}{void}{saveAsCSV}{}
        {Opens the OS File Manager in order to select where to save the Element as a CSV File. Updates Path on Library object.}{}{}
\end{itemize}}


\classdesc{OutputPin}{}
{\begin{itemize}
    \item \atrdesc{\private}{}{\type{ArrayList<TimingSense>}}{availableTimSen}{}
    \item \atrdesc{\private}{}{\type{ArrayList<TimingGroup>}}{availableTimGr}{}
    \item \atrdesc{\private}{}{\type{ArrayList<TimingType>}}{availableTimType}{}
    \item \atrdesc{\private}{}{\type{ArrayList<OutputPower>}}{outputPowers}{}
    \item \atrdesc{\private}{}{\type{ArrayList<Timing>}}{timings}{}
    \item \atrdesc{\private}{}{\type{String}}{outputFunction}{}
\end{itemize}}
{\begin{itemize}
        \item \constrdesc{\public}{OutputPin}{}{}{}
    \end{itemize}}
{\begin{itemize}
        
        \item \methoddesc{\public}{}{\type{ArrayList<Timing>}}{getTimings}{}{}{}{}
        \item \methoddesc{\public}{}{void}{setTimings}{\type{ArrayList<Timing>} timings}{}{}{}
        
        \item \methoddesc{\public}{}{\type{ArrayList<OutputPower>}}{getOutputPowers}{}{}{}{}
        \item \methoddesc{\public}{}{void}{setOutputPowers}{\type{ArrayList<OutputPower>} outputPowers}{}{}{}
        
        \item \methoddesc{\public}{}{\type{ArrayList<PowerGroup>}}{getAvailablePower}{}{}{}{}
        \item \methoddesc{\public}{}{void}{setAvailablePower}{\type{ArrayList<PowerGroup>} availablePowers}{}{}{}
        
        \item \methoddesc{\public}{}{\type{ArrayList<TimingSense>}}{getAvailableTimSen}{}{}
        {}{}
        \item \methoddesc{\public}{}{void}{setAvailableTimSen}{\type{ArrayList<TimingSense>} availableTimSen}{}{}{}
        
        \item \methoddesc{\public}{}{\type{ArrayList<TimingGroup>}}{getAvailableTimGr}{}{}
        {}{}
        \item \methoddesc{\public}{}{void}{setAvailableTimGr}{\type{ArrayList<TimingGroup>} availableTimGr}{}{}{}
        
        \item \methoddesc{\public}{}{\type{ArrayList<TimingType>}}{getAvailableTimType}{}{}
        {}{}
        \item \methoddesc{\public}{}{void}{setAvailableTimType}{\type{ArrayList<TimingType>} availableTimType}{}{}{}
        
        \item \methoddesc{\public}{}{void}{calculatePower}{}{}{}{}
        
        \item \methoddesc{\public}{}{void}{calculateTiming}{}{}{}{}
        
        \item \methoddesc{\public}{}{void}{interpolate}{\type{float[]} index1, \type{float[]} index2 }{}{}{}
        \item \methoddesc{\public}{\static}{void}{saveAsCSV}{}
        {Opens the OS File Manager in order to select where to save the Element as a CSV File. Updates Path on Library object.}{}{}
\end{itemize}}


\classdesc{Stat}{}
{\begin{itemize}
    \item \atrdesc{\private}{}{\type{float}}{min}{}
    \item \atrdesc{\private}{}{\type{float}}{max}{}
    \item \atrdesc{\private}{}{\type{float}}{avg}{}
    \item \atrdesc{\private}{}{\type{float}}{med}{}
\end{itemize}}
{\begin{itemize}
        \item \constrdesc{\public}{Stat}{\type{float} min, \type{float} max, \type{float} avg, \type{float} med}{}{}
    \end{itemize}}
{\begin{itemize}
        
        \item \methoddesc{\public}{}{\type{float}}{getMin}{}{}{}{}
        \item \methoddesc{\public}{}{void}{setMin}{\type{float} min}{}{}{}
        
        \item \methoddesc{\public}{}{\type{float}}{getMax}{}{}{}{}
        \item \methoddesc{\public}{}{void}{setMax}{\type{float} max}{}{}{}
        
        \item \methoddesc{\public}{}{\type{float}}{getAvg}{}{}{}{}
        \item \methoddesc{\public}{}{void}{setAvg}{\type{float} avg}{}{}{}
        
        \item \methoddesc{\public}{}{\type{float}}{getMed}{}{}{}{}
        \item \methoddesc{\public}{}{void}{setMed}{\type{float} med}{}{}{}
        
\end{itemize}}
}



\packagedesc{model.commands}{
\interfacedesc{Command}
    {
    An interface that is implemented by all commands.
    }
    {}
    {
    \begin{itemize}
        \item \implclsdec{OpenFileAction}
        \item \implclsdec{RemoveAction}
        \item \implclsdec{AddFilterAction}
        \item \implclsdec{RemoveFilterAction}
        \item \implclsdec{DeleteAction}
        \item \implclsdec{MergeAction}
        \item \implclsdec{MoveAction}
        \item \implclsdec{RenameAction}
        \item \implclsdec{SelectAction}
        \item \implclsdec{CreateLibraryAction}
        \item \implclsdec{ScaleAction}
        \item \implclsdec{UndoAction}
        \item \implclsdec{TextEditAction}
    \end{itemize}
    }
    {
    \begin{itemize}
        \item \methoddesc{\public}{}{void}{execute}{}
        {}{}{}{}
        \item \methoddesc{\public}{}{void}{undo}{}
        {}{}{}
    \end{itemize}
    }
\classdesc{OpenFileAction}{Opens a file.} {   \begin{itemize}
        \item \atrdesc{\private}{}{\type{Library}}{openedLibrary}{}
    \end{itemize}
}
{
    {\begin{itemize}
        \item \constrdesc{\public}{OpenFileAction}{}{}{}
    \end{itemize}
    }}
    
    {\begin{itemize}
        \item \methoddesc{\public}{}{void}{execute}{}
        {}{}{}
        \item \methoddesc{\public}{}{void}{undo}{}
        {}{}{}
    \end{itemize}
    }

\classdesc{RemoveAction}{Removes a library from the view.}
    {\begin{itemize}
        \item \atrdesc{\private}{}{\type{Library}}{removedLibrary}{}
    \end{itemize}} 
{   \begin{itemize}
        \item \constrdesc{\public}{RemoveAction}{\type{Library} library}{}
        {
            \begin{itemize}
            \item \paramdesc{library}{The library that is going to be removed from the view.}
            \end{itemize}
        }
    \end{itemize}
}
    {\begin{itemize}
        \item \methoddesc{\public}{}{void}{execute}{}
        {}{}{}
        \item \methoddesc{\public}{}{void}{undo}{}
        {}{}{}
    \end{itemize}
    }
    
    
\classdesc{AddFilterAction}{Adds a filter.}
    {\begin{itemize}
        \item \atrdesc{\private}{}{\type{Filter}}{addedFilter}{}
    \end{itemize}} 
{   \begin{itemize}
        \item \constrdesc{\public}{AddFilterAction}{\type{Filter} filter}{}
        {
            \begin{itemize}
            \item \paramdesc{filter}{The filter that is going to be added.}
            \end{itemize}
        }
    \end{itemize}
}
    {\begin{itemize}
        \item \methoddesc{\public}{}{void}{execute}{}
        {}{}{}
        \item \methoddesc{\public}{}{void}{undo}{}
        {}{}{}
    \end{itemize}
    }



\classdesc{RemoveFilterAction}{Removes a filter.}
    {\begin{itemize}
        \item \atrdesc{\private}{}{\type{Filter}}{removedFilter}{}
    \end{itemize}} 
{   \begin{itemize}
        \item \constrdesc{\public}{RemoveFilterAction}{\type{Filter} filter}{}
        {
            \begin{itemize}
            \item \paramdesc{filter}{The filter that is going to be removed.}
            \end{itemize}
        }
    \end{itemize}
}
    {\begin{itemize}
        \item \methoddesc{\public}{}{void}{execute}{}
        {}{}{}
        \item \methoddesc{\public}{}{void}{undo}{}
        {}{}{}
    \end{itemize}
    }


\classdesc{DeleteAction}{Deletes a cell from a library.}
    {\begin{itemize}
        \item \atrdesc{\private}{}{\type{Cell}}{deletedCell}{}
    \end{itemize}} 
{   \begin{itemize}
        \item \constrdesc{\public}{DeleteAction}{\type{Cell} cell}{}
        {
            \begin{itemize}
            \item \paramdesc{cell}{The cell that is going to be deleted.}
            \end{itemize}
        }
    \end{itemize}
}
    {\begin{itemize}
        \item \methoddesc{\public}{}{void}{execute}{}
        {}{}{}
        \item \methoddesc{\public}{}{void}{undo}{}
        {}{}{}
    \end{itemize}
    }


\classdesc{MergeAction}{Merges multiple selected libraries to a single library.}
    {\begin{itemize}
        \item \atrdesc{\private}{}{\type{Library[]}}{mergedLibraries}{}
        \item
        \atrdesc{\private}{}{\type{Library}}{productLibrary}{}
    \end{itemize}} 
{   \begin{itemize}
        \item \constrdesc{\public}{MergeAction}{\type{String} name}{}
        {
            \begin{itemize}
            \item \paramdesc{name}{Name of the resulting library}
            \end{itemize}
        }
    \end{itemize}
}
    {\begin{itemize}
        \item \methoddesc{\public}{}{void}{execute}{}
        {}{}{}
        \item \methoddesc{\public}{}{void}{undo}{}
        {}{}{}
        \item \methoddesc{\proted}{}{void}{continue}{\type{NameConflictResolver} resolver}
        {}{
            \begin{itemize}
            \item \paramdesc{resolver}{Object that handled the naming conflict resolution}
            \end{itemize}
        }{}
        \item \methoddesc{\proted}{}{void}{cancel}{}
        {}{}{}
    \end{itemize}
    }
\classdesc{PasteAction}{Moves copied cells to a desired library.}
    {\begin{itemize}
        \item \atrdesc{\private}{}{\type{HashMap<\type{Cell}, \type{String}>}}{renamedCellsOldNames}{}
        \item \atrdesc{\private}{}{\type{ArrayList<\type{Cell}>}}{pastedCells}{}
        \item \atrdesc{\private}{}{\type{ArrayList<\type{Cell}>}}{deletedCells}{}
        \item \atrdesc{\private}{}{Library}{destinationLibrary}{}
    \end{itemize}} 
{   \begin{itemize}
        \item \constrdesc{\public}{PasteAction}{\type{Library} library}{}
        {
            \begin{itemize}
            \item \paramdesc{library}{The library that the copied cells are going to be pasted to.}
            \end{itemize}
        }
        \item \constrdesc{\public}{PasteAction}{\type{Library} library, \type{ArrayList<\type{Cell}>, cells}}{
            This constructor is mainly here to be used by MoveAction instead of pasting copied cells.
        }{
            \begin{itemize}
            \item \paramdesc{library}{The library that the copied cells are going to be pasted to.}
            \item \paramdesc{cells}{The cells that are going to be moved to the library.}
            \end{itemize}
        }
    \end{itemize}
}
    {\begin{itemize}
        \item \methoddesc{\public}{}{void}{execute}{}
        {}{}{}
        \item \methoddesc{\public}{}{void}{undo}{}
        {}{}{}
        \item \methoddesc{\proted}{}{void}{continue}{\type{NameConflictResolver} resolver}
        {}{
            \begin{itemize}
            \item \paramdesc{resolver}{Object that handled the naming conflict resolution}
            \end{itemize}
        }{}
        \item \methoddesc{\proted}{}{void}{cancel}{}
        {}{}{}
    \end{itemize}
    }

\classdesc{MoveAction}{Moves selected cells to a desired library.}
    {\begin{itemize}
        \item \atrdesc{\private}{}{\type{HashMap<\type{Cell}, \type{Library}>}}{initialPositions}{}\item \atrdesc{\private}{}{\type{HashMap<\type{Cell}, \type{String}>}}{renamedCellsOldNames}{}
        \item \atrdesc{\private}{}{\type{ArrayList<\type{Cell}>}}{deletedCells}{}
        \item
        \atrdesc{\private}{}{Library}{destinationLibrary}{}
    \end{itemize}} 
{   \begin{itemize}
        \item \constrdesc{\public}{MoveAction}{\type{ArrayList<\type{Cell}>} cells, \type{Library} library}{}
        {
            \begin{itemize}
            \item \paramdesc{cells}{Array of the selected cells that are going to be moved.}
            \item \paramdesc{library}{The library that the cells are going to be moved to.}
            \end{itemize}
        }
    \end{itemize}
}
    {\begin{itemize}
        \item \methoddesc{\public}{}{void}{execute}{}
        {}{}{}
        \item \methoddesc{\public}{}{void}{undo}{}
        {}{}{}
        \item \methoddesc{\proted}{}{void}{continue}{\type{NameConflictResolver} resolver}
        {}{
            \begin{itemize}
            \item \paramdesc{resolver}{Object that handled the naming conflict resolution}
            \end{itemize}
        }{}
        \item \methoddesc{\proted}{}{void}{cancel}{}
        {}{}{}
    \end{itemize}
    }

\classdesc{RenameAction}{Changes the name of an element.}
    {\begin{itemize}
        \item \atrdesc{\private}{}{\type{Element}} {renamedElement}{}
        \item \atrdesc{\private}{}{\type{String}} {oldName}{}
        \item
        \atrdesc{\private}{}{\type{String}}{newName}{}
    \end{itemize}} 
{   \begin{itemize}
        \item \constrdesc{\public}{RenameAction}{\type{Element} element, \type{String} name}{}
        {
            \begin{itemize}
            \item \paramdesc{element}{Element that is going to be renamed.}
            \item \paramdesc{name}{The new name of the element.}
            \end{itemize}
        }
    \end{itemize}
}
    {\begin{itemize}
        \item \methoddesc{\public}{}{void}{execute}{}
        {}{}{}
        \item \methoddesc{\public}{}{void}{undo}{}
        {}{}{}
    \end{itemize}
    }


\classdesc{SelectAction}{Selects an element.}
    {\begin{itemize}
        \item \atrdesc{\private}{}{\type{HashSet<Element>}} {selectedElements}{}
        \item \atrdesc{\private}{}{\type{HashSet<Element>}} {deselectedElements}{}
    \end{itemize}} 
{   \begin{itemize}
        \item \constrdesc{\public}{SelectAction}{\type{Element} element}{}
        {
            \begin{itemize}
            \item \paramdesc{element}{Element that is going to be selected.}
            \end{itemize}
        }
    \end{itemize}
}
    {\begin{itemize}
        \item \methoddesc{\public}{}{void}{execute}{}
        {}{}{}
        \item \methoddesc{\public}{}{void}{undo}{}
        {}{}{}
    \end{itemize}
    }

\classdesc{CreateLibraryAction}{Creates a library.}
    {\begin{itemize}
        \item \atrdesc{\private}{}{\type{Library}} {createdLibrary}{}
    \end{itemize}} 
{   \begin{itemize}
        \item \constrdesc{\public}{CreateLibraryAction}{\type{String} name}{}
        {
            \begin{itemize}
            \item \paramdesc{name}{Name of the created library.}
            \end{itemize}
        }
    \end{itemize}
}
    {\begin{itemize}
        \item \methoddesc{\public}{}{void}{execute}{}
        {}{}{}
        \item \methoddesc{\public}{}{void}{undo}{}
        {}{}{}
    \end{itemize}
    }{}
\classdesc{ScaleAction}{Scales the values of an attribute by a given float.}
    {\begin{itemize}
        \item \atrdesc{\private}{}{\type{Attribute}} {attribute}{}
        \item \atrdesc{\private}{}{\type{float}} {scale}{}
    \end{itemize}} 
{   \begin{itemize}
        \item \constrdesc{\public}{ScaleAction}{\type{Attribute} attribute, \type{float} scale}{}
        {
            \begin{itemize}
            \item \paramdesc{attribute}{Attribute that is going to be scaled.}
            \item \paramdesc{scale}{The value of scale.}
            \end{itemize}
        }
    \end{itemize}
}
    {\begin{itemize}
        \item \methoddesc{\public}{}{void}{execute}{}
        {}{}{}
        \item \methoddesc{\public}{}{void}{undo}{}
        {}{}{}
    \end{itemize}
    }{}
\classdesc{UndoAction}{Undoes an action.}{}
{\begin{itemize}
        \item \constrdesc{\public}{UndoAction}{}{}{}
    \end{itemize}
}
    {\begin{itemize}
        \item \methoddesc{\public}{}{void}{execute}{}
        {}{}{}
        \item \methoddesc{\public}{}{void}{undo}{}
        {}{}{}
    \end{itemize}
    }{}
\classdesc{TextEditAction}{Makes changes on the text.}
    {\begin{itemize}
        \item \atrdesc{\private}{}{\type{String}} {oldContent}{}
        \item \atrdesc{\private}{}{\type{String}} {newContent}{}
        \item \atrdesc{\private}{}{\type{Element}} {element}{}
    \end{itemize}} 
{   \begin{itemize}
        \item \constrdesc{\public}{TextEditAction}{\type{String} oldContent, \type{String} newContent, \type{Element} element}{}
        {
            \begin{itemize}
            \item \paramdesc{oldContent}{Old content of the changed text.}
            \item \paramdesc{newContent}{New content of the changed text.}
            \item \paramdesc{element}{The text of this element is going to be changed.}
            \end{itemize}
        }
    \end{itemize}
}
    {\begin{itemize}
        \item \methoddesc{\public}{}{void}{execute}{}
        {}{}{}
        \item \methoddesc{\public}{}{void}{undo}{}
        {}{}{}
    \end{itemize}
    }{}
\classdesc{ActionHistory}{Keeps the history of the executed Commands and manages it.}
    {\begin{itemize}
        \item \atrdesc{\private}{}{\type{Command[]}} {actions}{}
        \item \atrdesc{\private}{}{\type{Command[]}} {undoneActions}{}
        \item \atrdesc{\private}{}{\type{int}} {undoCount}{}
    \end{itemize}} 
{   \begin{itemize}
        \item \constrdesc{\public}{ActionHistory}{}{}{}
    \end{itemize}
}{
    \begin{itemize}
        \item \methoddesc{\public}{}{void}{setUndoCount}{\type{int} undoCount}{}
        {{\begin{itemize}
        \item \paramdesc{undoCount} {The number that decides how many undo operations can be made.}
        \end{itemize}}
        }{}
        \item \methoddesc{\private}{}{void}{resetUndoneActions}{}
        {}{}{}
        \item \methoddesc{\public}{}{void}{AddAction}{\type{Command} action}
        {}{{\begin{itemize}
        \item \paramdesc{action} {An action that is added to the undo history.}
        \end{itemize}}
        }{}
        \item \methoddesc{\public}{}{void}{removeLatestAction}{}
        {}{}{}
        \item \methoddesc{\public}{}{void}{resetActions}{}
        {}{}{}
        \item \methoddesc{\public}{}{\type{Command}}{getLatestAction}{}
        {}{}{{\begin{itemize}
        \item \paramdesc{Command} {Latest action will be returned.}
        \end{itemize}}}
    \end{itemize}
    }{}
    
\classdesc{NameConflictResolver}{Resolves naming conflicts while keeping track of the taken actions}
    {\begin{itemize}
        \item \atrdesc{\private}{}{\type{HashMap<\type{Cell}, \type{String}>}} {renamedCells}{}
        \item \atrdesc{\private}{}{\type{ArrayList<\type{Cell}>}} {deletedCells}{}
        \item \atrdesc{\private}{}{\type{ArrayList<\type{Cell}>}} {cells}{}
        \item \atrdesc{\private}{}{\type{Command}} {action}{}
        \item \atrdesc{\private}{}{\type{Cell[]}} {currentConflict}{}
    \end{itemize}} 
{   \begin{itemize}
        \item \constrdesc{\public}{NameConflictResolver}{\type{ArrayList<\type{Cell}> cells}}{}{}
    \end{itemize}
}{
    \begin{itemize}
        \item \methoddesc{\public}{}{void}{keepLeft}{}
        {Keeps the left Cell and deletes the right one}{}{}
        \item \methoddesc{\public}{}{void}{keepRight}{}
        {Keeps the right Cell and deletes the left one}{}{}
        \item \methoddesc{\public}{}{void}{renameLeft}{String name}
        {Keeps both cells and renames the left one}{
        \begin{itemize}
            \item \paramdesc{name}{New name}
        \end{itemize}
        }{}
        \item \methoddesc{\public}{}{void}{renameRight}{String name}
        {Keeps both cells and renames the right one}{
        \begin{itemize}
            \item \paramdesc{name}{New name}
        \end{itemize}
        }{}
        \item \methoddesc{\public}{}{void}{continue}{}
        {Continues with the conflict resolution}{}{}
        \item \methoddesc{\public}{}{void}{cancel}{}
        {Cancels the conflict resolution}{}{}
    \end{itemize}
}{}
}
\packagedesc{model.parsers}{
\classdesc{LibertyParser}{
Provides functionality to parse pieces of Liberty File text format into their corresponding data objects.
}{
    \begin{itemize}
        \item \atrdesc{\private}{\static \final}{\type{JsonParser}} {PARSER}{The JSON Parser}
    \end{itemize}
}{
    \begin{itemize}
        \item \constrdesc{\private}{LibertyParser}{}{}{}
    \end{itemize}
}{
    \begin{itemize}
        \item \methoddesc{\public}{\static}{Library}{parseLibrary}{String content}
        {Parses the content in Liberty format into a library object}{
        \begin{itemize}
            \item \paramdesc{content}{String Content to be parsed}
        \end{itemize}
        }{}
        \item \methoddesc{\public}{\static}{Cell}{parseCell}{String content}
        {Parses the content in Liberty format into a Cell object}{
        \begin{itemize}
            \item \paramdesc{content}{String Content to be parsed}
        \end{itemize}
        }{}
        \item \methoddesc{\public}{\static}{Pin}{parsePin}{String content}
        {Parses the content in Liberty format into a Pin object}{
        \begin{itemize}
            \item \paramdesc{content}{String Content to be parsed}
        \end{itemize}
        }{}
    \end{itemize}
}{}
}
\packagedesc{model.compiler}{

\classdesc{CSVCompiler}{
Provides functionality to compile Element Data Objects into their corresponding CSV text format.
}{}{
    \begin{itemize}
        \item \constrdesc{\private}{CSVCompiler}{}{}{}
    \end{itemize}
}{
    \begin{itemize}
        \item \methoddesc{\public}{\static}{String}{compile}{Library library}
        {Compiles a library into CSV Format}{
        \begin{itemize}
            \item \paramdesc{library}{Library object to be compiled}
        \end{itemize}
        }{}
        \item \methoddesc{\public}{\static}{String}{compile}{Cell cell}
        {Compiles a cell into CSV Format}{
        \begin{itemize}
            \item \paramdesc{cell}{Cell object to be compiled}
        \end{itemize}
        }{}
        \item \methoddesc{\public}{\static}{String}{compile}{Pin pin}
        {Compiles a pin into CSV Format}{
        \begin{itemize}
            \item \paramdesc{pin}{Pin object to be compiled}
        \end{itemize}
        }{}
    \end{itemize}
}{}
\classdesc{LibertyCompiler}{
Provides functionality to compile Element Data Objects into their corresponding Liberty File text format.
}{}{
    \begin{itemize}
        \item \constrdesc{\private}{LibertyCompiler}{}{}{}
    \end{itemize}
}{
    \begin{itemize}
        \item \methoddesc{\public}{\static}{String}{compile}{Library library}
        {Compiles a library into Liberty Format}{
        \begin{itemize}
            \item \paramdesc{library}{Library object to be compiled}
        \end{itemize}
        }{}
        \item \methoddesc{\public}{\static}{String}{compile}{Cell cell}
        {Compiles a cell into Liberty Format}{
        \begin{itemize}
            \item \paramdesc{cell}{Cell object to be compiled}
        \end{itemize}
        }{}
        \item \methoddesc{\public}{\static}{String}{compile}{Pin pin}
        {Compiles a pin into Liberty Format}{
        \begin{itemize}
            \item \paramdesc{pin}{Pin object to be compiled}
        \end{itemize}
        }{}
    \end{itemize}
}{}
}

\packagedesc{model.project}{
\classdesc{Model}{
Keeps track of the objects of the entire model package by ensuring the singularity of Project, Settings and Shortcuts classes as well as managing their Files.
}{
    \begin{itemize}
        \item \atrdesc{\private}{\static}{Model}{instance}{The single instance of the Model class}
        \item \atrdesc{\private}{}{Project}{currentProject}{The active Project class}
        \atrdesc{\private}{}{Settings}{currentSettings}{The active Settings class}
        \atrdesc{\private}{}{Project}{currentShortcuts}{The active Shortcuts class}
    \end{itemize}
}{
    \begin{itemize}
        \item \constrdesc{\private}{Model}{}{}{}
    \end{itemize}
}{  
    \begin{itemize}
        \item \methoddesc{\public}{\static}{Model}{getInstance}{}
        {Returns the sole instance of the Model}{}{The sole instance of the Model}
        \item \methoddesc{\public}{\static}{Project}{getCurrentProject}{}
        {Returns the sole instance of the Project}{}{The sole instance of the Project}
        \item \methoddesc{\public}{}{void}{saveProject}{}
        {Opens the OS File Manager in order to select where to save the current Project Object in JSON format.}{}{}
        \item \methoddesc{\public}{}{void}{loadProject}{}
        {Opens the OS File Manager in order to select a File in JSON Format that replaces the current Project Object if the format fits.}{}{}
        \item \methoddesc{\public}{\static}{Settings}{getCurrentSettings}{The sole instance of the Settings}
        {Returns the sole instance of the Settings}{}{}
        \item \methoddesc{\public}{}{void}{saveSettings}{}
        {Saves the Settings Object in the Program files (so that it reloads on rerun).}{}{}
        \item \methoddesc{\public}{}{void}{resetSettings}{}
        {Loads the default Settings Object from the Program Files and calls saveSettings.}{}{}
        \item \methoddesc{\public}{\static}{Shortcuts}{getCurrentShortcuts}{}
        {Returns the sole instance of the Shortcuts}{}{The sole instance of the Shortcuts}
        \item \methoddesc{\public}{}{void}{saveShortcuts}{}
        {Saves the Shortcuts Object in the Program files (so that it reloads on rerun).}{}{}
        \item \methoddesc{\public}{}{void}{resetShortcuts}{}
        {Loads the default Shortcuts Object from the Program Files and calls saveShortcut.}{}{}
        \item \methoddesc{\public}{}{void}{notify}{}
        {Notifies the view of made changes}{}{}
    \end{itemize}
}
\classdesc{Project}{
Keeps track of the Elements loaded into the program
}{
    \begin{itemize}
        \item \atrdesc{\private}{}{ArrayList<Library>}{libraries}{The array that keeps track of the libraries loaded into the program}
        \item \atrdesc{\private}{}{HashSet<Element>}{copiedElements}{The Set that keeps track of the copied elements}
        \item \atrdesc{\private}{}{HashSet<Element>}{openedInTextElements}{The Set that keeps track of the elements opened in a text editor}
        \item \atrdesc{\private}{}{ArrayList<Filter>}{filters}{The array that keeps track of the filters active in the project}
    \end{itemize}
}{
    \begin{itemize}
        \item \constrdesc{\public}{Project}{}{}{}
    \end{itemize}
}{
    \begin{itemize}
        \item \methoddesc{\public}{}{\type{ArrayList<\type{Library}>}}{getLibraries}{}
        {Returns the list of Libraries loaded into the project}{}{The list of Libraries loaded into the project}
        \item \methoddesc{\public}{}{\type{ArrayList<\type{Filter}>}}{getFilters}{}
        {Returns the list of Filters loaded into the project}{}{The list of Filters loaded into the project}
        \item \methoddesc{\public}{}{\type{HashSett<\type{Library}>}}{getOpenedInTextElements}{}
        {Returns the list of Elements opened in the Text Editor}{}{The list of Elements opened in the Text Editor}
        \item \methoddesc{\public}{}{\type{HashSett<\type{Library}>}}{getCopiedElements}{}
        {Returns the list of copied Elements}{}{The list of copied Elements}
        \item \methoddesc{\public}{}{void}{notify}{}
        {Notifies the view of made changes}{}{}
        \item \methoddesc{\public}{\static}{void}{saveDefaultFilters}{}
        {Saves the current array of Filters in the program files.}{}{}
        \item \methoddesc{\public}{\static}{void}{loadDefaultFilters}{}
        {Loads the saved array of Filters from program files into the current Filter class.}{}{}
    \end{itemize}
}{}
\classdesc{Interpolator}{
Provides interpolation functionality.
}{}{
    \begin{itemize}
        \item \constrdesc{\public}{Intepolator}{}{}{}
    \end{itemize}
}{
    \begin{itemize}
        \item \methoddesc{\public}{\static}{\type{float[]}}{interpolator}{\type{float[]} indexes, \type{float[]} values, \type{float[]} newIndexes}
        {Interpolates a set of 2D coordinates (index1 and value) and gives a set of values for a given set of indexes.\newline
        Imports org.apache.commons.math.analysis.interpolation}{
        \begin{itemize}
            \item \paramdesc{indexes}{Original indexes}
            \item \paramdesc{values}{Original values}
            \item \paramdesc{newIndexes}{New required indexes}
        \end{itemize}
        }{}
        \item \methoddesc{\public}{\static}{\type{float[]}}{bicubicInterpolate}{\type{float[]} indexes1, {float[]} indexes2 \type{float[]} values, \type{float[]} newIndexes1, \type{float[]} newIndexes1}
        {Interpolates a set of 3D coordinates (index1, index2 and value) and gives a set of values for a given set of indexes. \newline
        Imports org.apache.commons.math3.analysis.interpolation}{
        \begin{itemize}
            \item \paramdesc{indexes1}{Original indexes1}
            \item \paramdesc{indexes2}{Original indexes2}
            \item \paramdesc{values}{Original}
            \item \paramdesc{newIndexes1}{New required indexes1}
            \item \paramdesc{newIndexes2}{New requires indexes2}
        \end{itemize}
        }{}
    \end{itemize}
}{}

\classdesc{Shortcuts}{
Maps the characters from pressed Key events with their corresponding Actions with the Event enum.}{
    \begin{itemize}
        \item \atrdesc{\private}{}{HashMap<char, Event>}{commands}{The map of characters and their corresponding actions}
    \end{itemize}
}{
    \begin{itemize}
        \item \constrdesc{\public}{Shortcuts}{}{}{}
    \end{itemize}
}{
    \begin{itemize}
        \item \methoddesc{\public}{}{void}{setKey}{\type{char} key, \type{Event} action}
        {Binds a char key to the event in the controller}{
        \begin{itemize}
                \item \paramdesc{key}{The pressed key.}
                \item \paramdesc{action}{The action that corresponds to the key.}
        \end{itemize}
        }{}
        \item \methoddesc{\public}{}{void}{removeKey}{\type{char} key}
        {Removes a key char from the binding}{
        \begin{itemize}
                \item \paramdesc{key}{The pressed key.}
        \end{itemize}
        }{}
        \item \methoddesc{\public}{}{ArrayList<char>}{getKeys}{}
        {Returns all the set char keys}{}{}
        \item \methoddesc{\public}{}{Event}{getAction}{\type{char} key}
        {Returns the action corresponding to a keystroke char}{
        \begin{itemize}
                \item \paramdesc{key}{The pressed key.}
        \end{itemize}
        }{}
    \end{itemize}
}{}
\classdesc{FileManager}{
Provides functionality to save and load Files.
}{}{
    \begin{itemize}
        \item \constrdesc{\public}{FileManager}{}{}{}
    \end{itemize}
}{
    \begin{itemize}
        \item \methoddesc{\public}{\static}{\type{File}}{openFile}{}
        {Opens the OS File Manager in order to select which file will be opened}{}{}
        \item \methoddesc{\public}{\static}{\type{File}}{openFile}{\type{String} path}
        {Opens the file in the specified path}{}{}
        \item \methoddesc{\public}{\static}{void}{saveFile}{\type{String} content, \type{String[]} extensions}
        {Opens the OS File Manager in order to select where the file will be saved}{
        \begin{itemize}
                \item \paramdesc{content}{The content of the file.}
                \item \paramdesc{extensions}{The extensions the file can be saved as.}
        \end{itemize}
        }{}
        \item \methoddesc{\public}{\static}{void}{saveFile}{\type{String} content, \type{String} extension, \type{String} path}
        {Saves the file in the specified path}{
        \begin{itemize}
                \item \paramdesc{content}{The content of the file.}
                \item \paramdesc{extension}{The extension the file will be saved as.}
                \item \paramdesc{path}{The path where the file will be saved.}
        \end{itemize}
        }{}
    \end{itemize}
}{}
\classdesc{Filter}{
Keeps track of the filters, a filters specific data and provides filtering functionality.
}{
    \begin{itemize}
        \item \atrdesc{\private}{\static}{ArrayList<Filter>}{filters}{The array list of all currently active filters}
        \item \atrdesc{\private}{}{Attribute}{attribute}{The attribute that is being filtered.}
        \atrdesc{\private}{}{Mode}{mode}{The way the filtered attribute is calculated.}
        \atrdesc{\private}{}{float}{value}{The value that the attribute is being compared to.}
        \atrdesc{\private}{}{Operation}{operation}{The operation that is being executed in the filter.}
    \end{itemize}
}{
    \begin{itemize}
        \item \constrdesc{\public}{Filter}{\type{Attribute} attribute, \type{Mode} mode, \type{float} value, \type{Operation} operation}{}{
        \begin{itemize}
            \item \paramdesc{attribute}{}
            \item \paramdesc{mode}{}
            \item \paramdesc{value}{}
            \item \paramdesc{operation}{}
        \end{itemize}
        }
    \end{itemize}
}{
    \begin{itemize}
        \item \methoddesc{\public}{}{void}{filter}{}
        {Filters all Elements by changing their Filtered attribute.}{}{}
    \end{itemize}
}{}
\classdesc{Settings}{
Keeps track off the settings
}{
    \begin{itemize}
        \item \atrdesc{\private}{}{Language}{currentLanguage}{The currently active Language}
        \item \atrdesc{\private}{}{ColorTheme}{colors}{The currently active ColorTheme}
        \atrdesc{\private}{}{String}{fontType}{The set Font type}
        \atrdesc{\private}{}{int}{fontSize}{Ther set Font Size}
        \atrdesc{\private}{}{int}{barCount}{The number of bars set to be displayed in a bar chart}
        \atrdesc{\private}{}{int}{undoCount}{The set number of undoable Actions saved in the ActionHistory}
        \atrdesc{\private}{}{boolean}{editorOpenedFirst}{If true, the Text Editor opens first upon opening an Element in the works space.}
    \end{itemize}
}{
    \begin{itemize}
        \item \constrdesc{\public}{Settings}{}{}{}
    \end{itemize}
}{
    \begin{itemize}
        \item \methoddesc{\public}{}{void}{setLanguage}{\type{LanguageEnum} language}
        {}{
        \begin{itemize}
            \item \paramdesc{language}{}
        \end{itemize}
        }{}
        \item \methoddesc{\public}{}{void}{setColorTheme}{\type{ColorThemeEnum} colorTheme}
        {}{
        \begin{itemize}
            \item \paramdesc{colorTheme}{}
        \end{itemize}
        }{}
        \item \methoddesc{\public}{}{void}{setFontType}{\type{String} fontType}
        {}{
        \begin{itemize}
            \item \paramdesc{fontType}{}
        \end{itemize}
        }{}
        \item \methoddesc{\public}{}{void}{setFontSize}{\type{int} fontSize}
        {}{
        \begin{itemize}
            \item \paramdesc{fontSize}{}
        \end{itemize}
        }{}
        \item \methoddesc{\public}{}{void}{setBarCount}{\type{int} barCount}
        {}{
        \begin{itemize}
            \item \paramdesc{barCount}{}
        \end{itemize}
        }{}
        \item \methoddesc{\public}{}{void}{setUndoCount}{\type{int} undoCount}
        {}{
        \begin{itemize}
            \item \paramdesc{undoCount}{}
        \end{itemize}
        }{}
        \item \methoddesc{\public}{}{void}{setEditorOpenedFirst}{\type{boolean} editorOpenedFirst}
        {}{
        \begin{itemize}
            \item \paramdesc{editorOpenedFirst}{}
        \end{itemize}
        }{}
        \item \methoddesc{\public}{}{LanguageEnum}{getLanguage}{}
        {}{}{}
        \item \methoddesc{\public}{}{colorTheme}{getColorTheme}{}
        {}{}{}
        \item \methoddesc{\public}{}{String}{getFontType}{}
        {}{}{}
        \item \methoddesc{\public}{}{int}{getFontSize}{}
        {}{}{}
        \item \methoddesc{\public}{}{int}{getBarCount}{}
        {}{}{}
        \item \methoddesc{\public}{}{int}{getUndoCount}{}
        {}{}{}
        \item \methoddesc{\public}{}{boolean}{getEditorOpenedFirst}{}
        {}{}{}
        \item \methoddesc{\private}{}{void}{notify}{}
        {}{}{}
    \end{itemize}
}{}

\classdesc{ColorTheme}{
Stores the colors included in the color theme.
}{
    \begin{itemize}
        \item \atrdesc{\private}{}{int[]}{colors}{The colors corresponding to the theme}
    \end{itemize}
}{
    \begin{itemize}
        \item \constrdesc{\public}{ColorTheme}{}{}{}
    \end{itemize}
}{
    \begin{itemize}
        \item \methoddesc{\public}{}{int[]}{getColors}{}
        {}{}{}
    \end{itemize}
}
\classdesc{Language}{
Stores the snippets of text that correspond to a specific language
}{
    \begin{itemize}
        \item \atrdesc{\private}{}{String[]}{librety}{The element, attributes, modes names in the specified language}
        \atrdesc{\private}{}{String[]}{menus}{The interface menu names in the specified language}
        \atrdesc{\private}{}{String[]}{methods}{The method names in the specified language}
        \atrdesc{\private}{}{String[]}{attributes}{The attribute names in the specified language}
        \atrdesc{\private}{}{String[]}{errors}{The error texts in the specified language}
        \atrdesc{\private}{}{String[]}{misc}{any other text element in the specified language}
    \end{itemize}
}{
    \begin{itemize}
        \item \constrdesc{\public}{Language}{}{}{}
    \end{itemize}
}{}
\enumdesc{Mode}{
Keeps track of the mode used to calculate the attribute used on the filter
}{
    \begin{itemize}
        \item \fielddesc{Mode}{MAX}{Maximum: Takes the maximum value for the attribute}
        \item \fielddesc{Mode}{MIN}{Minimum: Takes the minimal value for the attribute}
        \item \fielddesc{Mode}{MIN}{Average: Takes the average value for the attribute}
        \item \fielddesc{Mode}{Med}{Median: Takes the median value for the attribute}
    \end{itemize}
}{}
\enumdesc{Operation}{Keeps track of the type of operation done by the Filter}{
    \begin{itemize}
        \item \fielddesc{Operation}{LESS}{Checks if attribute is less than value}
        \item \fielddesc{Operation}{EQUAL}{Checks if attribute is equal than value}
        \item \fielddesc{Operation}{BIGGER}{Checks if attribute is bigger than value}
    \end{itemize}
}{}
\enumdesc{ColorThemeEnum}{Keeps track of the Color themes}{
    \begin{itemize}
        \item \fielddesc{ColorThemeEnum}{THEME1}{Theme name 1}
        \item \fielddesc{ColorThemeEnum}{THEME2}{Theme name 2}
        \item \fielddesc{ColorThemeEnum}{AND-SO-ON}{And further theme names}
    \end{itemize}
}{}
\enumdesc{LanguageEnum}{Keeps track of the languages available for the program}{
    \begin{itemize}
        \item \fielddesc{LanguageEnum}{ENGLISH}{The default English language}
        \item \fielddesc{LanguageEnum}{DEUTSCH}{German}
        \item \fielddesc{LanguageEnum}{TURK}{Turkish}
        \item \fielddesc{LanguageEnum}{SHQIP}{Albanian}
        \item \fielddesc{LanguageEnum}{FRANCAIS}{French}
    \end{itemize}
}{}
}

\packagedesc{model.exceptions}{
\classdesc{TooManyPanelsOpenedException}{}{}{}{}{}
\classdesc{SearchedStringNotFoundException}{}{}{}{}{}
\classdesc{InvalidNameException}{}{}{}{}{}
\classdesc{InvalidComparisonException}{}{}{}{}{}
\classdesc{InvalidFileFormatException}{}{}{}{}{}
\classdesc{ExceedingFileSizeException}{}{}{}{}{}
\classdesc{TooManySelectedException}{}{}{}{}{}
}


\packagedesc{controller.listeners}{
\classdesc{EventManager}{\public \class EventManager \decl{}{Holds lists of all listeners and manages their subscription.}}{
    \begin{itemize}
        \item \atrdesc{\private}{}{MainWindow}{view}{View component}
        \item \atrdesc{\private}{}{Model}{model}{Main model data}
        \item \atrdesc{\private}{}{Map<Event, EventListener>}{listeners}{Map of all view listeners for subscription and initialization.}
        
    \end{itemize}
    }{
    \begin{itemize}
        \item \constrdesc{\public}{EventManager}{\type{MainWindow} {view}, \type{Model} model}{}{}
    \end{itemize}
}{ 
    \begin{itemize}
        \item \methoddesc{\public}{}{Map<Event, EventListener>}{getListeners}{}{Returns a list of the listeners}{}{}
        \item \methoddesc{\public}{}{void}{subscribeListener}{\type{EventListener} listener}{Registers a listener}{\begin{itemize}
            \item \paramdesc{listener}{Listener which will be registered.}
        \end{itemize}}{}
        \item \methoddesc{\public}{}{void}{removeListener}{\type{EventListener} listener}{Removes the given listener}{\begin{itemize}
            \item \paramdesc{listener}{Listener which will be removed.}
        \end{itemize}}{}
    \end{itemize}}{}

\enumdesc{Event}
{enum Event \decl{}{Identifies each view listener for subscription.}
}{
    \begin{itemize}
        \item \fielddesc{Event}{LOAD}{}
        \item\fielddesc{Event}{OPEN}{}
        \item \fielddesc{Event}{DELETE}{}
        \item \fielddesc{Event}{REMOVE}{}
        \item \fielddesc{Event}{SAVE}{}
        \item \fielddesc{Event}{SAVEAS}{}
        \item \fielddesc{Event}{SELECT}{}
        \item \fielddesc{Event}{CREATE}{}
        \item \fielddesc{Event}{SEARCH}{}
        \item \fielddesc{Event}{EDIT}{}
        \item \fielddesc{Event}{RENAME}{}
        \item \fielddesc{Event}{MERGE}{}
        \item \fielddesc{Event}{COPY}{}
        \item \fielddesc{Event}{MOVE}{}
        \item \fielddesc{Event}{PASTE}{}
        \item \fielddesc{Event}{UNDO}{}
        \item \fielddesc{Event}{REDO}{}
        \item \fielddesc{Event}{SCALE}{}
        \item \fielddesc{Event}{COMPARE}{}
        \item \fielddesc{Event}{INTERPOLATE}{}
        \item \fielddesc{Event}{ADDFILTER}{}
        \item \fielddesc{Event}{REMOVEFILTER}{}
        \item \fielddesc{Event}{LOADPROJECT}{}
        \item \fielddesc{Event}{EXPORTPROJECT}{}
        \item \fielddesc{Event}{SETTINGS}{}
        \item \fielddesc{Event}{SCSETTINGS}{}
        \item \fielddesc{Event}{PRSETTINGS}{}
        \item \fielddesc{Event}{LASETTINGS}{}
    \end{itemize}
}{}


\classdesc{LoadLibraryListener}{public class LoadLibraryListener implements ActionListener \decl{}{Listener for loading a liberty file into the application.}}{
    \begin{itemize}
        \item \atrdesc{\private}{}{Command}{command}{Command for loading a liberty file}
    \end{itemize}
}{
    \begin{itemize}
        \item \constrdesc{\public}{LoadLibraryListener}{}{}{}
    \end{itemize}
}{
    \begin{itemize}
        \item \methoddesc{\public}{}{void}{actionPerformed}{\type{ActionEvent} e}
        {}{\begin{itemize}
            \item \paramdesc{e}{Performed action event on the component.}
            \end{itemize}}{}
    \end{itemize}
}{}
\classdesc{OpenElementListener}{public class OpenElementListener implements ActionListener, TreeSelectionListener \decl{}{Listener for opening an element in the working area of the application.}
}{
    \begin{itemize}
        \item \atrdesc{\private}{}{Command}{command}{Command for opening a liberty file in working area}
        \item \atrdesc{\private}{}{Element}{element}{Element which will be opened in the working area.}
    \end{itemize}
}{
    \begin{itemize}
        \item \constrdesc{\public}{OpenElementListener}{}{}{}
    \end{itemize}
}{
    \begin{itemize}
        \item \methoddesc{\public}{}{void}{actionPerformed}{\type{ActionEvent} e}
        {}{\begin{itemize}
            \item \paramdesc{e}{Performed action event on the component.}
            \end{itemize}}{}
        \item
        \methoddesc{\public}{}{void}{valueChanged}{\type{TreeSelectionEvent} e}
        {}{\begin{itemize}
            \item \paramdesc{e}{Performed selection event on the component.}
            \end{itemize}}{}
    \end{itemize}
    }{}
\classdesc{DeleteCellListener}{public class DeleteCellListener implements ActionListener, TreeSelectionListener \decl{}{Listener for deleting a cell permanently from a library.}}{
    \begin{itemize}
        \item \atrdesc{\private}{}{Command}{command}{Command for deleting a cell.}
        \item \atrdesc{\private}{}{Cell[]}{cells}{Cells to be deleted.}
    \end{itemize}
}{
    \begin{itemize}
        \item \constrdesc{\public}{DeleteListener}{}{}{}
    \end{itemize}
}{  
    \begin{itemize}
        \item \methoddesc{\public}{}{void}{actionPerformed}{\type{ActionEvent} e}
        {}{\begin{itemize}
            \item \paramdesc{e}{Performed action event on the component.}
            \end{itemize}}{}
         \item
        \methoddesc{\public}{}{void}{valueChanged}{\type{TreeSelectionEvent} e}
        {}{\begin{itemize}
            \item \paramdesc{e}{Performed selection event on the component.}
            \end{itemize}}{}
    \end{itemize}
}{}
\classdesc{RemoveListener}{public class RemoveListener implements ActionListener, TreeSelectionListener \decl{}{Listener for removing an element from the project.}}{
    \begin{itemize}
        \item \atrdesc{\private}{}{Command}{command}{Command for removing a library}
        \item \atrdesc{\private}{}{Element[]}{elements}{Libraries to be removed.}
    \end{itemize}
}{
    \begin{itemize}
        \item \constrdesc{\public}{RemoveListener}{}{}{}
    \end{itemize}
}{
    \begin{itemize}
        \item \methoddesc{\public}{}{void}{actionPerformed}{\type{ActionEvent} e}
        {}{\begin{itemize}
            \item \paramdesc{e}{Performed action event on the component.}
            \end{itemize}}{}
         \item
        \methoddesc{\public}{}{void}{valueChanged}{\type{TreeSelectionEvent} e}
        {}{\begin{itemize}
            \item \paramdesc{e}{Performed action event on the component.}
            \end{itemize}}{}
    \end{itemize}
}{}
\classdesc{EditListener}{public class EditListener \decl{}{Observes the changes in the text editor.}}{
    \begin{itemize}
        \item \atrdesc{\private}{}{Command}{command}{Command for editing a liberty file in the text editor}
        \item \atrdesc{\private}{}{Element}{element}{Edited liberty element}
        \item \atrdesc{\private}{}{String}{newText}{Changed text}
        \item \atrdesc{\private}{}{String}{oldText}{Old content of the text.}
    \end{itemize}
}{
    \begin{itemize}
        \item \constrdesc{\public}{EditListener}{}{}{}
    \end{itemize}
}{
    \begin{itemize}
        \item \methoddesc{\public}{}{void}{actionPerformed}{\type{ActionEvent} e}
        {}{\begin{itemize}
            \item \paramdesc{e}{Performed action event on the component.}
            \end{itemize}}{} 
        \item
        \methoddesc{\public}{}{void}{changesUpdated}{\type{DocumentEvent} e}
        {}{\begin{itemize}
            \item \paramdesc{e}{Performed document event on the component.}
            \end{itemize}}{}
    \end{itemize}}{}
\classdesc{RenameListener}{\public \class RenameListener implements ActionListener, TreeSelectionListener \decl{}{Listener for renaming a liberty file in the outliner.}}{
    \begin{itemize}
        \item \atrdesc{\private}{}{Command}{command}{Command for renaming an element.}
        \item \atrdesc{\private}{}{String}{newName}{New name of the element.}
        \item \atrdesc{\private}{}{Element}{element}{To be renamed element.}
    \end{itemize}
}{
    \begin{itemize}
        \item \constrdesc{\public}{RenameListener}{}{}{}
    \end{itemize}
}{
    \begin{itemize}
        \item \methoddesc{\public}{}{void}{actionPerformed}{\type{ActionEvent} e}
        {}{\begin{itemize}
            \item \paramdesc{e}{Performed action event on the component.}
            \end{itemize}}{}
         \item
        \methoddesc{\public}{}{void}{valueChanged}{\type{TreeSelectionEvent} e}
        {}{\begin{itemize}
            \item \paramdesc{e}{Performed selection event on the component.}
            \end{itemize}}{}
    \end{itemize}}{}
\classdesc{SearchListener}{public class SearchListener implements TextListener \decl{}{Listener for search bar.}}{
    \begin{itemize}
        \item \atrdesc{\private}{}{Model}{model}{Model holding the project data.}
        \item \atrdesc{\private}{}{Panel}{panel}{Panel which contains the search bar}
    \end{itemize}
}{
    \begin{itemize}
        \item \constrdesc{\public}{SearchListener}{\type{Model} model}{}{}
    \end{itemize}
}{
    \begin{itemize}
        \item \methoddesc{\public}{}{void}{textValueChanged}{\type{TextEvent} e}
        {}{\begin{itemize}
            \item \paramdesc{e}{Performed text event on the component.}
            \end{itemize}}{}
    \end{itemize}}{}
\classdesc{CreateLibraryListener}{public class CreateLibraryListener implements ActionListener \decl{}{Listener for creating a new library.}}{
    \begin{itemize}
        \item \atrdesc{\private}{}{Model}{model}{Model holding the project data.}
    \end{itemize}
}{
    \begin{itemize}
        \item \constrdesc{\public}{CreateLibraryListener}{\type{Model} model}{}{}
    \end{itemize}
}{
    \begin{itemize}
        \item \methoddesc{\public}{}{void}{actionPerformed}{\type{ActionEvent} e}
        {}{\begin{itemize}
            \item \paramdesc{e}{Performed action event on the component.}
            \end{itemize}}{}
    \end{itemize}}{}
\classdesc{SaveListener}{public class SaveListener implements ActionListener \decl{}{Listener for save button.}}{
    \begin{itemize}
        \item \atrdesc{\private}{}{Model}{model}{Model holding the project data.}
    \end{itemize}
}{
    \begin{itemize}
        \item \constrdesc{\public}{SaveListener}{\type{Model} model}{}{}
    \end{itemize}
}{
    \begin{itemize}
        \item \methoddesc{\public}{}{void}{actionPerformed}{\type{ActionEvent} e}
        {}{\begin{itemize}
            \item \paramdesc{e}{Performed action event on the component.}
            \end{itemize}}{}
    \end{itemize}}{}
\classdesc{SaveAsListener}{public class SaveAsListener implements ActionListener \decl{}{Listener for saving changes made in the liberty file as a new file.}}{
    \begin{itemize}
        \item \atrdesc{\private}{}{Model}{model}{Model holding the project data.}
    \end{itemize}
}{
    \begin{itemize}
        \item \constrdesc{\public}{SaveAsListener}{\type{Model} model}{}{}
    \end{itemize}
}{
    \begin{itemize}
        \item \methoddesc{\public}{}{void}{actionPerformed}{\type{ActionEvent} e}
        {}{\begin{itemize}
            \item \paramdesc{e}{Performed action event on the component.}
            \end{itemize}}{}
    \end{itemize}}{}
\classdesc{MergeListener}{public class MergeListener implements ActionListener, TreeSelectionListener \decl{}{Listener for merging multiple libraries.}}{
    \begin{itemize}
        \item \atrdesc{\private}{}{Command}{command}{Command for merging action}
        \item \atrdesc{\private}{}{Library[]}{libraries}{Libraries selected for merge action.}
    \end{itemize}
}{
    \begin{itemize}
        \item \constrdesc{\public}{MergeListener}{}{}{}
    \end{itemize}
}{
    \begin{itemize}
        \item \methoddesc{\public}{}{void}{actionPerformed}{\type{ActionEvent} e}
        {}{\begin{itemize}
            \item \paramdesc{e}{Performed action event on the component.}
            \end{itemize}}{}
         \item
        \methoddesc{\public}{}{void}{valueChanged}{\type{TreeSelectionEvent} e}
        {}{\begin{itemize}
            \item \paramdesc{e}{Performed selection event on the component.}
            \end{itemize}}{}
    \end{itemize}}{}
\classdesc{ScaleListener}{public class ScaleListener implements ActionListener \decl{}{Listener for scaling.}}{
    \begin{itemize}
        \item \atrdesc{\private}{}{Command}{command}{Command for scaling action}
        \item \atrdesc{\private}{}{Attribute}{attribute}{Attribute to scale.}
        \item \atrdesc{\private}{}{float}{value}{Scaling value}
    \end{itemize}
}{
    \begin{itemize}
        \item \constrdesc{\public}{ScaleListener}{}{}{}
    \end{itemize}
}{
    \begin{itemize}
        \item \methoddesc{\public}{}{void}{actionPerformed}{\type{ActionEvent} e}
        {}{\begin{itemize}
            \item \paramdesc{e}{Performed action event on the component.}
            \end{itemize}}{}
    \end{itemize}}{}
\classdesc{InterpolationListener}{public class InterpolationListener implements ActionListener \decl{}{Listener for interpolation action.}}{
    \begin{itemize}
        \item \atrdesc{\private}{}{Command}{command}{Command for interpolation action.}
    \end{itemize}
}{
    \begin{itemize}
        \item \constrdesc{\public}{InterpolationListener}{}{}{}
    \end{itemize}
}{
    \begin{itemize}
        \item \methoddesc{\public}{}{void}{actionPerformed}{\type{ActionEvent} e}
        {}{\begin{itemize}
            \item \paramdesc{e}{Performed action event on the component.}
            \end{itemize}}{}
    \end{itemize}}{}
\classdesc{UndoListener}{public class UndoListener implements ActionListener\decl{}{Listener for the undo button.}}{
    \begin{itemize}
        \item \atrdesc{\private}{}{Command}{command}{Command for the undo action.}
    \end{itemize}
}{
    \begin{itemize}
        \item \constrdesc{\public}{UndoListener}{}{}{}
    \end{itemize}
}{
    \begin{itemize}
        \item \methoddesc{\public}{}{void}{actionPerformed}{\type{ActionEvent} e}
        {}{\begin{itemize}
            \item \paramdesc{e}{Performed action event on the component.}
            \end{itemize}}{}
    \end{itemize}}{}
\classdesc{RedoListener}{public class RedoListener implements ActionListener \decl{}{Listener for the redo button.}}{
    \begin{itemize}
        \item \atrdesc{\private}{}{Command}{command}{Command for the redo action.}
    \end{itemize}
}{
    \begin{itemize}
        \item \constrdesc{\public}{RedoListener}{}{}{}
    \end{itemize}
}{
    \begin{itemize}
        \item \methoddesc{\public}{}{void}{actionPerformed}{\type{ActionEvent} e}
        {}{\begin{itemize}
            \item \paramdesc{e}{Performed action event on the component.}
            \end{itemize}}{}
    \end{itemize}}{}
\classdesc{CompareListener}{public class CompareListener implements ActionListener, TreeSelectionListener \decl{}{Listener for the compare action.}}{
    \begin{itemize}
        \item \atrdesc{\private}{}{Command}{command}{Command for comparing.}
        \item \atrdesc{\private}{}{Element}{element1}{First element to compare}
        \item \atrdesc{\private}{}{Element}{element2}{Second element to compare}
    \end{itemize}
}{
    \begin{itemize}
        \item \constrdesc{\public}{CompareListener}{}{}{}
    \end{itemize}
}{
    \begin{itemize}
        \item \methoddesc{\public}{}{void}{actionPerformed}{\type{ActionEvent} e}
        {}{\begin{itemize}
            \item \paramdesc{e}{Performed action event on the component.}
            \end{itemize}}{}
         \item
        \methoddesc{\public}{}{void}{valueChanged}{\type{TreeSelectionEvent} e}
        {}{\begin{itemize}
            \item \paramdesc{e}{Performed selection event on the component.}
            \end{itemize}}{}
    \end{itemize}}{}
\classdesc{MoveListener}{public class MoveListener implements ActionListener, TreeSelectionListener \decl{}{Listener for moving selected cells to another library.}}{
    \begin{itemize}
        \item \atrdesc{\private}{}{Command}{command}{Move action.}
        \item \atrdesc{\private}{}{Element[]}{element}{Selected cells.}
    \end{itemize}
}{
    \begin{itemize}
        \item \constrdesc{\public}{MoveListener}{}{}{}
    \end{itemize}
}{
    \begin{itemize}
        \item \methoddesc{\public}{}{void}{actionPerformed}{\type{ActionEvent} e}
        {}{\begin{itemize}
            \item \paramdesc{e}{Performed action event on the component.}
            \end{itemize}}{}
         \item
        \methoddesc{\public}{}{void}{valueChanged}{\type{TreeSelectionEvent} e}
        {}{\begin{itemize}
            \item \paramdesc{e}{Performed selection event on the component.}
            \end{itemize}}{}
    \end{itemize}}{}
\classdesc{CopyListener}{public class CopyListener implements ActionListener, TreeSelectionListener \decl{}{Listener for copying selected cells to another library.}}{
    \begin{itemize}
        \item \atrdesc{\private}{}{Command}{command}{Copy action}
        \item \atrdesc{\private}{}{Element[]}{element}{Elements selected for copying.}
    \end{itemize}
}{
    \begin{itemize}
        \item \constrdesc{\public}{CopyListener}{}{}{}
    \end{itemize}
}{
    \begin{itemize}
        \item \methoddesc{\public}{}{void}{actionPerformed}{\type{ActionEvent} e}
        {}{\begin{itemize}
            \item \paramdesc{e}{Performed action event on the component.}
            \end{itemize}}{}
         \item
        \methoddesc{\public}{}{void}{valueChanged}{\type{TreeSelectionEvent} e}
        {}{\begin{itemize}
            \item \paramdesc{e}{Performed selection event on the component.}
            \end{itemize}}{}
    \end{itemize}}{}
\classdesc{PasteListener}{public class PasteListener implements ActionListener, TreeSelectionEvent \decl{}{Listener for copying selected cells to another library.}}{
    \begin{itemize}
        \item \atrdesc{\private}{}{Command}{command}{Copy action}
        \item \atrdesc{\private}{}{Library}{targetLibrary}{Target library for copy and move actions} 
    \end{itemize}
}{
    \begin{itemize}
        \item \constrdesc{\public}{PasteListener}{}{}{}
    \end{itemize}
}{
    \begin{itemize}
        \item \methoddesc{\public}{}{void}{actionPerformed}{\type{ActionEvent} e}
        {}{\begin{itemize}
            \item \paramdesc{e}{Performed action event on the component.}
            \end{itemize}}{}
         \item
        \methoddesc{\public}{}{void}{valueChanged}{\type{TreeSelectionEvent} e}
        {}{\begin{itemize}
            \item \paramdesc{e}{Performed selection event on the component.}
            \end{itemize}}{}
    \end{itemize}}{}

\classdesc{AddFilterListener}{public class AddFilterListener implements ActionListener \decl{}{Listener for add filter button.}}{
    \begin{itemize}
        \item \atrdesc{\private}{}{Command}{command}{Filter adding action.}
        \item \atrdesc{\private}{}{Filter}{filter}{Filter to be added.}
    \end{itemize}
}{
    \begin{itemize}
        \item \constrdesc{\public}{AddFilterListener}{}{}{}
    \end{itemize}
}{
    \begin{itemize}
        \item \methoddesc{\public}{}{void}{actionPerformed}{\type{ActionEvent} e}
        {}{\begin{itemize}
            \item \paramdesc{e}{Performed action event on the component.}
            \end{itemize}}{}
    \end{itemize}}{}
\classdesc{RemoveFilterListener}{public class RemoveFilterListener implements ActionListener \decl{}{Listener for remove filter button.}}{
    \begin{itemize}
        \item \atrdesc{\private}{}{Command}{command}{Filter removing action.}
        \item \atrdesc{\private}{}{Filter}{filter}{Filter to be removed.}
    \end{itemize}
}{
    \begin{itemize}
        \item \constrdesc{\public}{RemoveFilterListener}{}{}{}
    \end{itemize}
}{
    \begin{itemize}
        \item \methoddesc{\public}{}{void}{actionPerformed}{\type{ActionEvent} e}
        {}{\begin{itemize}
            \item \paramdesc{e}{Performed action event on the component.}
            \end{itemize}}{}
    \end{itemize}}{}

\classdesc{LoadProjectListener}{public class LoadProjectListener implements ActionListener \decl{}{Listener for importing a project}}{
    \begin{itemize}
        \item \atrdesc{\private}{}{Command}{command}{Command for importing a project.}
        \item \atrdesc{\private}{}{Model}{model}{Model instance}
    \end{itemize}
}{
    \begin{itemize}
        \item \constrdesc{\public}{LoadProjectListener}{Model model}{}{}
    \end{itemize}
}{
    \begin{itemize}
        \item \methoddesc{\public}{}{void}{actionPerformed}{\type{ActionEvent} e}
        {}{\begin{itemize}
            \item \paramdesc{e}{Performed action event on the component.}
            \end{itemize}}{}
    \end{itemize}}{}
    
\classdesc{ExportProjectListener}{public class ExportProjectListener implements ActionListener \decl{}{Listener for exporting a project as a CSV file.}}{
    \begin{itemize}
        \item \atrdesc{\private}{}{Command}{command}{Command for exporting a project.}
        \item \atrdesc{\private}{}{Model}{Model}{Model instance}
    \end{itemize}
}{
    \begin{itemize}
        \item \constrdesc{\public}{ExportProjectListener}{Model model}{}{}
    \end{itemize}
}{
    \begin{itemize}
        \item \methoddesc{\public}{}{void}{actionPerformed}{\type{ActionEvent} e}
        {}{\begin{itemize}
            \item \paramdesc{e}{Performed action event on the component.}
            \end{itemize}}{}
    \end{itemize}}{}
    
\classdesc{SettingsListener}{public class SettingsListener implements ActionListener \decl{}{Listener for main settings window.}}{
    \begin{itemize}
        \item \atrdesc{\private}{}{Settings}{settings}{Settings data in the model.}
    \end{itemize}
}{
    \begin{itemize}
        \item \constrdesc{\public}{SettingsListener}{Model model}{}{}
    \end{itemize}
}{
    \begin{itemize}
        \item \methoddesc{\public}{}{void}{actionPerformed}{\type{ActionEvent} e}
        {}{\begin{itemize}
            \item \paramdesc{e}{Performed action event on the component.}
            \end{itemize}}{}
    \end{itemize}}{}

\classdesc{ShortcutSettingsListener}{public class ShortcutSettingsListener implements ActionListener \decl{}{Listener for changing shortcuts component.}}{
    \begin{itemize}
        \item \atrdesc{\private}{}{Shortcuts}{shortcuts}{Shortcut data in the model.}
    \end{itemize}
}{
    \begin{itemize}
        \item \constrdesc{\public}{ShortcutSettingsListener}{Model model}{}{}
    \end{itemize}
}{
    \begin{itemize}
        \item \methoddesc{\public}{}{void}{actionPerformed}{\type{ActionEvent} e}
        {}{\begin{itemize}
            \item \paramdesc{e}{Performed action event on the component.}
            \end{itemize}}{}
    \end{itemize}}{}
\classdesc{PreferencesSettingsListener}{public class PreferencesSettingsListener implements ActionListener \decl{}{Listener for application settings}}{
    \begin{itemize}
        \item \atrdesc{\private}{}{Model}{model}{Preferences data in the model.}
    \end{itemize}
}{
    \begin{itemize}
        \item \constrdesc{\public}{PreferencesSettingsListener}{Model model}{}{}
    \end{itemize}
}{
    \begin{itemize}
        \item \methoddesc{\public}{}{void}{actionPerformed}{\type{ActionEvent} e}
        {}{\begin{itemize}
            \item \paramdesc{e}{Performed action event on the component.}
            \end{itemize}}{}
    \end{itemize}}{}
\classdesc{LanguageSettingsListener}{public class LanguageSettingsListener implements ActionListener \decl{}{Listener for language dropdown component.}}{
    \begin{itemize}
        \item \atrdesc{\private}{}{Model}{Model}{Model holding the language data.}
    \end{itemize}
}{
    \begin{itemize}
        \item \constrdesc{\public}{LanguageSettingsListener}{Model model}{}{}
    \end{itemize}
}{
    \begin{itemize}
        \item \methoddesc{\public}{}{void}{actionPerformed}{\type{ActionEvent} e}
        {}{\begin{itemize}
            \item \paramdesc{e}{Performed action event on the component.}
            \end{itemize}}{}
    \end{itemize}}{}
    
}

\chapter{Class Diagrams}
\section{Model}
\includeimage{}{0.1}{Model}{Model}{Model}
\section{View}
\includeimage{}{0.035}{View}{View}{View}
\section{Controller}
\includeimage{}{0.062}{Controllerv}{Controller}{Controller}
\chapter{Architecture}{
The Model-View-Controller architecture is used in the application to separate the internal data regarding liberty files
from how they will be presented to the user. This allows an organized implementation for the GUI.\newline

The view component presents the user a visualization of the liberty file data
such the hierarchical structure of the library and the by using diagrams, bar charts, heat maps, cell representations and a hierarchy view.
It also contains the UI logic, which the user interacts with.\newline 

Controller is responsible for observing the changes made in the view and responding to these changes by sending 
the user input to the corresponding components of the model. \newline

Model stores and manages the liberty file data. Once model takes the user inputs from the controller, it makes the necessary changes
in the liberty data and updates the view accordingly.\newline}

\chapter{Design Patterns}
\pattern{Strategy}{
    The strategy pattern is used by the given below, along with their roles:
    \leavevmode \\ \leavevmode \\
    \textbf{Abstract strategy:} Provides an interface for the available strategies. \leavevmode \\
    \textbf{Concrete strategy:} Contains the concrete implementation of the strategy it represents. \leavevmode \\
    \begin{itemize}
        \item In \refer{\lblroot:view.diagrams.overlayer}{view.diagrams.overlayer} package:
        \begin{itemize}
            \item \refer{\lblroot:view.diagrams.overlayer:IDiagramOverlayStrategy}{IDiagramOverlayStrategy} as abstract strategy.
            \item \refer{\lblroot:view.diagrams.overlayer:FunctionGraphOverlayStrategy}{FunctionGraphOverlayStrategy}, \refer{\lblroot:view.diagrams.overlayer:HistogramOverlayStrategy}{HistogramOverlayStrategy}, \refer{\lblroot:view.diagrams.overlayer:BarChartOverlayStrategy}{BarChartOverlayStrategy} as concrete strategies.
        \end{itemize}
        \item In \refer{\lblroot:view.diagrams.data}{view.diagrams.data} package:
        \begin{itemize}
            \item \refer{\lblroot:view.diagrams.data:DiagramDataFormatter}{DiagramDataFormatter} as abstract strategy.
            \item \refer{\lblroot:view.diagrams.data:ArrayListDataFormatter}{ArrayListDataFormatter} and \refer{\lblroot:view.diagrams.data:ArrayDataFormatter}{ArrayDataFormatter} as concrete strategies.
        \end{itemize}
    \end{itemize}
}
\pattern{Builder}{
    The builder pattern is used by the given below, along with their roles:
    \leavevmode \\ \leavevmode \\
    \textbf{Director:} The manager of builders. \leavevmode \\
    \textbf{Abstract builder:} Provides an interface for the available builders. \leavevmode \\
    \textbf{Builder:} Contains the concrete implementation for construction steps. \leavevmode \\
    \textbf{Product:} The products of the builders. \leavevmode \\
    \begin{itemize}
        \item \refer{\lblroot:view.diagrams:DiagramDirector}{DiagramDirector} as director.
        \item \refer{\lblroot:view.diagrams.builder:DiagramBuilder}{DiagramBuilder} as abstract builder.
        \item \refer{\lblroot:view.diagrams.builder:BarChartBuilder}{BarChartBuilder}, \refer{\lblroot:view.diagrams.builder:HistogramBuilder}{HistogramBuilder}, \refer{\lblroot:view.diagrams.builder:FunctionGraphBuilder}{FunctionGraphBuilder}, \refer{\lblroot:view.diagrams.builder:HeatMapBuilder}{HeatMapBuilder} as builders.
        \item \refer{\lblroot:view.diagrams.type:Histogram}{Histogram}, \refer{\lblroot:view.diagrams.type:BarChart}{BarChart}, \refer{\lblroot:view.diagrams.type:HeatMap}{HeatMap}, \refer{\lblroot:view.diagrams.type:FunctionGraph}{FunctionGraph} as products.
    \end{itemize}
}
\pattern{Singleton}{
    The singleton pattern is used by the given below:
    \leavevmode \\
    \begin{itemize}
        \item \refer{\lblroot:view.diagrams:SettingsProvider}{SettingsProvider}
        \item \refer{\lblroot:view.diagrams:DiagramDirector}{DiagramDirector}
        \item \refer{\lblroot:view.diagrams.components:DiagramComponentFactory}{DiagramComponentFactory}
        \item \refer{\lblroot:view.diagrams.indicator:DiagramViewHelperFactory}{DiagramViewHelperFactory}
        \item \refer{\lblroot:view.diagrams.components:HoverLabel}{HoverLabel}
    \end{itemize}
}
\pattern{Facade}{
    The facade pattern is used by the given below:
    \leavevmode \\
    \begin{itemize}
        \item \refer{\lblroot:view.diagrams:DiagramDirector}{DiagramDirector}
        \item \refer{\lblroot:view.diagrams.components:DiagramComponentFactory}{DiagramComponentFactory}
        \item \refer{\lblroot:view.diagrams.indicator:DiagramViewHelperFactory}{DiagramViewHelperFactory}
    \end{itemize}
}
\pattern{Observer}{The listeners in the \refer{\lblroot:controller.listeners}{Controller} package are all implementing several subclasses of \type{EventListener} from java.util package. These listeners are observing the events on the components they were added to while passing the input to the Model package and it's corresponding components.}
\pattern{Command}
{The Command design pattern is used in the model package.\newline
All the action classes in the commands package use the Command interface to easily undo or redo executed actions.}
\pattern{Composite}{
    The Swing library makes use of the Composite Pattern in order to enable a hierarchical structure of components. The composite/component packages make use of this pattern and extend the classes with additional functionality.
}

\chapter{Sequence Diagrams}
\section{Mandatory Functional Requirements}
\subsection{FR-1 and FR-2}
\includeimage{}{0.23}{FR1-2}{Sequence Diagram showing interactions for FR-1 and FR-2}{FR-1 and FR-2}
\subsection{FR-4}
\includeimage{}{0.32}{FR-4}{Sequence Diagram showing interactions for FR-4}{FR-4}
\subsection{FR-5, FR-6 and FR-7}
\includeimage{}{0.32}{DiagramCreation}{Sequence Diagram that shows how \refer{\lblroot:view.diagrams:IDiagram}{IDiagrams} are created}{Diagram Creation}
\subsection{FR-8}
\includeimage{}{0.32}{DiagramComparison}{Sequence Diagram that shows how \refer{\lblroot:view.diagrams:IDiagram}{IDiagrams} are overlaid for comparison.}{Diagram Comparison}
\subsection{FR-9}
\includeimage{}{0.42}{FR-9}{Sequence Diagram showing interactions for FR-9}{FR-9}
\subsection{FR-10}
\includeimage{}{0.35}{FR10}{Sequence Diagram showing interactions for FR-10}{FR-10}
\subsection{FR-12 and FR-13}
\includeimage{}{0.20}{FR-12,13}{Sequence diagram showing interactions for FR-12 and FR-13}{Merging Libraries and Resolving Merge Conflicts}
\subsection{FR-14}
\includeimage{}{0.27}{FR-14}{Sequence diagram showing interactions for FR-14}{Copying Cells into another Library}
\subsection{FR-15}
\includeimage{}{0.30}{removingalibrary}{Sequence diagram showing interactions for FR-15}{Removing Libraries}
\subsection{FR-16}
\includeimage{}{0.25}{deletingcell}{Sequence diagram showing interactions for FR-16}{Deleting Cells}
\subsection{FR-17}
\includeimage{}{0.35}{scalingvalues}{Sequence diagram showing interactions for FR-17}{Scaling Values}
\subsection{FR-19}
\includeimage{}{0.30}{DiagramStatistics}{Sequence Diagram that shows how statistics are shown on top of \refer{\lblroot:view.diagrams:IDiagram}{IDiagrams}.}{Diagram Statistics}
\subsection{FR-21}
\includeimage{}{0.35}{FR-21}{Sequence diagram showing interactions for FR-21}{Saving a Library as a new File}
\section{Optional Functional Requirements}


\chapter{Requirements Changes}
\pattern{Mandatory Requirements}
{All mandatory functional requirements are covered by the design with class diagrams and sequence diagrams. \newline


A change has been made in the mandatory requirement "FR-5: Displaying a Library". As in the requirement phase decided, library level display would show statistics of the library for every index. However, it was decided during the design phase, that it would be more plausible, if the statistics were shown for each cell instead of for each index in the library level display. This would offer the user a better overview of the statistics. \newline

This change would mean that there would only be bar charts (a bar for each cell) in the library level display instead of the mentioned "histogram" and "heatmap" in FR-5. This of course also changes the way libraries are compared (in FR-8) - the comparison of libraries would only feature bar charts. }
\pattern{Optional Requirements}
{The following optional requirements are either covered completely, or fit into existing classes and structures. 
\begin{itemize}
    \item {FRO-1: Moving cells to another library}
    \item {FRO-2: Renaming elements}
    \item {FRO-3: Creating Liberty files from scratch}
    \item {FRO-5: Exporting data as CSV}
    \item {FRO-6: Resizing visualizer}
    \item {FRO-11: Saving the state of the project} 
    \item {FRO-12: Loading a project}
    \item {FRO-16: Interpolating}
    \item {FRO-17: Changing GUI appearance} 
    \item {FRO-18: Changing the preferences}
    \item {FRO-20: Shortcuts} 
    \item {FRO-21: Undo/Redo} 
    \item {FRO-24: Setting filters for a search} 
    \item {FRO-25: Setting default filters} 
\end{itemize}}
The remaining optional requirements are not covered in the design phase, but the current design allows for easy extendability.

\chapter{Project Schedule}
\end{document}
